\A
{RE-PENSANDO UN ESTUDIO CONFIRMATORIO COMO “ADAPTATIVO” USANDO EL ENFOQUE DE LA PROBABILIDAD PREDICTIVA Y LOS MODELOS DE REGRESSION POR CUANTILES}
{\Presenting{CARMEN VALENZUELA}$^1$\index{VALENZUELA, C}, PATRICIA LORENZO-LUACES$^1$\index{LORENZO-LUACES, P}, CAMILO RODRÍGUEZ$^1$\index{RODRÍGUEZ, C}, ELIA NENINGER$^2$\index{NENINGER, E}, ARASAY MONTES$^1$\index{MONTES, A} y TANIA CROMBET$^1$\index{CROMBET, T}}
{\Afilliation{$^1$CENTRO DE INMUNOLOGÍA MOLECULAR (CIM), CUBA.}
\Afilliation{$^2$HOSPITAL CLÍNICO QUIRÚRGICO HERMANOS AMEIJEIRAS, CUBA}
\\\Email{carmenv@cim.sld.cu}}
{diseños adaptativos; probabilidad predictiva; regresión por cuantiles} 
 {Otras ciencias de la salud} 
 {Modelos de regresión} 
 {189} 
 {905-1}
{Introducción: Los Ensayos Clínicos Adaptativos son aceptados y reconocidos actualmente. Sin embargo, algunas veces las agencias reguladoras no son entusiastas con las decisiones basadas en estrategias de simulación. El enfoque de la Probabilidad Predictiva ofrece un modo natural de estimar el éxito en el análisis futuro, condicionado a los datos acumulados. Por otro lado, la regresión por cuantiles es de gran flexibilidad en la evaluación del efecto de covariables sobre una variable “tiempo hasta la ocurrencia de un evento”, como análisis complementario a los modelos de regresión de Cox. Desarrollo: Los datos acumulados de 6 estudios exploratorios fueron considerados como información previa para re-pensar el estudio fase III como adaptativo, estimando iterativamente la probabilidad predictiva de éxito basados en las distribución final (análisis bayesiano) y usando el enfoque frecuentista. Se compararon los resultados del modelo de regresión de Cox tradicional con los resultados de la regresión por cuantiles. Principales resultados: Usando ambos enfoques para estimar la probabilidad predictiva, la probabilidad de éxito fue mayor de 0.95 con el 25\% del tamaño de muestra planificado para el estudio fase III, consistente con los resultados finales. El análisis bayesiano muestra algunas ventajas respecto al enfoque frecuentista, considerando diferentes maneras de usar la información previa. Por otro lado, la regresión por cuantiles fue de gran utilidad para comprender el efecto de las covariables en el tiempo de supervivencia a los diferentes niveles de los cuantiles. La influencia de algunas covariables no se mantiene a lo largo de toda la distribución. Conclusiones: Estas herramientas pueden ser cruciales en el proceso de toma de decisiones en enfermedades raras, cuando las alternativas de tratamiento son no efectivas o no accesibles, o cuando existe hetetocedasticidad en los datos. El enfoque de la probabilidad predictiva puede optimizar el diseño de futuros ensayos y contribuir a una mayor eficiencia de los mismos. El supuesto de riesgos proporcionales constituye una restricción al uso de los modelos de Cox y en ese sentido la regresión por cuantiles es más flexible, natural e informativa, permitiendo identificar y modelar las causas de heterogeneidad que impactan en la variable dependiente. Introducción: Los Ensayos Clínicos Adaptativos son aceptados y reconocidos actualmente. Sin embargo, algunas veces las agencias reguladoras no son entusiastas con las decisiones basadas en estrategias de simulación. El enfoque de la Probabilidad Predictiva ofrece un modo natural de estimar el éxito en el análisis futuro, condicionado a los datos acumulados. Por otro lado, la regresión por cuantiles es de gran flexibilidad en la evaluación del efecto de covariables sobre una variable “tiempo hasta la ocurrencia de un evento”, como análisis complementario a los modelos de regresión de Cox. Desarrollo: Los datos acumulados de 6 estudios exploratorios fueron considerados como información previa para re-pensar el estudio fase III como adaptativo, estimando iterativamente la probabilidad predictiva de éxito basados en las distribución final (análisis bayesiano) y usando el enfoque frecuentista. Se compararon los resultados del modelo de regresión de Cox tradicional con los resultados de la regresión por cuantiles. Principales resultados: Usando ambos enfoques para estimar la probabilidad predictiva, la probabilidad de éxito fue mayor de 0.95 con el 25\% del tamaño de muestra planificado para el estudio fase III, consistente con los resultados finales. El análisis bayesiano muestra algunas ventajas respecto al enfoque frecuentista, considerando diferentes maneras de usar la información previa. Por otro lado, la regresión por cuantiles fue de gran utilidad para comprender el efecto de las covariables en el tiempo de supervivencia a los diferentes niveles de los cuantiles. La influencia de algunas covariables no se mantiene a lo largo de toda la distribución. Conclusiones: Estas herramientas pueden ser cruciales en el proceso de toma de decisiones en enfermedades raras, cuando las alternativas de tratamiento son no efectivas o no accesibles, o cuando existe hetetocedasticidad en los datos. El enfoque de la probabilidad predictiva puede optimizar el diseño de futuros ensayos y contribuir a una mayor eficiencia de los mismos. El supuesto de riesgos proporcionales constituye una restricción al uso de los modelos de Cox y en ese sentido la regresión por cuantiles es más flexible, natural e informativa, permitiendo identificar y modelar las causas de heterogeneidad que impactan en la variable dependiente. }
