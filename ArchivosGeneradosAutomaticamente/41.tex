\A
{MODELOS DE CRECIMIENTO PARA GOMPHRENA PULCHELLA Y GOMPHRENA PERENNIS}
{\Presenting{ERICA RAÑA}$^1$\index{RANA, E@RAÑA, E}, MARÍA DEL C OCHOA$^2$\index{OCHOA, M}, SALVERA PRIETO ANGUEIRA$^3$\index{PRIETO ANGUEIRA, S} y SALVADOR CHAILA$^2$\index{CHAILA, S}}
{\Afilliation{$^1$CÁTEDRA DE BIOESTADÍSTICA Y DISEÑO-FACULTAD DE AGRONOMÍA Y AGROINDUSTRIAS. UNIVERSIDAD NACIONAL DE SANTIAGO DEL ESTERO}
\Afilliation{$^2$CÁTEDRA DE MATOLOGÍA - FACULTAD DE AGRONOMÍA Y AGROINDUSTRIAS. UNIVERSIDAD NACIONAL DE SANTIAGO DEL ESTERO}
\Afilliation{$^3$CÁTEDRA DE AGROCLIMATOLOGÍA - FACULTAD DE AGRONOMÍA Y AGROINDUSTRIAS. UNIVERSIDAD NACIONAL DE SANTIAGO DEL ESTERO}
\\\Email{erica.ra511@gmail.com}}
{modelos; crecimiento; gomphrena} 
 {Ciencias agropecuarias} 
 {Modelos de regresión} 
 {41} 
 {94-1}
{El propósito de éste trabajo fue encontrar una función que permita estimar la biomasa (expresada como peso seco en g.pl-1) en función de unidades térmicas (UT) de dos malezas: Gomphrena pulchella y Gomphrena perennis. El peso seco fue determinado en distintos momentos del crecimiento de plantas de G. pulchella emergidas el 19/01/14 y cosechada el 20/05/14 y de plantas de G. perennis emergidas el 02/12/15 y cosechadas el 03/03/16. El conocimiento de dicha función es importante porque permite explicar biológicamente la asignación de la biomasa por planta en función de la acumulación de temperatura y la modificación de las etapas vegetativas y reproductivas entre especies y años. Se ajustaron modelos no lineales para el Peso Seco de G. pulchella y G. perennis usando el módulo de Modelos no lineales Mixtos de Infostat v. 2017. Para corregir la falta de homogeneidad de varianzas, se probaron modelos con distintas funciones de varianza: VarPower, que la relaciona con la media a través de una función de potencia, Var ConstPower, similar a la anterior, pero corrida por una constante y Var Exp, una función exponencial. El modelo adecuado se seleccionó en base al criterio de información de Akaike (AIC), y teniendo en cuenta también el error estándar relativo de los estimadores (error estándar dividido por la estimación del parámetro). En el caso de G. pulchella se seleccionó el modelo de Gompertz para expresar acumulación de materia seca (PS) en función de UT, con una función de varianza Var Power, que resultó la mas adecuada para corregir la heterocedasticidad. PS= alfa*exp(-beta*exp(-gamma*UT))= 21.46*exp(-28.09*exp(-0.00028*UT)) Siendo PS: peso seco de la biomasa; alfa: el valor máximo que alcanza la biomasa; beta: número positivo que desplaza el modelo de izquierda a derecha, gamma: tasa de aumento de la biomasa y UT unidades térmicas. En el caso de G.perennis, el modelo seleccionado es el logístico con función de varianza exponencial. PS=alfa/(1+exp(-(UT-beta)/gamma)) PS=39.52/(1+exp(-(UT-1325.95)/198.04)) Siendo PS: peso seco de la biomasa; alfa: el valor máximo que alcanza la biomasa; beta: número positivo, gamma: tasa de aumento de la biomasa y UT unidades térmicas. }
