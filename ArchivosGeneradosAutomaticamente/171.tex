\A
{ESTUDIO COMPARATIVO SOBRE HÁBITOS ALIMENTARIOS Y ACTIVIDAD FÍSICA DE ALUMNOS DE LA LICENCIATURA EN NUTRICIÓN, BIOQUÍMICA Y BIOTECNOLOGÍA DE LA UNIVERSIDAD NACIONAL DEL LITORAL}
{\Presenting{EUGENIA EMILIA BERTA}\index{BERTA, E} y OLGA BEATRIZ AVILA\index{AVILA, O}}
{\Afilliation{DEPARTAMENTO DE MATEMATICA-FACULTAD BIOQUIMICA Y CIENCIAS BIOLOGICAS-UNL}
\\\Email{berta.eugenia@hotmail.com}}
{alimentación; actividad física; estudiantes universitarios} 
 {Otras aplicaciones} 
 {Otras categorías metodológicas} 
 {171} 
 {347-2}
{Introducción: La etapa universitaria es un periodo decisivo en la consolidación de un estilo de vida saludable, y es de esperar que cuando mayor es la información en nutrición, mejor son los hábitos alimentarios. Sin embargo, a medida que el estudiante adquiere autonomía para decidir comidas y horarios, los factores sociales, culturales y económicos, además de las preferencias alimentarias, van a contribuir al establecimiento del patrón de consumo alimentario, no traduciéndose estos conocimientos en el seguimiento de una dieta equilibrada. El objetivo de este trabajo fue describir los hábitos alimentarios y de actividad física (AF) de estudiantes y su relación con la carrera elegida. Material y método: Estudio transversal realizado el presente año, a través de un cuestionario anónimo autoadministrado a 164 estudiantes del segundo año las carreras de Lic. en Nutrición, Bioquímica y Biotecnología, pertenecientes a la Facultad de Bioquímica y Ciencias Biológicas, con preguntas cerradas acerca de los hábitos y patrones de consumo alimentario, la actividad física y sus factores de influencia. Resultados: La muestra estuvo representada por un 77\% de mujeres y 23\% hombres, con edad promedio de 20,7 años (DE = 2,46). Se observó que la mayoría presenta conductas saludables en las áreas evaluadas, encontrándose una diferencia estadísticamente significativa entre carreras en relación a la realización diaria del desayuno (p=0,043), de las cuatro comidas al día (p=0,009) y de la incorporación de azúcar al mate en quienes lo consumen (p=0,014), resultando los estudiantes de nutrición quienes presentaban los patrones más saludables. Además, en relación a los motivos de elección alimentaria, el valor nutritivo de los alimentos fue un factor a tener en cuenta preferentemente por los estudiantes de Nutrición tanto para el desayuno (p=0,003), almuerzo (p=0,007) y merienda (p=0,000), no así en la cena momento en el cual un mayor porcentaje de alumnos la realizan en familia, dedicándole más tiempo respecto a las demás comidas. En relación a la AF, no se observaron diferencias entre carreras respecto a los minutos semanales realizados, pero sí entre el tipo de actividad elegida, siendo la caminata la preferida entre estudiantes de nutrición (p=0,003). Conclusión: es recomendable propiciar la educación en alimentación saludable y actividad física que beneficien a toda la comunidad universitaria, a través de proyectos de salud y, preferentemente, dentro de los planes curriculares de todas las carreras como parte de una formación integral. }
