\A
{ÍNDICES DE DESARROLLO PSICOMOTOR POR PROVINCIA EN NIÑOS MENORES DE SEIS AÑOS. DATOS DE LA ENCUESTA NACIONAL DE SALUD 2005}
{\Presenting{DIANA M KELMANSKY}$^1$\index{KELMANSKY, DM}, HORACIO LEJARRAGA$^2$\index{LEJARRAGA, H}, ALICIA MASAUTIS$^3$\index{MASAUTIS, A} y FERNANDO NUNES$^4$\index{NUNES, F}}
{\Afilliation{$^1$INSTITUTO DE CALCULO, UBA}
\Afilliation{$^2$UBA}
\Afilliation{$^3$MINISTERIO DE EDUCACIÓN}
\Afilliation{$^4$AUTORIDAD CUENCA MATANZA RIACHUELO (ACUMAR)}
\\\Email{dkelmansky@gmail.com}}
{indicador; desarrollo psicomotor; regresión} 
 {Salud humana} 
 {Modelos de regresión} 
 {17} 
 {39-1}
{Se calculó un índice de desarrollo psicomotor (IDP) para cada provincia en base a datos de la encuesta Nacional de Nutrición y Salud (ENNyS) realizada en 2005. El tamaño muestral con información sobre pautas de desarrollo fue de 13,323 niños (6536 varones). Para cada provincia se estimó la mediana de la edad de cumplimiento de cada pauta ajustando una regresión logística con la variable de respuesta cumple /no cumple y edad cronológica como variable explicativa. El IDP fue calculado sobre la edad de cumplimiento de diez pautas de desarrollo utilizando la recta estimada y= a + bx, donde y es: la diferencia entre la mediana de la edad de cumplimiento para la Referencia Nacional (x) y la mediana de la edad cumplimiento de una pauta. Para facilitar la interpretación de la pendiente el IDP fue calculado IDP = 100*(1 + b). El valor teórico esperable de la pendiente es 0 si los niños no difieren en su desarrollo con los de la Referencia Nacional. En ese caso el IDP = 100. Se calculó el IDP para cada provincia; el rango fue entre 72.1 en el Chaco, hasta 106.4 en Tierra del Fuego. En la mayoría de las provincias los coeficientes de regresión estimados fueron negativos, indicando un aumento progresivo del retraso en la edad de cumplimiento de pautas con la edad de los niños. El coeficiente de correlación entre el IDP por provincia y la mortalidad infantil en 2005 fue extremadamente alto: -0.85, lo que sugiere que ambos indicadores comparten similares determinantes bio- sociales. El signo es negativo porque cuanto mayor es la mortalidad, menor es el IDP. El país dispone ahora de un indicador positivo de salud: el Índice de Desarrollo Psicomotor, simple de recoger, confiable y de bajo costo para ser incorporado a las estadísticas nacionales de salud. }
