\A
{FACTORES DETERMINANTES DE LA ADOPCIÓN DE TICS EN EMPRESAS MANUFACTURERAS DE LA PROVINCIA DE CÓRDOBA}
{\Presenting{PABLO ORTIZ}\index{ORTIZ, P}, MARIANA GUARDIOLA\index{GUARDIOLA, M} y NORMA P. CARO\index{CARO, N}}
{\Afilliation{FACULTAD DE CIENCIAS ECONÓMICAS, UNIVERSIDAD NACIONAL DE CÓRDOBA}
\\\Email{pabort@eco.uncor.edu}}
{tics; componentes principales; análisis de correspondencias múltiples; empresas manufactureras} 
 {Economía} 
 {Métodos multivariados} 
 {49} 
 {112-1}
{El propósito del presente trabajo es indagar sobre los distintos factores que inciden en la adopción de Tecnologías de la Información y la Comunicación (TICs) en empresas manufactureras de la provincia de Córdoba. Dada la gran cantidad de variables, tanto de naturaleza cuantitativa como cualitativa, que influyen en este aspecto resulta menester su análisis conjunto, considerando las correlaciones que puedan existir y, eventualmente, las dimensiones que puedan constituir. A tal efecto, se utilizaron datos de la Encuesta de Innovación y Conducta Tecnológica del año 2014, diseñada e implementada por la Dirección General de Estadística y Censos de Córdoba, la cual cuenta con un módulo específico referido a las TICs. Particularmente, en 2014 fueron relevadas 462 empresas con representatividad provincial. Entre las variables consideradas, un grupo se corresponde con habilidades o capacidades para el uso de las TICs, las cuales fueron construidas como indicadores cuantitativos. En este caso, se empleó la técnica de Análisis de Componentes Principales y se identificaron dos dimensiones de relevancia: la primera caracteriza las habilidades derivadas de la experiencia de los trabajadores en la propia empresa relacionada al uso de PC e internet en sus tareas, al igual que el nivel de disponibilidad de PC; la segunda dimensión, por su parte, está determinada por la formación profesional individual de los trabajadores. Por otro lado, se tomaron en cuenta doce variables vinculadas a la infraestructura en TICs y los usos de Internet dentro de las empresas. Dada la naturaleza cualitativa de estas variables, se utilizó Análisis de Correspondencias Múltiples, definiéndose cuatro dimensiones de relevancia. La primera dimensión se constituye como un factor global de tamaño dada la alta correlación entre las variables, donde los usos de Internet como búsqueda de información y banca electrónica, junto a la disponibilidad de Intranet o Red de Área local, son las de mayor peso. La segunda dimensión engloba la conectividad a Internet y la comunicación online; la tercera, responde a aspectos comerciales: venta, compra y atención al cliente. Finalmente, la cuarta dimensión refiere principalmente al tipo de conexión a Internet.  Esta primera caracterización de las dimensiones que constituyen la adopción de TICs por parte de las empresas, es una primera aproximación para la construcción de indicadores sintéticos que capten, entre otros aspectos, la complejidad intrínseca que significa la consideración de la infraestructura requerida, las políticas internas de las empresas y las habilidades y la formación necesarias por parte de los trabajadores.}
