\A
{HACIA UN APRENDIZAJE PROFUNDO EN LA ASIGNATURA PROBABILIDAD Y ESTADÍSTICA NIVELES ALCANZADOS POR LOS ESTUDIANTES EN UNA ACTIVIDAD EVALUATIVA}
{\Presenting{NOEMÍ MARÍA FERRERI}$^1$\index{FERRERI, N}, RAÚL DAVID KATZ$^1$\index{KATZ, R} y MARÍA EVANGELINA ÁLVAREZ$^2$\index{ALVAREZ, M}}
{\Afilliation{$^1$FACULTAD DE CIENCIAS EXACTAS, INGENIERÍA Y AGRIMENSURA, UNR}
\Afilliation{$^2$FACULTAD DE CIENCIAS ECONÓMICAS Y ESTADÍSTICA, UNR}
\\\Email{nferreri@fceia.unr.edu.ar}}
{probabilidad y estadística; evaluación; variancia; niveles de aprendizaje} 
 {Enseñanza de la estadística} 
 {Otras categorías metodológicas} 
 {81} 
 {183-4}
{En la asignatura Probabilidad y Estadística, venimos trabajando para mejorar los procesos de ense- ñanza y de aprendizaje, ideando permanentemente innovaciones en nuestra práctica docente. Es así que impulsamos en el aula una mayor participación de los estudiantes a través de la realización de actividades en el marco de un trabajo colaborativo, propiciando el planteo de preguntas, la confrontación de diferentes suposiciones y predicciones y la reestructuración de nociones erróneas. Esta forma de trabajo nos ofrece a los docentes interesantes elementos para actuar didácticamente, en beneficio de un mejor aprendizaje. Desde nuestro acompañamiento a los estudiantes en el desarrollo de conocimientos, habilidades, actitudes, etc. y desde las distintas instancias evaluativas, emerge la necesidad de innovaciones en nuestras prácticas, que posibiliten a los estudiantes reflexionar sobre su propio desempeño y tener conciencia sobre el alcance de la comprensión lograda. En este trabajo mostramos una actividad propuesta con el objetivo de indagar sobre la profundidad del aprendizaje alcanzado, en relación al concepto de variancia de un conjunto finito de datos. La misma involucró la definición, el cálculo, la interpretación y el reconocimiento de la misma a partir del gráfico de una distribución de frecuencias. Las respuestas dadas por los estudiantes revelan distintos niveles de aprendizajes alcanzados. Algunos testimonian un aprendizaje profundo por cuanto no sólo escriben correctamente la definición y la calculan correctamente, sino que además hacen una interpretación exhaustiva de la misma caracterizándola como un valor que cuantifica la variación de los datos respecto de la media. Más aún, hacen referencia al promedio de las desviaciones al cuadrado de los datos respecto a su media y logran caracterizarla a partir de un histograma. En contraposición encontramos estudiantes que no escriben la definición o lo hacen con errores, determinan a la variancia pero sólo dicen que la misma caracteriza la variación de los datos, sin especificar respecto de quién. Así mismo, para un histograma que sugiere una distribución uniforme de los valores, sostienen que la variancia es muy pequeña, por cuanto observan que las frecuencias relativas de los diferentes intervalos de clase presentan escasa variación entre ellas. El diálogo con los estudiantes, especialmente con los de menor desempeño, nos permitió comprender mejor el porqué de sus dificultades. Al finalizar esta actividad, nos encontramos ante nuevos desafíos: promover procesos que le permitan a los estudiantes reflexionar sobre sus propios saberes, formular nuevos interrogantes e integrar los conceptos a su estructura cognitiva. }
