\A
{BONDAD DE AJUSTE DE LA DISTRIBUCIÓN DE UN ÍNDICE DE EXPOSICIÓN A PLAGUICIDAS APLICACIÓN EN TRABAJADORES RURALES DE LA PROVINCIA DE CÓRDOBA}
{\Presenting{OLGA PADRÓ}$^1$\index{PADRÓ, O} y MARÍA DEL PILAR DÍAZ$^2$\index{DIAZ, MARIA@DÍAZ, MARÍA}}
{\Afilliation{$^1$FACULTAD DE CIENCIAS ECONÓMICAS- UNIVERSIDAD NACIONAL DE CÓRDOBA}
\Afilliation{$^2$INICSA-FACULTAD DE CIENCIAS MÉDICAS –UNIVERSIDAD NACIONAL DE CÓRDOBA}
\\\Email{olgapadro@gmail.com}}
{índices de exposición; bondad de ajuste; distribuciones de vida; tasas de riesgo; ttt plot} 
 {Ecología y medio ambiente} 
 {Teoría sobre distribuciones de probabilidad} 
 {151} 
 {324-1}
{El estudio de los problemas ambientales, productos de fenómenos que implican acumulación de algún tipo de tóxicos, o la concentración de contaminantes en el aire y en el agua (Vilca et al., 2010), como así también la exposición acumulada a plaguicidas que afecta a los trabajadores rurales, constituye un desafío analítico y estadístico. Estos fenómenos involucran variables de vida caracterizadas por ser asimétricas, unimodales, sesgadas positivamente y de dos parámetros, con momentos de cualquier orden (Marshall \& Olkin, 2007). Tradicionalmente, distribuciones como la exponencial, Weibull, lognormal, gamma, Birnbaum-Saunders y Gaussiana inversa, son las comúnmente utilizadas a pesar de presentar bondad de ajuste deficiente en los eventos extremos o en las colas de la distribución. Para mejorar el ajuste diversos modelos de vida, como los modelos GBS (Generalized Birnbaum-Saunders) (Leiva et al. 2008) y los modelos IGT (Inverse Gaussian Type) (Sanhuenza et al., 2008), han sido propuestos. Los primeros se caracterizan por admitir diferentes grados de curtosis y asimetría, así como unimodalidad y bimodalidad; los segundosson considerados robustos, desde el enfoque de Lange et al. (1989) ya que es una clase de modelos que incluye diferentes grados de curtosis. El presente trabajo aborda dos índices para la valoración de la exposición a plaguicidas: el Nivel de Intensidad a la Exposición (IE) y la Exposición Acumulada (EAC), construidos por el Grupo de Epidemiología Ambiental del Cáncer en Córdoba (GEACC, Lantieri et al., 2011). El objetivo fue definir el modelo probabilístico de mejor ajuste para cada uno, utilizando y comparando las distribuciones tradicionales con los modelos GBS e IGT, y seleccionando los más adecuados según sus propiedades teóricas e interpretaciones. Con el fin de validar la distribución seleccionada como de mejor ajuste, se estudian los comportamientos de las tasas de riesgo de dichos índices (IE y EAC), utilizando TTT Plots (Aarset, 1987). Una vez elegido el modelo de distribución se obtienen los percentiles que determinan los niveles de baja, media y alta exposición a plaguicidas. }
