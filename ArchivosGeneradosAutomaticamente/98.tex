\A
{USO DE TÉCNICAS GEOESTADÍSTICAS PARA CUANTIFICAR EL EFECTO DE LA INTENSIDAD DE TRÁNSITO SOBRE LA DISTRIBUCIÓN HORIZONTAL Y VERTICAL DE LA COMPACTACIÓN EN SUELOS AGRÍCOLAS}
{\Presenting{C.A. ALESSO}$^1$\index{ALESSO, C}, P.A. CIPRIOTTI$^2$\index{CIPRIOTTI, P}, M.J. MASOLA$^3$\index{MASOLA, M}, ME CARRIZO$^3$\index{CARRIZO, M}, DL ANTILLE$^4$\index{ANTILLE, D} y SC IMHOFF$^3$\index{IMHOFF, S}}
{\Afilliation{$^1$DPTO. DE CIENCIAS BÁSICAS, FACULTAD DE CIENCIAS AGRARIAS, UNIVERSIDAD NACIONAL DEL LITORAL. (FCA-UNL)}
\Afilliation{$^2$DPTO. DE MÉTODOS CUANTITATIVOS Y SISTEMAS DE INFORMACIÓN, FACULTAD DE AGRONOMÍA, UNIVERSIDAD DE BUENOS AIRES (FAUBA). INSTITUTO DE INVESTIGACIONES FISIOLÓGICAS Y ECOLÓGICAS VINCULADAS A LA AGRICULTURA}
\Afilliation{$^3$DPTO. DE BIOLOGÍA VEGETAL, FACULTAD DE CIENCIAS AGRARIAS, UNIVERSIDAD NACIONAL DEL LITORAL. (FCA-UNL)}
\Afilliation{$^4$UNIVERSITY OF SOUTHERN QUEENSLAND, NATIONAL CENTRE FOR ENGINEERING IN AGRICULTURE, TOOWOOMBA, QLD, AUSTRALIA.}
\\\Email{calesso@fca.unl.edu.ar}}
{compactación; geoestadística; resistencia mecánica a la penetración; tránsito controlado; variabilidad espacial} 
 {Ciencias agropecuarias} 
 {Estadística espacial} 
 {98} 
 {217-1}
{El tránsito controlado es un sistema de mecanización en donde la maquinaria agrícola transita dentro del lote únicamente por sendas predeterminadas para reducir la superficie compactada por las ruedas y maximizar las condiciones del suelo para la exploración radical y transitabilidad. La magnitud y distribución espacial de la compactación, estimada por la resistencia mecánica a la penetración (RP), en torno a las sendas podría explicar la respuesta de los cultivos en las hileras próximas a las huellas. El objetivo de este trabajo fue modelar la distribución espacial de la RP en el plano vertical en torno a las sendas de circulación sobre un suelo Argiudol Típico Serie Rafaela de la localidad de Aurelia, Santa Fe (Argentina). Se simularon cuatro intensidades de tránsito (0, 6, 12 y 18 pasadas de un tractor Pauny 230C) en parcelas de 6 m x 30 m, siguiendo un diseño en bloques completos al azar con tres réplicas. Previo a la siembra de soja y trigo, en cada parcela se registró la RP de los primeros 30 cm del suelo cada 2 cm de profundidad (dirección vertical) insertando un penetrómetro digital de punta cónica cada 5,5 cm en transectas de 110 y 187 cm perpendiculares a las sendas (dirección horizontal). La distribución espacial de los datos en la grilla vertical XZ de 5,5 cm x 2 cm se modeló mediante técnicas geoestadísticas y se obtuvieron superficies interpoladas por kriging universal. Se estimó la proporción de área con valores de RP $>$ 2 MPa (límite para el crecimiento radical) en las zonas bajo huella y entre huellas la cual se utilizó como variable respuesta ajustando un modelo con medidas repetidas en el espacio. En todos los casos, los valores de RP mostraron tendencia en ambas direcciones la cual explicó 50-84\% de la variación total. La variación residual mostró una estructura espacial esférica con alcances de 15 a 36 cm. Previo a la siembra de la soja no se detectaron diferencias significativas en la proporción de área compactada. En cambio, previo a la siembra del trigo se detectaron diferencias significativas entre las intensidades y entre zonas. La proporción de área compactada siguió una tendencia cuadrática en función del número de pasadas ($Y = 0.25 + 0.035X - 0.00176X^2$). El 45\% del área bajo la huella tuvo RP $>$ 2 MPa mientras que sólo el 22\% mostró compactación fuera de la huella.}
