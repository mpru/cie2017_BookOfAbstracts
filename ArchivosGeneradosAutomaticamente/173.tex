\A
{DIMENSIONES DE LA POBREZA RELATIVA}
{\Presenting{ALFREDO BARONIO}\index{BARONIO, A}, FAVIO NICOLÁS D'ERCOLE\index{D'ERCOLE, F} y ANA MARÍA VIANCO\index{VIANCO, A}}
{\Afilliation{FACULTAD DE CIENCIAS ECONÓMICAS – UNIVERSIDAD NACIONAL DE RÍO CUARTO}
\\\Email{anavianco@yahoo.com.ar}}
{educación; umbral; diferencias; comparación} 
 {Economía} 
 {Métodos multivariados} 
 {173} 
 {350-1}
{La medición multidimensional de la pobreza incorpora aspectos no monetarios y evalúa si las personas logran alcanzar umbrales mínimos de bienestar en cada una de las dimensiones consideradas. En acuerdo a CEPAL (2013), identificar las dimensiones y establecer los umbrales son instancias que generan controversias; aunque mediciones en Colombia, México, Chile y Argentina coinciden en considerar educación, salud y nivel de vida. El relevamiento realizado por el Consejo Económico y Social en el año 2014 sobre 654 hogares, permitió conocer que en Río Cuarto la pobreza multidimensional relativa se asocia a nivel educativo que sólo alcanza a secundario incompleto, nivel de ingresos que sitúa a sus miembros en situación de riesgo alimentario y disponibilidad de bienes que no le permite acceder al confort derivado de las nuevas tecnologías; se consideró que el factor principal y discriminante de la pobreza está representado por la educación, siendo el umbral a superar el de la escuela secundaria. El estudio se repite en el año 2017 sobre 718 hogares, utilizando similar instrumento de recolección de información y recurriendo a estadística descriptiva, análisis factorial de correspondencias múltiples y clasificación jerárquica de la información para su análisis. El objetivo de este trabajo es comparar las modalidades predominantes que alcanza cada dimensión entre 2014 y 2017 y proponer los umbrales a considerar en cada una de ellas. Ambos estudios se realizaron en el segundo trimestre del año respectivo; a través de muestreo probabilístico de etapas múltiples, donde el radio censal tiene probabilidad 1 de formar parte de la muestra; en los hogares seleccionados se observaron las dimensiones ambiente, educación, empleo, salud, violencia, opinión e ingresos. Resultados preliminares identifican, nuevamente, a las dimensiones educación, empleo, ingresos y vivienda discriminando a la población, dando la posibilidad de hablar de pobreza relativa. A tres años del primer estudio, la situación social se ha mantenido estable con señales de mejoras en la dirección deseada, pero estas diferencias no resultan significativas.}
