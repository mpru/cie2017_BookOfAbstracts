\A
{ENCUESTA DE ACTIVIDADES DE NIÑOS, NIÑAS Y ADOLESCENTES – RURAL (EANNA- RURAL) – DISEÑO MUESTRAL}
{\Presenting{HUGO DELFINO}$^1$\index{DELFINO, H}, ANA MARÍA CATALANO$^2$\index{CATALANO, A}, GERMAN ROSATI$^2$\index{ROSATI, G}, JOSÉ ANCHORENA$^2$\index{ANCHORENA, J} y EQUIPO DE EANNA RURAL$^2$}
{\Afilliation{$^1$UNLU}
\Afilliation{$^2$SUBSECRETARIA DE POLÍTICAS, ESTADÍSTICAS Y ESTUDIOS LABORALES. MTEYSS}
\\\Email{hugo.delfino@dataexplorer.com.ar}}
{radios con población agrupada; radios con población dispersa; sistema de información geográfica; imágenes satelitales; trabajo infantil; población rural} 
 {Estadísticas oficiales} 
 {Muestreo} 
 {6} 
 {11-1}
{En el año 2017, el MTEySS diseñó la muestra para la Encuesta de Actividades de Niños, Niñas y Adolescentes (EANNA) Rural. El objetivo de la EANNA Rural es analizar las actividades económicas y no económicas que realizan los niños/as y adolescentes de entre 5 y 17 años que habitan en viviendas particulares de zonas rurales, enfocando en las características, los contextos y los principales condicionantes del trabajo infantil. Desde un punto de vista de estadístico es un estudio basado en una muestra probabilística compleja. Se utilizó información de radios censales provistos por INDEC. Como la incidencia de la población rural dispersa varía según la región, se definieron dos estratos: a) radios con población agrupada y b) radios con población dispersa. Teóricamente corresponde a un: Diseño Estratificado de conglomerados a dos o tres etapas, con probabilidades variables de selección de las unidades de primera etapa (UPE) y probabilidad igual para las de segunda etapa (USE) en el caso de radios agrupados y variable en el caso de los radios dispersos. Se seleccionaron como UPE a radios dentro de cada estrato-dominio regional. Las USE fueron en el caso de los radios agrupados una muestra de viviendas presuntamente habitadas, en los radios dispersos, dada su alta dispersión en la mayoría de los casos y debido a que no se dispone de un marco de muestreo ni es posible construir dicho marco, se procedió a marcar posibles viviendas (techos) utilizando sistemas de información geográfica e imágenes satelitales de Google Earth y Bing. En una etapa posterior, estos “techos” fueron clusterizados dentro de cada radio, para ello se buscó el número óptimo de clusters utilizando el Gap Statistic, luego utilizando el número óptimo del paso 1 se construyen los segmentos utilizando las coordenadas (X,Y) de los techos por medio del algoritmo K-means. y se seleccionaron dos de dichos clusters para, en una tercera etapa (UTE) seleccionar viviendas con probabilidades iguales. En los radios dispersos, se realiza un recorrido del área seleccionada para identificar si los “techos”, corresponden realmente a viviendas habitadas y detectar posibles omisiones, luego de actualizado el marco se procede a seleccionar viviendas de manera sistemática. La presente metodología muestra una alternativa novedosa para trabajar con áreas donde no es posible construir un marco de muestreo de manera tradicional. Los resultados suministraran información fehaciente a nivel nacional y regional que podrá ser utilizada para la formulación políticas de protección social basadas en evidencia.}
