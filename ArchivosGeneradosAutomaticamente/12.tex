\A
{REDES NEURALES APLICADAS AL ESTUDIO DE LA CRIMINALIDAD EN EL DEPARTAMETO DE CUNDINAMARCA, COLOMBIA}
{\Presenting{JULIAN DAVID REALPE CAMPAÑA}\index{REALPE CAMPAÑA, JD} y FREDSON GUERRERO\index{GUERRERO, F}}
{\Afilliation{SERVICIO NACIONAL DE APRENDIZAJE - SENA}
\\\Email{jdrealpec@gmail.com}}
{redes neuronales; multivariado; criminalidad; correlación espacial; cundinamarca; colombia} 
 {Estadísticas oficiales} 
 {Métodos multivariados} 
 {12} 
 {21-2}
{El crimen es un problema generalizado en todas las ciudades de Colombia y del mundo en general. Las causas y efectos del crimen en la sociedad han sido ampliamente estudiados y documentados, no obstante, en los últimos años con la ayuda de herramientas estadísticas computacionales, se han realizado importantes estudios que permiten brindar a las autoridades locales y sociedad en general de elementos alternativos de caracterización. En este trabajo se realizó el análisis estadístico de los datos de criminalidad entre los años 2008 a 2014 en el departamento de Cundinamarca, Colombia, mediante la utilización de índices de correlación, modelos lineales multivariados y redes neuronales. Los datos trabajados fueron obtenidos de parte de los boletines anuales del Observatorio de Seguridad del Departamento de Cundinamarca, Colombia. El cálculo del índice de correlación de los datos de criminalidad de las provincias mostró que los datos estaban espacialmente correlacionados, mostrando valores entre 0.85 y 0.99 para las provincias que compartían límites geográficos. Esto fue una indicación para utilizar modelos lineales multivariados que pudieran relacionar la ocurrencia de diversos tipos de delitos en cada una de las provincias de la Región de Cundinamarca con su cercanía geográfica. Los modelos lineales generados mostraron un ajuste promedio de 85\% escogiendo para cada provincia los datos de criminalidad de sus vecinos inmediatos como variables independientes. Posteriormente, a fin de mejorar la capacidad predictiva, se utilizaron redes neuronales, las cuales se entrenaron con el 70\% de los datos y probado con el 30\% restante. La red neuronal mostró un valor de R-cuadrado de ajuste con los datos de prueba 0.95, mostrando una alta capacidad de predicción. Estos resultados apuntan a desarrollar metodologías alternativas para el estudio del crimen en la región, permitiendo a las autoridades locales proveer herramientas estadísticas que puedan ayudar a monitorear y predecir la distribución y evolución de la delincuencia en las distintas provincias de Cundinamarca.}
