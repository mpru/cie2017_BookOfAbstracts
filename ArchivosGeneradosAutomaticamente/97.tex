\A
{INCORPORACIÓN DE LOS PROCESOS KDD (KNOWLEDGE DISCOVERY IN DATABASES) EN LA CARRERA DE MEDICINA VETERINARIA}
{\Presenting{GABRIELA PILAR CABRERA}\index{CABRERA, G}, ANDREA PEÑA MALAVERA\index{PEÑA MALAVERA, A}, LAURA ANGÉLICA MARCO\index{MARCO, L} y ELINA DEL VALLE FERNÁNDEZ\index{FERNÁNDEZ, ELINA}}
{\Afilliation{UNIVERSIDAD NACIONAL DE VILLA MARÍA}
\\\Email{gabriela.pilar.cabrera@gmail.com}}
{minería de datos; pensamiento estadístico; carrera de grado} 
 {Enseñanza de la estadística} 
 {Minería de datos} 
 {97} 
 {215-1}
{En el marco de la asignatura “Metodología de la Investigación” de la carrera de Medicina Veterinaria de la Universidad Nacional de Villa María (UNVM), se desarrolló una experiencia didáctica que permitió introducir a los estudiantes en la aplicación del proceso de KDD (Knowledge Discovery in Data Bases); a la vez que vivenciar los conceptos y procedimientos ejes de dicha asignatura. El problema de estudio específico, se contextualizó en la implementación de los procesos KDD para la caracterización de calidad de leche bovina cruda en aspectos Higiénico–Sanitario, Composicional y Adulteración y sus variaciones estacionales de cinco empresas lácteas del departamento Rio II, Provincia de Córdoba en 2014. Con los estudiantes, se implementaron las etapas de limpieza e integración, obtención del Data Warehouse, evaluación y presentación de patrones. En particular, el pre-procesamiento de los datos provistos por los usuarios (Laboratorio de Diagnóstico Villa María), con la asistencia de expertos en Calidad de Leche Bovina, posibilitó comprender el dominio del problema de estudio (entendimiento del negocio). En tanto, la evaluación de la calidad de los datos generados y la identificación de patrones en los mismos, se realizó mediante un análisis exploratorio con el soporte del software estadístico InfoStat. Este entorno de trabajo, permitió a los estudiantes re-significar y consolidar conocimientos y herramientas abordados en Bioestadística y a su vez reconocerla como tecnología de la investigación científica. En la implicación de los estudiantes en un problema real de investigación, resultó significativa la asignatura Metodología de la Investigación. Esta experiencia didáctica, promueve la inclusión de contenidos introductorios a los procesos KDD, en los programas de estudio de las asignaturas Informática y Bioestadística; de manera conjunta e integrada}
