\A
{EXTENSIÓN DEL MENÚ EPIDEMIOLOGÍA DEL SOFTWARE INFOSTAT: PROCEDIMIENTOS PARA ANÁLISIS DE SUPERVIVENCIA}
{\Presenting{MARGOT TABLADA}$^1$\index{TABLADA, M}, LAURA ALICIA GONZALEZ$^1$\index{GONZALEZ, L}, PABLO PACCIORETTI$^1$\index{PACCIORETTI, P}, DUBERNEY MUÑOZ MORALES$^2$\index{MUÑOZ MORALES, D} y DAVID MORALES$^1$\index{MORALES, D}}
{\Afilliation{$^1$FCA}
\Afilliation{$^2$UNC}
\\\Email{gonzalez.lauraalicia@gmail.com}}
{bioestadística; estudios de seguimiento; software estadístico infostat} 
 {Biología} 
 {Datos de duración} 
 {114} 
 {247-1}
{El seguimiento de individuos para conocer el tiempo transcurrido hasta que se presenta un resultado de interés es una práctica en numerosas áreas como agronomía, ecología, medicina, psicología, economía, sociología, industria, entre otras. Los métodos para el tratamiento estadístico de los datos se conocen como análisis de supervivencia o de sobrevida. El evento de interés no necesariamente es la muerte y, por ejemplo, se puede estudiar el tiempo hasta que falla un producto o equipo, o el tiempo hasta la reincidencia en el delito, entre otros. Es frecuente que los registros de los tiempos estén acompañados de otras variables (edad, material, sexo, tratamiento, ingresos, estado civil, etc.), que se usan como clasificatorias de grupos o como covariables relacionadas con el evento de interés. Mediante el análisis se pretende estimar la función de supervivencia y la función de riesgo, y llevar a cabo comparaciones entre grupos. El módulo de aplicaciones en Epidemiología, del software estadístico InfoStat posee procedimientos para ajuste de tasas, obtención de índices de concordancia y consistencia, medidas de riesgos y asociación para diferentes tipos de muestreo, curva ROC y regresión de Cox. Mediante la extensión propuesta, se incorporan a éstos técnicas tradicionalmente utilizadas para la descripción, estimación y contraste de hipótesis en el análisis de sobrevida: estimación de la función de supervivencia (Kaplan-Meier, Flemming y Harrington), medidas resumen, intervalos de confianza, gráficos, comparación de curvas (prueba log-rank, de Wilcoxon, Tarone, Peto). También se incluyen rutinas que amplían los resultados obtenidos con el uso del módulo actual para la regresión de Cox: función de supervivencia ajustada mediante modelo de Cox, herramientas para verificar supuestos (riesgos proporcionales, residuos tipo score, deviance, martingala). La implementación se complementa con la elaboración de bases de datos y un tutorial, para ejemplificar el uso de los procedimientos, que incluye un marco conceptual introductorio. El desarrollo se realiza mediante la combinación de herramientas de programación de InfoStat y R que permite ofrecer aplicaciones, en base a paquetes internacionalmente conocidos y validados, en una interfaz amigable. El desarrollo del módulo Epidemiología es resultado de proyectos avalados por la Secretaría de Ciencia y Tecnología de la Universidad Nacional de Córdoba. }
