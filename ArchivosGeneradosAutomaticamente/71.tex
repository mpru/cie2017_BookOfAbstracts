\A
{UN MODELO DE REGRESIÓN PARA EL ANÁLISIS DE BIOMASA DE ALGAS MARINAS}
{JOSÉ SARAVIA\index{SARAVIA, J}}
{\Afilliation{SECRETARÍA DE PESCA DE LA PROVINCIA DEL CHUBUT}
\\\Email{josesaravia@speedy.com.ar}
}
{regresión heteroscedástica; biomasa; modelo lineal mixto} 
 {Biología} 
 {Modelos de regresión} 
 {71} 
 {162-1}
{Este trabajo presenta una propuesta de aplicación de regresión heteroscedástica para evaluar la biomasa de algas bentónicas marinas y su relación con diversas variables ambientales y nutrientes. Se evaluaron tres praderas existentes en la zona norte del golfo San Jorge, con diferentes cantidades de puntos de muestreo (estaciones). La biomasa se midió a los largo de 9 campañas, registrándose conjuntamente las variables pH, salinidad, oxígeno disuelto, temperatura, nitratos y fosfatos del agua de mar. Se realizó una regresión común y ante la desviación de los supuestos del modelo (homoscedacia y normalidad), se optó por aplicar modelos lineales mixtos con la variable clasificatoria “estaciones” como aleatoria. El gráfico correspondiente muestra que la distribución de los residuos es aproximadamente normal. Al analizar el diagrama de dispersión de los residuos condicionales de Pearson versus los Valores ajustados, se observa una clara tendencia de los residuos a aumentar su varianza a medida que aumenta el valor medio. Esto sugiere la necesidad de modelar esta falta de homogeneidad de varianza con una función que relacione las varianzas de los residuos con la media. Se plantea el mismo modelo pero con diversas funciones de varianzas. Se concluye que la regresión puede ser utilizada para describir el comportamiento de la biomasa de algas ya que el ajuste fue satisfactorio considerando que la varianza es distinta para cada estación. La biomasa de algas marinas está relacionada con pH, oxígeno disuelto y temperatura. No se observaron diferencias significativas entre praderas. La mayor variabilidad es atribuible a diferencias entre estaciones.}
