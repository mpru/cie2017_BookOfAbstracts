\A
{ELABORACIÓN DE INDICADORES PARA LA GESTIÓN PÚBLICA: EL CASO DE LOS INDICES DE OBRAS PÚBLICAS DE CÓRDOBA}
{\Presenting{LAURA ISABEL LUNA}\index{LUNA, L}, VERÓNICA ARIAS\index{ARIAS, V} y MARIANA DÍAZ\index{DIAZ, MARIANA@DÍAZ, MARIANA}}
{\Afilliation{DIRECCIÓN GENERAL DE ESTADÍSTICAS Y CENSOS}
\\\Email{lauraisabel.luna@gmail.com}}
{estadísticas oficiales; índices de obra pública; redeterminación de precios; provincia de córdoba} 
 {Estadísticas oficiales} 
 {Otras categorías metodológicas} 
 {66} 
 {156-1}
{El presente trabajo tiene como objetivo difundir y poner al alcance de los especialistas y usuarios en general los principales aspectos metodológicos y prácticos vinculados a la construcción de los índices de “Obra Pública” de Córdoba. Es fundamental la calidad, exactitud y oportunidad de esta importante herramienta para la gestión pública. El objetivo principal de estos índices es reflejar la variación de los precios medios de una serie de factores, constituyéndose en exclusivos indicadores de referencia para la redeterminación de Precios de Obras Públicas del Gobierno de la Provincia de Córdoba, tal como lo establece el Decreto 800/16. Cada índice se calcula en base a la variación del precio medio del respectivo insumo, ó bien, como un promedio ponderado de las variaciones de los precios medios de una canasta de insumos. Se relevan precios de 187 productos, que en distinto grado inciden en los indicadores que se publican, y se incluyen dos índices generales, el Índice del Costo de la Construcción de Córdoba y el Índice de Precios al Consumidor de Córdoba, ambos relevados mensualmente. En el presente trabajo se analizan las principales características de este conjunto de índices y se detallan los métodos y procedimientos utilizados para su cálculo, también se presentan los resultados para la serie 2015-2017. Consideramos que la publicación y discusión de los mismos es fundamental para la transparencia, confiabilidad y mejora continua de las estadísticas oficiales. }
