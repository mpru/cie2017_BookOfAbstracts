\clearpage
\newpage
\noindent\\
\thispagestyle{empty}
\begin{center}
\Large
\begin{flushright}
\vspace{17cm} {\Huge \em{ \textbf{\textcolor{ultramarine}{CONCURSOS\\PARA ESTUDIANTES}}}} \\ [0.5cm]
\addcontentsline{toc}{section}{Concursos para estudiantes}
\end{flushright}
\normalsize
\end{center}

\small
\clearpage
\pagestyle{fancy}
\SetHeader{Concursos para estudiantes}{\textit{Congreso Interamericano de Estadística}}

%--------------------------------------------------------------------------
\renewcommand{\titulo}{CONCURSO JÓVENES BIOMETRISTAS}
\addcontentsline{toc}{subsubsection}{\titulo}
\vspace*{1cm}

\begin{center}
\textbf{\textcolor{ultramarine}{CONCURSO JÓVENES BIOMETRISTAS\\}}
\bigskip
\textbf{\textit{Coordinadora:\\}}
Lic. Ivana Barbona\\
\textbf{\textit{Jurado:\\}}
Mg. Liliana Contini, Ing. Joaquín Llera, Ing. María Inés Urrutia, Dra. Laura Itatí Gimenez\\
\bigskip
\end{center}

\noindent Este concurso, orientado a estudiantes universitarios, consiste en la resolución de un problema de investigación provisto por los organizadores mediante la aplicación de herramientas estadísticas. Existen dos niveles de participación: para estudiantes de grado y estudiantes de postgrado. Los resultados se presentan en un informe que es evaluado por el jurado.\\
\noindent Este año el problema se trata sobre conservación de semillas de maíz. Luego de la cosecha en el campo, las espigas de maíz se transportan hasta la planta de procesamiento. Allí son sometidas a distintas etapas: deschalado, limpieza, selección, secado, desgranado y clasificación según tamaño de las semillas. Finalmente las semillas se almacenan en distintos envases y se fraccionan en bolsas de 80000 unidades. Durante este tiempo se producen procesos metabológicos propios de las semillas que conducen al deterioro de la calidad fisiológica original, condicionando su viabilidad y vigor. Estos procesos se ven afectados, a su vez, por procesos externos, tales como la temperatura y la humedad del ambiente donde las semillas son mantenidas hasta su próxima siembra. En un experimento se evaluaron dos tipos de envasado (vacío y abierto al ambiente) y dos condiciones de almacenamiento (cámara fría y galpón). A partir de los datos recolectados, el objetivo es estudiar el efecto de estos factores sobre el poder germinativo y el vigor de las semillas.

\bigskip
\noindent Equipos participantes:\\

\begin{itemize}
\item Nicolás Batisttón y Martín Prarizzi (Ingeniería Agronómica, Universidad Nacional de Villa María).
\item Tomás Capretto (Licenciatura en Estadística, Universidad Nacional de Rosario) y David Presutti (Ingeniería Agronómica, Universidad Nacional del Noroeste de la Provincia de Buenos Aires).
\item Martín Castro (Licenciatura en Estadística, Universidad Nacional de Rosario).
\end{itemize}
%--------------------------------------------------------------------------

\newpage
\clearpage
\renewcommand{\titulo}{TRABAJO EN PÓSTER DESTACADO PARA ESTUDIANTES}
\addcontentsline{toc}{subsubsection}{\titulo}
\vspace*{1cm}

\begin{center}
\textbf{\textcolor{ultramarine}{TRABAJO EN PÓSTER DESTACADO PARA ESTUDIANTES\\}}
\bigskip
\textbf{\textit{Coordinador:\\}}
Mg. Javier Bussi\\
\textbf{\textit{Jurado:\\}}
Mg. María Teresa Blaconá, Dra. Patricia Caro y Lic. Ernesto Rosa\\
\bigskip
\end{center}

\noindent Con la finalidad de promover la participación de estudiantes de las actividades de la Sociedad Argentina de Estadística y divulgar la Estadística como disciplina, se instituye el Premio al trabajo en Póster destacado presentado por Estudiantes. El trabajo debe tratar algún tema referido a la Estadística, teórica o aplicada a algún área en particular, pudiendo concursar los estudiantes de grado de cualquier carrera universitaria o cualquier terciario universitario de la República Argentina.

\bigskip
\noindent Participantes y trabajos presentados:\\

\begin{itemize}
\item \textbf{Asahan, Adriana}: Impacto del nivel educativo sobre los ingresos mensuales.
\item \textbf{Castro, Martín}: Prevalencia de anemia en niños menores de 42 meses usuarios de la red pública de pediatría ambulatoria en la provincia de Santa Fe.
\item \textbf{Cometto, Franco}: Alquileres en Rosario: análisis exploratorio de avisos en un portal de clasificados.
\item \textbf{Millanes, Iván}: Estudio sobre la mortalidad en bacteriemia nosocomial por staphylococcus aureus en pacientes del Hospital Español de la ciudad de Rosario.
\item \textbf{Molinari, Daniela Alejandra}: Análisis de la estructura espacial de parámetro magnético de la contaminación ambiental.
\item \textbf{Monteros, Noelia}: El gran dilema azucarero: rendimiento vs trash.
\item \textbf{Rampello, Yamila}: Factores pronósticos de presentación en cáncer de mama lobulillar.
\item \textbf{Trimarco, María Victoria}: Modelos, marcas y anchos de bandas para recapado.
\end{itemize}



% anterior, en una hoja con items

%\vspace*{1cm}
%\centerline{\textbf{\LARGE{\textcolor{ultramarine}{Concursos para estudiantes}}}}
%\vspace{1cm}

%\begin{itemize}
%\item \textbf{Concurso Jóvenes Biometristas} \\
%Coordinadora: Lic. Ivana Barbona \\
%Jurado: Mg. Liliana Contini, Ing. Joaquín Llera, Ing. María Inés Urrutia, Dra. Laura Itatí Gimenez \\
%Este concurso, orientado a estudiantes universitarios, consiste en la resolución de un problema de investigación provisto por los organizadores mediante la aplicación de herramientas estadísticas. Existen dos niveles de participación: para estudiantes de grado y estudiantes de postgrado. Los resultados se presentan en un informe que es evaluado por el jurado.

%\item \textbf{Trabajo en Póster Destacado para Estudiantes} \\
%Jurado: Mg. María Teresa Blaconá, Dra. Patricia Caro y Lic. Ernesto Rosa \\
%Con la finalidad de promover la participación de estudiantes de las actividades de la Sociedad Argentina de Estadística y divulgar la Estadística como disciplina, se instituye el Premio al trabajo en Póster destacado presentado por Estudiantes. El trabajo debe tratar algún tema referido a la Estadística, teórica o aplicada a algún área en particular, pudiendo concursar los estudiantes de grado de cualquier carrera universitaria o cualquier terciario universitario de la República Argentina.
%\end{itemize}
