\A
{DATOS ABIERTOS: PUBLICACIÓN DE DATOS EPIDEMIOLÓGICOS ROSARIO, PERÍODO 2015-2017}
{\Presenting{VALERIA AGUZZI}$^1$\index{AGUZZI, V}, LAURA BALPARDA$^2$\index{BALPARDA, L}, MARIELA CAMUSSI$^2$\index{CAMUSSI, M}, MARIO CHAVERO$^2$\index{CHAVERO, M}, ANALÍA CHUMPITAZ$^2$\index{CHUMPITAZ, A}, JORGE DIMARCO$^1$\index{DIMARCO, J}, PATRICIA GIARDINI$^1$\index{GIARDINI, P}, DIEGO LÓPEZ$^3$\index{LOPEZ, D@LÓPEZ, D}, ANDREA MORO$^2$\index{MORO, A}, MARÍA EUGENIA PERETTI$^4$\index{PERETTI, M}, CINTIA SILVA$^2$\index{SILVA, C} y MARINA SUAREZ$^2$\index{SUAREZ, M}}
{\Afilliation{$^1$DIRECCIÓN GENERAL DE INFORMÁTICA. SECRETARÍA GENERAL. MUNICIPALIDAD DE ROSARIO.}
\Afilliation{$^2$SISTEMA MUNICIPAL DE EPIDEMIOLOGÍA. SECRETARÍA DE SALUD PÚBLICA. MUNICIPALIDAD DE ROSARIO.}
\Afilliation{$^3$ÁREA DE SENSORES REMOTOS. ESCUELA DE AGRIMENSURA. FACULTAD DE CIENCIAS EXACTAS, INGENIERÍA Y AGRIMENSURA. UNIVERSIDAD NACIONAL DE ROSARIO}
\Afilliation{$^4$EQUIPO WEB. DIRECCIÓN GENERAL DE COMUNICACIÓN SOCIAL. SECRETARÍA DE GOBIERNO. MUNICIPALIDAD DE ROSARIO}
\\\Email{lbalparda@hotmail.com}}
{datos abiertos; epidemiología; datos agregados-desagregados} 
 {Estadísticas oficiales} 
 {Otras categorías metodológicas} 
 {125} 
 {278-1}
{Datos abiertos es una iniciativa a nivel mundial cuyo objetivo es la publicación de datos de la gestión pública, en forma completa y oportuna, en el marco de un proceso proactivo, “en y por los canales, medios, formatos y bajo la licencia que mejor facilite su ubicación, acceso, procesamiento, uso, reutilización y redistribución”1 . De este modo, se pretende que la difusión de datos públicos contribuya a fortalecer el “proceso democrático, el desarrollo de políticas públicas basadas en la evidencia, la provisión de servicios públicos, la promoción del desarrollo social, económico, científico y cultural”1 de una sociedad. En el año 2009, países como EEUU, Gran Bretaña y Canadá se sumaron a esta propuesta. En Argentina, la Constitución Nacional garantiza el principio de publicidad de los actos de gobierno y el derecho a la información pública (artículos 33, 41, 42 y 75 inciso 22); en tanto que, el Decreto 117/2016 del Plan Nacional de Apertura de Datos promulga que las áreas dependientes del Poder Ejecutivo Nacional realicen un detalle de los activos de los conjuntos de datos (datasets) y un cronograma de publicación. La Municipalidad de la ciudad de Rosario reglamentó la creación del programa “Open Data Rosario” (ordenanza 9279/2014). Actualmente, el portal “Rosario Datos” (datos.rosario.gob.ar) publica 510 archivos en formatos abiertos agrupados en 160 datasets de diversas temáticas: territorio, población, movilidad, salud, administración financiera, entre otros. El Sistema Municipal de Epidemiología, adhiriendo a esta política de socialización de datos, publicó en el portal municipal 29 datasets correspondientes a Eventos de Notificación Obligatoria (ENOs), tales como dengue, coqueluche, meningitis, hantavirus, leptospirosis, psitacosis, triquinosis y zika, correspondientes al período 04/01/2015 a 01/07/2017. Los datos, agregados a nivel de ciudad (unidad espacial) y por semana epidemiológica (unidad temporal), se actualizarán semestralmente. Cumplida esta primera etapa, el equipo de trabajo de Epidemiología se plantea superarla, publicando las características de la unidad de análisis original (paciente) con un mayor nivel de desagregación espacial (adicionando el radio/fracción censal, según el domicilio de residencia del paciente) y temporal (fecha de ocurrencia del caso). En todos los registros se omitirán los datos filiatorios (nombre, apellido, DNI y dirección), respetando el secreto estadístico (Ley 17.622, artículo 10). Esto permitiría a los usuarios disponer de más datos, ampliando las posibilidades de uso de los datos abiertos. 1 Boletín Oficial República Argentina. Buenos Aires [citado 02/08/2017]. Plan de Apertura de Datos, Decreto 117/2016. Disponible en: https://www.boletinoficial.gob.ar/.}
