\A
{REPRESENTAÇÃO DA ESTRUTURA DO PENSAMENTO COLETIVO SOBRE AS ENCHENTES DO RIO DOCE: CONECTANDO INDIVÍDUOS AFINS ATRAVÉS DA TEORIA DOS GRAFOS}
{\Presenting{WESLEY PEREIRA}$^1$\index{PEREIRA, W}, GILVAN GUEDES$^1$\index{GUEDES, G}, DENISE DUARTE$^2$\index{DUARTE, D} y RODRIGO RIBEIRO$^3$\index{RIBEIRO, R}}
{\Afilliation{$^1$UNIVERSIDADE FEDERAL DE MINAS GERAIS}
\Afilliation{$^2$UNIVERSIDADADE FEDERAL DE MINAS GERAIS}
\Afilliation{$^3$INSTITUTO NACIONAL DE MATEMÁTICA PURA E APLICADA}
\\\Email{denise@est.ufmg.br}}
{representações sociais; enchentes rio doce; pensamento coletivo; teoria de grafos; comunidades em grafo} 
 {Otras ciencias sociales y humanas} 
 {Métodos no paramétricos} 
 {92} 
 {207-1}
{Introdução O município brasileiro de Governador Valadares é historicamente impactado pelas i- nundações (enchentes) do Rio Doce. As enchentes são fenômenos naturais que normalmente são sinônimos de prosperidade, uma vez que estas contribuem para a fertilidade das terras próximas ao rio. Porém, as enchentes em áreas urbanas podem se transformar em eventos catastróficos quando combinadas com intervenções no meio ambiente capazes de catalisá-las. Segundo {Moscovici, 1961}, idealizador da Teoria das Representações Sociais, dependendo dos significados sobre o objeto enchentes difundidos na sociedade, as pessoas podem não considerar as enchentes como um perigo a elas mesmas. Representação da rede de indivíduos através da Teoria dos Grafos Nesta análise, cada indivíduo contribui com até cinco evocações para a estrutura coletiva de significados. Uma técnica de representação dessa estrutura é a representação da rede através de grafos. Na Teoria dos Grafos, um grafo G = (V,E) é definido como um conjunto de objetos de duas naturezas: o vértice (V), a unidade fundamental sob a qual o grafo é construído e as arestas (E), que representam a ligação entre pares de vértices. A essas ligações podem ser atribuídas muitas características, como direção, distância e peso. Representar a rede de indivíduos por intermédio de um grafo possibilita evidenciar indivíduos ``populares, os quais possuem um grande número de conexões, resumindo as ideias mais difundidas em sua comunidade. Outra propriedade desta abordagem consiste na possibilidade de encontrar a difusão espacial do pensamento coletivo, sob a condição de se conhecer o georreferenciamento dos indivíduos. Dessa forma, procura-se delimitar a área de propagação do pensamento, verificando se existe nele algum indício de influência geoespacial. Além disso, pode-se realizar a validação das partições encontradas, bem como a verificação sobre indícios de processos latentes resposáveis pelo regimento comportamental da rede. Conclusão O estudo realizado leva à conclusão de que existem diferenças evidentes na perspectiva que as pessoas têm ao significar as enchentes. Exemplificando, existem alguns subgrupos da sociedade que buscam identificar as causas das enchentes enquanto outros subgrupos se preocupam com suas consequências. Alguns subgrupos expressam os sentimentos ruins que estes eventos geram à sociedade, enquanto outros se concentram nos prejuízos resultantes. A união das perspectivas de cada um desses subgrupos formam o pensamento coletivo da sociedade: as enchentes do Rio Doce são um evento natural que, pela uma série de fatores, são significadas majoritariamente pelas pessoas com um evento prejudicial à população e ao município de Governador Valadares.}
