\A
{LA ORQUESTA ESTADISTICA AL SERVICIO DE LA PSICOMETRIA}
{\Presenting{MARIA PIA CAUMONT}$^1$\index{CAUMONT, M} y SILVIA MARCELA MARIANO$^2$\index{MARIANO, S}}
{\Afilliation{$^1$SAE}
\Afilliation{$^2$SAE - IASI}
\\\Email{mariapiacaumont@yahoo.com.ar}}
{psicometria; diferencial semántico; redes semánticas naturales} 
 {Otras aplicaciones} 
 {Otras categorías metodológicas} 
 {21} 
 {45-1}
{Se presenta la descripción del análisis estadístico de dos técnicas psicométricas que conjuntamente estudian el significado de las palabras. Objetivo: Describir los recursos estadísticos utilizados para dos instrumentos psicométricos. Materiales y métodos: Se recolectó información en dos instituciones públicas midiendo el significado de dos conceptos para obtener los sentidos tanto denotativos como connotativos de los mismos a través de dos instrumentos psicométricos: Redes Semánticas Naturales y Diferencial Semántico. La primera pide al participante que definan con la mayor precisión posible una palabra estimulo, a través de otra palabras con que la relaciona y se puede observar la riqueza de los conceptos, su densidad y las relaciones. Con la segunda se le pide a los sujetos que ubique un concepto en una serie de escalas cuyos extremos están marcados por adjetivos bipolares. Se mide esa significación a partir de la situación del concepto del objeto analizado en un espacio semántico de dimensiones valorativas. Resultados: A través de métodos estadísticos se desarrolla la presentación de los resultados obtenidos. Por medio de la Red Semántica Natural se mide la riqueza de la red, peso semántico de la definidora, distancia semántica, densidad de la red, se usa recuento, porcentuales, ponderaciones, medianas, test de diferencia de proporciones y Chi cuadrados. Por medio del Diferencial Semántico la "significación" del término para una determinada persona será dada por el perfil resultante en las diferentes escalas de adjetivos, que se reducen a menor cantidad de dimensiones a través del puntaje factorial. Se crea 3 factores: Evaluativo, Potencial y Actividad. Se mide Polarización que es la distancia entre el punto neutral del espacio del diferencial semántico y el concepto particular considerado. Se obtienen medias y medianas de los resultados. Para medir las reacciones asociadas con el estímulo se emplea un análisis factorial. En ambos casos apoyados por representaciones gráficas que sintetizan y facilitan la lectura de la información. Conclusiones: Con el apoyo de técnicas Estadísticas, se hace un aporte al campo de la Psicometría. Relacionando los dos instrumentos permiten el estudio del significado de los conceptos, estableciendo grados de semejanza o disparidad entre ellos y a partir de esta información se puede localizar grupos de entrevistados con perfiles análogos y relacionarlos en base a determinadas características sociales o personales con las respuestas a otras cuestiones. }
