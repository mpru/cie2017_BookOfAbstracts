\A
{MODELACIÓN DE LA EMERGENCIA DE PLÁNTULAS EN SITUACIÓN DE DESERTIFICACIÓN EN EL OESTE DE RÍO NEGRO: UN PROBLEMA DE SOBREDISPERSIÓN}
{\Presenting{GUSTAVO GIMÉNEZ}$^1$\index{GIMÉNEZ, G}, DANIEL ZÚÑIGA$^2$\index{ZUNIGA, D@ZÚÑIGA, D}, M FERNANDA LÓPEZ ARMENGOL$^3$\index{LOPEZ ARMENGOL, M@LÓPEZ ARMENGOL, M} y NATALIA RUBIO$^1$\index{RUBIO, N}}
{\Afilliation{$^1$DEPARTAMENTO DE ESTADÍSTICA, FACULTAD DE ECONOMÍA, UNIVERSIDAD NACIONAL DEL COMAHUE}
\Afilliation{$^2$FACULTAD DE CIENCIAS DEL AMBIENTE Y LA SALUD, UNIVERSIDAD NACIONAL DEL COMAHUE}
\Afilliation{$^3$CENTRO DE INVESTIGACIONES EN TOXICOLOGÍA AMBIENTAL Y AGROBIOTECNOLOGÍA DEL COMAHUE (CITAAC), INSTITUTO DE BIOTECNOLOGÍA AGROPECUARIA DEL COMAHUE (IBAC), FACULTAD DE CIENCIAS AGRARIAS, UNIVERSIDAD NACIONAL DEL COMAHUE}
\\\Email{gustavo.gimenez@faea.uncoma.edu.ar}}
{conteo; modelos generalizados; binomial negativa} 
 {Biología} 
 {Otras categorías metodológicas} 
 {104} 
 {229-1}
{En ecología es frecuente encontrar datos generados a partir de conteos, emergencia de plántulas, frecuencias, abundancia de especies o carga de parásitos. El método frecuentemente empleado para modelar estos datos es asumir que los mismos se aproximan a una distribución Poisson y especificar modelos estadísticos acordes a esta distribución. Sin embargo, un problema persistente con el modelo Poisson es que los datos exhiben una marcada sobredispersión, donde la varianza de la variable de respuesta es mayor a la esperanza, resultando un ajuste pobre de los datos. Esta situación se observó en los datos del presente trabajo: emergencia de plántulas registrada en un campo desertificado en el Oeste de Río Negro durante un año (septiembre de 2013 al 2014) con y sin pastoreo ovino. La emergencia se evaluó sobre dos transectas de 25 m de longitud en un área excluida al pastoreo y dos transectas de igual longitud en sitios pastoreados. Las plántulas se muestrearon mediante un cuadro de 0,1 m2 que se situó sobre el terreno a intervalos regulares de 2 m en cada transecta, las plántulas emergidas se contabilizaron y clasificaron de acuerdo al criterio de Raunkiaer que se basa en agrupar a las especies vegetales según su biotipo. Asimismo, se obtuvieron los valores de humedad del suelo en superficie como en profundidad. Inicialmente, se ajustó un modelo Poisson mostrando una marcada sobredispersión donde la devianza superaba ampliamente los grados de libertad del modelo, el test de sobredispersión indicaba un amplio rechazo. Luego, se utilizaó un modelo quasi-poisson y un modelo binomial negativo. El modelo que permitió ajustar adecuadamente los datos y contemplar la sobredispersión fue un modelo binomial negativo mixto considerando el efecto transecta y recuadro como efectos aleatorios. Se arribó a este modelo con una devianza de 53. 43 y donde el test de ajuste no fue rechazado (p-valor: 0.46), se comprobó a partir del test de sobredispersión que bajo este modelo no resultaba significativa (p-valor= 0,49). Los residuos del modelo se mostraron satisfactorios, sólo el gráfico residuos versus predichos indicaba una leve falta de ajuste en valores de emergencia bajos. Luego, mediante pruebas de cociente de verosimilitud y valores de Akaike, se dedujo que no existió efecto del pastoreo, aunque se destaca el efecto de los biotipos así como también un importante efecto del mes y del año, la humedad del suelo pudo verse confundido con el efecto mes y año no resultando significativa.}
