\A
{GAMMA MIXED MODEL FOR LONGITUDINAL DATA ANALYSIS: MEMORY EVALUATION OF RATS WITH CEREBRAL ISCHEMIA COMBINED WITH DIABETES}
{\Presenting{M.H.D. RIBEIRO}$^1$\index{RIBEIRO, M}, M. V. O. PERES$^2$\index{PERES, MVO}, A. NUNES$^3$\index{NUNES, A}, R.M.W. DE OLIVEIRA$^3$\index{DE OLIVEIRA, R}, H. MILANI$^3$\index{MILANI, H} y I. PREVIDELLI$^4$\index{PREVIDELLI, I}}
{\Afilliation{$^1$DEPARTMENT OF MATHEMATICS, FEDERAL TECHNOLOGICAL UNIVERSITY OF PARANÁ, PATO BRANCO, PARANÁ, BRAZIL}
\Afilliation{$^2$DEPARTMENT OF  SOCIAL MEDICINE, UNIVERSITY OF SÃO PAULO, RIBEIRÃO PRETO MEDICAL SCHOOL, BRAZIL}
\Afilliation{$^3$DEPARTMENT OF PHARMACOLOGY AND THERAPEUTICS, STATE UNIVERSITY OF MARINGA, MARINGÁ, PARANÁ, BRAZIL}
\Afilliation{$^4$DEPARTMENT OF STATISTICS, STATE UNIVERSITY OF MARINGA, MARINGÁ, PARANÁ, BRAZIL}
\\\Email{mribeiro@utfpr.edu.br}}
{correlation; gamma distribution; random effect} 
 {Otras ciencias de la salud} 
 {Otras categorías metodológicas} 
 {32} 
 {78-2}
{Studies with repeated measures have gained prominence in the scientific scenario in recent years, mainly in the areas of health and biological sciences. This is due to the interest of researchers evaluate repeatedly one or more experimental units in order to know the effectiveness of a treatment or the evolution of a patient through an intervention. In this sense, a methodology that can be used to model data thus characterized, when the distribution of the response variable belongs to the exponential family, is that of the generalized linear mixed models, whose focus is on accommodating the inter and intra individual variations of the longitudinal measurements. The estimation of the vector of fixed parameters of this model as well as the components of the variance-covariance matrix will be done through the maximization of the marginal likelihood function obtained through integration under the random effects . However, this integration have not analitical solution. Thus is necessary to use numerical integration methods, which provide approximate solutions to given integrals . Thus, in this study a mixed gamma model with logarithmic link function and random effects with normal distribution was used to evaluate longitudinal behavioral data, by means latency response variable. These measure is a positive real number, and data are assimetric. So the gamma distribution is a candidate to modeling data with these characteristics. In this case, the effects of cerebral ischemia, was induced by chronic cerebral hypoperfusion combined with diabetes, on the performance of long-term retrograde memory were studied in rats. The R software was used to fit the model. Based on the results obtained, the proposed model presented good fit (evaluation by means half-normal plot) and accommodated the correlation inherent to the data. Residual analysis was satisfactory (fitted values vs earson Residual). Through influence analysis influential points was not observed. Using this modelating was possible identify that physiologicly normal animals had better results than other animals. Therefore, this methodology becomes an alternative to the use of frequently used traditional methods, which disregard the effect of longitudinal dependence and may make inferences developed inappropriate.}
