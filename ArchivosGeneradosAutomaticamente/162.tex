\A
{ELABORACIÓN DE PERFILES EPIDEMIOLÓGICOS EN ESTUDIOS SANITARIOS MEDIANTE TÉCNICAS DE CLUSTERING DIFUSO Y ANÁLISIS DE REDES}
{\Presenting{RAMON ÁLVAREZ}\index{ALVAREZ, R} y FERNANDO MASSA\index{MASSA, F}}
{\Afilliation{INSTITUTO DE ESTADÍSTICA - DMMCC - FCEA}
\\\Email{ramon@iesta.edu.uy}}
{análisis de redes; clustering; entropía; factores de riesgo; variables binarias} 
 {Salud humana} 
 {Métodos multivariados} 
 {162} 
 {340-1}
{En los estudios epidemiológicos es práctica habitual trabajar con variables binarias que reflejan la presencia de determinadas enfermedades, las que a su vez se asocian con otro conjunto otras enfermedades, denominadas comorbilidades, medidas también a através de variables binarias y que en general se asumen como factores de riesgo de las primeras. En el ámbito de los estudios epidemiológicos existen situaciones donde se manejan enfermedades no transmisibles (ENT), en particular en salud bucal, donde ambos tipos de variables pueden ser intercambiables en cuanto a quien hace el rol de factor de riesgo. Teniendo en cuenta esta situación se proponen comparar dos análisis para la determinación de tipologías de encuestados en base a los atributos binarios, de forma de obtener perfiles epidemiológicos bien diferenciados. Los datos utilizados corresponden a un estudio en personas que demandan atención en la Facultad de Odontología-UDELAR durante el período 2015-2016. Por un lado se emplea un método mixto que combina clustering basado en medidas de entropía con una partición difusa estimada con el algoritmo c-modes y por otro lado, a través del análisis de redes (SNA), a partir de las mismas variables se construye la matriz de adyacencias sobre la que se aplican una batería de métricas (closeness, betweenness, modularity, clustering) sobre los nodos y enlaces, que permite detectar comunidades. Las comunidades detectadas mediante el SNA a través del uso de diferentes algoritmos de búsqueda, como el de fast greedy o de random walk o de particionado espectral, se cruzan con las tipologías creadas mediante el método mixto de clustering.}
