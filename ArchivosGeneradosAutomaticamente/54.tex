\A
{TRASTORNOS DEL ESPECTRO AUTISTA ESTIMACIÓN DE SU PREVALENCIA A EDADES TEMPRANAS EN LA CIUDAD DE SANTA FE, COSTOS DE TRATAMIENTO Y EFECTOS DE LA ACTIVIDAD FÍSICA SOBRE HABILIDADES MOTORAS GRUESAS}
{\Presenting{LILIANA ESTER CONTINI}$^1$\index{CONTINI, L}, FRANCISCO ASTORINO$^2$\index{ASTORINO, F}, DIEGO CARLOS MANNI$^{1,2}$\index{MANNI, D}, GABRIEL FESSIA$^2$\index{FESSIA, G} y ELENA FERNÁNDEZ DE CARRERA$^2$\index{FERNÁNDEZ DE CARRERA, E}}
{\Afilliation{$^1$FACULTAD DE BIOQUÍMICA Y CIENCIAS BIOLÓGICAS - UNL}
\Afilliation{$^2$FACULTAD DE CIENCIAS MÉDICAS - UNL}
\\\Email{lcontini@fbcb.unl.edu.ar}}
{tea; prevalencia; epidemiologia} 
 {Otras ciencias de la salud} 
 {Otras categorías metodológicas} 
 {54} 
 {116-1}
{El autismo es un complejo trastorno del desarrollo psicomotor que comienza antes de los tres años de edad y persiste durante toda la vida del individuo. La importancia de la detección precoz de los niños con trastornos del espectro autista (TEA) permite la intervención oportuna en etapas iniciales de la vida impactando positivamente en la calidad de vida del paciente y las de sus familias. No se conoce con exactitud la prevalencia de TEA. Según la OMS en las Américas es de de 65,5/10.000 (34-90/10.000). Esta estimación representa una cifra media, pues la prevalencia observada difiere considerablemente entre las distintas investigaciones, si bien, en estudios bien controlados se han registrado cifras notablemente mayores. En Argentina no se tienen datos al respecto. Esta información es primordial para establecer las bases de una propuesta de Salud Pública que permita lograr diagnósticos precoces e intervenciones oportunas, como lo expresan las leyes Nacional y de la Provincia de Santa Fe de Autismo. Por esta razón, un grupo de investigadores multidisciplinario de la Universidad Nacional del Litoral, realizó una investigación con el fin de determinar la prevalencia de TEA en niños pequeños en la ciudad de Santa Fe, así como también se propuso evaluar los beneficios de la actividad física en niños con TEA y los costos directos e indirectos destinados a diagnóstico, tratamiento, educación. El objetivo de este trabajo es presentar los hallazgos de esta investigación. Prevalencia: se realizó un muestreo estratificado con afijcación proporcional en la ciudad. Se evaluaron 506 niños de 18 a 36 meses de edad resultando una prevalencia temprana estimada de1 cada 128, valor que se encuentra dentro del rango dado por la OMS para las Américas. Costos: Si bien este ítem es muy difícil de determinar debido a la multiplicidad de terapias, sistemas de salud y de educación, se realizó una encuesta, de respuesta voluntaria, a padres de niños con TEA de provincias de Argentina. Sólo fueron respondidas 28 de más de 50 entregadas. De ellas se pudo concluir que el rango de los gastos por niño por mes es: $1809,44 - $24427,44. Información que no puede compararse, no se conocen estudios similares en Argentina. Actividad física: De la evaluación de la experiencia realizada en adolescentes con TEA, se encontró que en todos los casos hubo aumento de las habilidades motoras gruesas que variaron entre 40,0\%-85,7\% de mejoría respecto de las de habilidades iniciales. }
