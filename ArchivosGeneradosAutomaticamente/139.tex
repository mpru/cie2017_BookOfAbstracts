\A
{ANÁLISIS DE UN PROYECTO ESTADÍSTICO BASADO EN UN SISTEMA DE INDICADORES DE RAZONAMIENTO Y PENSAMIENTO ESTADÍSTICO}
{\Presenting{MARIEL LOVATTO}\index{LOVATTO, M} y LILIANA TAUBER\index{TAUBER, L}}
{\Afilliation{UNIVERSIDAD NACIONAL DEL LITORAL}
\\\Email{estadisticabiologiafhuc@gmail.com}}
{razonamiento estadístico; pensamiento estadístico; evaluación continua; formación de profesores} 
 {Enseñanza de la estadística} 
 {Otras categorías metodológicas} 
 {139} 
 {305-3}
{En el presente trabajo describimos el significado de los indicadores que consideramos dentro de un sistema. Estos indicadores nos permiten analizar las relaciones entre conceptos estocásticos, que pueden ponerse en práctica cuando se elabora una propuesta didáctica basada en el trabajo por proyectos. Tomando como fundamento el sistema de indicadores mencionado antes, describimos el análisis de contenido realizado sobre un proyecto para el aula de Estadística, elaborado por un docente que ejerce actualmente en el Nivel Secundario. Por último, realizamos una discusión de los resultados obtenidos a partir del análisis de contenido, problematizando sobre los conflictos que encontramos en nuestro caso de estudio en relación con el razonamiento y pensamiento estadísticos. Para la elaboración de los indicadores nos basamos en las dimensiones uno y dos del pensamiento estadístico de Wild \& Pfannkuch (1999), definidos como ciclo investigativo y tipos de pensamientos respectivamente. Dentro de la dimensión dos encontramos los cinco componentes que, según las autoras, son fundamentales para el desarrollo del pensamiento estadístico: Necesidad de los datos, Transnumeración, Variabilidad, utilización de modelos, contexto y conceptos. Dentro de cada componente definimos indicadores los cuales utilizamos para analizar un proyecto, resultado del Trabajo Final obligatorio para aprobar el Módulo de Enseñanza de la Probabilidad y la Estadística (Tauber e INFD, 2016), el cual es una instancia obligatoria en el cursado de la Especialización Docente de Nivel Superior en Enseñanza de la Matemática en la Educación Secundaria, ofrecida de manera virtual por el Instituto Nacional de Formación Docente (INFD). Logramos observar en el análisis del proyecto desajustes entre lo planteado por los docentes en cada actividad, objetivo y problemas, con respecto a los indicadores planteados. Se espera desarrollar análisis de nuevos proyectos para comparar entre ellos. Estos conflictos nos hacen reflexionar sobre la necesidad de promover la formación de los docentes que están encargados de alfabetizar estadísticamente a ciudadanos que puedan actuar críticamente y debatir de manera fundamentada ante la información estadística.}
