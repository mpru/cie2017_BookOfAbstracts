\A
{INTERVALOS DE CONFIANZA PARA LA RAZÓN COSTO-EFECTIVIDAD: UNA COMPARACIÓN DE TÉCNICAS PARAMÉTRICAS Y NO PARAMÉTRICAS MEDIANTE SIMULACIÓN}
{\Presenting{PABLO COTTET}\index{COTTET, P} y NORA ARNESI\index{ARNESI, N}}
{\Afilliation{INSTITUTO DE INVESTIGACIONES TEÓRICAS Y APLICADAS DE LA ESCUELA DE ESTADÍSTICA}
\\\Email{narnesi@fcecon.unr.edu.ar}}
{economía en salud; razón costo-efectividad; estimación por intervalos de confianza} 
 {Salud humana} 
 {Inferencia estadística} 
 {16} 
 {37-1}
{La evaluación económica en salud abarca diversas técnicas o procedimientos para recabar información sobre la relación que existe entre el costo y el resultado de implementar determinadas intervenciones en salud. Si la medida de resultado que se incorpora al proceso de toma de decisiones se relaciona con la experiencia del paciente ante el evento clínico de interés, la evaluación se denomina: Análisis de Costo-Efectividad. La medida de mayor interés en estas evaluaciones económicas en salud es la Razón de Costo-Efectividad Incremental conocida habitualmente por su sigla en inglés ICER, la cual se define como el cociente entre el incremento del valor esperado de los costos de las intervenciones a comparar y el incremento del valor esperado de las efectividades correspondientes. Resulta natural intentar cuantificar la incertidumbre en la estimación del ICER a través de la construcción de intervalos de confianza, sin embargo uno de los problemas que se presenta es la necesidad de realizar supuestos acerca de la distribución muestral del estimador del ICER sumado a complejidad de la obtención de la variancia del estimador de una razón. Para subsanar este inconveniente, recientemente, gran parte de la bibliografía sobre el tema se ha focalizado en el uso de métodos no paramétricos. De ahí que el objetivo de este trabajo es comparar los resultados hallados a partir de métodos derivados del enfoque paramétrico clásico y el correspondiente al método no paramétrico basado en Bootstrap. Se comparan los Intervalos de Confianza, calculados a partir de los datos obtenidos de la simulación de una muestra, bajo el esquema de un árbol de decisión de 200 pacientes, de una situación hipotética donde se comparan dos técnicas quirúrgicas. El valor estimado del ICER es aproximadamente igual a 33 y representa el costo de una unidad adicional de efectividad. El intervalo más estrecho se logra a partir de la utilización del método no paramétrico y resulta igual a [30.25 ; 35.47]. En cambio, los métodos tradicionalmente utilizados en el enfoque paramétrico arrojan intervalos de confianza menos precisos. El intervalo más amplio se obtiene por la aplicación del método de Fieller, [2.61 ; 160.02], seguido por el que utiliza la aproximación de la variancia por series de Taylor y el método Confidence Box. Los resultados hallados en este estudio por simulación, sumados a la ventaja de no requerir supuestos distribucionales apoyan la recomendación del uso del método Bootstrap para la determinación de los límites de confianza. }
