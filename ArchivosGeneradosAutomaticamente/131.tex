\A
{EVALUACIÓN DE LA EFICIENCIA REPRODUCTIVA EN VACAS LECHERAS CON MODELOS DE COX MIXTOS}
{\Presenting{MONICA PICCARDI}$^1$\index{PICCARDI, M}, GABRIEL BÓ$^2$\index{BO, G@BÓ, G} y MONICA BALZARINI$^1$\index{BALZARINI, M}}
{\Afilliation{$^1$FACULTAD DE CIENCIAS AGROPECUARIAS – UNC; CONICET}
\Afilliation{$^2$INSTITUTO DE REPRODUCCIÓN ANIMAL (IRAC); INSTITUTO DE CIENCIAS BÁSICAS Y APLICADAS – UNVM}
\\\Email{monicapiccardi@gmail.com}}
{días vacíos; efectos aleatorios; análisis de sobrevida} 
 {Ciencias agropecuarias} 
 {Modelos de regresión} 
 {131} 
 {293-1}
{Los establecimientos lecheros utilizan distintos indicadores para monitorear el desempeño reproductivo, como son los días vacíos (DV). Los DV son los días que transcurren desde que la vaca tiene un parto hasta que se preña nuevamente de manera efectiva. Existen diferentes herramientas metodológicas que pueden ser usadas para construir este indicador. El uso de medidas resúmenes como estadísticas descriptivas simples para estimar los DV a partir de los datos individuales de un rodeo puede introducir sesgos en la evaluación del desempeño reproductivo ya que éstas estadísticas sólo incluyen la información del tiempo en días desde parto a la concepción de vacas que se han preñado. Los animales que son descartados por no preñarse o por haber sufrido un aborto, o bien por haber sido descartados antes de tener la oportunidad de preñarse no aportan información para el cálculo descripto de los DV. Una mejor alternativa para la obtención de este indicador es su estimación a partir del análisis de sobrevida para el tiempo que transcurre hasta la preñez considerando todas las vacas. Esta técnica estadística no sólo incluye a los animales que experimentaron el evento, sino que también incluye a aquellos que no experimentan el evento o se pierden durante el periodo de observación. Los DV dependen del efecto del animal, pero también de la época del parto (época calurosa versus fresca) y del número de lactancia y en general del manejo reproductivo del tambo. El objetivo de este trabajo fue ajustar y comparar 6 modelos candidatos de regresiones de cox de efectos mixtos para modelar los DV. Para esto se utilizó una base de datos compuesta por 22.456 vacas de 35 tambos de la zona centro y sur de Santa Fe y provincia de Córdoba. Los efectos fijos del modelo con más parámetros fueron estación de parto, número de lactancia y su interacción; y los efectos aleatorios fueron el tambo y las vacas dentro de tambo. Usando la prueba de cociente de verosimilitud ajustada (LRT) se seleccionó el modelo donde el efecto aleatorio era las vacas independientemente de los tambos y ambos efectos fijos sin interacción. Los blup del efecto vaca fueron extraídos y usados para seleccionar a las mismas por fertilidad.}
