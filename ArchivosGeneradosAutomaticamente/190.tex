\A
{DESARROLLO SUSTENTABLE DE CULTIVOS HIDROPÓNICOS PARA CONSUMO HUMANO: ANÁLISIS DE TASAS DE CRECIMIENTO}
{\Presenting{JOSÉ M VARGAS}$^1$ $^2$\index{VARGAS, J} y RICARDO D JUAN$^1$\index{JUAN, R}}
{\Afilliation{$^1$UNIVERSIDAD NACIONAL DE VILLA MARÍA, INSTITUTO DE CIENCIAS BÁSICAS Y APLICADAS}
\Afilliation{$^2$UNIVERSIDAD NACIONAL DE CÓRDOBA, FACULTAD DE CIENCIAS ECONÓMICAS}
\\\Email{donjmvargas@gmail.com}}
{hidroponía; cultivo de hojas verdes; agua; agricultura sustentable; cultivo orgánico} 
 {Ciencias agropecuarias} 
 {Diseño de experimentos} 
 {190} 
 {906-1}
{La hidroponía es la ciencia del crecimiento de las plantas sin utilizar suelo. Consiste en el cultivo en un medio líquido donde las raíces absorben los nutrientes de una solución. El objetivo general es determinar la factibilidad de producción de alimentos de hojas verdes hidropónico (AHVH) sin la utilización de pesticidas, como tecnología apta para pequeños productores de verduras, cuantificando la optimalidad de producción, posibilidades y rentabilidad. Como objetivos específicos se plantean optimizar las formas de producción de AHVH, controlando la fertilización y reutilización del agua de riego y estudiar las tasas de crecimiento de cada planta, modelando curvas de crecimiento de las mismas. En un invernadero montado en la UNVM se realizó un primer experimento de cultivo hidropónico de lechuga en el cual se midieron el crecimiento de cada planta en tres densidades de cultivo. Sobre una estructura metálica, se dispusieron 12 tubos de PVC de 12 cm de diámetro por 5 mts de largo en cuatro estantes, tres por nivel, interconectados por mangueras. Con un diseño en bloques completamente aleatorizado, en cada tubo se dispusieron tres tramos, en cada uno de los cuales se realizaron agujeros para albergar plantines de hojas verdes a igual distancia, 20cm, 25cm y 30cm. Las tres distancias de separación entre plantines son considerados tratamientos; tubos y niveles de estantería, bloques. El agua, con los nutrientes necesarios, circula inyectada con temporizador por una bomba y drenaje.  Cada planta es una unidad experimental. Las variables de respuesta medidas fueron longitud de la hoja más larga, cantidad de hojas y peso fresco. Al término del experimento se pesó cada planta para medir el efecto de la densidad de cultivo y el efecto de la luz a distintos niveles  de la estantería, encontrando diferencias significativas entre las medias de los tratamientos con distancias 20cm y 30cm entre plantas. Contrastes entre las medias a distintos niveles de estanterías muestran diferencias significativas entre los tres niveles correspondientes a los estantes más altos, no así entre los dos niveles más bajos. Seguimiento de las longitudes de hojas más larga y la cantidad de hojas  de cada planta durante las últimas cuatro semanas del experimento ajustado por modelos longitudinales, muestra que el desarrollo de las hojas es lineal e independiente de la densidad de cultivo. La cantidad de hojas, en cambio, depende del nivel de la estantería, favoreciendo mayor producción de hojas a mayor iluminación.}