\A
{APLICACIÓN DEL ANÁLISIS DE COMPONENTES PRINCIPALES PARA LA OBTENCIÓN DE PATRONES DE CONSUMO ALIMENTARIO EN UN ESTUDIO DE CASOS Y CONTROLES DE ENFERMEDAD COLELITIÁSICA}
{\Presenting{GUILLERMINA ISERN}$^1$\index{ISERN, G}, AGUSTINA BERTOLA COMPAGNUCCI$^2$\index{BERTOLA COMPAGNUCCI, A} y STELLA MARIS PEZZOTTO$^2$\index{PEZZOTTO, S}}
{\Afilliation{$^1$FACULTAD DE CIENCIAS ECONÓMICAS Y ESTADÍSTICA. UNR}
\Afilliation{$^2$IDICER-CONICET. FACULTAD DE CIENCIAS MÉDICAS. UNR}
\\\Email{gisern@fcecon.unr.edu.ar}}
{análisis de componentes principales; patrones de consumo alimentario; casos y controles; enfermedad colelitiásica} 
 {Salud humana} 
 {Métodos multivariados} 
 {63} 
 {152-1}
{El análisis de componentes principales se ha convertido en una herramienta fundamental de la epidemiología nutricional para la obtención de patrones de consumo alimentario, ya que los mismos describen tendencias de ingestas características de diversos grupos. El objetivo de este trabajo fue determinar el patrón de consumo alimentario de los casos y de los controles en un estudio de identificación de factores de riesgo alimentarios de enfermedad colelitiásica (EC). Los casos son 51 pacientes con diagnóstico de EC, algunos de los cuales ya han sido colecistectomizados. Los controles son 69 personas a las que se les realizó una ecografía abdominal para descartar la presencia de cálculos asintomáticos. A todos los participantes se les realizó una entrevista personal y se aplicó un cuestionario de frecuencia de consumo (FFQ) con 210 items para la recolección de la ingesta alimentaria. Para establecer el tamaño de las porciones alimentarias se utilizó un atlas asociado al FFQ. Con los datos obtenidos se calcularon los gramos consumidos de los diferentes grupos de alimentos, generando 29 variables, que fueron incluidas en un análisis de componentes principales (ACP). De la aplicación del ACP se seleccionaron las 3 primeras componentes, que representan un 34\% de la variabilidad total. Entre las tres primeras componentes principales obtenidas, se determinaron dos que permiten diferenciar los casos de los controles: la primera y la tercera. Los casos se caracterizaron por consumir mayor cantidad de grasas animales, vísceras, azúcar, bebidas azucaradas, papas, cereales pastas y granos, fiambres y embutidos, pollo con piel y carne vacuna grasa, además de consumir poca cantidad de vegetales rojos y amarillos, coles, otras frutas, y pescado. En cambio, los controles se caracterizaron, en su mayoría, por condiciones contrarias a las descriptas para los casos. Además, según la tercera componente, los controles se caracterizan por consumir mayor cantidad de pollo sin piel, frutas secas, carne vacuna magra, lácteos enteros y frutas frescas y bajas cantidades de pollo con piel. Así, quedaron conformados 2 patrones de consumo alimentario. De esta manera queda evidenciado que el análisis de componentes principales es una herramienta útil y acorde para la generación de patrones de consumo alimentario en vista a su aplicabilidad en mensajes de prevención de las enfermedades. }
