\A
{LA ENTROPÍA COMO MÉTODO PARA COMPARAR IMÁGENES EN EL TIEMPO}
{\Presenting{SILVIA P MELO}$^1$\index{MELO, S} y RODOLFO C J SALOMÓN$^2$\index{SALOMÓN, R}}
{\Afilliation{$^1$DEPARTAMENTO DE MATEMÁTICA, UNIVERSIDAD NACIONAL DEL SUR, ARGENTINA}
\Afilliation{$^2$DEPARTAMENTO DE GEOLOGÍA, UNIVERSIDAD NACIONAL DEL SUR, ARGENTINA}
\\\Email{smelo@uns.edu.ar}}
{entropía; asociación; imagen; probabilidad} 
 {Otras aplicaciones} 
 {Probabilidad y procesos estocásticos} 
 {2} 
 {2-1}
{Una sucesión de imágenes, que ofrecen información respecto a un determinado proceso, pueden ser abordadas desde diversos puntos de vista. Al analizar los cambios que se producen a lo largo del tiempo, resulta de interés poder determinar en qué momento comienzan a aparecer diferencias que puedan ser consideradas significativas. En esta instancia, es deseable poder obtener resultados que no dependan de la experiencia y / o criterio del observador. En el presente trabajo se busca establecer un método automático y repetible en el tiempo, accesible en su interpretación, de fácil implementación y aplicación, tanto para el usuario entrenado como para el no entrenado. El principio es la aplicación del concepto de entropía . Se considera la información subyacente en cada imagen, como una matriz de m x n puntos de resolución, cuyos elementos corresponden a los valores de luminosidad tomados en escala de grises. En base a esta matriz, se estima la distribución de probabilidades de la escala de grises por medio de las frecuencias relativas observadas, lo que permite el cálculo de la entropía. El método propuesto es: tomando como punto de partida la entropía de la imagen inicial, considerar las diferencias con los correspondientes valores de entropía de las restantes imágenes. El criterio para tal comparación se fundamenta en que pequeños cambios generados por el tratamiento aplicado, no modificarán sustancialmente la distribución de probabilidades inicial. Cuando estos cambios sean mayores, la disociación entre las imágenes quedará evidenciada, tornándose significativa la diferencia entre las entropías. Se presentan resultados obtenidos mediante simulación y por aplicación a datos reales. }
