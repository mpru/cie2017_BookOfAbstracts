\A
{DISTRIBUCIÓN Y PROPORCIÓN DE TESTIGOS EN DISEÑOS CON TESTIGOS A LA PAR}
{\Presenting{ANDREA PIA}$^1$\index{PIA, A}, MARIANO CORDOBA$^2$\index{CORDOBA, M} y JULIO ALEJANDRO DI RIENZO$^2$\index{DI RIENZO, J}}
{\Afilliation{$^1$FACULTAD DE AGRONOMÍA, UNIVERSIDAD NACIONAL DE LA PAMPA, ARGENTINA}
\Afilliation{$^2$FACULTAD DE CIENCIAS AGROPECUARIAS, UNIVERSIDAD NACIONAL DE CÓRDOBA, ARGENTINA}
\\\Email{dirienzo.julio@gmail.com}}
{mejoramiento vegetal; testigos a la par; correlación espacial} 
 {Ciencias agropecuarias} 
 {Diseño de experimentos} 
 {115} 
 {250-1}
{La primera etapa en el desarrollo de nuevos híbridos y variedades es la evaluación de nuevos genotipos resultantes del cruzamiento de parentales seleccionados. Los programas modernos de selección están enfrentando el síndrome “big data”. Estos programas manejan miles de genotipos en pequeñas parcelas sin repeticiones. La única forma de estimar la varianza residual es incluir réplicas de testigos seleccionados. Debido a que el ensayo puede tener una gran dimensión en el terreno los testigos también tienen el propósito de capturar la variabilidad espacial. El número y ubicación de los testigos es usualmente decidida por razones económicas y operativas. Sin embargo, debido a que el proceso de selección depende de la calidad de las estimaciones del rendimiento de los nuevos genotipos, la decisión de cuántas repeticiones hacer de los testigos y dónde ubicarlos requiere un análisis estadístico. Presentamos los resultados de un estudio de simulación en el que se evalúan varias formas de ubicación sistemática de los testigos, así como la proporción de los mismos en dos tipos de arreglos: uni y bi-dimensionales. Lo escenarios simulados fueron generados teniendo en cuenta parámetros típicos que caracterizan estadísticamente la distribución de rendimientos en ensayos típicos de maíz, soja y girasol. Los datos simulados incluyeron errores correlacionados espacialmente. La correlación (modelo exponencial) se especificó mediante la varianza y el range del proceso gaussiano estacionario subyacente. La simulación se realizó mediante un script escrito en lenguaje R, e implementado alrededor de la librería RandomFields. Para cada escenario y repetición, los datos fueron evaluados suponiendo independencia y también modelando la correlación espacial. La variable evaluada fue la correlación de Spearman entre los efectos genotípicos simulados y los predichos por el modelo ajustado. Los resultados muestran que, como es de esperar, cuanto mayor es la proporción de testigos mejor es la predicción. Sin embargo, como el propósito de los ensayos no es evaluar testigos, y la “ganancia en predicción” es mayor cuando se pasa de una proporción baja a una proporción intermedia, que cuando se pasa de una proporción intermedia a un alta de testigos, la recomendación es utilizar proporciones intermedias (20\% del número total de parcelas en arreglos unidimensionales, y 10\% en arreglos bidimensionales). En los arreglos bidimensionales, no todos los arreglos son igualmente efectivos. }
