\A
{MODELOS PARA EL TRATAMIENTO DE SESGOS EN ESTUDIOS DE DATOS AGREGADOS EN ESPACIO Y/O TIEMPO}
{\Presenting{C CUESTA}\index{CUESTA, C}, I BARBONA\index{BARBONA, I}, G ISERN\index{ISERN, G} y N MEROI\index{MEROI, N}}
{\Afilliation{FACULTAD DE CIENCIAS ECONÓMICAS Y ESTADÍSTICA, UNR}
\\\Email{ccuesta@fcecon.unr.edu.ar}}
{falacia ecológica; estudios ecológicos; epidemiología} 
 {Otras ciencias de la salud} 
 {Modelos de regresión} 
 {10} 
 {16-1}
{La denominada “falacia ecológica” en estudios cuya unidad de observación es un área geográfica (o tiempo), y donde la variable observada es una estadística resumen (promedio, porcentaje, tasa, etc), se refiere al sesgo que se comete al querer extrapolar las asociaciones observadas a nivel de área, a los individuos de las mismas. Este tipo de sesgos puede producirse debido a diferentes motivos: la incapacidad de determinar la temporalidad de la variable respuesta y la explicativa, la posible presencia de variables confundentes, la imposibilidad de contar con las distribuciones de las medidas agrupadas, etc. A pesar del conocimiento de estos problemas, los estudios ecológicos siguen siendo muy utilizados en diversas áreas debido a su sencillez, bajo costo, fácil obtención de la información, etc. y son utilizados, generalmente en una faz exploratoria. A fin de controlar el posible sesgo asociado a estos estudios se han propuesto diferentes modelos, entre ellos los que tienen en cuenta estratos definidos de acuerdo a alguna posible variable de confusión. La ventaja de estos modelos estratificados es que se trabaja con grupos más pequeños que el total del área geográfica permitiendo tener en cuenta posibles variables de confusión (que son las que determinan los estratos). Sin embargo, esta metodología puede no ser adecuada cuando se carece de la información a nivel de estrato y/o cuando las áreas de estudio son demasiado pequeñas. En este trabajo se presenta un modelo que pretende controlar el mencionado sesgo combinando distintas fuentes de datos, una que incluye los datos agrupados y otra que incluye los datos a nivel individual (que puede no ser la misma que la anterior). Este modelo es un intermedio entre utilizar datos a nivel individuo y datos a nivel área geográfica. Se comparan los resultados de este modelo con los que resultan bajo un modelo estratificado y con uno que considera los individuos como unidad de análisis. Se ejemplifica la metodología utilizando datos de la Encuesta Nacional de Factores de Riesgo llevada a cabo por las Direcciones Provinciales de Estadística entre octubre y diciembre de 2013. Se estudia la asociación entre obesidad y diferentes factores tales como edad, sexo, nivel de instrucción, cobertura de salud, etc. }
