\A
{AN EXTENSION OF HORVITZ-THOMPSON ESTIMATOR APPLICATION: ESTIMATION OF NETWORK DEGREE DISTRIBUTION}
{\Presenting{ALEJANDRO IZAGUIRRE}$^1$\index{IZAGUIRRE, A} y GABRIEL MONTES ROJAS$^2$\index{MONTES ROJAS, G}}
{\Afilliation{$^1$UNIVERSIDAD DE SAN ANDRÉS, BUENOS AIRES, ARGENTINA}
\Afilliation{$^2$INSTITUTO INTERDISCIPLINARIO DE ECONOMÍA POLITICA DE BUENOS AIRES AND CONICET, FACULTAD DE CIENCIAS ECONÓMICAS, UBA}
\\\Email{izaguirre.ale@gmail.com}}
{horvitz-thompson estimation; networks; network sampling designs; degree distribution; average degree} 
 {Otras aplicaciones} 
 {Inferencia estadística} 
 {27} 
 {74-1}
{Most networks investigated today are parts of a much larger networks. There are many papers in which the focus is on understanding the extent to which characteristics of a sampled network are reflective of the complete network. A central statistical question in those works is, how much the properties of the sampled network reflect those of the true network. Typical characteristics of interest include degree distribution, density, diameter, clustering coeficient, average path length, among others. In general,under many sampling schemes, this measures are biased when we use sampled networks as if they were the true. In this paper we focus our attention on the degree distribution, this measure is one of the most fundamental characteristic associated with a network and,it may be afected by sampling, sometimes dramatically. In literature there exist many different networks sampling designs which can be probabilistic or nonprobabilistic, here we restrict our attention on probabilistic samples, as in the rest of the works related with this topic. In the first part we present an extension of Horvitz-Thompson estimator(HT) to be used under more general context, basically with partial information on the variable of interest. Briefly, suppose you have a population N and your interest is in variable yi, suppose you want to estimate the total of yi, Ty, by a sample of N, under some assumptions you could use the HT estimator, but to do so you have to observe yi in the sample. What we do in the first part of the work is to propose an extension of HT estimator to make it feasible under partial information on yi. Based on the previous methodology, in a second part, we propose an estimator for degree distribution (and its variance) to be used under any probabilistic sampling designs, and then we apply it for two widely used probabilistic sampling designs known as induced subgraph (nodes sampling) and incident subgraph (edges sampling). Finally we do a Monte Carlo simulation to assess some aspects of the theoretical results. }
