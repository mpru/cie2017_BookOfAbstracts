\A
{EVALUACIÓN DE PAPA POS COSECHA MEDIANTE MODELOS MIXTOS LINEALES}
{\Presenting{NELIDA DEL VALLE ORTIZ}\index{ORTIZ, N} y ELIZABETH L. VILLAGRA\index{VILLAGRA, E}}
{\Afilliation{UNIVERSIDAD NACIONAL DE TUCUMÁN}
\\\Email{ortiznelidadelvalle@gmail.com}}
{modelo mixto lineal; simex; r} 
 {Ciencias agropecuarias} 
 {Otras categorías metodológicas} 
 {182} 
 {368-2}
{El correcto equilibrio nutricional recibido durante el cultivo de papa, incide en el potencial de conservación del tubérculo en almacenaje. El balance de nutrientes afecta brotación, rendimientos, respiración e incide en la pérdida de agua por lo cual, afecta directamente la calidad de los tubérculos. La biofertilización constituye un método nutricional alternativo y ofrece perspectivas para un manejo sustentable. El objetivo de este trabajo es contrastar efectos de la fertilización química estándar con el manejo combinado de fertilización química y biofertilización sobre poscosecha de tubérculos de papa-consumo producidos en el Pedemonte tucumano. En este estudio se aplicaron siete tratamientos incluido un control. Los tubérculos se trataron en laboratorio bajo condiciones homogéneas, sin embargo, alguno se perdieron en el tiempo por putrefacción de manera aleatoria. La variable respuesta fue el peso de los tubérculos registrados durante diez observaciones en el tiempo con intervalos de diez días y las variables aleatorias de predicción fueron los días de observación posteriores a la cosecha y la referida a los tratamientos aplicados al cultivo. Se estima que podría tratarse de error de medida en el ensayo por la pérdida de tubérculos. Los datos se evaluaron mediante el ajuste de modelos mixtos lineales para datos continuos mediante el paquete estadístico nlme de R. Luego se sometió a la evaluación de modelación de variables con error de medida mediante la inferencia del método de simulación extrapolación SIMEX para variable continua. Se obtuvo diferencias entre los tratamientos utilizados. Se registraron algunas leves diferencias entre errores estándares de los estimadores. Se concluye que el ajuste de modelos mixtos lineales fue apropiado.}
