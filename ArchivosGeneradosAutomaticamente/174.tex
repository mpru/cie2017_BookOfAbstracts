\A
{EPIDEMIOLOGÍA DE LA MALARIA EN PANAMÁ DESDE LA PERSPECTIVA DEL NIÑO OSCILACIÓN DEL SUR}
{\Presenting{LISBETH AMARILIS HURTADO}$^1$\index{HURTADO, L}, JOSÉ CALZADA$^1$\index{CALZADA, J}, CHYSTRIE RIGG$^1$\index{RIGG, C}, MILIXA PEREA$^1$\index{PEREA, M}, SAHIR DUITARY$^1$\index{DUITARY, S}, MILAGROS CASTILLO$^1$\index{CASTILLO, M}, ELSA BENAVIDES$^2$\index{BENAVIDES, E} y LUIS CHAVES$^3$\index{CHAVES, L}}
{\Afilliation{$^1$INSTITUTO CONMEMORATIVO GORGAS DE ESTUDIOS DE LA SALUD}
\Afilliation{$^2$MINISTERIO DE SALUD DE PANAMÁ}
\Afilliation{$^3$PROGRAMA DE INVESTIGACIÓN EN ENFERMEDADES TROPICALES. ESCUELA DE MEDICINA VETERINARIA, UNIVERSIDAD NACIONAL, COSTA RICA}
\\\Email{lhurtado@gorgas.gob.pa}}
{malaria; plasmodium vivax; enso; panamá} 
 {Salud humana} 
 {Modelos de regresión} 
 {174} 
 {351-1}
{Introducción. En Panamá la malaria continúa siendo un reto para la salud pública; que con el tiempo ha ocupado una importante dimensión social de difícil intervención. En la epidemiología de la malaria se observa una tendencia descendente desde el 2005, cuando ocurrió la notificación de casos más alta (5095) y los últimos registros de mortalidad por esa enfermedad. En este lapso la transmisión de la malaria ha mostrado variaciones entre provincias y de un año a otro. Aún así, la morbilidad se mantiene focalizada y relativamente baja. Al igual que el resto de Centroamérica, la mayoría de las infecciones por malaria han sido por Plasmodium vivax (97\%). Geográficamente los casos se concentran en poblados indígenas. Varios estudios han sugerido que en años recientes el ambiente se ha comportado de manera anómala cuando se compara con las décadas pasadas. Esto indica que los programas de control de la malaria se enfrentarán ahora a una mayor incertidumbre en los patrones de la enfermedad. En consecuencia, saber hasta qué grado los factores ambientales posiblemente han cambiado patrones de transmisión e inhibido la probabilidad de éxito de Programas de Control es una incógnita importante. Desarrollo. Desde esta perspectiva, se realiza un estudio de tipo ecológico, en el cual se busca analizar la asociación entre los cambios ambientales y la transmisión espacio-temporal de la malaria en Panamá desde 1990 a 2015. Específicamente estudiamos los impactos de la variabilidad climática, medido por las diferentes fases del Niño Oscilación Sur (ENOS). Para ello, se aplicó un análisis de sincronía para estudiar los cambios en la incidencia de los casos de malaria entre las distintas provincias. Resultados. Los resultados sugieren que el Niño tiene un impacto sobre la epidemiología de la malaria en Panamá, sin embargo varia geográficamente.Conclusión. La última década ha sido testigo de los avances contra la eliminación de la malaria. La información epidemiológica muestra una reducción que vislumbra alcanzar esta meta. El éxito ha estado supeditado a lograr arraigar en los poblados indígenas el compromiso de eliminar la malaria. Hoy el escenario involucra, frenar los casos de Plasmodium falciparum que podrían deterior aún más la frágil condición de vida de los pueblos originarios de Panamá y establecer las medidas para reducir la vulnerabilidad de estas comunidades ante los cambios del ambiente.}
