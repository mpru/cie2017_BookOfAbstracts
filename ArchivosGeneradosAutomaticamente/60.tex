\A
{EL ROL DEL REFERENTE ESTADÍSTICO EN EL AREA DE SALUD PUBLICA}
{\Presenting{JUAN PABLO INCOCCIATI}$^1$\index{INCOCCIATI, J} y DIANA ROSANA KUCUKBEYAZ$^2$\index{KUCUKBEYAZ, D}}
{\Afilliation{$^1$MUNICIPIO DE LA MATANZA}
\Afilliation{$^2$UNTREF}
\\\Email{jpabloinco@gmail.com}}
{rol; lider; referente} 
 {Salud humana} 
 {Otras categorías metodológicas} 
 {60} 
 {143-1}
{¿QUE ES UN LIDER ¿QUIEN ES UN REFERENTE Hay lideres formales que solamente ocupan un cargo, pero el liderazgo autentico es básicamente la sumisión de significados y valores que obran como líneas de fuerza, estimulantes en el corto plazo e irradiadas hacia el largo plazo. El líder no tiene que ser excepcional, sino simplemente debe generar una relación excepcional con la organización que permita COMPROMETER más que involucrar. Los que ejecutan las tareas son los que mejor conocen los detalles del trabajo, sus posibles errores y las mejores alternativas para solucionarlo Todo miembro de una organización esta en condiciones y generalmente desea contribuir con la realización de una buena función. Las contribuciones proveen un sentido de pertenencia y favorecen la integración entre distintos niveles. La utilización de procesos estructurados para la mejora y resolución de los problemas usando medios estadísticos de control, produce a largo plazo mejores resultados que el control no estructurado, la incorporación de personal con formación estadistica en los efectores de salud , genera un aporte de calidad y confianza a los datos y su tratamiento Gerenciamiento, gestión y coordinación. Actitud de respeto y valoración a todas las actividades del área que se hicieran antes de su incorporación. Adaptación local de las propuestas acorde a la realidad de la región, respetando pautas culturales, sociales, económicas y políticas. Evaluación en forma permanente para producir reformas rápidas y acordes a los requerimientos locales. Facilitador de propuestas, utilizando a la comunicación como una herramienta de conducción participativa y abierta, dispuesto al aprendizaje continuo y a la innovación como respuesta a la variabilidad del sistema. Palabras claves Rol , Líder, Referente Eje temático: Salud Humana Conclusiones: Solucionar en parte la crisis de valores, colaborar en el descubrimiento de las verdaderas necesidades de la población. en el área de referencia, generar datos confiables que ayuden a la toma de decisiones en un área tan sensible como la salud publica. Palear a partir de esto las Fallas en la comunicación entre distintos niveles y con la comunidad. Utilización de un enfoque sistemático en el análisis de las situaciones}
