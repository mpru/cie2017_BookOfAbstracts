\A
{UN ESTUDIO SOBRE INDICADORES DE CARENCIAS CRÍTICAS A TRAVÉS DE UN ANÁLISIS DE COMPONENTES PRINCIPALES ROBUSTAS}
{\Presenting{PATRICIA CICCIOLI}\index{CICCIOLI, P} y JAVIER BUSSI\index{BUSSI, J}}
{\Afilliation{INSTITUTO DE INVESTIGACIONES TEÓRICAS Y APLICADAS DE LA ESCUELA DE ESTADÍSTICA, UNIVERSIDAD NACIONAL DE ROSARIO}
\\\Email{cicciolipatricia@gmail.com}}
{indicadores carencia; componentes principales; minimum covariance determinant; spherical principal components} 
 {Economía} 
 {Métodos robustos} 
 {121} 
 {260-1}
{Los indicadores socio-económicos de carencias críticas para ciudades y comunas de la provincia de Santa Fe permiten caracterizar estas poblaciones de acuerdo a 10 variables que describen aspectos relacionados con la conformación y las condiciones presentes en los hogares, como así también características de las viviendas y cuestiones referidas a escolaridad y situación laboral de la población joven. Estas variables se refieren a 158 ciudades y comunas de más de 2000 habitantes de la provincia de Santa Fe, los cuales provienen del Censo Nacional de Población, Hogares y Viviendas 2010. Para poder describir y resumir las diferencias existentes entre las ciudades y comunas con respecto a estos indicadores socio-económicos es posible utilizar el Análisis de Componentes Principales (ACP) que permite representar un conjunto de n observaciones con p variables a través de un número menor de variables construidas como combinaciones lineales de las originales y conservar la mayor variabilidad posible de los datos. Este método clásico utiliza medidas de variabilidad (como por ejemplo, la matriz de variancias y covariancias), las cuales están influenciadas por la presencia de valores extremos (outliers), lo que puede generar una distorsión en la representación. Debido a esta cuestión, en este trabajo se presentan dos métodos robustos para el ACP, el correspondiente a Matriz de Covariancia de Determinante Mínimo (Minimum Covariance Determinant, MCD ), y el método de Componentes Principales Esféricas (Spherical Principal Components, SPC). El objetivo de este trabajo es comparar estos dos métodos con el método clásico aplicándolos a estos indicadores socio-económicos. Para poder resumir las diferencias existentes entre las ciudades y comunas es necesario retener un número mayor de componentes principales en los métodos robustos (MCD y SPC) que en el método clásico. Se nota además que las variables que más influyen para resumir las diferencias de carencia social-económica son las mismas, siendo en total 4 de las 10 originales: Porcentaje de jefes de hogar con educación primaria incompleta, Porcentaje de hogares sin caño de agua dentro de la vivienda, porcentaje de hogares en vivienda sin retrete con descarga de agua y Porcentaje de población de 14 a 19 años que asiste a nivel de instrucción primario. Sin embargo, estas variables tienen una distinta participación en las componentes principales, según el método utilizado.}
