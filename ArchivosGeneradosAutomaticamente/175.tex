\A
{APLICACIÓN DE MODELOS LINEALES MIXTOS PARA EVALUAR LA CAPACIDAD PULMONAR DE ESCOLARES DE LA CIUDAD DE ARTIGAS (URUGUAY) Y LA RELACIÓN CON SU ENTORNO 2015-2016}
{\Presenting{RAMÓN ÁLVAREZ}$^1$\index{ALVAREZ, R}, YOHANA ALTEZ DE CASTRO\index{ALTEZ DE CASTRO, Y} y VIRGINIA BURGUETE EGUREN \index{BURGUETE EGUREN, V}}
{\Afilliation{$^1$INSTITUTO DE ESTADÍSTICA - DMMCC - FCEA}
\\\Email{altezyohana@gmail.com}}
{capacidad pulmonar; datos longitudinales; modelos lineales mixtos} 
 {Otras aplicaciones} 
 {Modelos de regresión} 
 {175} 
 {352-1}
{Los modelos lineales mixtos son una herramienta eficaz que permite el análisis de datos longitudinales. En la presente investigación se utilizará esta herramienta para analizar la capacidad pulmonar de los niños de la ciudad de Artigas (Uruguay), en cuatro etapas de monitoreo a lo largo de un año, contemplando periodos de intensificación de la actividad industrial (dentro o fuera de zafra) de la zona. OBJETIVOS: Evaluar diferentes tipos de modelos lineales mixtos para datos correlacionados temporal y espacialmente, que permitan evaluar si la capacidad pulmonar medida a través del pico flujo espiratorio se asocia con la edad, sexo, zona geográfica y la etapa de monitoreo, entre otras. MÉTODOS: El estudio consiste en monitorear la funcionalidad respiratoria de una muestra representativa de la población, que consta de 714 niños de 4 a 7 años inclusive, matriculados en las escuelas urbanas de la ciudad de Artigas. Se analiza la evolución de los individuos muestreados evaluando su capacidad pulmonar y su relación con otras variables. Se realizan comparaciones de distintos modelos lineales mixtos en los cuales se quiere explicar una variable cuantitativa (capacidad pulmonar) y su relación lineal con factores fijos, factores aleatorios y las covariables. RESULTADOS PRELIMINARES: Se analizaron perfiles medios y patrones de no respuesta, así como trayectorias individuales de cada escolar. CONCLUSIONES: Mediante el análisis descriptivo se muestra una diferenciación notoria entre niñas y varones sin identificar patrones con respecto a la zafra. Tampoco se aprecian asociaciones entre zona de residencia con respecto a la capacidad pulmonar, las cuales sí podrían detectarse con la aplicación de modelos lineales mixtos.}
