\A
{PRONÓSTICO DEL PRECIO DE ACCIONES EMPLEANDO MODELOS OCULTOS DE MARKOV}
{LUIS DAMIANO\index{DAMIANO, L}}
{\Afilliation{UNR}
\\\Email{damiano.luis@gmail.com}
}
{input-output hidden markov model; series de tiempo; finanzas; pronósticos para el precio de acciones; métodos bayesianos; markov-chain monte carlo} 
 {Otras ciencias económicas, administración y negocios} 
 {Métodos bayesianos} 
 {75} 
 {167-1}
{El pronóstico fuera de la muestra es posiblemente el problema estadístico de mayor interés en el ámbito de las series de tiempo financieras. Se emplea la inferencia Bayesiana para estimar los parámetros de un Input-Output Hidden Markov Model (IOHMM) aplicado al pronóstico de precios de acciones argentinas. IOHMM es una arquitectura que relaciona una secuencia de entrada, o señal de control, con una secuencia de salida. Se trata de una variante con supervisión de los Modelos Ocultos de Markov (HMM), uno de los métodos de aprendizaje automático no supervisado (Unsupervised Machine Learning) más empleados para series de tiempo. Se asume la existencia de estados latentes para los precios, los cuales se modelan con una mezcla de gaussianas. Estimados los parámetros, se aplica una medida de distancia sobre la log-verosimilitud estimada de cada elemento dentro de la muestra a fin de identificar patrones en la secuencia histórica y producir un pronóstico replicando este comportamiento fuera de la muestra. La estimación bayesiana se realiza a través del método de cadenas de Markov Monte Carlo (MCMC). Se implementa el algoritmo de simulación de muestras en Stan, el lenguaje de programación probabilística más moderno e influyente. Los estados latentes dan lugar a una función de densidad a posteriori con marcada multimodalidad, agravando las frecuentes dificultades que las mezclas de distribuciones presentan por la falta de identificabilidad. Se mejora la convergencia incorporando creencias a priori en la estructura del modelo estableciendo una relación jerárquica en los parámetros de locación de los componentes que conforman cada estado latente. Se divide la muestra en los conjuntos de datos de entrenamiento y prueba, se estiman los parámetros del modelo y se realizan las predicciones fuera de la muestra empleando una ventana rodante de datos (walking-forward). Se obtienen medidas de error de pronóstico satisfactorias. Se observa con particular interés que los estados latentes resultan persistentes, una característica valiosa en el ámbito de las finanzas que modelos econométricos como los cambios de regímenes con cadenas de no logran replicar. Se destaca especialmente que la especificación es muy flexible, con gran potencial para adaptarse a otros requerimientos típicos en el análisis de series de tiempo financieras. Se insta a futuros trabajos a experimentar con otras variables de entrada como ser el volumen negociado, la volatilidad o indicadores de tendencia. Nota al evaluador: Para más información, https://goo.gl/Sr2x1c. Aplicación similar que desarrollé para Google Summer of Code.}
