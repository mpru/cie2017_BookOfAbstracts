\A
{ALGORITMOS DE AGRUPAMIENTO MULTIVARIADOS PARA GRANDES BASES DE DATOS}
{\Presenting{MARÍA EUGENIA VIDELA}$^1$\index{VIDELA, M}, JULIANA IGLESIAS$^2$\index{IGLESIAS, J} y CECILIA BRUNO$^3$\index{BRUNO, C}}
{\Afilliation{$^1$ESTADÍSTICA Y BIOMETRÍA. FCA. UNC. CONICET. UNIVERSIDAD DE VILLA MARÍA, CÓRDOBA}
\Afilliation{$^2$INTA PERGAMINO}
\Afilliation{$^3$CONICET. ESTADÍSTICA Y BIOMETRÍA. FACULTAD DE CIENCIAS AGROPECUARIAS. UNIVERSIDAD NACIONAL DE CÓRDOBA}
\\\Email{eugeniavidela12@gmail.com}}
{estructura genética; marcadores snp; clasifiación multivariada} 
 {Ciencias agropecuarias} 
 {Métodos multivariados} 
 {165} 
 {343-2}
{La estructura genética poblacional entre genotipos permite conocer relaciones entre los individuos e incorporar dicha información en modelos de asociación y/o predicción genómica. Sin embargo, aplicar métodos de agrupamiento en una gran colección de datos implica un alto costo computacional y un incremento en la complejidad para la búsqueda de estructura genética. Diferentes algoritmos de agrupamiento han surgido como una alternativa para analizar con precisión el volumen masivo de datos generados por las aplicaciones modernas. El objetivo de este trabajo fue evaluar algoritmos de agrupamiento en grandes bases de datos genéticas obtenidas a partir de marcadores moleculares del tipo Single Nucleotide Polymorphism (SNP). En el presente trabajo se ilustra la comparación de diez algoritmos de clasificación: Divisive Analysis Clustering (DIANA), Partitioning Around Medoids (PAM), Clustering Large Applications (CLARA), Agglomerative Nesting (AGNES), Unweighted Pair Group Method with Arithmetic Mean (UPGMA), K-means, Kernel K-means, Fuzzy K-means, Admixture y Self Organizing Tree Algorithm (SOTA), sobre una base de datos de 270 líneas estabilizadas de maíz genotipadas con 55769 marcadores moleculares tipo SNP pertenecientes al Grupo de Mejoramiento de maíz de INTA Pergamino. Los marcadores fueron codificados en forma binaria (donde 1 representa alelos heterocigotas y 0 alelos homocigotas), se eliminaron aquellos marcadores con frecuencia alélica menor a 0.01 y aquellos con más del 30\% de datos faltantes, obteniendo así un total de 51576 marcadores moleculares. La validación del número óptimo de grupos para cada algoritmo comparado se realizó mediante los índices de Dunn, Conectividad y Silueta. Todos los métodos sugirieron la existencia de dos subpoblaciones, excepto los métodos basados en K-means cuyos índices de Silueta indican un número mayor de grupos. De la comparación entre métodos se concluyó que DIANA, AGNES y UPGMA presentaron mejor desempeño en la clasificación de la base de datos genética.}
