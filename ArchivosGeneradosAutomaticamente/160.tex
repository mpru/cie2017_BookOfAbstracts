\A
{UN ESTUDIO COMPARATIVO DE TRES CRITERIOS DE D-OPTIMALIDAD BAYESIANOS EN UN MODELO NO LINEAL}
{VÍCTOR IGNACIO LÓPEZ RÍOS\index{LOPEZ RIOS, V@LÓPEZ RÍOS, V}}
{\Afilliation{UNIVERSIDAD NACIONAL DE COLOMBIA SEDE MEDELLÍN}
\\\Email{vilopez@unal.edu.co}
}
{distribuciones a priori; diseños d-óptimos bayesianos; criterios de optimalidad; eficiencia; teorema de equivalencia} 
 {Ciencias exactas y naturales} 
 {Diseño de experimentos} 
 {160} 
 {338-1}
{Los diseños de experimentos D-óptimos Bayesianos dan las condiciones experimentales donde un investigador debe conducir un experimento para estimar los parámetros de un modelo bajo estudio. Estos diseños resultan de la minimización de un criterio de optimalidad obtenido a partir de una función de utilidad definida apropiadamente. Bajo ciertos supuestos de normalidad e independencia en el término de error, dicho criterio se reduce a la optimización de un funcional de valor real de la matriz de información de Fisher. En la literatura de diseños óptimos existen distintas formulaciones para el criterio D-optimalidad Bayesiano. En este trabajo se comparan los diseños óptimos obtenidos bajo tres formulaciones distintas del criterio D-optimalidad, usando cuatro distribuciones a priori: uniforme discreta, uniforme continua, gamma y lognormal en un modelo exponencial de dos parámetros. Como estrategia metodológica se seleccionan los parámetros de las distribuciones consideradas de tal forma que la media y la varianza de éstas coincidan con la media y varianza de la distribución uniforme continua, ésta última justificada en la viabilidad que el investigador proporcione al menos los valores mínimo y máximo donde pueden variar los parámetros del modelo. Se obtienen los siguientes resultados: Se hallan todos los diseños óptimos para las distribuciones a prioris consideradas usando diferentes escenarios para el soporte de los parámetros del modelo; adicionalmente, se obtienen las eficiencias de los diseños obtenidos y también se verifica que los diseños obtenidos son diseños óptimos usando el respectivo teorema de equivalencia. Se concluye que, para cada una de las distribuciones consideradas, los diseños obtenidos difieren considerablemente dependiendo de la versión del criterio de D-optimalidad usado. Se lograron obtener diseños de dos y tres puntos de soporte.}
