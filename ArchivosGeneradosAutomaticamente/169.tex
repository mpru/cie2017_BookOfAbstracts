\A
{VALIDACIÓN DE UN INSTRUMENTO DISEÑADO EN UNA CÁTEDRA DE CIRUGÍA VETERINARIA PARA INNOVAR EN LA TOMA DE EXÁMENES ORALES}
{\Presenting{ELENA FERNÁNDEZ DE CARRERA}$^1$\index{FERNÁNDEZ DE CARRERA, E} y JORGE FIORENTINI$^2$\index{FIORENTINI, J}}
{\Afilliation{$^1$FACULTAD DE CIENCIAS MÉDICAS, UNL}
\Afilliation{$^2$FACULTAD DE CIENCIAS VETERINARIAS, UNR}
\\\Email{elenacarrera2@gmail.com}}
{exámenes orales; cirugía veterinaria; validación de instrumentos} 
 {Educación, ciencia y cultura} 
 {Otras categorías metodológicas} 
 {169} 
 {346-1}
{En las cátedras de Cirugía I y II de la Facultad de Ciencias Veterinarias (FCV) de la Universidad Nacional de Rosario, se planteó la necesidad de modificar la metodología de la evaluación final de ambas materias que fundamentalmente eran exámenes orales. Estos se realizaban en la forma tradicional siguiendo las pautas puestas en práctica desde el comienzo. Pero surgieron los interrogantes: ¿el método para evaluar estos exámenes finales era el mismo en todos los llamados y para todos los alumnos o dependen de la subjetividad? ¿Todos los docentes asignan la misma jerarquía a todos los ítems a evaluar? Para ello se realizó una investigación de la temática señalada que además propondría la confección de una planilla que guiara a los docentes en la forma de evaluar para homogeneizar criterios y minimizar el impacto de la subjetividad. Objetivo: Evaluar la validez de la planilla diseñada como instrumento de evaluación para guiar y ordenar el desarrollo del examen final oral de las asignaturas Cirugía I y II. Metodología y resultados: Se realizó un análisis de la concordancia de las calificaciones de 100 exámenes tomados por dos evaluadores distintos con empleo de la planilla diseñada al efecto mediante la aplicación del análisis de regresión entre las notas de ambos evaluadores, obteniéndose un coeficiente de correlación lineal de Pearson de 0,929 (p<0,001). A continuación se analizaron los indicadores empleados en la planilla para identificar aquellos que presentaban mayor diferencia. Interesa también investigar la “concordancia” de las notas puestas por cada evaluador. Por último se los analiza como variable continua y resulta que no hay acuerdo en las notas de sólo 5 alumnos dado que la media de la diferencia de notas entre ambos evaluadores es 0,9 con un intervalo de confianza del 95\%, según proponen Bland y Altam de (-10,8; 12,7). Conclusión: El uso de la planilla es concordante entre ambos evaluadores pero existen problemas en alguno de los indicadores que convendría revisar. Los más frecuentes son el indicador 2, el 3 y el 4. Se sugiere unificar la puntuación de los indicadores para facilitar el uso de la planilla. }
