\A
{MODELOS ESTADÍSTICOS EN ENSAYOS MULTIAMBIENTALES  PARA EVALUAR ASOCIACIÓN GENÓMICA}
{\Presenting{MARÍA ANGÉLICA RUEDA CALDERÓN}\index{RUEDA CALDERÓN, M}, CECILIA BRUNO\index{BRUNO, C} y MÓNICA BALZARINI\index{BALZARINI, M}}
{\Afilliation{CONICET. ESTADÍSTICA Y BIOMETRÍA. FACULTAD DE CIENCIAS AGROPECUARIAS. UNIVERSIDAD NACIONAL DE CÓRDOBA}
\\\Email{angelica8511@gmail.com}}
{componentes de varianza; blup; interacción g×a; gwas} 
 {Ciencias agropecuarias} 
 {Otras categorías metodológicas} 
 {166} 
 {343-5}
{En los ensayos agrícolas multiambientales interesa la estimación de la contribución de los efectos de genotipo (G), ambiente (A) e interacción genotipo-ambiente (G×A) en la variación de caracteres cuantitativos. El modelo lineal mixto (MLM) de componentes de varianza es comúnmente usado para este fin. Con la creciente disponibilidad de datos de marcadores moleculares, el efecto global de G puede ser expresado como un modelo aditivo de los efectos de múltiples marcadores y, en consecuencia, también es necesario modelar la interacción G×A desde la información molecular. Debe considerarse también que en estudios de asociación genómica (Genome-wide association study, GWAS) es común que los individuos de la población bajo análisis se encuentren emparentados genéticamente. Esta relación puede estimarse con coeficientes de coancestría a través del conocimiento del pedigree o bien desde los mismos datos moleculares. El objetivo de este trabajo fue comparar estrategias estadísticas para la estimación de modelos asociación en presencia de interacción G×A. Las estrategias de análisis son comparadas con el MLM de efectos global de G e interacción, cuyas componentes G, G×A son aleatorias y A es fija. Se trabajó con un conjunto de datos público conformado por 599 genotipos de trigo, genotipados con 1279 SNP y evaluados fenotípicamente en 4 ambientes. La interacción G×A es la componente de varianza más importante. La primera estrategia fue estimar modelos GWAS por ambiente, considerando la estructura genética a través de la matriz de pedigree y, alternativamente, mediante la similitud molecular (EMMA). La segunda estrategia fue ajustar un modelo GWAS para el total de datos donde se incorpora G×A, usando la correlación de efectos genómicos entre ambientes para contemplar la interacción; nuevamente la matriz de relaciones aditivas fue calculada a partir del pedigree y de la matriz de marcadores moleculares. El Mejor Predictor Lineal Insesgado (BLUP) de los efectos de G en cada A fue derivado desde cada modelo. Se correlacionaron los BLUPs obtenidos bajo distintas estrategias de modelación con los obtenidos en el modelo de referencia. Con la primera estrategia implementada, para ambas formas de estimar la matriz de relaciones aditivas, se obtuvieron altas correlaciones de la performance estimada de cada G en cada A con la performance observada a partir del modelo de referencia.}
