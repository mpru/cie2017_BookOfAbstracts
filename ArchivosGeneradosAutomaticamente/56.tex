\A
{INFERENCIA Y BONDAD DE AJUSTE PARA DATOS CON CENSURAS HÍBRIDA Y PROGRESIVA}
{\Presenting{CLAUDIA CASTRO KURISS}$^1$\index{CASTRO KURISS, C} y VÍCTOR LEIVA$^2$\index{LEIVA, V}}
{\Afilliation{$^1$DEPARTAMENTO DE MATEMÁTICA, INSTITUTO TECNOLÓGICO DE BUENOS AIRES, ARGENTINA}
\Afilliation{$^2$ESCUELA DE INGENIERÍA INDUSTRIAL, PONTIFICIA UNIVERSIDAD CATÓLICA DE VALPARAÍSO, CHILE}
\\\Email{ccastrok@gmail.com}}
{distribuciones de probabilidad; censuras tipo i y tipo ii; censuras híbrida y progresiva} 
 {Industria y mejoramiento de la calidad} 
 {Datos de duración} 
 {56} 
 {132-1}
{Datos censurados surgen en forma natural en ciencias de la salud y de la ingeniería, entre otras áreas. La censura ocurre cuando algunas unidades bajo análisis no han presentado el evento de interés antes de finalizado el estudio. En censura Tipo I, el tiempo de estudio es fijo y el número de unidades que fallan es aleatorio. En censura Tipo II, el tiempo del estudio es aleatorio y el número de unidades que fallan es fijo. Existe un tipo de censura llamada híbrida que mezcla los dos tipos mencionados. La censura híbrida se ha usado con éxito, por ejemplo, en modelos de riesgos competitivos. Los esquemas de censura híbridos se han extendido recientemente también a un tipo de censura llamado progresiva que ha sido discutida ampliamente por varios conocidos autores. Estos nuevos esquemas de censura han producido la apertura de un vasto campo para la investigación, ya que incluso para distribuciones muy conocidas, como la Weibull o la gamma, no se han obtenido todavía resultados de bondad de ajuste y de inferencia bajo censura híbrida. La censura progresiva es más alentadora en este aspecto, puesto que se han obtenido estimaciones de parámetros para algunas distribuciones conocidas, como la exponencial. El objetivo de este trabajo es discutir distintos métodos de censura, algunos resultados inferenciales obtenidos para esos métodos y los desafíos que se presentan al estimar parámetros distribucionales bajo estos métodos de censura. En este trabajo se introducen también algunos de resultados sobre bondad de ajuste y estimación bajo censura progresiva que se encuentran en estudio por los autores. }
