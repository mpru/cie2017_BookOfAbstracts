\A
{EMPLEO DE MODELOS DE MEZCLA CON DISTRIBUCIONES BETA PARA LA IDENTIFICACIÓN DE GENES DIFERENCIALMENTE METILADOS EN UN ESTUDIO PAN-CÁNCER}
{\Presenting{MARCOS PRUNELLO}$^1$\index{PRUNELLO, M} y OLIVIER GEVAERT$^2$\index{GEVAERT, O}}
{\Afilliation{$^1$FACULTAD DE CIENCIAS BIOQUÍMICAS Y FARMACÉUTICAS, UNIVERSIDAD NACIONAL DE ROSARIO; CONICET}
\Afilliation{$^2$BIOMEDICAL INFORMATICS RESEARCH, DEPARTMENT OF MEDICINE, STANFORD UNIVERSITY}
\\\Email{mprunello@fbioyf.unr.edu.ar}}
{modelos de mezcla; distribución beta; conglomerados; metilacion de adn; cáncer; expresión génica} 
 {Genética} 
 {Otras categorías metodológicas} 
 {8} 
 {14-1}
{La metilación del ADN es la transferencia de un grupo metílico en sitios CpG en la cadena del ADN que produce importantes efectos regulatorios. Se conoce que tanto la hipo- como la hiper-metilación juegan un rol significativo en cáncer por lo cual resulta de interés identificar genes con metilación anormal. A través de un algoritmo que modela la metilación a través del ajuste de modelos de mezcla para distribuciones beta y que ajusta la relación entre la metilación y la expresión génica con modelos lineales, identificamos genes diferencialmente metilados y transcripcionalmente predictivos en un estudio pan-cáncer consistente de 26 tipos de tumores y más de 9000 muestras. Encontramos 146 genes con metilación diferencial en muestras de cáncer con respecto a muestras normales. Caracterizamos a cada muestra de acuerdo con su perfil de metilación para estos genes y a través de la técnica de consensus clustering identificamos 15 conglomerados de pacientes. Algunos grupos están principalmente constituidos por muestras del mismo cáncer o de cánceres relacionados, mientras que otros están formados por muestras de distintos tipos de tejidos reflejando similitudes en sus perfiles de metilación. Los conglomerados asociados a tumor cerebral presentaron el mayor número de genes hiper-metilados y los relacionados a cáncer de ovarios y tiroides, el menor. La supervivencia resultó significativamente diferente entre los conglomerados pero también entre muestras del mismo tejido que fueron agrupadas en distintos conglomerados. Analizamos también variables clínicas de los pacientes y encontramos diferencias significativas en cuanto a la edad, hábito de fumar, género, estado de la enfermedad, grado del tumor e histología. Para algunos cánceres, la separación de muestras de un mismo tejido en distintos conglomerados está asociada a tipificaciones basadas en expresión génica. Empleamos SAM (Significance Analysis of Microarrays) para identificar genes sobre-expresados en cada conglomerado, los cuales fueron posteriormente analizados con GSEA (Gene Set Enrichment Analysis). Varios conjuntos de genes conocidos por estar regulados por metilación mostraron enriquecimiento y en muchos casos el conjunto de genes sobre-expresados en un conglomerado particular resultó estar vinculado a conjuntos de genes previamente asociados al tipo de cáncer dominante en ese conglomerado. Este análisis revela nuevas similitudes entre tejidos malignamente transformados basadas en patrones de metilación comunes y puede ser útil para re describir aspectos del cáncer en función de la metilación en lugar de sitio de origen.}
