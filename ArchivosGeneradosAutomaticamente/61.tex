\A
{CENSO NACIONAL DE FAMILIAS CON MIEMBROS CON DISCAPACIDAD}
{\Presenting{LUIS CASTIGLIONI}$^1$\index{CASTIGLIONI, L}, GRACIELA GAZZERA$^1$\index{GAZZERA, G} y ROSANA KUCUKBEYAZ$^2$\index{KUCUKBEYAZ, D}}
{\Afilliation{$^1$OSDOP}
\Afilliation{$^2$UNTREF}
\\\Email{ranush2003@yahoo.com.ar}}
{censo; discapacidad; osdop} 
 {Salud humana} 
 {Otras categorías metodológicas} 
 {61} 
 {144-1}
{La obra social OSDOP (Obra Social de Docentes Privados) desde su Gerencia de Prestaciones de Salud decidió realizar un Censo de las familias con miembro/miembros con discapacidad por considerar indispensable conocer la realidad cotidiana de vida y la dinámica de su grupo familiar. Comenzó este proyecto con una experiencia piloto en la delegación Mar del Plata donde se probó la herramienta censal construida desde la OSDOP. La encuesta relevó información sobre las condiciones de vida de las familias que tiene uno o mas integrante con discapacidad. La cedula censal contiene aspectos cuantitativos y cualitativos. La entrevista fue realizada por una pareja de encuestadores formados en la institución : un Licenciado en Trabajo Social – personal externo- y un Agente Sanitarios –personal interno- La cedula censal presenta dos dimensiones : La familia y el afiliado/s con discapacidad En número total de afilados a censar fue de 740, llevándose a cabo 660. No contamos con antecedentes sobre la realización de un Censo de estas características lo cual posibilitó concretar desde foja cero, datos que permitan contar con información confiable y útil. Este Censo es el inicio de un Proyecto mayor que permitirá realizar un trabajo más amplio sobre el total de nuestra población de afiliados con discapacidad y sus familias. Fundamentalmente pondrá en evidencia las debilidades y fortalezas de la Obra Social y sus prestaciones. Identificar situaciones invisibilizadas en el hacer diario de la actividad de la OSDOP. Conocer y captar las sugerencias que aporten las familias sobre nuestra calidad de atención. Construir un puente permanente de contacto. Verificar los datos del padrón. }
