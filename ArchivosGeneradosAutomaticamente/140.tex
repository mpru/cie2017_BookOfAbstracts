\A
{UNA PROPUESTA DE EVALUACIÓN DE CONCEPTOS ESTADÍSTICOS EN CARRERAS DEL ÁREA SOCIALES}
{\Presenting{SILVANA SANTELLÁN}\index{SANTELLÁN, S}, LILIANA TAUBER\index{TAUBER, L} y MARIELA CRAVERO\index{CRAVERO, M}}
{\Afilliation{UNIVERSIDAD NACIONAL DEL LITORAL}
\\\Email{estadisticamatematicafhuc@gmail.com}}
{cultura estadística; razonamiento estadístico; evaluación continua; estudiantes universitarios} 
 {Enseñanza de la estadística} 
 {Otras categorías metodológicas} 
 {140} 
 {305-2}
{Compartimos una propuesta de evaluación continua que pretende enriquecer las prácticas de enseñanza de Estadística para Ciencias Sociales, teniendo como premisa desarrollar el pensamiento estadístico crítico. Nuestra Propuesta Uno de los principales desafíos a los que nos enfrentamos los educadores estadísticos, en carreras sociales, es proponer una enseñanza que logre desarrollar aprendizajes a largo plazo. En este sentido, coincidimos con Behar y Pere Grima (2014), Herrera y Konic (2017), quienes proponen algunos lineamientos para introducir contenidos de Estadística que sean significativos para los estudiantes. Este desafío es el que nos hemos propuesto desde la cátedra “Métodos estadísticos para Ciencias Sociales”, en la Facultad de Humanidades y Ciencias de la Universidad Nacional del Litoral. Desde el inicio de este espacio observamos que los alumnos comienzan el cursado indicando que aprender los contenidos estadísticos es tarea difícil y, una baja proporción de alumnos lograba realizar exitosamente el recorrido. Si bien estos alumnos evidencian mucha autonomía en la lectura de textos específicos de las Ciencias Sociales, no la tienen cuando deben leer textos de carácter más técnico como puede ser uno de Estadística. Además, los estudiantes no acostumbran a cursar regularmente, lo cual es un problema agregado para la planificación de las clases, ya que una gran proporción abandona rápidamente. Este contexto nos llevó a diseñar una propuesta que priorice la participación activa de nuestros estudiantes y cuyo eje mentor sea la continuidad de los contenidos desarrollados. Delineamos una evaluación continua atravesada por tres trabajos prácticos basados en un texto de Escudero (2014), el software Gapminder y los últimos datos disponibles de la Encuesta Permanente de Hogares (EPH) del aglomerado Gran Santa Fe. El objetivo de estos prácticos es promover el razonamiento crítico trabajando las ideas estadísticas fundamentales que plantea Batanero (2013), para fomentar la formación de profesionales cultos estadísticamente. Reflexiones Finales Consideramos que nuestros estudiantes han tenido la posibilidad de lograr aprendizajes más significativos de los contenidos estadísticos; principalmente debido al importante rol del análisis sobre sobre situaciones reales y el reconocimiento de determinados conceptos estadísticos en estas situaciones y la interpretación de medidas obtenidas sobre estos datos. }
