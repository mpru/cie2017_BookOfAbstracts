\A
{COMPARACIÓN DE DIFERENTES METEDOLOGÍAS PARA LA PREDICCIÓN DEL VALOR CONTADO DE UN INMUEBLE}
{\Presenting{LEONARDO MORENO}\index{MORENO, L}, MARCO SCAVINO\index{SCAVINO, M} y JUAN JOSÉ GOYENECHE\index{GOYENECHE, J}}
{\Afilliation{UDELAR}
\\\Email{mrleo@iesta.edu.uy}}
{método de stacking; modelos espacio-temporales; precio hedónico; procesos autoregresivos} 
 {Otras aplicaciones} 
 {Estadística espacial} 
 {167} 
 {344-1}
{El objetivo del trabajo es comparar la eficiencia predictiva de diferentes metodologías basadas en información espacio-temporal de un conjunto de inmuebles. Se tiene como insumo información espacial y temporal sobre un conjunto de inmuebles y ciertas variables intrínsecas a cada bien, por ejemplo, superficie, antiguedad, número de dormitorios. En tal sentido es conocido el valor contado (en dólares norteamericanos) de ciertos inmuebles de la ciudad de Montevideo (Uruguay) en diferentes fechas, entendiendo por valor contado a aquel que es asignado por el tasador, donde cada propiedad puede haber sido tasada en diferentes momentos, problema denominado en la literatura como ventas repetidas, [Bailey et al., 1963]. La información espacial, coordenadas de cada inmueble, permite la construcción de modelos autoregresivos espaciales donde el precio de un inmueble se encuentra correlacionado con el de sus vecinos, [Anselin, 1988]. La información temporal puede ser modelada mediante un modelo autoregresivo temporal, en este caso la dependencia entre una tasación y la siguiente disminuye en función al lapso de tiempo transcurrido entre una y otra, [Nagaraja et al., 2011]. Las variables hedónicas se modelan mediante un modelo de regresión lineal dinámico, donde los coeficientes de la regresión son función de la ubicación del inmueble, [Sun et al., 2014]. Es posible mediante la metodología de Stacking, [Breiman, 1996] encontrar un procedimiento que permita construir un único algoritmo que determine predicciones más precisas en términos del error cuadrático medio. En este caso, otras metodologías como los modelos mixtos, [McCulloch and Searle, 2001], y el modelo de Case-Shiller, [Shiller and Case, 1987], pueden ser incluídas en la agregación. Se evalúa la performance del modelo en datos simulados y en datos reales mediante validación cruzada. Es desafiante la búsqueda de un único modelo con una mejor performance que los ya desarrollados.}
