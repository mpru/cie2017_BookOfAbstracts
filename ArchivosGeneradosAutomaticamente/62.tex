\A
{MÉTODOS DE ANÁLISIS MULTIVARIADO A TRES VÍAS PARA IDENTIFICACIÓN DE ASOCIACIONES ENTRE CARACTERES FENOTÍPICOS Y MARCADORES MOLECULARES}
{\Presenting{SERGIO BRAMARDI}$^1$\index{BRAMARDI, S}, SABRINA COSTA TÁRTARA$^2$\index{COSTA TÁRTARA, S} y RAMIRO CURTI$^3$\index{CURTI, R}}
{\Afilliation{$^1$DEPARTAMENTO DE ESTADÍSTICA, UNIVERSIDAD NACIONAL DEL COMAHUE}
\Afilliation{$^2$DEPARTAMENTO DE TECNOLOGÍA, UNIVERSIDAD NACIONAL DE LUJÁN}
\Afilliation{$^3$CÁTEDRA DE ESTADÍSTICA Y DISEÑO EXPERIMENTAL, FAC. DE CS. NATURALES, UNIVERSIDAD NACIONAL DE SALTA (SEDE METÁN)}
\\\Email{sbramardi@gmail.com}}
{quinoa; datos de conjuntos múltiples; análisis de procrustes generalizados; análisis factorial múltiple; discretización de escofier; microsatélites} 
 {Ciencias agropecuarias} 
 {Métodos multivariados} 
 {62} 
 {151-1}
{En la conservación de recursos fitogenéticos, la evaluación del material vegetal es una etapa crítica para su aprovechamiento, la que generalmente culmina con una descripción y clasificación de las poblaciones bajo estudio. Es común que esta caracterización se realice en función de atributos morfológicos-agronómicos y marcadores moleculares. Utilizar esta información para avanzar en el establecimiento de relaciones entre caracteres fenotípicos y los marcadores moleculares puede permitir una incipiente detección de loci asociados a estos caracteres con potencial uso en posteriores procesos de mejora. Estos análisis involucran estructuras de datos a tres vías, más precisamente, conjunto de datos múltiples, por cuanto un mismo conjunto de individuos son descriptos por distintos tipos de variables: cuantitativas, cualitativas y moleculares. Existen diferentes alternativas de estudio para estas situaciones como el Análisis de Procrustes Generalizados (APG), Análisis Factorial Múltiple (AFM) y Análisis Factorial de Correspondencias aplicado sobre los datos discretizados según Escofier. En estos tres métodos, con sustentos teóricos totalmente diferenciados, se presenta una etapa de análisis de las interrelaciones de las tres vías, para luego, construir una estructura de consenso o compromiso de los individuos caracterizados a través de los distintos tipos de variables. Quizás para el objetivo de establecer relaciones entre individuos en base a variables mixtas la técnica más difundida sea el APG, pero ésta no permite una representación de los caracteres como si lo hacen las otras dos. En el presente trabajo, a partir de datos correspondientes a la caracterización de 25 poblaciones de quinoa de la región del Noroeste de Argentina a las que se les observaron 23 variables cuantitativas, 11 cualitativas y 22 loci microsatélites (SSR) que originaron 313 alternativas alélicas, se compararon las configuraciones de individuos obtenidos por las tres técnicas y se establecieron posibles relaciones entre caracteres fenotípicos y marcadores en las configuraciones de variables obtenidas del AFM y el AFC. Estas relaciones fueron verificadas a través de Pruebas chi-cuadrado y Análisis de la Variancia.}
