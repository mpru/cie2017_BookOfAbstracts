\A
{LA EXTENSIÓN DE LA ENCUESTA PERMANENTE DE HOGARES AL ÁMBITO RURAL: DESAFÍOS PARA LA FORMULACIÓN DE LA ESTRATEGIA DE MEDICIÓN URBANO-RURAL}
{\Presenting{CYNTHIA POK}$^1$\index{POK, C}, MARISA DUARTE$^2$\index{DUARTE, M}, SOLEDAD TRIANO$^2$\index{TRIANO, S} y DALILA VADELL$^3$\index{VADELL, D}}
{\Afilliation{$^1$INDEC}
\Afilliation{$^2$INDEC/UBA}
\Afilliation{$^3$INDEC/UBA/UNR}
\\\Email{dvadell@fcecon.unr.edu.ar}}
{estadísticas oficiales; encuestas a hogares; ámbito rural} 
 {Estadísticas oficiales} 
 {Otras categorías metodológicas} 
 {154} 
 {328-1}
{Las Encuestas a Hogares que llevan a cabo los Institutos Oficiales de Estadística, forman parte de las investigaciones orientadas al conocimiento de la realidad económico-social, generando indicadores intercensales que permiten caracterizar la situación social de los individuos y de los hogares, en un todo de acuerdo con las recomendaciones de los organismos internacionales y garantizando su comparabilidad. En nuestro país, la Encuesta Permanente de Hogares, programa de relevamiento continuo del Instituto Nacional de Estadística y Censos (INDEC), comenzó a relevar información sobre los hogares de Capital Federal en 1973, contemplando desde sus orígenes la extensión al total del país a través de sucesivas etapas. Desde el punto de vista temático se diferenciaron dos etapas claramente definidas, la primera supuso la elaboración de un modelo adecuado al tratamiento de indicadores sociales en el contexto de mercados de trabajo urbanos de carácter permanente y la segunda al ámbito de los mercados de trabajo urbano-rurales de carácter estacional. En su extensión espacial se fueron incorporando progresivamente aglomerados urbanos de más de 100.000 habitantes y todas las capitales provinciales, al tiempo que se iniciaba el tratamiento de indicadores sociales en el contexto de los mercados de trabajo urbano-rurales de carácter no permanente, a partir de dos experiencias: Alto Valle del Río Negro y provincia de Tucumán. En la actualidad se cubre como dominios de estimación 31 aglomerados y el total urbano en el tercer trimestre de cada año. La etapa de extensión iniciada en 2016, pretende avanzar en la caracterización, medición y análisis del mercado de trabajo para el conjunto de la población del país. El presente trabajo aborda los principales nudos conceptuales que plantea el entorno de los mercados laborales afectados por la estacionalidad, muestra los avances en términos de pruebas temáticas en curso, ahondando sobre las características que requieren ser visibilizadas en cuanto a las estrategias de reproducción y de inserción laboral propia de los contextos semiurbanos y rurales (trabajo de producción para el autoconsumo, antigüedad cíclica, ingresos no monetarios entre otros aspectos), expone los resultados iniciales de dichas pruebas y los desafíos a encarar para la formulación de la estrategia de medición urbano-rural. }
