\A
{NUEVO ENFOQUE PARA LA BÚSQUEDA DE GENES ISOFUNCIONALES Y SU APLICACIÓN EN EL ESTUDIO EN BACILLUS PROMOTORES DEL CRECIMIENTO VEGETAL}
{\Presenting{MARIANO TORRES MANNO}$^1$\index{TORRES MANNO, M}, MARÍA DOLORES PIZARRO$^{2,3}$\index{PIZARRO, M}, PAMELA BELÉN BARENGO$^2$\index{BARENGO, P}, LUCAS DAURELIO$^{2,3}$\index{DAURELIO, L} y MARTÍN ESPARIZ$^1$\index{ESPARIZ, M}}
{\Afilliation{$^1$IBR-CONICET}
\Afilliation{$^2$CÁTEDRA DE FISIOLOGÍA VEGETAL, FACULTAD DE CIENCIAS AGRARIAS, UNL}
\Afilliation{$^3$CONICET}
\\\Email{mariano.atm@gmail.com}}
{pgpr; bacillus; bayes; genómica comparativa} 
 {Biología} 
 {Métodos bayesianos} 
 {108} 
 {235-1}
{Las rizobacterias promotoras del crecimiento de plantas (PGPR por “Plant Growth Promoting Rhizobacteria”) son bacterias beneficiosas que tienen la capacidad de colonizar las raíces de las plantas y promover su crecimiento. Los Bacillus son preferidos a la hora de formular productos comerciales dado que al ser formadores de esporas tienen mayor viabilidad en el tiempo. Las vías por las cuales las bacterias promueven de diferentes formas el crecimiento en las plantas son diversas. Una de las más importantes es la síntesis de metabolitos secundarios (SMS) que incluyen polímeros como los policétidos y pequeñas estructuras tales como bacteriocinas, oligopéptidos y lipopéptidos. Dada la complejidad de secuencia de los genes que codifican para estas vías su asignación suele ser errónea. Aunque existen buenos algoritmos para la identificación de clusters génicos de SMS, dada la similitud de secuencia y naturaleza modular de los genes que componen cada cluster, la predicción de isofuncionalidad entre clusters codificados en distintas cepas sigue representando un verdadero desafío. Por ello se ha desarrollado un flujo de trabajo, basado en scripts ad hoc de R que utilizan BLAST, un enfoque inferencial Bayesiano y análisis de sintenia, para estimar la probabilidad de que una dada vía de SMS sea isofuncional a aquellas usados como carnada. Se ha realizado la búsqueda y caracterización de 67 genes de las 10 vías SMS más importantes en 729 Bacillus con genoma no redundante disponibles al momento. Por medio de alineamientos por BLAST identificamos 27.038 genes que compondrían 6.834 vías SMS. Utilizando histogramas de los porcentajes de identidad de poblaciones de ortólogos y parálogos se infirió densidad a priori. Con las curvas de densidad se obtuvo tanto el factor bayesiano como la probabilidad de ortólogo o parálogo de cada alineamiento de BLAST y se determinó que solo 17.854 genes serían isofuncionales a los usados como carnada. Estos conforman solo 4.299 vías lo que supone que la herramienta BLAST genera un 40\% de falsos positivos en este tipo de análisis. Analizando la sintenia de todas las 4.299 vías utilizando un script de R ad hoc pudimos determinar que 2.312 presentaban una estructura similar a las de las vía query. Los pipelines y scripts generados pueden ser utilizados para múltiples propósitos que involucren búsqueda de genes con actividades deseadas en diferentes genomas en forma automática y altamente eficiente lo que permitirá profundizar y sistematizar estudios de genómica comparativa. }
