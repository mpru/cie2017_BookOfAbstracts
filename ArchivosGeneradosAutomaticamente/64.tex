\A
{GESTIÓN EFICIENTE DE RIESGOS FINANCIEROS UTILIZANDO UN MODELO ESTOCÁSTICO DE DOS FACTORES}
{\Presenting{MARÍA TERESA CASPARRI}\index{CASPARRI, M}, MARTÍN EZEQUIEL MASCI\index{MASCI, M} y JOAQUÍN BOSANO\index{BOSANO, J}}
{\Afilliation{UNIVERSIDAD DE BUENOS AIRES. FACULTAD DE CIENCIAS ECONÓMICAS. IADCOM, CENTRO DE INVESTIGACIÓN EN MÉTODOS CUANTITATIVOS APLICADOS A LA ECONOMÍA Y LA GESTIÓN.}
\\\Email{martinmasci@economicas.uba.ar}}
{procesos estócasticos; modelos multi-factor; derivados financieros; futuros} 
 {Otras ciencias económicas, administración y negocios} 
 {Probabilidad y procesos estocásticos} 
 {64} 
 {154-1}
{Los instrumentos derivados financieros están sujetos a condiciones de incertidumbre que justifican la importancia de tener modelos de valuación que sean a la vez precisos y parsimoniosos. En la actualidad existen un gran número de ellos, por lo tanto, resulta importante disponer de un criterio de validación exploratorio para asistir al analista en el proceso de selección del mejor modelo. En este sentido, la eficiencia del modelo debe basarse en la decisión financiera asociada y las características intrínsecas de la herramienta. Respecto de la primera característica, el uso de modelos estocásticos en ciencias económicas representa una herramienta para el análisis crítico de las decisiones. Por otro lado, las características intrínsecas de dichos modelos estadísticos facilitan la implementación en términos de velocidad de procesamiento y parsimonia de las variables involucradas. Este trabajo implementa un modelo canónico de valuación de futuros sobre petróleo crudo -Gibson-Schwartz (1990)- y busca validar los estimadores de sus parámetros para explorar su precisión y eficiencia. Dada la naturaleza de los datos, el método de estimación elegido es un modelo de seemingly unrelated regression (SUR), una generalización del modelo de regresión lineal multiecuacional que permite que los residuos estén correlacionados. Para hacer esto, se desarrolla un proceso de validación basado en casos, implementado en un código de STATA. Dichos casos contemplan la valuación del precio de los derivados, en distintas ventanas temporales. De esta manera, el grado de eficiencia que se le exige al modelo se basa en la parsimonia del mismo, a la vez que identifica las dificultades de cada variable en términos de su bondad del ajuste. Esta investigación concluye que el modelo propuesto es satisfactorio, dada su naturaleza parsimoniosa, pero existe lugar para mejoras si se consideran algunos modelos alternativos. }
