\A
{COMPETENCIAS DIGITALES PARA LA EDUCACIÓN SUPERIOR A DISTANCIA. ADAPTACIÓN Y APLICACIÓN DEL CUESTIONARIO ACUTIC AL CASO DE LA FCE-UNC}
{\Presenting{ADRIÁN MAXIMILIANO MONETA PIZARRO}\index{MONETA PIZARRO, A}, LAURA MONTERO\index{MONTERO, L}, MARÍA ALEJANDRA JUÁREZ\index{JUAREZ, M@JUÁREZ, M}, MARIANA LASPINA\index{LASPINA, M}, JOSEFINA DEPETRIS\index{DEPETRIS, J}, BRUNO FAGNOLA\index{FAGNOLA, B} y FABRIZIO SOFFIETTI\index{SOFFIETTI, F}}
{\Afilliation{UNIVERSIDAD NACIONAL DE CÓRDOBA}
\\\Email{amoneta@eco.uncor.edu}}
{competencias digitales; tecnologías de la información y comunicación; educación superior a distancia; ecuaciones estructurales} 
 {Educación, ciencia y cultura} 
 {Métodos multivariados} 
 {55} 
 {121-1}
{La Facultad de Ciencias Económicas (FCE) de la Universidad Nacional de Córdoba (UNC) ofrece desde 2002 un Ciclo Básico a Distancia (CBD) para sus carreras de grado. Esta propuesta consiste en disponer, para cada asignatura del CBD, de una comisión cuyo cursado se realiza a distancia con el apoyo de una plataforma virtual. El objetivo de este CBD es reforzar estrategias encaminadas a disminuir problemas y dificultades con el acceso, la deserción, el retraso y el fracaso académico en los primeros años de cursado (Moneta Pizarro et al., 2017). Sin embargo, estudios exploratorios previos muestran que menos del 40\% de los alumnos que optan por esta modalidad logran la regularidad y que entre las posibles causas de este fenómeno se encontrarían problemas relacionados con la preparación, uso y actitud de los estudiantes para las tecnologías de la información y comunicación (Blanch et al., 2013 octubre). De acuerdo a la literatura, el dominio tecnológico y las competencias digitales de los alumnos son fundamentales para el éxito de la educación a distancia (Moore y Kearsley, 2011). Resulta entonces importante contar con escalas de medición de los constructos correspondientes y que éstas sean válidas para identificar la asociación con el desempeño académico de los estudiantes en esta modalidad. El objetivo de este trabajo es adaptar el cuestionario de actitud, conocimiento y uso de tecnologías de la información y la comunicación (ACUTIC) en Educación Superior propuesto por Mirete Ruiz, García-Sánchez y Hernández Pina (2015) y validarlo estadísticamente sobre una muestra del curso de Microeconomía I del CBD de la FCE-UNC mediante técnicas de análisis factorial y modelación de ecuaciones estructurales. Los resultados demuestran consistencia interna de la escala total adaptada (=0.9474) y de cada una de las sub-escalas ( actitud=0.9393, conocimiento=0.9045 y uso=0.8868). Las estimaciones del modelo de medida resultaron satisfactorias reportando mayoría de índices con bondad de ajuste aceptable (RMSEA=0.069; CFI= 0.931; SRMR=0.079). Esto evidencia que el instrumento propuesto es confiable y útil para obtener indicadores de actitud, conocimiento y uso de tecnologías de la información y la comunicación en cursos similares del CBD de la FCE-UNC y para ser aplicado en investigaciones que busquen identificar el impacto de estos constructos sobre otras variables relevantes.}
