\A
{ANÁLISIS ESTADÍSTICO PARA EVALUAR UN MÉTODO DE ACREDITACIÓN DE IDENTIDAD EN RECIÉN NACIDOS}
{\Presenting{MARIA DE LA PAZ GUILLON}$^1$\index{GUILLON, M}, LILIANA AMALIA GARCÍA$^1$\index{GARCÍA, L} y MARIA DEL CARMEN COVAS$^2$\index{COVAS, M}}
{\Afilliation{$^1$UNIVERSIDAD NACIONAL DEL SUR}
\Afilliation{$^2$HOSPITAL PRIVADO DEL SUR, BAHÍA BLANCA}
\\\Email{mguillon@criba.edu.ar}}
{inferencia estadística; identificación del recién nacido; pulsera de identificación} 
 {Otras ciencias de la salud} 
 {Inferencia estadística} 
 {122} 
 {263-1}
{La correcta identidad de las personas desde su nacimiento es un derecho inalienable que fija nuestra Constitución: “Todo niño nacido vivo o muerto y su madre deben ser identificados de acuerdo con las disposiciones de la ley 24540”. Según las instituciones que asisten nacimientos, para identificar al recién nacido se siguen diferentes procedimientos, los cuales se aplican simultáneamente en la madre y su hijo antes de abandonar la sala de partos (incluye quirófano). La colocación y mantenimiento de pulseras de identificación es la conducta más utilizada. Sin embargo, es un método que puede resultar no confiable en algunos casos, dado que en la práctica se sabe que la pulsera puede removerse o desprenderse de manera natural del recién nacido. Dada la importancia que reviste entonces la pulsera en la identificación de un recién nacido, surgió en enfermeras y médicos del Servicio de Neonatología del Hospital Privado del Sur de Bahía Blanca, Argentina, la inquietud por averiguar cuán confiable es la pulsera como elemento de identificación. Con este fin, el Centro de Estudios de Calidad Total fue convocado por dicho Servicio para realizar el asesoramiento estadístico que el problema requería. Conjuntamente se decidió realizar un estudio de observación del comportamiento de la pulsera, prospectivo, tipo cohorte, aleatorizado, en 914 recién nacidos en el citado hospital, lo que se completó a lo largo de diez meses entre 2013 y 2014. En este trabajo se describe la planificación de las actividades requeridas para el citado estudio; la organización del equipo que lo llevó a cabo; modalidad, capacitación brindada e inconvenientes surgidos en la recolección de los datos; la base de datos diseñada; las pruebas realizadas y conclusiones alcanzadas. Se observaron, entre otras variables, el lugar de colocación de la pulsera (pierna o brazo) en el recién nacido, la presencia de la pulsera al momento del alta, las causas y la incidencia del familiar o personal de enfermería en caso de su ausencia. El análisis estadístico fue realizado mediante el software SPSS 15.0 y Statgraphics Centurion XV. La conclusión más relevante fue que la colocación de la pulsera de identificación no resulta un método seguro para la identificación del recién nacido. La tercera parte de los mismos, al momento del alta, no la tenía en el lugar (brazo o pierna) donde había sido colocada. La salida espontánea fue la causa más frecuente de su ausencia, con una mayor permanencia en la pierna. }
