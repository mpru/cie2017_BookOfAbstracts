\A
{USO DE INDICADORES PARA EL ANÁLISIS DE PROBLEMAS PROPUESTOS EN TEXTOS DE PROBABILIDAD Y ESTADÍSTICA PARA INGENIERÍA}
{\Presenting{NOEMÍ MARÍA FERRERI}$^1$\index{FERRERI, N}, GRACIELA HAYDÉE CARNEVALI$^1$\index{CARNEVALI, G} y MARÍA EVANGELINA ÁLVAREZ$^2$\index{ALVAREZ, M}}
{\Afilliation{$^1$FACULTAD DE CIENCIAS EXACTAS, INGENIERÍA Y AGRIMENSURA, UNR}
\Afilliation{$^2$FACULTAD DE CIENCIAS ECONÓMICAS Y ESTADÍSTICA, UNR}
\\\Email{nferreri@fceia.unr.edu.ar}}
{resolución de problemas; ciclo ppdac; indicadores; textos de probabilidad y estadística} 
 {Enseñanza de la estadística} 
 {Otras categorías metodológicas} 
 {82} 
 {183-1}
{La resolución de problemas de naturaleza estadística, es una competencia importante a desarrollar en los futuros ingenieros. Esta tarea implica un conjunto de procesos que se articulan formando un ciclo, al que, por ejemplo, Wild y Pfannkuch, denominan “Ciclo PPDAC”. La sigla corresponde a las etapas “planteo del problema” (P), “planificación del estudio estadístico” (P), “recolección de los datos” (D), “análisis de los mismos” (A) y “obtención de conclusiones”, en contexto (C). Con el objetivo de que los futuros ingenieros adquieran esta competencia, es deseable que en los cursos de Estadística se resuelvan frecuentemente problemas diseñados de tal manera que, en el proceso de resolución, los alumnos deban transitar la mayor cantidad de etapas y se enfrenten a diferentes situaciones. Para facilitar el diseño de problemas que tengan estas características, se cuenta con un conjunto de indicadores propuestos en trabajos anteriores, el cual se viene elaborando en el marco del proyecto de investigación ING525 “El Pensamiento Estadístico en el control y la mejora de los procesos: diseño, aplicación y evaluación de propuestas didácticas para su desarrollo en alumnos de Ingeniería Industrial”. En muchas ocasiones los problemas se toman de textos de referencia, que no siempre contienen todos los elementos necesarios para las diferentes etapas del ciclo PPDAC. El uso de los indicadores facilita el análisis de estos problemas y orienta a los docentes para que los enriquezcan, favoreciendo así el desarrollo de la competencia deseada. En el presente trabajo se analizan algunos problemas tomados de textos de Probabilidad y Estadística de uso habitual en carreras de Ingeniería, aplicando los indicadores mencionados y, a modo de ejemplo, se hacen algunas propuestas para enriquecerlos. }
