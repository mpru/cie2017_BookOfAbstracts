\A
{ANALISIS Y EVALUACIÓN DE LA PRODUCCIÓN DE CAÑA DE AZUCAR MEDIANTE SIMEX}
{\Presenting{NELIDA DEL VALLE ORTIZ}\index{ORTIZ, N}, PATRICIA ANDREA DIGONZELLI,\index{DIGONZELLI, P}, MARIA BEATRIZ GARCÍA\index{GARCÍA, M B}, JUAN FERNÁNDEZ DE ULLIVARRI\index{FERNÁNDEZ DE ULLIVARRI, J} y EDUARDO ROMERO\index{ROMERO, E}}
{\Afilliation{UNIVERSIDAD NACIONAL DE TUCUMÁN}
\\\Email{ortiznelidadelvalle@gmail.com.ar}}
{datos longitudinales; ecuaciones de estimación generalizadas; simex; variable discreta} 
 {Ciencias agropecuarias} 
 {Otras categorías metodológicas} 
 {181} 
 {368-3}
{La caña de azúcar es un cultivo perenne cuya mejor producción es durante los cinco primeros años del cultivo. Concluida la cosecha los residuos agrícolas pueden dejarse en la superficie o eliminarse. Para comparar las producciones de ambas prácticas agrícolas se registró el número de tallos a lo largo del ciclo del cultivo. Por tratarse de datos longitudinales de conteo el análisis se realizó mediante ecuaciones de estimación generalizadas con estructura de varianza intercambiable. El objetivo de este trabajo fue analizar el error de medida en el número tallos y evitar estimaciones sesgadas de los parámetros. Las observaciones se realizaron durante el ciclo del cultivo en días posteriores a la cosecha en diferentes frecuencias de tiempo generando desiguales intervalos aleatorios. El análisis descriptivo de los datos mostró que la el número de tallos en parcelas sin residuos agrícolas es superior hasta, aproximadamente, la mitad del ciclo y decrece manteniéndose entre valores próximos hasta el final del ciclo. Simultáneamente, la producción correspondiente a parcelas con residuos agrícolas permanece con valores inferiores y también decrece hacia el final. Por lo tanto, se decidió analizar los datos desde las últimas instancias del cultivo marcado por el comienzo del decrecimiento de la producción del número de tallos en parcelas sin cobertura agrícola. Se determinó como variables de predicción al tiempo transcurrido desde la cosecha, es decir, a días posteriores a la cosecha y la ausencia o presencia de la cobertura de residuos agrícola como el tratamiento aplicado. Naturalmente, la variable respuesta fue el número de tallos. Los resultados mostraron homogeneidad entre las producciones de ambas prácticas agrícolas. Para el análisis de error en la variable dependiente se aplicó el método de inferencia para modelos con error de medida basado en simulación y extrapolación SIMEX, que por tratarse de datos de conteo se usó MCSIMEX. Se compararon las desviaciones estándar de las estimaciones de los parámetros del modelo de ecuaciones de estimación generalizadas en el que se asume que no existe error de medida, es decir modelo naive, y el modelo mcsimex generado desde la aplicación del método MCSIMEX en el modelo naive. De la comparación se observaron mínimas diferencias entre las desviaciones estándar.}
