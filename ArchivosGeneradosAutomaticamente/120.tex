\A
{EVALUACIÓN DE LA DELIMITACIÓN DE ZONAS HOMOGÉNEAS EN SUELOS SÓDICOS INCLUYENDO INFORMACIÓN ESPACIAL}
{\Presenting{MARTIN PRARIZZI}\index{PRARIZZI, M}, NICOLAS BATTISTON\index{BATTISTON, N}, MICAELA MANZOTTI\index{MANZOTTI, M}, CECILIA MILÁN\index{MILÁN, C} y PAOLA SALVATIERRA\index{SALVATIERRA, P}}
{\Afilliation{UNIVERSIDAD NACIONAL DE VILLA MARÍA}
\\\Email{martinprari@gmail.com}}
{multispati-pca; fuzzi-kmeans; pca} 
 {Ciencias agropecuarias} 
 {Estadística espacial} 
 {120} 
 {258-1}
{En el marco de este trabajo se realizó un ensayo con la finalidad de comparar los ANÁLISIS DE COMPONENTES PRINCIPALES CLÁSICO (PCA) y ANÁLISIS DE COMPONENTES PRINCIPALES ESPACIALES (MULTISPATI-PCA). Se recolectaron datos en un lote de 41 ha. con valores geo-referenciados de conductividad eléctrica aparente CEa (mS m-1) en dos profundidades 0-30 cm (CE30) y 0-90cm (CE90) y altimetría (m) a través de un sensor de arrastre (modelo 3100, VerisTech Inc., EEUU) y un DGPS (Raven P300 con corrección de señal Omnistar, de precisión submétrica) que además les asocio coordenadas geográficas (sistema Lat Lon-WGS84). Como punto de partida se llevó a cabo la transformación de coordenadas y depuración de datos outliers. Para la delimitación de las zonas homogéneas en primer lugar se realizaron MULTISPATI-PCA Y PCA para asociar los datos de las diferentes variables relevadas. Los resultados obtenidos por ambos métodos se compararon a través de la pérdida de inercia (varianza espacial vs varianza total) y el aumento de la información espacial (Índice de Moran de MULTISPATI-PCA vs Índice de Moran del PCA). Posteriormente con las primeras dos componentes espaciales y clásicas se realizaron clasificaciones usando el algoritmo de FUZZY K-MEANS y con el agrupamiento obtenido, en cada caso, se realizó la visualización grafica utilizando un suavizado. Para la determinación del número óptimo de grupos se utilizaron los índices de “Xie.Beni”, Fukuyama Sugeno, Coeficiente de partición y entropía de partición. Los resultados mostraron que con MULTISPATI-PCA se explicó una menor proporción de varianza acumulada en la primer componente respecto de PCA, mientras que en la segunda componente esta relación se invierte. En cuanto al índice de Moran calculado sobre cada una de las componentes, se observó que la autocorrelación no fue más alta al incluir información espacial. En cuanto a la clasificación, ambos casos, los resultados obtenidos fueron dos zonas bien delimitadas y homogéneas dentro del lote. Esto indica que, en esta aplicación, la introducción de información espacial no proporciona una ventaja adicional.}
