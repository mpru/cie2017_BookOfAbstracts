\A
{EVALUACIÓN DE LA EYACULACIÓN EN LESIONADOS MEDULARES TRAUMÁTICOS MEDIANTE ESTADÍSTICA MULTIVARIADA Y REGRESIÓN}
{\Presenting{LAURA CACHEIRO}\index{CACHEIRO, L} y ALBERTO RODRIGUEZ VELEZ\index{RODRIGUEZ VELEZ, A}}
{\Afilliation{IREP/GCBA}
\\\Email{laura.cacheiro@gmail.com}}
{lesión medular; eyaculación; descriptiva multivariada; regresión logística} 
 {Salud humana} 
 {Modelos de regresión} 
 {74} 
 {166-1}
{La eyaculación está definida fisiológicamente como la expulsión enérgica y rítmica del semen hacia el meato uretral, ocurre como resultado de un reflejo medular que involucra la indemnidad de los segmentos medulares de Torácico 10 a Sacro 4 , controlado por el Sistema Nervioso Central supramedular. Sin embargo la eyaculación es posible aun en ausencia de estas conexiones superiores por la existencia de un centro espinal generador reflejo de la eyaculación localizado en la región toracolumbar entre Toracico 10 – Lumbar 3. La lesión medular implica la pérdida total o parcial de las funciones motoras, sensitivas y autonómicas, por debajo del nivel de lesión. Luego de la lesión medular la mayoría de los hombres no pueden alcanzar una eyaculación durante la masturbación o el coito. Existe amplia controversia respecto a las posibilidades eyaculatorias en pacientes con lesión medular, y muy poca investigación al respecto, dada la importancia que tiene , es necesario conocer cuales variables están implicadas en la presencia de eyaculación. Se desea conocer cuáles son las características de la lesión medular más fuertemente vinculadas con una eyaculación exitosa con o sin asistencia médica. Además de obtener información específica sobre la función sexual de los hombres con lesiones de la médula espinal. Para ello se registraron los siguientes datos de 185 pacientes: Nivel de lesión motora, Nivel de lesión sensitiva, erección psicogénica, erección refleja, contracción de esfínter anal y Escala ASIA. La aplicación de metodología estadística multivariada como, Análisis de Correspondencias múltiple, Regresión Logística y el manejo de datos faltantes permitió en primera instancia tener una idea del comportamiento de todas las variables en conjunto y luego ayudo a entender variables mas asociadas con la Eyaculacion. Este estudio confirma los hallazgos anteriores que la capacidad de alcanzar la eyaculación es deficiente en los hombres con lesión de la médula espinal. La discapacidad locomotora más severa podría afectar negativamente la vida sexual de las personas con lesión medular. Estas conclusiones se traducen en poder predecir la capacidad eyaculatoria y en consecuencia poder emprender acciones que permitan a los pacientes acceder a la paternidad y una mejora en su calidad de vida sexual.  }
