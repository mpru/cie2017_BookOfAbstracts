\A
{CARACTERIZACION GENOTOXICA DE COMPUESTOS PEPTIDICOS ANTIMICROBIANOS UTILIZANDO EL ENSAYO COMETA EN LEUCOCITOS HUMANOS}
{\Presenting{GINA DE GIUSTO}$^1$\index{DE GIUSTO, G}, MARÍA VERÓNICA HÚMPOLA$^2$\index{HUMPOLA, M@HÚMPOLA, M}, GEORGINA TONARELLI$^2$\index{TONARELLI, G}, MARÍA FERNANDA SIMONIELLO$^1$\index{SIMONIELLO, M}, LILIANA ESTER CONTINI$^3$\index{CONTINI, L} y OLGA BEATRIZ ÁVILA$^3$\index{AVILA, O}}
{\Afilliation{$^1$CÁTEDRA DE TOXICOLOGÍA, FARMACOLOGÍA Y BIOQUÍMICA LEGAL, FBCB, UNL}
\Afilliation{$^2$LABORATORIO DE PÉPTIDOS BIOACTIVOS. DEPARTAMENTO DE QUÍMICA ORGÁNICA, FBCB-UNL}
\Afilliation{$^3$DPTO. DE MATEMÁTICA, FBCB - UNL}
\\\Email{lcontini@fbcb.unl.edu.ar}}
{péptidos antimicrobianos; genotoxicidad; ensayo cometa; lipopéptidos; modelo factorial; medidas repetidas} 
 {Salud humana} 
 {Diseño de experimentos} 
 {53} 
 {116-2}
{La aparición y rápida propagación de bacterias resistentes a los antibióticos convencionales se ha convertido en una problemática mundial que ha llevado a la búsqueda de nuevos agentes antimicrobianos. En este sentido, los péptidos antimicrobianos (PAs) han surgido como una alternativa para el tratamiento de enfermedades infecciosas. En el diseño de fármacos, la caracterización toxicológica frente a células de mamíferos resulta esencial a fin de evaluar el potencial terapéutico de nuevas drogas. El objetivo de este trabajo fue evaluar las propiedades genotóxicas de compuestos peptídicos antimicrobianos. La genotoxicidad de los 4 péptidos y 9 lipopéptidos se evaluó in vitro, utilizando el ensayo cometa en leucocitos humanos de donantes sanos. Este primer análisis estadístico se realizó con dos experimentos de dos factores con medidas repetidas en uno de ellos. Los factores analizados fueron: concentración con cuatro niveles (medidas repetidas) y el otro, tipo de péptido o lipopéptido. La variable respuesta fue genotoxicidad respecto del valor inicial de cada muestra. Para ambos experimentos se observó que al aumentar la concentración aumentó la genotoxicidad. La interacción concentración*péptido y concentración*lipopéptido, resultó estadísticamente significativa. Del análisis de las comparaciones múltiples considerando todas las combinaciones y de los gráficos de perfil, se concluye que: Lipopéptidos: se observan dos grupos, uno con tres, que es más genotóxico y la interacción se manifiesta en todas las concentraciones; el otro grupo, con seis, menos genotóxico y la interacción se manifiesta solo en las concentraciones bajas. Péptidos: se observa que uno de ellos es más genotóxico en todas las concentraciones; los otros tres, a bajas concentraciones, se comportan de manera similar, a partir del tercer nivel de concentración uno de ellos es tan genotóxico como el primero, mientras que los otros dos, si bien aumenta la genotoxicidad no llegan a los niveles del primero. Esta evaluación permitió seleccionar los compuestos peptídicos que presentan mejor perfil toxicológico para su potencial uso terapéutico. }
