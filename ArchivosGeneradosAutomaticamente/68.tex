\A
{EL ARTE DE LOS NÚMEROS EN LA TRIBUTACIÓN}
{ANA OÑA\index{ONA, A@OÑA, A}}
{\Afilliation{UNIVERSIDAD CATÓLICA DE CHILE}
\\\Email{alona@.uc.cl}
}
{aplicación ley de benford; fraude fiscal; anomalías datos} 
 {Economía} 
 {Minería de datos} 
 {68} 
 {160-1}
{La Ley de Benford, conocida también como la ley del primer dígito a lo largo del tiempo ha sido empleada para descubrir anomalías en la información en datos numéricos como por ejemplo declaraciones de impuestos. Su aplicación permite detectar datos erróneos que no siguen un patrón teórico, de acuerdo a esta ley la distribución de los números del 1 al 9 como primer dígito de una cifra es asimétrica. La frecuencia con la que inician las cifras con el número 1 es aproximadamente el 30\% mientras que iniciar con el número 2 ya es del 17\% y así va descendiendo. Esta Ley se ha empleado para detectar el fraude fiscal y aunque en concepto parece intuitiva da luces de posibles actos de evasión y mala declaración. El boom tecnológico, la globalización y las tendencias de internacionalización hacen que cada vez se tenga más información de las empresas tanto de forma directa como por información de terceros. Al mismo tiempo, esto provoca que con tanta información los procedimientos tradicionales de control de las Administraciones Tributarias se vuelvan obsoletos y poco oportunos. Contar con herramientas que permitan que este control sea oportuno y eficiente se vuelve imprescindible. El realizar controles en un periodo de tiempo más cercano incrementa la percepción de riesgo y disminuye la opción de manipular los datos. La detección del fraude fiscal debe realizarse desde la declaración, de manera inmediata, no costosa y de fácil implantación. La presente investigación busca corroborar la distribución del primer dígito con las declaraciones de impuesto a la renta de las sociedades en Ecuador. A su vez, determinar esquemas para utilizarla como herramienta para detectar anomalías en las declaraciones de impuestos.}
