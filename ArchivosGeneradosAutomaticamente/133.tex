\A
{DIAGNÓSTICO DE MODELOS DE MÚLTIPLES FACETAS DE RASCH EN ESTUDIOS OBSERVACIONALES}
{ROBERTO DANIEL CÁCERES BAUER\index{CACERES BAUER, R@CÁCERES BAUER, R}}
{\Afilliation{DEPARTAMENTO DE EDUCACIÓN MÉDICA, FACULTAD DE MEDICINA, UNIVERSIDAD DE LA REPÚBLICA, URUGUAY;UNIDAD DE ESTADÍSTICA, ESCUELA DE NUTRICIÓN, ÁREA INVESTIGACIÓN, UNIVERSIDAD DE LA REPÚBLICA, URUGUAY.}
\\\Email{rcaceresb@gmail.com}
}
{modelos de múltiples facetas de rasch; redes de evaluación; diagnóstico de modelos; estudios observacionales; gráficas no dirigidas; conexión} 
 {Educación, ciencia y cultura} 
 {Datos categóricos} 
 {133} 
 {301-1}
{El modelo de múltiples facetas de Rasch (MFRM) tiene aplicaciones en diversas áreas (Bond y Fox, 2015; Engelhard, 2013; Linacre, 2015). MFRM permite modelar la respuesta a ítems de una escala como un fenómeno probabilístico influenciado por elementos de varias facetas. Cada faceta corresponde a factores que afectan la probabilidad de la respuesta descripta mediante una variable ordinal. Un aspecto clave en la aplicación del MFRM es la conexión entre los elementos de las facetas. Este aspecto del diseño afecta la exactitud y precisión de las medidas y la aplicación de los procedimientos de diagnóstico del modelo. En estudios experimentales la conexión es definida a priori. No obstante, en estudios observacionales la conexión emergente puede ser compleja y deficitaria. Por tanto, se requieren métodos especializados de diagnóstico con los que actualmente no se cuenta. Los objetivos de este trabajo son: (a) caracterizar la conexión emergente en un estudio observacional; (b) comparar un método de diagnóstico corrientemente usado (Linacre, 2015), con un método alternativo. Los datos considerados provienen de un cuestionario de evaluación estudiantil de la calidad en la enseñanza empleada en un estudio observacional, con 3243 estudiantes, 98 docentes, 10 actividades de enseñanza, en una carrera de médico. El modelo MFRM incluyó tres facetas de medición (docentes, estudiantes e ítems). Para caracterizar la conexión emergente se aplicó un modelo de gráfica no dirigida (Trudeau, 2013) y se determinó el tipo de red de evaluación (Engelhard, 1996). En la comparación de los métodos de diagnóstico se consideró: la estabilización de las mediciones; el ajuste global; y las medidas de invarianza. La conexión en la faceta docentes resultó mayor que la observada en la faceta estudiantes. La red de evaluación encontrada resultó ser una red de evaluación conectada incompleta e incluyó conjuntos de elementos que son claves para la conexión de la red. Los criterios de comparación considerados sugieren que el método propuesto funciona mejor.}
