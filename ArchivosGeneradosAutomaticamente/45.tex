\A
{CONSTRUCCIÓN DE UNA ESCALA PARA MEDIR LA PROPENSIÓN DE USO DE PLATAFORMAS DE FINANCIAMIENTO MASIVO EN ESTUDIANTES DE CIENCIAS ECONÓMICAS}
{\Presenting{MARIANA GUARDIOLA}\index{GUARDIOLA, M} y NORMA PATRICIA CARO\index{CARO, N}}
{\Afilliation{FACULTAD DE CIENCIAS ECONÓMICAS, UNIVERSIDAD NACIONAL DE CÓRDOBA}
\\\Email{marianaguardiola@eco.unc.edu.ar}}
{ecuaciones estructurales; crowdfunding; variable latente} 
 {Otras ciencias económicas, administración y negocios} 
 {Modelos de regresión} 
 {45} 
 {100-1}
{El crowdfunding o financiamiento masivo es una manera de conseguir recursos financieros a través de plataformas en Internet a fin de que un gran número de individuos realicen aportes a una causa, logrando en su conjunto una contribución relevante. Debido a que en la literatura no existe consenso sobre escalas que midan la propensión al uso de este tipo de financiamiento, se definió un constructo referido a la valoración de plataformas de crowdfunding para el financiamiento de proyectos de calidad. A través de un modelo de medida, en el marco de los modelos de ecuaciones estructurales, se estimaron las relaciones causales del comportamiento de los potenciales usuarios de crowdfunding y el efecto entre los indicadores propuestos y la variable latente. En el año 2016 se realizó una encuesta piloto a 377 estudiantes avanzados de la Facultad de Ciencias Económicas de la Universidad Nacional de Córdoba (UNC), a fin de evaluar su propensión al uso de plataformas de financiamiento masivo para proyectos de calidad. En base a la literatura existente se definieron indicadores de satisfacción, reconocimiento, valoración, aporte y confianza en sí mismo y a este medio de financiamiento, obteniéndose relaciones significativas entre cada uno de los indicadores y la variable latente. El estadístico Chi cuadrado indicó que el modelo es adecuado (p-value: 0,079), al igual que las medidas de bondad de ajuste (RMSE del 0,05 y SRMR del 0,04). A partir de los resultados obtenidos se pretende extender el análisis a todos los estudiantes de grado de la UNC, considerando otras dimensiones de relevancia.}
