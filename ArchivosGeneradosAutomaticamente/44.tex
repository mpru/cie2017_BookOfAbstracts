\A
{ESTUDIO SOBRE LA DEMORA EN LA ELECCIÓN DE CARRERA EN UNA FACULTAD DE ROSARIO MEDIANTE ANALISIS DE SUPERVIVENCIA PARA TIEMPOS AGRUPADOS}
{\Presenting{JULIA ANGELINI}\index{ANGELINI, J} y GABRIELA BOGGIO\index{BOGGIO, G}}
{\Afilliation{INSTITUTO DE INVESTIGACIONES TEÓRICAS Y APLICADAS DE LA ESCUELA DE ESTADÍSTICA, FACULTAD DE CIENCIAS ECONÓMICAS Y ESTADÍSTICA, UNIVERSIDAD NACIONAL DE ROSARIO.}
\\\Email{jangelini\_93@hotmail.com}}
{censura por intervalos; modelos lineales generalizados; razón de hazards} 
 {Educación, ciencia y cultura} 
 {Datos de duración} 
 {44} 
 {97-2}
{Uno de los problemas que enfrentan las instituciones de Educación Superior relacionado con el rendimiento académico se refiere a la excesiva demora para alcanzar el título. En particular, se ha percibido en la Facultad de Ciencias Económicas y Estadística de la Universidad Nacional de Rosario un retraso en los tiempos previstos por los planes de estudio para cumplir una primera meta relacionada con la posibilidad de elegir carrera. El Plan de Estudios de las carreras de Cs. Económicas de esta Facultad comienza con un Ciclo Introductorio Común y la elección de la carrera es posible previa aprobación de cuatro de las seis asignaturas que conforman ese ciclo. Se ha notado una demora importante en cumplir este requisito, por lo que interesa investigar cuáles son los determinantes que tienen impacto directo sobre ella. Para dar respuesta a este objetivo, se recurre a técnicas específicas de análisis de datos de supervivencia para tiempos discretos o agrupados, debido a que, si bien el cumplimiento del requisito para poder elegir carrera se puede dar en diferentes momentos durante el año académico, en este trabajo se utiliza la información disponible al inicio de cada año académico, es decir, la demora se mide en cantidad de años. El enfoque elegido consiste en tratar cada tiempo de supervivencia individual como un conjunto de observaciones dicotómicas que consideran si un alumno presenta o no el evento en cada año académico hasta que experimente el evento o sea censurado. Esta reestructuración de los datos permite ajustar un modelo lineal generalizado para respuesta binaria eligiendo, en este caso, el enlace “log-log del complemento”, de manera de interpretar los coeficientes en términos de razones de hazards como en el clásico modelo de Cox. Este enfoque permite, además, la inclusión de covariables que dependan del tiempo sin dificultades y la obtención de una medida del efecto de que el evento ocurra en un determinado tiempo en comparación con que ocurra en otro tiempo. Los resultados hallados muestran que los estudiantes que tienen mayor demora en el cumplimiento de los requisitos para elegir carrera son los varones, los egresados de una escuela secundaria pública, los estudiantes que no trabajan, los que tienen un promedio menor a seis y aquéllos cuyo padre tiene un nivel educativo bajo. Por último, presentan mayor demora los que han terminado la educación secundaria antes de 2007, lo que se ve potenciado si además las madres trabajan.}
