\A
{ESTIMACIÓN Y PRONÓSTICOS DE TASAS BÁSICAS EN EL AGLOMERADO GRAN ROSARIO UTILIZANDO MÉTODOS DE SUAVIZADO EXPONENCIAL PROBABILÍSTICOS}
{\Presenting{SABRINA SILVA QUINTANA}$^1$\index{SILVA QUINTANA, S}, LUCÍA ANDREOZZI$^2$\index{ANDREOZZI, L} y MARÍA TERESA BLACONÁ$^2$\index{BLACONÁ, M}}
{\Afilliation{$^1$ÁREA ESTADÍSTICA Y PROCESAMIENTO DE DATOS. FACULTAD DE CIENCIAS BIOQUÍMICAS Y FARMACÉUTICAS. UNIVERSIDAD NACIONAL DE ROSARIO. ROSARIO, ARGENTINA}
\Afilliation{$^2$FACULTAD DE CIENCIAS ECONÓMICAS Y ESTADÍSTICA. UNIVERSIDAD NACIONAL DE ROSARIO. ROSARIO, ARGENTINA}
\\\Email{sabrinasilvaq@gmail.com}}
{series; tiempo; tasa; actividad; empleo; desocupación} 
 {Otras aplicaciones} 
 {Otras categorías metodológicas} 
 {51} 
 {113-2}
{En estudios de mercado laboral, uno de los principales intereses es identificar aquellos grupos que tienen mayores inconvenientes a la hora de desempeñarse, lo cual puede lograrse a través del análisis de las condiciones de ocupación de las personas -si consiguen o buscan trabajo o si están desocupados- y, consecuentemente, llevar a cabo políticas activas de empleo para enfrentarlos y mejorar la situación actual. En este trabajo, se estudian las series de los principales indicadores del mercado de trabajo para Rosario y la región: tasas de Actividad, Empleo y Desocupación; calculadas de manera general, por género y por tres grandes grupos de edad definidos como: Jóvenes (18 a 24 años), Adultos (25 a 45 años) y Adultos Mayores (46 a 65 años). El objetivo de este trabajo es analizar el comportamiento de las tasas trimestrales en el pasado para pronosticar valores futuros. Se proponen a tal fin, los Modelos de Espacio de Estado de Innovación de Hyndman (Hyndman, Koehler, Ord y Snyder, 2008) y los modelos ARIMA. En cada serie, se ha optado por el modelo que mejor ajusta los datos presentados, en otras palabras el modelo que exhibe menor AIC (AKAIKE) entre todos los posibles y se busca detectar aquellos grupos que necesitan mayor atención para acceder a un empleo. Una medida del error de pronóstico de cada modelo es el MAPE “Mean Absolute Percentage Error” de capacidad predictiva, cuando el valor del mismo sea del 10 \% o menos, se concluirá que dicho modelo es óptimo para pronosticar. En el 27,78\% de las tasas los modelos de Hyndman no detectaron estacionalidad que los modelos ARIMA sí. Cuando la capacidad predictiva de un modelo fue aceptable, la del otro modelo también lo fue y viceversa.Para las tasas de actividad y empleo se obtienen buenos ajustes, mientras que para las tasas de desocupación los modelos no representan adecuadamente los datos y los pronósticos resultan no satisfactorios, esto mismo puede deberse a la gran variabilidad que posee dicha tasa. Por otra parte los grupos donde es necesario centrarse para mejorar su situación laboral son los jóvenes de 18 a 24 años y el género femenino. }
