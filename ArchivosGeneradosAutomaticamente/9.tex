\A
{EVALUACIÓN DEL EFECTO DE LA CIRUGÍA DE PTERIGIÓN EN LA EVOLUCIÓN DEL ASTIGMATISMO CORNEAL A TRAVÉS DEL USO DE MODELOS LINEALES GENERALES MIXTOS}
{\Presenting{MARCOS PRUNELLO}$^1$\index{PRUNELLO, M}, EZEQUIEL ARANDA$^2$\index{ARANDA, E}, JÉSICA DIMATTIA$^2$\index{DIMATTIA, J}, RODRIGO MANTARAS$^2$\index{MANTARAS, R}, ELINA CIARLO$^2$\index{CIARLO, E}, BRUNO ZEGA$^2$\index{ZEGA, B}, LUCIANA MICHELETTI$^2$\index{MICHELETTI, L} y NADIA SICA$^2$\index{SICA, N}}
{\Afilliation{$^1$ÁREA ESTADÍSTICA Y PROCESAMIENTO DE DATOS, FACULTAD DE CIENCIAS BIOQUÍMICAS Y FARMACÉUTICAS, UNIVERSIDAD NACIONAL DE ROSARIO}
\Afilliation{$^2$CÁTEDRA DE OFTALMOLOGÍA DE LA UNIVERSIDAD NACIONAL DE ROSARIO, HOSPITAL PROVINCIAL DEL CENTENARIO}
\\\Email{mprunello@fbioyf.unr.edu.ar}}
{pterigión; astigmatismo; modelo lineal general mixto; oftalmología; cirugía} 
 {Salud humana} 
 {Modelos de regresión} 
 {9} 
 {14-2}
{El pterigión es un crecimiento fibrovascular del tejido conectivo conjuntival, localizado en forma de ala sobre la córnea. Puede causar aplanamiento de la córnea e inducir astigmatismo. El objetivo de este trabajo es evaluar el efecto de la excéresis del pterigión en el astigmatismo corneal. Se evaluaron de manera prospectiva 102 pacientes que asistieron a la consulta al Servicio de Oftalmología del Hospital provincial del Centenario de Rosario, y que fueron sometidos a cirugía de pterigión según la técnica de injerto conjuntival libre entre agosto de 2014 y agosto de 2016. Se registró la edad, sexo, ubicación de la degeneración (nasal o temporal), aspecto biomicroscópico (rosado, hiperémico o blanco-atrófico) y clasificación en grados de acuerdo a su extensión sobre la córnea. Todos los pacientes incluidos en el estudio fueron sometidos a un examen oftalmológico completo incluyendo agudeza visual y topografía corneal para cuantificar el nivel de astigmatismo en la primera consulta, y luego de uno y tres meses de la cirugía. La evolución en el tiempo del astigmatismo fue estudiada con un modelo lineal general mixto para datos longitudinales. El modelo incluyó efectos fijos para el grado (1, 2, 3 o 4), el aspecto (blanco, rosado o hiperémico), la lateralidad (ojo izquierdo o derecho) y el tipo (sólo nasal/temporal o ambos), y efectos aleatorios para la identificación del paciente. Se encontró que las variables aspecto, lateralidad y tipo del pterigión no influyen significativamente en los valores medios de astigmatismo (p = 0.1966, p = 0.6306 y p = 0.5975, respectivamente). Por el contrario, los efectos del tiempo, del grado y de la interacción entre ambos resultaron estadísticamente significativos (p = <0.0001 en los tres casos). Para los ojos con el grado más leve, no se observan diferencias significativas en el promedio de astigmatismo a través de los tres controles. En cada uno de los restantes grados, los valores promedios en las dos evaluaciones post-operatorios no difieren entre sí, pero son significativamente menores que al promedio del pre-operatorio. Finalmente, 49 casos (48.0\%) evidenciaron una mejora en la agudeza visual, 47 (46.1\%) permanecieron igual y 6 (5.9\%) empeoraron, siendo estos casos con recidivas o con grado inicial I o 2. Este estudio corrobora que la cirugía exitosa de excéresis de pterigión reduce el astigmatismo inducido por el pterigión y mejora la agudeza visual, siendo el efecto beneficioso más notable en los pacientes con pterigión de grados más avanzados.}
