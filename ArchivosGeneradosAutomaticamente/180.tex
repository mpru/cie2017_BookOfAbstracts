\A
{MODELOS ESTADÍSTICOS INFLADOS EN CERO PARA ANALIZAR DATOS BIOLÓGICOS: MICROCAVIA AUSTRALIS COMO CASO DE ESTUDIO}
{\Presenting{CLAUDIA DE LOS RÍOS}\index{DE LOS RÍOS, C} y NATALIA ANDINO\index{ANDINO, N}}
{\Afilliation{DEPARTAMENTO DE BIOLOGÍA. FACULTAD DE CIENCIAS EXACTAS FÍSICAS Y NATURALES. UNIVERSIDAD NACIONAL DE SAN JUAN}
\\\Email{cfdelorios@unsj-cuim.edu.ar}}
{ecología comportamental; modelos inflados en cero; microcavia australis} 
 {Biología} 
 {Modelos de regresión} 
 {180} 
 {367-1}
{Tradicionalmente, en la ciencias biológicas, los datos comportamentales se analizaban utilizando ANOVA. Sin embargo, las variables respuesta que resulta de un proceso de conteo, tales como la frecuencia de un comportamiento particular, son mejor modeladas usando distribuciones de probabilidad discretas tales como la distribución de Poisson siempre que sea posible. Algunas veces al usar esta distribución Poisson se presenta sobredispersión. En este caso, el término de error debe incluirse en el modelo dando lugar a un Modelo Binomial Negativo. Sin embargo, en ecología comportamental una característica distintiva de estos sets de datos comportamentales es su tendencia a contener una gran cantidad de ceros, por lo que se ha sugerido que el uso de Modelos Inflados en cero es el mejor enfoque para este tipo de datos ya que permite analizar datos con exceso de ceros. El objetivo de este estudio es realizar un trabajo comparativo desarrollando modelos tradicionales y modelos inflados en cero que permitan describir mejor datos biológicos en ecología comportamental, y que puedan utilizarse para realizar inferencias y predicciones de eventos futuros. Se trabajó con datos de un proyecto de ecología comportamental de un roedor social (Microcavia australis) que habita la provincia de San Juan. El proyecto busca conocer si las variables sociales (número de machos o número de hembras) o la época de reproducción influyen en el comportamiento agonístico de los machos de esta especie. Para esto se contó con datos de frecuencia de dicho comportamiento. Los datos presentaban un exceso de ceros (78\%) y se analizaron considerando distribución Gaussiana, Poisson, Binomial Negativa, Inflado Cero Poisson e Inflado Cero Binomial Negativa. Los resultados mostraron que para este conjunto de datos los Modelos Inflados en Cero obtuvieron mejores resultados que los modelos tradicionales. Esto fue evidente a partir de los valores de AIC y de gráficos de ajuste de los residuos. El Modelo Inflado en Cero muestra que el número de hembras y la época reproductiva son las variables que mejor explican los comportamientos agonísticos en este roedor. Dado que los resultados de los Modelos Inflados en Cero proporcionaron una resolución estadística que correspondía con las respuestas biológicas esperadas, se sugiere que este análisis es una opción óptima para este tipo de datos.}
