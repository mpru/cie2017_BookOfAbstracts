\A
{ESTRATEGIAS DE ANÁLISIS CON MODELOS LINEALES GENERALIZADOS MIXTOS PARA PREDICCIÓN DE NITRÓGENO MINERALIZADO APARENTE EN CULTIVO DE MAÍZ EN LA PROVINCIA DE BUENOS AIRES}
{\Presenting{OLGA SUSANA FILIPPINI}$^1$\index{FILIPPINI, O}, VANINA MARIANA DELFINO$^2$\index{DELFINO, V} y MERCEDES ZUBILLAGA$^3$\index{ZUBILLAGA, M}}
{\Afilliation{$^1$UNLU CS.BÁSICAS - FAUBA}
\Afilliation{$^2$UNTREF}
\Afilliation{$^3$FAUBA - FERTILIDAD}
\\\Email{sfilippini@unlu.edu.ar}}
{modelo lineal generalizado; nitrógeno aparente; funciones de enlace} 
 {Ciencias agropecuarias} 
 {Modelos de regresión} 
 {70} 
 {161-1}
{Con el objeto de ajustar modelos predictivos del Nitrógeno mineralizado aparente (Nap), durante el ciclo del cultivo, a partir de variables climáticas, edáficas y espectrales a escala de parcela, se llevaron a cabo diversas estrategias de análisis Estadístico. La información de las variables edáficas se obtuvo a partir de un muestreo aleatorio de las parcelas cultivadas con maíz. Se consolida la base de datos, aplicando Modelos que puedan predecir la Mineralización del Nitrógeno evaluado en suelos argentinos de la planicie de la provincia de Buenos Aires, durante el ciclo del crecimiento del maíz. El tamaño de la muestra es de 139 datos, extraídos durante 6 campañas, considerando en este caso los años de cada campaña cómo el factor aleatorio, en 6 parcelas de producción de donde se escogieron, microparcelas para evaluar la mineralización del Nitrógeno. Cómo co-variables, se consideraron variables climáticas cómo las precipitaciones en distintos períodos y las temperaturas. Las variables edáficas incluidas en el modelo fueron, arena, arcilla y limo y la variable topográfica (loma, media loma o bajo) fue considerada factor fijo. Algunas variables edáficas y la topográfica resultaron estadísticamente significativas. Se comparan las posiciones topográficas, que representan el tipo de ambiente, al momento de medir la cantidad de nitrógeno mineralizado en la parcela, detectando diferencias significativas. Las predicciones del Nitrógeno Mineralizado aparente a partir de la selección de alguna de las variables relevadas y de Modelos Lineales Generalizados con diferentes funciones de enlace permite estudiar la variabilidad espacial y utilizar las mismas en las recomendaciones de fertilización con Nitrógeno en la agricultura de precisión.}
