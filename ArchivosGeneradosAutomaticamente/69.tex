\A
{DIMENSIONES CONCEPTUALES Y METODOLÓGICAS DEL ESTUDIO POR MUESTREO A PRODUCTORES HORTÍCOLAS EN LOS DEPARTAMENTOS DE LUJÁN Y EXALTACIÓN DE LA CRUZ, PROVINCIA DE BUENOS AIRES}
{\Presenting{HUGO DELFINO}$^1$\index{DELFINO, H}, VIVIANA ESCANES$^2$\index{ESCANES, V} y OLGA SUSANA FILIPPINI$^1$\index{FILIPPINI, O}}
{\Afilliation{$^1$UNLU - CS. BÁSICAS}
\Afilliation{$^2$UNLU - CS. SOCIALES}
\\\Email{h\_delfino@yahoo.com.ar}}
{muestreo probabilístico; periurbano; productores hortícolas} 
 {Otras ciencias sociales y humanas} 
 {Muestreo} 
 {69} 
 {161-2}
{La producción hortícola argentina se caracteriza por su amplia distribución geográfica y por la diversidad de especies producidas. Analizar la implementación de “buenas prácticas agropecuaria y las características de la producción hortícola del partido de Luján y Exaltación de la Cruz resulta de suma utilidad para los organismos que difunden Tecnologías Agropecuarias y los municipios del periurbano bonaerense. En torno a los grandes núcleos urbanos se han desarrollado los llamados “cinturones verdes”: territorios ocupados por quintas o huertas familiares y comerciales que rodean a las ciudades y donde se producen hortalizas para abastecer a la población urbana. Estos cinturones se caracterizan por su tamaño relativamente pequeño o mediano de las explotaciones (1 a 40 hectáreas) y por la amplia variedad de especies cultivadas. La expansión urbana expulsó a las “familias productoras” a regiones más alejadas y precarias, con escasa planificación territorial y menor acceso a servicios básicos. Esta nueva población periurbana también incluye a las familias de productores y trabajadores rurales provenientes de migraciones internas y externa, principalmente del Norte argentino y de Bolivia. Describir la producción hortícola en relación con el uso del suelo, tipo de producción y régimen de tenencia, conocer las estrategias y formas de inserción en el área productiva y relevar la utilización de Buenas Prácticas aplicables a la producción y comercialización por parte de los actores claves, fue el objetivo de este estudio. Se realiza una encuesta por muestreo, con selección al azar de productores a partir de un listado proporcionado por el Instituto Nacional de Tecnología Agropecuaria. El relevamiento de la información permite identificar aspectos socio-demográficos, nivel de Buenas Prácticas de producción Agrícolas, origen y destino de la producción, cambios posibles en la dinámica productiva de la zona bajo estudio que sirve de insumo para investigadores de la Universidad Nacional de Luján y el Instituto Nacional de tecnología Agropecuario. Los resultados que se muestran tienen carácter de provisorios pero la representatividad de la muestra se está consolidadando, ya que “en grupos bien seleccionados, con percepciones e interéses comunes sobre un tema, con un tamaño adecuado de entrevistas, se pueden detectar patrones comunes de la población bajo estudio, ya que la curva de aparición de nuevos conceptos alcanza una meseta.}
