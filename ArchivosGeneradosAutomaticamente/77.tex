\A
{INCLUYENDO PROPIEDADES EMERGENTES AL ESTUDIO DE MOVIMIENTO ANIMAL: UN ENFOQUE ESPACIO ESTADO CON AJUSTE POR ABC}
{\Presenting{SOFIA RUIZ SUAREZ}\index{RUIZ SUAREZ, S} y JUAN MANUEL MORALES\index{MORALES, J}}
{\Afilliation{INIBIOMA—CONICET, UNIVERSIDAD NACIONAL DEL COMAHUE, BARILOCHE, ARGENTINA}
\\\Email{sofi.ruizsuarez@gmail.com}}
{approximate bayesian computation; modelos espacio estado; movimiento animal} 
 {Ecología y medio ambiente} 
 {Métodos bayesianos} 
 {77} 
 {172-1}
{La forma en que se mueven los organismos es un proceso de fundamental importancia en ecología. El mismo determina el destino de los individuos en el sentido que interfiere en su adecuación biológica, en la evolución de las especies, en la estructura de las poblaciones y comunidades, y en la transmisión de enfermedades. En los últimos años, a partir de los nuevos avances en tecnología de rastreo y monitoreo, distintos métodos de estadística computacional han permitido ajustar modelos de movimiento cada vez más complejos. Sin embargo, cuando se simulan modelos ajustados a datos, las trayectorias resultantes raramente se parecen a las observadas. Una posible causa de este problema puede ser que estos modelos basan el ajuste solo en propiedades de las trayectorias a pequeña escala como son el largo de los pasos o los ángulos de giro, sin considerar propiedades emergentes del comportamiento de los animales en el paisaje como el uso del espacio. De forma de contemplar este punto, en este trabajo se presenta un modelo de movimiento basado en caminatas aleatorias dentro de un marco espacio estado, cuyo ajuste se plantea a partir de técnicas de Approximate Bayesian Computation (ABC) en donde los valores resumen contemplados en el ajuste, buscan reflejar propiedades emergentes del movimiento. Con este propósito, se consideraron estadísticos resumen tales como la medida de tortuosidad de la trayectoria, la función de auto correlación entre el largo de los pasos, el índice de sinuosidad, así como también se incluyeron la media del largo de paso y de los ángulos de giro, entre otros. Como medida de distancia para la comparación entre las simulaciones y las observaciones se consideró la distancia de Mahalanobis. Los modelos espacio-estado permiten enmarcar este análisis en un contexto donde es posible contemplar tanto el proceso de movimiento del individuo como el proceso de observación. A partir de datos simulados se exploraron distintas posibilidades bajo diferentes escenarios, de forma de destacar las ventajas y limitaciones de este enfoque. Se encontró que dependiendo en parte de la escala de observación y del tipo de movimiento puede resultar beneficioso incorporar este tipo de información al ajuste. Consideramos útil seguir trabajando en esta línea de investigación, contemplando modelos de movimiento con más estructura que permitan mejores distinciones entre trayectorias. }
