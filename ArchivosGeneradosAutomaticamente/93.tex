\A
{RENDIMIENTO ACADÉMICO: ANÁLISIS DE TRAYECTORIAS EN MATEMÁTICA Y BIOESTADÍSTICA}
{\Presenting{MARCELA FERNICOLA}\index{FERNICOLA, M@FERNÍCOLA, M}, MYRIAM NÚÑEZ\index{NUÑEZ, M}, CHRISTIANE PONTEVILLE\index{PONTEVILLE, C} y FERNANDA RUSSO\index{RUSSO, F}}
{\Afilliation{FACULTAD DE FARMACIA Y BIOQUÍMICA. UBA}
\\\Email{chponteville@gmail.com}}
{bioestadística; matemática; rendimiento académico; prueba diagnóstica} 
 {Educación, ciencia y cultura} 
 {Otras categorías metodológicas} 
 {93} 
 {208-1}
{El objetivo del presente trabajo es analizar la evolución del rendimiento académico de los estudiantes de las carreras de Farmacia y Bioquímica de la Universidad de Buenos Aires, inscriptos a las asignaturas Matemática y Bioestadística en ambos cuatrimestres del año 2015. Para este análisis se recabaron datos sobre el resultado de la prueba diagnóstica inicial que se realiza el primer día de clase, confeccionada para evaluar los saberes previos necesarios para la cursada de la asignatura correspondiente. Se efectuó una partición de los resultados en tres grupos, considerando la proporción de ejercicios bien resueltos. Asimismo, se relevaron datos relativos a la trayectoria de cursada: calificaciones parciales y resultado en instancias de recuperación, condición final de cursada y calificación en exámenes finales. Se observó que en Matemática los niveles de promoción (nota promedio de parciales 7 o más) se mantuvieron prácticamente dentro de los mismos márgenes en ambos cuatrimestres: 10\%, 36\% y 45\% en el primero, frente al 13\%, 33\% y 52\% respectivamente en el segundo, sobre el total de aprobados por grupo, mientras que en el primer cuatrimestre alcanzaron la regularidad (promocionaron o aprobaron los trabajos prácticos) un 47\%, 69\% y 81\% de los alumnos, frente al 36\%, 69\% y 76\% respectivamente del segundo. Por su parte, en Bioestadística, promocionaron el 34\%, 58\% y 66\% del total de regularizados de cada grupo en el primer cuatrimestre, mientras que el 40\%, 36\% y 59\% lo hizo en el segundo. Los niveles de regularización fueron del orden del 72\%, 80\% y 84\% en el primer cuatrimestre, contra un 58\%, 65\% y 78\% en el segundo. Se concluyó que en ambas asignaturas la evaluación inicial proporciona una herramienta de suma utilidad para pronosticar el rendimiento académico posterior, por ejemplo, se observó que en aquellos casos con pobre desempeño inicial, la cursada continúa presentándose con dificultad. La falta de conocimientos adquiridos en los niveles anteriores de formación permite pensar en el uso de algunas estrategias pedagógicas diferentes para este grupo. }
