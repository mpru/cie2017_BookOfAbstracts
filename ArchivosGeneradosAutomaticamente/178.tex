\A
{INTERPRETACIÓN Y LECTURA DE GRÁFICOS ESTADÍSTICOS EN LOS INGRESANTES A LOS PROFESORADOS DE BIOLOGÍA Y MATEMÁTICA}
{\Presenting{CELINA CORRÍAS}\index{CORRÍAS, C}, ADRIANA D'AMELIO\index{D'AMELIO, A} y LAURA ROSSI\index{ROSSI, L}}
{\Afilliation{UNIVERSIDAD NACIONAL DE CUYO}
\\\Email{estat0@hotmail.com}}
{lectura de gráficos; alfabetización; educación} 
 {Enseñanza de la estadística} 
 {Otras categorías metodológicas} 
 {178} 
 {365-1}
{Los gráficos estadísticos son muy útiles para resumir y comunicar información en forma precisa, permiten visualizar con mayor claridad y facilidad conceptos o relaciones entre variables muy complejas. Existe actualmente consenso entre los estadísticos e investigadores en educación acerca de la importancia de que un ciudadano pueda leer de forma crítica y comprensiva los gráficos estadísticos que proporcionan los medios de comunicación, internet o ámbitos de trabajo. Arteaga; Batanero; Díaz; (2009) y Monroy Santana (2007), coinciden en que se suele caer en la falsa apreciación de que la comprensión de gráficos estadístico es una tarea sencilla que no requiere de una especialización. Según estos autores la lectura e interpretación de gráficos estadísticos es una habilidad que no se adquiere de forma espontánea debido a su gran complejidad. Concluyeron que en muchos casos los futuros profesores carecían de los conocimientos matemáticos suficientes para la lectura de dichos gráficos. El objetivo principal de este trabajo, es identificar el nivel de comprensión de gráficos estadísticos de ingresantes a los profesorados de educación media en Biología y Matemática del Instituto de Educación Superior de Formación Docente y Técnica Nº 9-002 Tomás Godoy Cruz. Se plantea como hipótesis que los alumnos presentarían dificultades en el reconocimiento de los elementos que componen un gráfico estadístico, en la comparación e interpretación de valores y en la realización de proyecciones e inferencias. Los elementos de análisis se tomarán del instrumento aplicado a la muestra que consiste en la resolución de un cuestionario. El análisis y discusión de resultados tendrá como marco teórico los niveles de comprensión de gráficos estadísticos de Curcio (1989). La categorización en niveles de comprensión sobre gráficos estadísticos de los alumnos proporcionará información sobre el estado actual del conocimiento de dichos alumnos, dando lugar a la reflexión sobre múltiples aspectos: estrategias de enseñanza, tipo de actividades proporcionadas a los alumnos, instrumentos didácticos, tipo de evaluación, etc. En consecuencia el presente proyecto podrá sentar bases para futuras investigaciones y proyectos educativos que favorezcan la alfabetización estadística de los alumnos. La sociedad moderna produce y maneja gran cantidad de datos útiles para la toma de decisiones de los gobiernos, empresas y ciudadanos en general. Por lo tanto desarrollar capacidades para leer y comprender gráficos estadísticos es una necesidad social.}
