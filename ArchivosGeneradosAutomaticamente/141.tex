\A
{UNA PROPUESTA DIDÁCTICA QUE PROPICIA EL TRABAJO CON IDEAS ESTOCÁSTICAS FUNDAMENTALES Y EL RAZONAMIENTO INFERENCIAL INFORMAL}
{\Presenting{YANINA REDONDO}\index{REDONDO, Y}, LARISA ZILLONI\index{ZILLONI, L} y LILIANA TAUBER\index{TAUBER, L}}
{\Afilliation{UNIVERSIDAD NACIONAL DEL LITORAL}
\\\Email{estadisticamatematicafhuc@gmail.com}}
{ideas estocásticas fundamentales; razonamiento estadístico; inferencia estadística} 
 {Enseñanza de la estadística} 
 {Otras categorías metodológicas} 
 {141} 
 {305-1}
{A partir de trabajos realizados en el marco del proyecto CAI+D: “La inferencia informal como objetivo central de la educación estadística en estudiantes universitarios”, hemos identificado una serie de conceptos asociados a ideas fundamentales del razonamiento informal y de la inferencia estadística informal, estableciendo categorizaciones de dichos conceptos de acuerdo al grado de influencia sobre la comprensión de estas ideas fundamentales (Bianchi y Tauber, 2014; Gioria, 2015). Por otra parte, a través de experiencias previas (Tauber, 2014), hemos detectado un escaso nivel en la formación estadística de los profesores de matemática. Esto evidencia un grave problema, fundamentalmente si consideramos que la educación estadística puede brindar elementos importantes para el pensamiento crítico de cualquier ciudadano. A partir de la identificación de los elementos principales de nuestro marco teórico y del análisis previo de la situación, hemos podido delimitar los dos objetivos que perseguimos en este trabajo: Presentar una propuesta de enseñanza para el aula secundaria que permita construir significados acerca de las ideas fundamentales de aleatoriedad, variabilidad, distribución y modelo, y Realizar un análisis de contenido basado en la caracterización de razonamientos informales que puedan observarse al desarrollar una actividad que integra conceptos del Análisis Exploratorio de Datos y la Inferencia Estadística Informal. Como las actividades propuestas aún no han sido implementadas, aún no podemos hablar de resultados en ese sentido. Pero sí podemos indicar que, dado que el presente forma parte de las tareas programadas en el plan de trabajo de una adscripción en investigación desarrollada por una de las autoras, el resultado de este trabajo es la elaboración de la actividad que presentamos. Por otra parte, en base al análisis de contenido hemos podido establecer una red conceptual en la que se describen las relaciones entre los conceptos e ideas fundamentales que se ponen en juego a partir del desarrollo y ejecución de la propuesta didáctica y, también hemos podido establecer una categorización de los tipos de razonamientos estocásticos informales que se podrían poner en correspondencia A modo de conclusión, consideramos que hemos aportado una propuesta que tiene un enorme potencial didáctico ya que permite poner en relación diversas ideas estocásticas fundamentales y tipos de razonamientos que pueden propiciar la generación de los fundamentos de la Inferencia Estadística Informal. }
