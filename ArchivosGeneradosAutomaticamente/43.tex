\A
{ANÁLISIS DEL CUMPLIMIENTO DE LOS SUPUESTOS DEL MODELO AMMI Y SU INFLUENCIA EN LA SELECCIÓN DE GENOTIPOS DE DURAZNERO}
{\Presenting{JULIA ANGELINI}$^1$\index{ANGELINI, J}, LUIS ARROYO$^2$\index{ARROYO, L}, MARÍA ELENA DAORDEN$^2$\index{DAORDEN, M}, GABRIEL HUGO VALENTINI$^2$\index{VALENTINI, G}, MARTA BEATRIZ QUAGLINO$^3$\index{QUAGLINO, M} y GERARDO DOMINGO LUCIO CERVIGNI$^1$\index{CERVIGNI, G}}
{\Afilliation{$^1$CEFOBI-CONICET}
\Afilliation{$^2$INTA SAN PEDRO}
\Afilliation{$^3$ESCUELA DE ESTADÍSTICA - UNR}
\\\Email{angelini@cefobi-conicet.gov.ar}}
{adaptabilidad; estabilidad; métodos multivariados; multiambiente; prunus persica} 
 {Genética} 
 {Métodos multivariados} 
 {43} 
 {97-1}
{La identificación de genotipos superiores en diferentes ambientes es el principal objetivo del mejoramiento genético. En duraznero, la interacción genotipo x ambiente (IGA) del tipo compleja, altera el desempeño de genotipos haciendo que la selección sea una tarea compleja. El modelo AMMI (Additive Main effects and Multiplicative Interactions), que combina el análisis de la variancia (ANOVA) de efectos genotipo y ambiente, seguido de un análisis de componentes principales (ACP) de los residuos resultantes, ha ganado importancia en ensayos multiambientales. La validez del modelo depende del cumplimiento de dos supuestos, que los errores posean distribución normal y que la varianza de los mismos, entre campañas, sean homogéneas, los que raramente se evalúan o verifican en datos agronómicos. En este trabajo se analizó la robustez de AMMI y del parámetro ASV (AMMI Stability Value) y GSI (Genotype Selection Index), derivados del AMMI, para la selección del 20\% de los genotipos evaluados. La eficiencia obtenida fue comparada con la de los índices de superioridad (Pi) y de confianza (Ii) los cuales no son derivados de AMMI y cuyo buen desempeño en duraznero fue demostrado previamente. Se analizaron datos de rendimiento (kg/planta) de 29 genotipos de duraznero, obtenidos de siete campañas en INTA San Pedro. Como los supuestos de AMMI no se cumplieron, se utilizó la transformación indicada por el método de Box-Cox, potencia de 0,28. En el ANOVA inicial la IGA explico el 63,63\% de la variación de los efectos (P<0,0001), aunque los supuestos no se cumplieron debido a la presencia de valores atípicos. Cuando en el ANOVA se incluyó solamente ambiente y genotipo, la transformación logró una distribución normal de los errores con variancias constantes Las dos primeras componentes explicaron 57,20\% y 55,86\% de la variabilidad residual considerando la variable original y trasformada, respectivamente. Al transformar la variable, ASV coincidió en la selección de 1 de los 6 genotipos seleccionados con la variable original; mientras que con GSI se seleccionaron 4 de 6. Los métodos Pi e Ii seleccionaron los mismos seis genotipos con ambos tipos de variables. Cinco de los seleccionados con Pi e Ii coincidieron con GSI y dos con ASV usando la variable transformada, y de cuatro y uno para GSI y ASV, respectivamente, cuando se usó la variable original. Por lo tanto, al transformar la variable, se mejora levemente la eficiencia de selección de los parámetros derivados de AMMI.}
