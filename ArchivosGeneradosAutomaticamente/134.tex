\A
{ANÁLISIS ESTADÍSTICO DE LOS PATRONES DE COMPORTAMIENTO DE LA SUCCIÓN NUTRITIVA EN RECIÉN NACIDOS A TÉRMINO Y PREMATUROS}
{\Presenting{PATRICIA GIRIMONTE}$^1$\index{GIRIMONTE, P}, LAURA ALVAREZ$^2$\index{ALVAREZ, L}, NATALIA ELISEI$^2$\index{ELISEI, N}, SILVIA JURY$^3$\index{JURY, S} y ANIBAL PEDRO LAQUIDARA$^4$\index{LAQUIDARA, A}}
{\Afilliation{$^1$UBA - CÁTEDRA DE MATEMÁTICA, FFYB, Y DEPARTAMENTO PEDAGÓGICO DE MATEMÁTICA, FCE}
\Afilliation{$^2$UBA - FACULTAD DE MEDICINA}
\Afilliation{$^3$IDIP}
\Afilliation{$^4$UNLP, CIOP-CICBA}
\\\Email{patriciagirimonte@gmail.com}}
{succión nutritiva; recién nacidos pre término; análisis multivariado; correlación} 
 {Salud humana} 
 {Métodos multivariados} 
 {134} 
 {303-1}
{La cuantificación y caracterización de los patrones de succión nutritiva (SN) resultan interesantes y pertinentes clínicamente; siendo esta una actividad que puede ser fácilmente observable cuando se alimenta a un bebé. En Argentina existen antecedentes de evaluación de recién nacidos a término (RNT) pero hasta el momento no en recién nacidos pre término (RNPT). El objetivo del presente trabajo es analizar los patrones de comportamiento de la SN en RNPT y su comparación con los RNT. El estudio consta de una muestra que incluyó 45 RNT sin patología crónica asociada y 75 RNPT concurrentes al Hospital Sor María Ludovica de la ciudad de La Plata. Los recién nacidos fueron evaluados en condiciones similares, con posición de $45^\circ$, en su toma habitual (mamadera estándar y leche que bebía diariamente). Se utilizó un equipo electrónico de manometría (transductor de presión y conexión a computadora) que permite observar y registrar la amplitud [mmHg] y la frecuencia [Succ/seg] de succión. Luego se procesaron las señales con un algoritmo semi-automático. Las mediciones las realizó un solo evaluador en la zona de regularidad de la curva. Se analizó la relación entre las variables amplitud [mmHg] y frecuencia [Succ/seg] de succión. Para el análisis se aplicó la transformación logaritmo a ambas variables. Además de un análisis descriptivo dentro de cada grupo, se realizaron elipses de confianza de nivel 0.95 basadas en la distancia de Mahalanobis; se validó el supuesto de normalidad con el test multivariado de Shapiro Wilks y los test de Mardia. Se aplicó el test T2 de Hotelling resultando la diferencia entre los vectores de medias de ambos grupos estadísticamente significativa. Dentro de cada grupo se analizó la correlación entra amplitud y frecuencia, resultando negativo el coeficiente de correlación de Pearson, pero solo significativo en los RNT. Estos resultados sugieren que existirían diferentes aspectos del patrón de succión que maduran diferencialmente. Estos hallazgos coinciden con lo referido por otros autores interpretando que el comportamiento del prematuro muestra cierta desincronización e inmadurez con respecto al RNT. El análisis estadístico se realizó con R 3.4.0. }
