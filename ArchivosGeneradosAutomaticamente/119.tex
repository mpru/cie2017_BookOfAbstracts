\A
{APLICACIÓN DE MODELOS LINEALES MIXTOS PARA EVALUAR LA VARIABILIDAD ESPACIAL EN UN SUELO SODICO}
{\Presenting{NICOLÁS BATTISTON}$^1$\index{BATTISTON, N}, MICAELA MANZOTTI$^1$\index{MANZOTTI, M}, JUAN PABLO VELEZ$^2$\index{VELEZ, J}, CECILIA MILÁN$^3$\index{MILÁN, C} y PAOLA SALVATIERRA$^1$\index{SALVATIERRA, P}}
{\Afilliation{$^1$UNIVERSIDAD NACIONAL DE VILLA MARIA}
\Afilliation{$^2$INTA- MANFREDI}
\Afilliation{$^3$UNIVERSIDAD NACIONAL DE VILLA MARÍA}
\\\Email{nicolasjbattiston@gmail.com}}
{distribuciones espaciales; suelos alcalinos; geoestadistica moderna} 
 {Ciencias agropecuarias} 
 {Estadística espacial} 
 {119} 
 {258-2}
{La variabilidad de las propiedades edáficas, es entendida como el producto de factores formadores que operan e interactúan en una escala espacial y temporal continua. Describir esta variabilidad a través de técnicas geoestadísticas es de suma importancia para generar estrategias de muestreo, diseños de experimentos, como así también mejorar la capacidad productiva y tomar decisiones sustentables en el uso y manejo de suelos. El objetivo fue caracterizar la distribución espacial de conductividad eléctrica (CE) y pH en un suelo sódico. El estudio se llevó en un lote de 63,5 ha cercano a la localidad de Arroyo Algodón. Se realizó un muestreo de suelo a través dos grillas de 10x10 m georefenciadas ubicadas en zonas diferenciadas por altimetría y conductividad eléctrica aparente (CEa) evaluada por un equipo Veris. Por cada punto georeferenciado se determinó CE1-1 (dSm-1) y pH1-2,5 a tres profundidades diferentes 0-20 cm (P1), 20-40 cm (P2) y 40-60cm (P3). Se ajustaron diferentes modelos lineales mixtos con estructuras de correlación espacial y se utilizó el criterio de Akaike para seleccionar el mejor ajuste. Posteriormente con los parámetros estimados se calcularon índices de varianza estructural relativa (RSV) y mapas de predicción. Las funciones de semivarianza espacial para ambas grillas fueron principalmente Lineal, Exponencial y Gaussiana. Los rangos de distribución tanto para pH1-2,5 como para CE1-1 oscilaron entre 10 y 18 m. Estos valores no se vieron modificados en gran medida ni por la profundidad de la muestra, ni por la altimetría y CEa. En cuanto al RSV, fueron considerados altos, medios y bajos variando según la propiedad y profundidad. Se pudo observar que, cuando el grado de estructuración espacial fue alto para CE 1:1 para pH 1:2,5 fue medio y bajo, en las mismas profundidades analizadas. A través del análisis de mapas de predicción se podría concluir que ambas propiedades presentan una estructura de variación tipo vertical y horizontal.}
