\A
{PROMOVIENDO EL DESARROLLO DEL PENSAMIENTO ESTADÍSTICO EN LA EDUCACIÓN MEDIA A TRAVÉS DEL ABORDAJE DE PROBLEMÁTICAS COMUNITARIAS}
{\Presenting{GILDA GARIBOTTI}$^1$\index{GARIBOTTI, G}, DANIELA ZACHARÍAS$^1$\index{ZACHARÍAS, D}, CLAUDIA HUAYLLA$^2$\index{HUAYLLA, C}, MAILEN LALLEMENT$^3$\index{LALLEMENT, M} y GRUPO ESTADÍSTICA PARA RESOLVER PROBLEMAS DE LA COMUNIDAD: PERROS DE LA CALLE$^4$}
{\Afilliation{$^1$DEPARTAMENTO DE ESTADÍSTICA, CENTRO REGIONAL UNIVERSITARIO BARILOCHE, UN COMAHUE}
\Afilliation{$^2$DEPARTAMENTO DE MATEMÁTICA, CENTRO REGIONAL UNIVERSITARIO BARILOCHE, UN COMAHUE}
\Afilliation{$^3$GRUPO DE EVALUACIÓN Y MANEJO DE RECURSOS ÍCTICOS, CENTRO REGIONAL UNIVERSITARIO BARILOCHE, UN COMAHUE}
\Afilliation{$^4$UNCO CRUB}
\\\Email{garibottig@comahue-conicet.gob.ar}}
{educación en contexto; diseño de experiencia; obtención de datos; estadística descriptiva} 
 {Enseñanza de la estadística} 
 {Diseño de experimentos} 
 {26} 
 {73-2}
{Los recursos didácticos usualmente utilizados para la enseñanza de estadística están enfocados en cálculos que no permiten comprender el amplio espectro de instancias involucradas en un trabajo estadístico. El desarrollo del pensamiento y razonamiento estadístico es complejo e incluye múltiples aspectos: diseño de experimentos, análisis de datos e interpretación de los resultados en relación a la problemática social o experimental abordada. Este desarrollo se adquiere con mayor profundidad e intuición a través de la realización de proyectos interdisciplinarios donde se aborden problemas que movilicen a los alumnos. El objetivo de este trabajo es proponer una metodología multidisciplinaria, basada en una problemática comunitaria, orientada tanto a contribuir directamente a la formación estadística de alumnos del nivel medio como a brindar herramientas a los profesores. Además, se busca acercar a los alumnos del nivel medio a la universidad, de manera de facilitar su posible inserción en este nivel educativo. Se abordó la temática de la tenencia responsable de perros y su relación con la salud humana. Se diseñó un cuestionario para obtener información demográfica de los perros, características de su cuidado, conocimiento sobre zoonosis y episodios de mordeduras o accidentes de tránsito a causa de perros. En visitas a las escuelas se explicó el procedimiento para realizar un trabajo de investigación, con énfasis en el diseño de experimentos y la obtención de datos. Se capacitó a los alumnos para realizar las entrevistas (cada alumno entrevistó y fue entrevistado por un compañero) y el llenado de la base de datos. Los alumnos de las escuelas concurrieron a la universidad para la presentación del análisis estadístico de la información de la encuesta, las zoonosis más frecuentes en la región y las características principales de la tenencia responsable de perros. A manera de cierre, se realizó un debate sobre alternativas para contribuir a resolver el problema y presentación de propuestas elaboradas por los alumnos de las escuelas. Participaron aproximadamente 200 alumnos de cuarto y quinto año de 4 escuelas de San Carlos de Bariloche, entre mayo y junio de 2017. Se presentó una metodología de acercamiento de la estadística a alumnos de nivel medio recurriendo al estudio multidisciplinario de una problemática social e integrando diferentes materias, facilitando la comprensión de la aplicación de la estadística en contexto y no separada de la realidad. Esta estrategia de trabajo puede ser abordada con temáticas propuestas por los mismos alumnos, incluso referidas a la institución educativa.}
