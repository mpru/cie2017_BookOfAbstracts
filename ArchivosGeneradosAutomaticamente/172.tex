\A
{ELABORACIÓN DE INDICADORES PARA LA GESTIÓN PÚBLICA: EL CASO DEL FACTOR MANO DE OBRA DE CÓRDOBA}
{\Presenting{PATRICIA MASERA}\index{MASERA, P} y MARIANA DIAZ\index{DIAZ, MARIANA@DÍAZ, MARIANA}}
{\Afilliation{GOBIERNO DE LA PROVINCIA DE CÓRDOBA, SECRETARÍA DE FORTALECIMIENTO INSTITUCIONAL, DIRECCIÓN GENERAL DE ESTADÍSTICA Y CENSOS}
\\\Email{mariana.diaz@cba.gov.ar}}
{estadísticas oficiales; índice de salarios; provincia de córdoba; redeterminación de precios} 
 {Estadísticas oficiales} 
 {Otras categorías metodológicas} 
 {172} 
 {349-1}
{El presente trabajo tiene como objetivo difundir y poner al alcance de los especialistas y usuarios en general los principales aspectos metodológicos y prácticos vinculados a la construcción del índice “Factor Mano de Obra” de Córdoba. Es fundamental la calidad, exactitud, y oportunidad de esta importante herramienta para la gestión pública. Este indicador es un insumo para el sistema de redeterminación de precios por reconocimiento de variación de costos para la provisión de bienes o prestaciones de servicios del Régimen de Compras y Contrataciones de la Administración Pública Provincial establecido por Ley N° 10.155 y su Decreto Reglamentario N° 305/2014. A través del Decreto 1160/16 se designó a la Dirección General de Estadística y Censos como el organismo encargado de relevar, calcular y publicar mensualmente la evolución de los cambios en los precios medios de los insumos más relevantes utilizados y consumidos en los distintos tipos de contrataciones de Bienes y Servicios llevadas a cabo en la provincia de Córdoba. Uno de estos indicadores es el “Factor Mano de Obra” el cual se desagrega en “sector privado registrado” y “sector público provincial”. Su objetivo principal es reflejar la variación de los salarios medios en las principales actividades productivas de la Provincia. En lo que refiere al sector privado registrado, el cálculo contempla aquellas actividades económicas más relevantes en la provincia, según su masa salarial bruta. Asimismo se incluyeron, independientemente de la masa salarial, aquellas actividades más representativas en las contrataciones relacionadas a la compra y adquisición de bienes y servicios por parte del Estado Provincial. En el presente trabajo se analizan las principales características del índice Factor Mano de Obra Córdoba y se detallan los métodos y procedimientos utilizados para su cálculo, también se presentan los resultados para la serie 2016-2017. Consideramos que la publicación y discusión de los mismos es fundamental para la transparencia, confiabilidad y mejora continua de las estadísticas oficiales }
