\A
{ESTIMACIÓN ROBUSTA EN MODELOS DE REGRESIÓN CON ERRORES EN LAS VARIABLES FUNCIONALES}
{\Presenting{PATRICIA GIMÉNEZ}\index{GIMÉNEZ, P} y LUCAS GUARRACINO\index{GUARRACINO, L}}
{\Afilliation{DEPTO DE MATEMÁTICA - FCEYN. UNIVERSIDAD NACIONAL DE MAR DEL PLATA}
\\\Email{pcgimene@mdp.edu.ar}}
{modelos con errores en las variables; parámetros incidentales; estimación robusta; q-divergencia; lq-verosimilitud} 
 {Ciencias exactas y naturales} 
 {Modelos de regresión} 
 {100} 
 {224-1}
{Un modelo de regresión con errores de medición substanciales en las variables, principalmente en las covariables es conocido en la literatura como un modelo con errores en las variables (MEV). El caso funcional es obtenido cuando las verdaderas covariables son consideradas constantes fijas desconocidas o parámetros incidentales, cuyo número aumenta con el tamaño muestral. Los enfoques usuales de estimación de los parámetros de regresión o parámetros estructurales del modelo, tales como métodos de momentos, máxima verosimilitud, escore condicional o escore corregido entre otros, producen estimadores altamente sensibles ante contaminación de los datos y/o violación de los supuestos del modelo. Luego, métodos robustos en MEV funcionales son necesarios en varias aplicaciones. La literatura en este campo es escasa debido a la dificultad que acarrea la estimación en presencia de parámetros incidentales. En este trabajo proponemos un procedimiento de estimación para un MEV funcional lineal normal a partir del uso de una medida de divergencia cuasi logarítmica o q-divergencia, dependiendo de un parámetro real q. La motivación del enfoque se basa en la cuantificación de la discrepancia entre la densidad obtenida de los datos y la densidad del modelo. El estimador del parámetro de interés es obtenido minimizando esta divergencia como función de este parámetro desconocido. El método resulta equivalente a la minimización de una versión empírica de la entropía de Tsallis-Havdra-Charvat o a la maximización de la Lq-verosimilitud, siendo totalmente paramétrico y evitando los problemas que acarrea la estimación no parámetrica de densidades. La consistencia de Fisher es preservada a partir de una transformación para reescalar el parámetro. El procedimiento propone reemplazar los parámetros incidentales por estimadores que dependen de los parámetros estructurales. La constante q, con 0<q<1, controla el balance entre eficiencia y robustez. El estimador de máxima verosimilitud es obtenido como caso particular para q=1. Como estrategia de selección del parámetro q se considera la minimización del error cuadrático medio calculado por bootstrap paramétrico. Resultados asintóticos y propiedades de robustez de los estimadores son presentados. Su comportamiento es analizado además mediante estudios de simulación bajo diferentes escenarios de contaminación, elecciones de q y varianza del error de medición.}
