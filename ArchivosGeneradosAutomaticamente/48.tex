\A
{FACTORES DETERMINANTES EN LA PERCEPCIÓN DE LOS EGRESADOS EN CIENCIAS ECONÓMICAS EN RELACIÓN A SU CONSEJO PROFESIONAL APLICANDO UN MODELO DE ECUACIONES ESTRUCTURALES}
{\Presenting{MARIANA VERÓNICA GONZALEZ}$^1$\index{GONZALEZ, M}, ROCÍO M. CERINO$^2$\index{CERINO, R}, FERNANDO A. PRONELLO$^2$\index{PRONELLO, F} y FRANCO D. VICO$^2$\index{VICO, F}}
{\Afilliation{$^1$INSTITUTO DE ESTADÍSTICA Y DEMOGRAFÍA. FACULTAD DE CIENCIAS ECONÓMICAS. UNIVERSIDAD NACIONAL DE CÓRDOBA. ARGENTINA}
\Afilliation{$^2$FACULTAD DE CIENCIAS ECONÓMICAS. UNIVERSIDAD NACIONAL DE CÓRDOBA. ARGENTINA}
\\\Email{gonzalezmariana77@gmail.com}}
{ecuaciones estructurales; imagen; valoración} 
 {Otras ciencias económicas, administración y negocios} 
 {Otras categorías metodológicas} 
 {48} 
 {110-1}
{Esta investigación plantea una aproximación exploratoria a la vez que innovadora, construyendo un modelo integrado de relaciones para explicar la percepción de los egresados en Ciencias Económicas de la Provincia de Córdoba con relación a su Consejo Profesional (CPCE). Los Consejos Profesionales, como entidades de derecho público que nuclean a los egresados, reglamentan y ordenan el ejercicio de las profesiones, necesitan identificar los factores que determinan la decisión de los profesionales de incorporarse o permanecer en la organización. En efecto, la consolidación de relaciones estables entre cualquier tipo de organización y sus principales clientes se ha convertido en una herramienta imprescindible para garantizar la supervivencia de dichas organizaciones. En este sentido, interesa analizar la Imagen de CPCE de Córdoba, entendida como la manera en que el profesional percibe a la organización, de acuerdo a sus experiencias y la Valoración de la Matrícula, exigida para el ejercicio de la profesión. La muestra estuvo formada por 533 egresados de universidades públicas y privadas de la Provincia de Córdoba, en carreras vinculadas a las Ciencias Económicas. Del total de profesionales encuestados, el 50,2\% se encontraba matriculado en el CPCE y el resto no, aunque de éste segundo grupo un 23\% estaba considerando matricularse. Se aplicaron modelos de ecuaciones estructurales (structural equations models SEM), con dos variables latentes o constructos: Imagen del CPCE y Valoración de la Matricula. Luego de varias corridas de los modelos se estimó, por Máxima Verosimilitud, un ajuste con siete indicadores significativos para cada constructo. En el caso de la variable latente Valoración de la Matricula aparece como factor determinante que tiene un fin recaudatorio. En la Imagen, por su parte, influyen significativamente la mejora en las condiciones de la profesión y el hecho de que la pone en valor frente a la sociedad, entre otras. Se identificó, además, una correlación estadísticamente significativa entre las dos variables latentes, como así también correlaciones entre algunas indicadoras, que se incluyeron en el modelo a fin de mejorar el ajuste. Los indicadores de bondad de ajuste RMSEA, SRMR y CFI resultaron adecuados. Finalmente, se realizaron ajustes por sexo, concluyendo que no hay diferencias significativas, estadísticamente, entre hombres y mujeres en el sentido de la Imagen y Valoración para los indicadores considerados. }
