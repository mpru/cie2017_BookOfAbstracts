\A
{PREDICCIÓN DE LA DISTRIBUCIÓN DEL TAMAÑO DE FRUTOS DE TANGOR 'MURCOTT' A COSECHA}
{\Presenting{G R R BÓBEDA}$^1$\index{BOBEDA, G@BÓBEDA, G}, L I GIMÉNEZ$^1$\index{GIMÉNEZ, L}, S M MAZZA$^1$\index{MAZZA, S} y S J BRAMARDI$^2$\index{BRAMARDI, S}}
{\Afilliation{$^1$CÁTEDRA DE CÁLCULO ESTADÍSTICO Y BIOMETRÍA. FACULTAD DE CIENCIAS AGRARIAS. UNIVERSIDAD NACIONAL DEL NORDESTE. CORRIENTES, ARGENTINA}
\Afilliation{$^2$DEPARTAMENTO DE ESTADÍSTICA. UNIVERSIDAD NACIONAL DE COMAHUE. NEUQUÉN, ARGENTINA. CITAAC (CENTRO DE INVESTIGACIONES EN TOXICOLOGÍA AMBIENTAL Y AGROBIOTECNOLOGÍA DEL COMAHUE) CONICET-UNCO}
\\\Email{griseldabobeda@gmail.com}}
{prueba de bondad de ajuste; calibres comerciales; curvas de crecimiento} 
 {Ciencias agropecuarias} 
 {Probabilidades } 
 {184} 
 {900-1}
{La estimación temprana del volumen de producción y distribución de tamaños de frutos en cada temporada es una herramienta fundamental en la definición de prácticas de manejo de cultivos, planificación de cosecha y destinos de comercialización de frutos cítricos. Existen diferentes formas de realizar dicha estimación., una de ellas es analizar la distribución de probabilidad que presenta el tamaño con dos meses de anticipación y utilizar el ajuste de dicha distribución para el cálculo de probabilidades por calibre comercial. El presente trabajo se realizó con el objetivo de analizar la distribución que adoptan los calibres comerciales en 8 lotes de tangor 'Murcott' en las localidades de Bella Vista, Concepción y San Lorenzo, durante 5 campañas en la provincia de Corrientes, Argentina. La información se obtuvo de las mediciones del diámetro ecuatorial (mm) de frutos de mandarino tangor Murcott, al momento de la cosecha a partir de una muestra de 50 frutos de 10 árboles por lote comercial. Los datos se agruparon según las categorías de calibres comerciales utilizadas para el mercado externo según el boletín oficial de la Unión Europea, que varían de 53 a 93.5 mm y el número de frutos promedios oscila entre 48 a 130 frutos para cajas plató de 10 kg. Luego de la categorización, se analizó la variabilidad de las distribuciones obtenidas para cada lote y campaña. Mediante herramientas de Estadísticas Descriptiva y para evaluar el ajuste de la distribución de tamaños a una distribución normal se utilizó la Prueba de Bondad de ajuste de X2 (Chi-Cuadrado). Luego se procedió a construir tablas de predicción de porcentajes de la cantidad de frutos en función de la media y desvío estándar del diámetro ecuatorial. Las tablas construidas contienen medias de diámetros que van desde 53 a 82 mm, con desvíos que varían de 4 a 6 mm. que podrían estar asociados a diferentes condiciones ambientales y de manejo. Este ajuste a la distribución normal del diámetro ecuatorial permitió la construcción de tablas de predicción de porcentajes de frutas de cada calibre comercial. Para estimar el porcentaje de calibres de calibres comerciales, se medirá los diámetros con 2 meses de anticipación en cada campaña, y posteriormente predecir la distribución de los tamaños a cosecha. Esto constituye una herramienta útil para productores y sus futuras planificaciones de manejos culturales y comercialización de la fruta.}
