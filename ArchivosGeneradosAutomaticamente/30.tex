\A
{EVALUACIÓN DEL CRECIMIENTO DE POLLOS PARA CARNE A PARTIR DEL ANÁLISIS CONJUNTO DE DOS VARIABLES MEDIANTE UN MODELO NO LINEAL MULTIVARIADO}
{\Presenting{CECILIA RAPELLI}$^1$\index{RAPELLI, C}, RICARDO DI MASSO$^2$\index{DI MASSO, R} y MARÍA DEL CARMEN GARCÍA$^{1,3}$\index{GARCÍA, M D C}}
{\Afilliation{$^1$INSTITUTO DE INVESTIGACIONES TEÓRICAS Y APLICADAS DE LA ESCUELA DE ESTADÍSTICA. FACULTAD DE CIENCIAS ECONÓMICAS Y ESTADÍSTICA. UNIVERSIDAD NACIONAL DE ROSARIO}
\Afilliation{$^2$INSTITUTO DE GENÉTICA EXPERIMENTA. FACULTAD DE CIENCIAS MÉDICAS. UNIVERSIDAD NACIONAL DE ROSARIO. CIC-UNR}
\Afilliation{$^3$CIC-UNR}
\\\Email{cecirapelli@hotmail.com}}
{datos longitudinales; curva crecimiento; modelos no lineales} 
 {Ciencias agropecuarias} 
 {Modelos de regresión} 
 {30} 
 {77-1}
{En las investigaciones biológicas es de frecuente interés describir el proceso de crecimiento de los integrantes de una población mediante un modelo estadístico que explique la modificación de un indicador dado en función de la edad cronológica. Para caracterizar ese proceso se efectúan mediciones en forma repetida de dicho indicador. Los datos longitudinales multivariados surgen cuando un conjunto de diferentes respuestas se mide repetidamente en el tiempo sobre una misma unidad. Resulta de interés para tales datos conocer cómo la evolución de una respuesta está relacionada con la evolución de otra respuesta y/o como la asociación entre las distintas respuestas evoluciona con el tiempo. Por otro lado, la dinámica del crecimiento puede ser descripta mediante curvas de crecimiento sigmoideas. Los modelos no lineales mixtos multivariados proveen una herramienta útil para analizar este tipo de datos en los cuales la relación entre las variables explicativas y las variables respuesta puede modelarse con una función no lineal, permitiendo a los parámetros diferir entre las unidades. En la Cátedra de Genética de la Facultad de Ciencias Veterinaria se llevó a cabo un estudio con pollos destinados a la producción de carne con el fin de estudiar el comportamiento del peso corporal –estimador de la biomasa sustentada- y la longitud de la caña –estimador de la base ósea de sustentación- en función de la edad cronológica. Para esto se registraron ambas variables a intervalos semanales entre el nacimiento y la faena a las 12 semanas de edad. El estudio de la modificación del peso en función de la modificación de la longitud de la caña tiene como particularidad que, dado que la caña tiene un crecimiento finito, la máxima longitud que puede alcanzar limita el rango de variación del peso corporal. Se ajustó un modelo no lineal mixto multivariado utilizando la curva de Gompertz asumiendo que sus tres parámetros varían estocásticamente entre las unidades. El análisis de los resultados obtenidos permite concluir que la tasa de crecimiento de la longitud de la caña y el peso máximo alcanzado por los pollos están negativamente correlacionados (-0,45) indicando que cuanto más rápido crezca la caña menor será el peso máximo. Se observa, además, una fuerte correlación (0,75) entre el comportamiento de la longitud de la caña y el correspondiente al peso corporal de los pollos en función del tiempo, consistente con la condición de indicadores de crecimiento dimensional de ambas variables.}
