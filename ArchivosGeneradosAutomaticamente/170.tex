\A
{APLICACIÓN DEL MODELO DE REGRESIÓN LOGÍSTICA PARA EL ESTUDIO DE LA DESERCIÓN DE LOS INGRESANTES 2016 DE LAS CARRERAS DE GRADO DE LA FACULTAD DE CIENCIAS ECONÓMICAS DE LA UNER}
{\Presenting{OLGA BEATRIZ AVILA}\index{AVILA, O} y STEFANÍA D'IORIO\index{D'IORIO, S}}
{\Afilliation{FACULTAD DE CIENCIAS ECONÓMICAS- UNER}
\\\Email{olga.beatriz.avila@gmail.com}}
{regresión logistica; deserción; ingresantes} 
 {Educación, ciencia y cultura} 
 {Modelos de regresión} 
 {170} 
 {347-1}
{El problema de la deserción estudiantil es uno de los problemas más complejos y frecuentes que enfrentan las Universidades de nuestro país, Argentina, lo que resulta en la necesidad de realizar investigaciones para identificar los factores que influyen en el abandono de los ingresantes a las carreras universitarias. Los resultados de estas investigaciones servirían para identificar y atender las causas que intervienen en el abandono, y poder ejecutar políticas universitarias tendientes a la retención de los ingresantes. En este marco, el presente trabajo tuvo como objetivo identificar los factores que influyen en la deserción de los ingresantes 2016 a las carreras de grado de la Facultad de Ciencias Económicas de la Universidad Nacional de Entre Ríos. Para ello se utilizó un modelo de regresión logística donde se considera la influencia de diferentes variables en la reinscripción de los estudiantes al segundo año de la carrera, en este caso, la reinscripción al año académico 2017, considerando esta variable como posible indicador de deserción. El modelo logístico resulta apropiado cuando la variable dependiente es dicotómica. También llamado modelo de respuesta cualitativa, tiene utilidad para pronosticar qué sucederá cuando existan dos posibilidades, en nuestro caso, se reinscribió o no al siguiente año académico en dos de las carreras que se dictan en la Facultad, que son Contador Público y Licenciatura en Economía. Como posibles variables predictoras se consideraron, entre otras, género, materias aprobadas, cantidad de materias cursadas en el año 2016 y edad al finalizar el periodo. Se trabajó con el total de ingresantes a las carreras mencionadas del año académico 2016, que fueron 375 estudiantes. Por un lado, para ver la de bondad de ajuste del modelo se consideró la prueba de Hosmer y Lemeshow, que indicó un buen ajuste del modelo. Por otro lado, del modelo obtenido cuyas variables estadísticamente significativas resultaron materias aprobadas, materias aprobadas el año anterior, cantidad de materias cursadas en el año 2016, edad al finalizar el periodo, se obtuvo una clasificación para los casos estudiados del 81,6\%. Es necesario continuar con el estudio de las bases de datos de que se dispone, correspondientes a los alumnos ingresantes. para detectar un mayor número de variables que permitan identificar posibles causas de malos rendimientos en el área y, en consecuencia, tomar las medidas académicas necesarias para atenuar esta problemática. En este sentido, se debe mencionar que la Facultad cuenta con un Gabinete Psicopedagógico que trabaja en estos aspectos. }
