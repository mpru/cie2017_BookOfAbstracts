\A
{AJUSTE DE MODELOS POISSON Y BINOMIAL NEGATIVO TRUNCADOS EN CERO MEDIANTE UN ENFOQUE BAYESIANO PARA DATOS DE CONTEO DE MEDICAMENTOS UTILIZADOS POR ADULTOS MAYORES}
{\Presenting{LUCIANA CARLA CHIAPELLA}\index{CHIAPELLA, L} y MARÍA EUGENIA MAMPRIN\index{MAMPRIN, M}}
{\Afilliation{ÁREA FARMACOLOGÍA, FACULTAD DE CIENCIAS BIOQUÍMICAS Y FARMACÉUTICAS, UNIVERSIDAD NACIONAL DE ROSARIO, CONICET}
\\\Email{lucianachiapella@conicet.gov.ar}}
{binomial negativo; mcmc; datos de conteo; medicamentos} 
 {Salud humana} 
 {Métodos bayesianos} 
 {40} 
 {92-1}
{Introducción: Los adultos mayores conforman una población caracterizada por el uso de un alto número de medicamentos de forma diaria y crónica. En los estudios de utilización de medicamentos, la cantidad de fármacos utilizados por paciente es una variable de conteo que podría analizarse mediante modelos Poisson. Sin embargo, el supuesto de equidispersión es muy restrictivo, más aún cuando se está en presencia de datos truncados. Para estos casos, se sugiere utilizar modelos de regresión considerando la distribución Binomial Negativa truncada para modelar la variable respuesta. Objetivos: Comparar los resultados obtenidos mediante el ajuste de modelos Poisson y Binomial Negativo mediante técnicas bayesianas, considerando que los datos se encuentran truncados en cero. Evaluar el posible efecto del sexo y la edad de los pacientes sobre la cantidad de medicamentos utilizados en forma crónica y simultánea. Materiales y métodos: Se registraron los medicamentos dispensados a pacientes mayores de 65 años en 10 farmacias comunitarias de Rosario, entre los meses de abril y septiembre de 2015. Dicho periodo fue dividido en dos trimestres y se contabilizaron los medicamentos dispensados a cada paciente en ambos ciclos. Mediante un enfoque bayesiano, se ajustaron modelos Poisson y Binomial Negativo con distintas distribuciones a priori para el parámetro de dispersión (log-normal y uniforme). Para ello, se realizaron simulaciones mediante el método Monte Carlo Markov Chain (MCMC), utilizando el software Winbugs. Se evaluó la relación entre la edad y el sexo de los pacientes y la cantidad de medicamentos utilizados. Resultados: Según el Deviance Information Criterion (DIC), se verificó que el modelo Binomial Negativo truncado en cero ofrece mejor ajuste que aquel que considera la distribución Poisson para los datos de conteo, incluyendo la edad como variable explicativa (DIC 9315 versus 9761). El efecto del sexo de los pacientes sobre la cantidad de medicamentos utilizados resultó no significativo. La edad de los individuos presentó una relación estadísticamente significativa con la variable respuesta, evidenciando un crecimiento en el número de medicamentos empleados de acuerdo con el aumento en la edad. Conclusiones: Para este caso particular de análisis de datos de conteo positivos, se verificó la utilidad de los modelos binomiales negativos truncados para la modelización de los mismos. Como era previsible este modelo permitió establecer una relación significativa entre el número de medicamentos utilizados y la edad de los pacientes.}
