\A
{CUMPLIMIENTO DE LA LEY NACIONAL DE CONTROL DE TABACO EN PUNTOS DE VENTA CERCANOS A ESCUELAS SECUNDARIAS Y SU RELACIÓN CON EL CONSUMO DE TABACO EN ADOLESCENTES}
{\Presenting{ADRIANA PÉREZ}$^1$\index{PEREZ, A@PÉREZ, A}, GERARDO CUETO$^2$\index{CUETO, G}, SANDRA BRAUN$^3$\index{BRAUN, S} y RAÚL MEJÍA$^3$\index{MEJÍA, R}}
{\Afilliation{$^1$FACULTAD DE CIENCIAS EXACTAS Y NATURALES (UBA) - CEDES}
\Afilliation{$^2$FACULTAD DE CIENCIAS EXACTAS Y NATURALES (UBA)}
\Afilliation{$^3$HOSPITAL DE CLÍNICAS JOSÉ DE SAN MARTÍN - CEDES}
\\\Email{adrianaperez000@gmail.com}}
{tabaquismo; adolescentes; modelos multinivel} 
 {Salud humana} 
 {Modelos de regresión} 
 {99} 
 {221-1}
{Introducción: La promoción del tabaco aumenta el riesgo de que los jóvenes se inicien en el consumo de tabaco. Los puntos de venta de tabaco (PVT) representan un espacio de marketing de interés para la industria tabacalera. Objetivos: 1- Describir la cantidad y proximidad de PVT alrededor de escuelas secundarias. 2- Determinar el cumplimiento de la Ley Nacional de Control de Tabaco, que regula la promoción y publicidad de tabaco, en dichos PVT. 3- Determinar si el consumo de tabaco de los estudiantes se asocia a la cantidad, proximidad y grado de cumplimiento de la ley de los PVT cercanos a las escuelas a las que concurren. Métodos: Una encuesta autoadministrada fue aplicada a 3113 estudiantes de 33 escuelas secundarias públicas y privadas de Buenos Aires, Córdoba y Tucumán y se registró el consumo de tabaco. Se identificaron todos los PVT ubicados en un radio de 300 metros alrededor de las escuelas (n=319) y observadores entrenados determinaron el cumplimiento de siete artículos de la Ley Nacional de Control de Tabaco. Se aplicaron modelos de regresión multinivel para estudiar la relación entre ser fumador actual (últimos 30 días) o experimentador (fumó alguna vez pero no actualmente) en función de la cantidad y proximidad de los PVT y la cantidad de artículos de la ley incumplidos por los PVT. Resultados: Se encontraron entre 0 y 26 PVT alrededor de las escuelas (media=10,3, DE: 6,9). El 70\% de las escuelas tuvo al menos un PVT a menos de 100 metros. El 99\% de los PVT presentaron al menos una violación a alguno de los 7 artículos evaluados y el 45\% incumplió al menos 4 artículos. Las violaciones más habituales fueron el exceso de tamaño y cantidad de letreros publicitarios de tabaco (63\%) y la ausencia de advertencias sanitarias en los mismos (59\%). El 17,7\% de los estudiantes fue fumador actual y el 18,9\% fue experimentador. No se encontró asociación significativa entre uso de tabaco en los estudiantes y las variables asociadas a PVT. Conclusiones: Si bien la Ley Nacional de Control de Tabaco fue promulgada en 2011 se observa un fuerte incumplimiento de la misma en áreas cercanas a escuelas secundarias, exponiendo a los jóvenes a mayor exposición de marketing de tabaco. Si bien nuestros resultados no muestran una mayor prevalencia de tabaquismo asociada a publicidad en PVT, no se puede descartar un efecto de la promoción de tabaco a mediano plazo.}
