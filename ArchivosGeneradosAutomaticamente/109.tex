\A
{ANÁLISIS DE DATOS MULTIVARIADOS CON COORDENADAS PARALELAS: UNA APLICACIÓN AL ESTUDIO DE CAMBIO CLIMÁTICO MEDIANTE EL ANÁLISIS DE LA TEMPERATURA}
{CARLOMAGNO ARAYA ALPÍZAR\index{ARAYA ALPÍZAR, C}}
{\Afilliation{UNIVERSIDAD DE COSTA RICA}
\\\Email{carlomagnocr@gmail.com}
}
{coordenadas paralelas; visualización multidimensional; cambio climático} 
 {Ecología y medio ambiente} 
 {Métodos multivariados} 
 {109} 
 {236-2}
{En esta ponencia, se estudia las Coordenadas Paralelas (Coords$||$), que son un sistema de visualización que permite representar n-dimensiones en un sistema bidimensional. En este sistema, cada eje vertical (ordenada) representa un atributo (dimensión) que puede ser continuo o categórico. Cada uno de los ejes verticales de un sistema de Coords$||$ puede tener su propia escala o definirse todos con una sola escala. La primera forma nos permite la visualización de híper-superficies y el análisis del comportamiento del conjunto de datos, con la segunda podemos hacer un análisis de las relaciones entre las variables. En general, las Coords$||$ son una técnica de visualización donde las dimensiones son simbolizadas como una serie de ejes paralelos, con la misma separación entre ellos (equidistantes) y donde los valores son representados. Cada eje representa una coordenada en la dimensión correspondiente. Uniendo con líneas los ejes, podemos simbolizar los puntos en n-dimensiones. Asimismo, un punto en un espacio n-dimensional es transformado en una línea poligonal a través de n ejes paralelos como n-1 segmentos de línea. La exploración visual de datos multidimensionales es de gran interés en estadística y en la visualización de información. Para tal efecto, se aplican las Coords$||$ al estudio del cambio de la temperatura en la comunidad de San Ramón (Costa Rica), usando los datos de dos estaciones meteorológicas. El estudio se realizará para temperaturas medias diarias de cada estación. En conclusión, las Coordenadas Paralelas son un método multivariado para visualizar en un plano n-dimensional los cambios de las temperaturas por mes y año entre 2008-2017. Como resultado hemos obtenido a través del Modelo Lineal General que la temperatura promedio por mes y año ha variado significativamente, mostrando patrones que son un reflejo del cambio climático. }
