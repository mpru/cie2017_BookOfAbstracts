\A
{CARACTERIZACIÓN CLIMATOLÓGICA PARA UNA REGIÓN DE LA PROVINCIA DE SANTA CRUZ DESPROVISTA DE INFORMACIÓN}
{\Presenting{JEANETTE GONZÁLEZ CASTRO}$^1$\index{GONZÁLEZ CASTRO, J}, LUIS RUBIO JACOBO$^1$\index{RUBIO JACOBO, L} y YHENI FARFÁN MACHACA$^2$\index{FARFÁN MACHACA, Y}}
{\Afilliation{$^1$Universidad Nacional de Trujillo}
\Afilliation{$^2$Universidad Nacional San Antonio Abad del Cusco}
\\\Email{jeanette\_jbgc@hotmail.com}}
{satisfacción; lealtad; turistas; agencias viajes; cusco} 
 {Otras Ciencias Económicas, Administración y Negocios} 
 {Métodos multivariados} 
 {192} 
 {908-1}
{El flujo de los turistas ha producido cambios en los paradigmas de gestión no sólo de las empresas de turismo, sino de las economías de los países en que se desarrolla, razón por la que en esta investigación aplicada con diseño correlacional, nos interesó conocer la satisfacción de los turistas libres y la relación con su lealtad actitudinal. Para ello se entrevistaron a 448 turistas libres; los que llegan a la ciudad sin un paquete turístico comprado previamente a una agencia de viajes mayorista o  minorista desde su origen, sino que les compran a los operadores de turismo que se ubican en el centro histórico del Cusco; a quienes a la hora del retorno de su tour, se les aplicó una encuesta conteniendo las variables satisfacción del turista con la agencia de viajes, intención de recomendación, destino, duración, precio y modalidad de turismo, durante las 83 semanas comprendidas entre el 1 de Enero del 2016 hasta el 31 de Julio del 2017, cambiando cada semana los días de recolección de datos. Los turistas libres utilizaron 110 operadores turísticos, de un total de 636; es decir se trabajó con el 17\% de operadores turísticos. Los datos fueron procesados utilizándose el Análisis Factorial por Componentes Principales, determinándose 3 factores latentes en las variables estudiadas; un Factor de Optimización del Servicio Turístico, relacionado con las variables duración, precio y modalidad de turismo; un Factor de Calidad del Servicio Turístico, relacionado con las variables satisfacción e intención de recomendación  y un tercer Factor de Destino del Tour, con una explicación del 78\% de variabilidad de todas las variables; se encontró que los operadores turísticos más solicitados fueron SAS Travel, con el 37.1\% de turistas; Wayki Trek, con el 8.7\%; Andean Life, con el 6.3\% y Ayni Peru Expeditions, con el  el 5.2\%;  asimismo que  el 71.9\% de turistas están al menos muy satisfechos con el servicio brindado;  que el 90.4\% de turistas si tendrían la intención de recomendación del operador turístico; que el 85.9\% de turistas prefieren la modalidad de turismo de naturaleza y aventura; y que existe una relación altamente significativa entre satisfacción e intención de recomendación, como una medida de la lealtad actitudinal, ya que la significación de la prueba de chi cuadrado (recodificada) = 0.000 y de la prueba aproximada TauB de Kendall, significación aproximada = 0.000.}
