\A
{CARACTERIZACIÓN DE LOS ESTUDIANTES DE BIOESTADÍSTICA Y RELACIÓN CON LA CONDICIÓN FINAL. CARRERA DE NUTRICIÓN. FACULTAD DE SALUD. UNSA}
{\Presenting{SERGIO LEONARDO FONTEÑEZ}\index{FONTEÑEZ, S}, SILVIA ROCIO ECHALAR\index{ECHALAR, S} y MARIA DEL CARMEN HERRERA\index{HERRERA, M}}
{\Afilliation{CÁTEDRA DE BIOESTADÍSTICA. FACULTAD DE CIENCIAS DE LA SALUD. UNIVERSIDAD NACIONAL DE SALTA}
\\\Email{slfsergioslf@gmail.com}}
{bioestadística; estadística descriptiva; estudiantes de nutrición; condición final} 
 {Enseñanza de la estadística} 
 {Inferencia estadística} 
 {89} 
 {199-1}
{INTRODUCCIÓN: Toda Cátedra tiene como objetivo el conocer a los sujeto del aprendizaje y favorecer el éxito académico que se verá reflejado en la condición final del cursado. OBJETIVOS: Caracterizar a los estudiantes que cursaron la asignatura de Bioestadística. Relacionar la condición de cursado de Estadística Descriptiva con la condición final de Bioestadística. DESARROLLO: Se realizó un estudio de corte transversal, descriptivo y analítico. Se aplicó una encuesta online en la plataforma Moodle, la muestra estuvo constituida por 99 estudiantes, se evaluaron variables sociales, de percepción y rendimiento académico. Se aplicaron análisis descriptivos y pruebas inferenciales con InfoStat v.2012. RESULTADOS: El promedio de edad fue 21.34 ±3.13 años, con una edad mínima de 18 y máxima de 41 años. El 89\% de los estudiantes fueron mujeres. Del estado Civil el 95\% fue soltero. Con respecto a la procedencia se observó que el 64.29\% procede de Salta Capital y un 16,33\% de la Provincia de Jujuy. En el ciclo lectivo 2017 se registró que el 55.56\% eran Ingresantes, de los cuales el 87\% finalizó Estadística Descriptiva, y entre los recursantes el 84\%. Los temas que resultaron más complejos en su aprendizaje fueron Correlación y Regresión (41.41\%) y Pruebas de Hipótesis (27.27\%), cabe destacar que un 15\% de los estudiantes manifestaron ninguna dificultad. En cuanto al tema que consideran de mayor utilidad en su profesión encontramos que el 25.25\% considera a todos de igual importancia. El 69.70\% califica como normal su nivel de conocimiento en Estadística Inferencial una vez finalizado el programa. Al finalizar el cursado 49.49\% promocionó, el 38.38\% regularizó y el 10.10\% no regularizó. La condición final de los estudiantes de Bioestadística (Promocionó y Regularizó) es independiente de tener finalizada Estadística Descriptiva (p(v)= 0.1533). CONCLUSIONES: Las características de los estudiantes de Bioestadística obtenidas vienen con una tendencia similar en los últimos años, tener o no finalizada estadística descriptiva no condiciona un buen rendimiento del estudiante.}
