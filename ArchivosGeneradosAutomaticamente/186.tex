\A
{ON THE ROBUSTNESS OF THE PRINCIPAL VOLATILITY COMPONENTES}
{\Presenting{CARLOS TRUCÍOS}$^1$\index{TRUCÍOS, C}, LUIZ K HOTTA$^2$\index{HOTTA, L} y PEDRO VALLS$^1$\index{VALLS, P}}
{\Afilliation{$^1$SÃO PAULO SCHOOL OF ECONOMICS, FGV - BRAZIL}
\Afilliation{$^2$UNIVERSITY OF CAMPINAS - BRAZIL}
\\\Email{carlos.trucios@fgv.br}}
{covariance matrix; forecast; high-dimensional data; principal components; outliers; volatility} 
 {Economía} 
 {Series de tiempo} 
 {186} 
 {902-1}
{In this work, we analyze the effects of additive outliers on the recently proposed procedure of Hu and Tsay (2014) (Principal volatility component analysis. Journal of Business \& Economic Statistics, v32.2) and Li et al. (2016) (Modeling multivariate volatilities via latent common factors. Journal of Business \& Economic Statistics, v34.4) called principal volatility components. This procedure overcome several difficulties in modeling and forecasting the conditional covariance matrix in large dimensions arising from the curse of dimensionality. We show that outliers have a devastating effect on the construction of the principal volatility components, in the estimation of the number of components and on the forecast of the conditional covariance matrix in both small and large dimensions and in consequence in economic and financial applications. In fact, we show that just a few outliers are necessary to affect drastically the volatility components and the estimation of the conditional covariance matrix. We propose a robust procedure based on a robust estimator of the unconditional covariance matrix, on a weighted estimator that capture the volatility dynamic and in a robust filter. We analyze the finite sample properties by means of Monte Carlo experiments considering several patterns of contamination and show that the robust procedure outperforms the classical method in contaminated series and has a similar performance in uncontaminated ones. The new procedure is also applied to 73 stock returns of the Nasdaq 100 index and a minimum variance portfolio is constructed. Results are compared with the non-robust version. The new procedure showed a better performance in both simulated and real data.}
