\A
{COMPARACIÓN DE TÉCNICAS NO PARAMÉTRICAS PARA EL ANÁLISIS DE DISEÑOS DE EXPERIMENTOS BIFACTORIALES CON TAMAÑOS MUESTRALES PEQUEÑOS A TRAVÉS DE UN ESTUDIO POR SIMULACIÓN}
{\Presenting{MARCOS PRUNELLO}\index{PRUNELLO, M}, MARÍA BELÉN ALLASIA\index{ALLASIA, M}, JUAN JOSÉ IVANCOVICH\index{IVANCOVICH, J}, SABRINA SILVA QUINTANA\index{SILVA QUINTANA, S}, LAURA PISKULIC\index{PISKULIC, L}, HEBE BOTTAI\index{BOTTAI, H} y LILIANA RACCA\index{RACCA, L}}
{\Afilliation{ÁREA ESTADÍSTICA Y PROCESAMIENTO DE DATOS, FACULTAD DE CIENCIAS BIOQUÍMICAS Y FARMACÉUTICAS, UNIVERSIDAD NACIONAL DE ROSARIO}
\\\Email{estadistica.fbioyf@gmail.com}}
{diseño bifactorial; muestras pequeñas; métodos no paramétricos; simulación; interacción} 
 {Biología} 
 {Diseño de experimentos} 
 {50} 
 {113-1}
{El uso de diseños factoriales se ha generalizado en el ámbito científico por su capacidad para detectar efectos de los factores en estudio y sus interacciones. En las ciencias biológicas son frecuentes el pequeño número de observaciones, el incumplimiento de los supuestos requeridos por la técnica clásica y la presencia de observaciones atípicas (outliers). Este trabajo compara cinco técnicas: el método clásico (MC), Aligned Rank Transform (ART), Puri-Sen (PS), Van der Waerden (VDW) y una alternativa robusta (R). Se efectuó un estudio por simulación en contextos donde no se cumple alguno de los supuestos para un diseño bifactorial balanceado con hasta diez unidades experimentales por tratamiento (n). Los datos fueron generados a partir de distintos escenarios: ningún efecto significativo (E1), un efecto principal (E2), ambos efectos principales (E3), sólo efecto interacción (E4), un efecto principal y efecto interacción (E5), todos los efectos significativos (E6). Se evaluó la capacidad de las técnicas para ensayar el efecto de la interacción a través de la estimación de la probabilidad de error de tipo I (P(EI)) en los escenarios E1, E2 y E3, y de la potencia en E4, E5 y E6, considerando un nivel de significación $\alpha$=0,05. Cuando se viola el supuesto de normalidad, ART arrojó una P(EI) mayor a 0,05 aproximándose al valor nominal cuando n aumenta. Las restantes técnicas resultaron más conservadoras. Si bien PS y VDW mostraron mayor potencia en E4, ART fue superadora ante la presencia de al menos un efecto principal (E5 y E6) mientras MC y R tuvieron peor desempeño. En el caso de heterocedasticidad, ninguna técnica produjo P(EI) empíricas cercanas a la nominal, con excepción de MC y VDW en E1. Para E4, E5 y E6, ART presentó potencias mayores que el resto. Ante la presencia de outliers para E1 y E2 la P(EI) fue aproximadamente igual a la nominal para todas las técnicas y cualquier n, a excepción de R cuyo desempeño fue inferior. Para E4, E5 y E6, PS, VDW y ART produjeron potencias similares y mayores a las obtenidas con R y MC. En estas dos últimas la potencia no cambió al aumentar n. En conclusión, no fue posible identificar una técnica que mantenga la P(EI) bajo control o provea una potencia satisfactoria en todas las situaciones consideradas. No obstante, en general se observó un mejor desempeño de ART, especialmente frente a la existencia de al menos un efecto principal significativo. }
