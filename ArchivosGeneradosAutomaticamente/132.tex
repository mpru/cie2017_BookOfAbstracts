\A
{DIFERENCIACIÓN POR CALIDAD EN EL COMERCIO GLOBAL DE ALIMENTOS PESQUEROS: UN MODELO GRAVITACIONAL A NIVEL DE PRODUCTO}
{\Presenting{MARÍA VICTORIA LACAZE}$^1$\index{LACAZE, M} y OSCAR MELO$^2$\index{MELO, O}}
{\Afilliation{$^1$UNIVERSIDAD NACIONAL DE MAR DEL PLATA}
\Afilliation{$^2$PONTIFICIA UNIVERSIDAD CATÓLICA DE CHILE}
\\\Email{mvlacaze@mdp.edu.ar}}
{modelo gravitacional; comercio; alimentos pesqueros; diferenciación; eco-etiquetado; arancel equivalente} 
 {Economía} 
 {Modelos de regresión} 
 {132} 
 {296-1}
{Los esquemas de eco-etiquetado constituyen una importante herramienta de gestión de la calidad en el mercado global de alimentos pesqueros. Recientemente, se ha discutido si estos esquemas obstaculizarían el acceso al mercado por parte de productos procedentes de pesquerías (y países) del hemisferio sur. Sin embargo y hasta el presente, no hay análisis econométricos que brinden evidencia a favor del desempeño del eco-etiquetado como catalizador del o barrera al comercio de estos alimentos. Este trabajo contribuye a esa área de vacancia, mediante la estimación de un modelo gravitacional estructural, sobre un panel multidimensional de datos de comercio bilateral de alimentos pesqueros entre 38 países que, en términos de valor y para el período enero 2010–diciembre 2014, concentran el 80\% de los flujos comerciales globales. La fuente principal de información es la base de exportaciones mensuales UN-Comtrade, a cuatro dígitos de desagregación del sistema armonizado (HS). Dado que el HS no efectúa desagregaciones por estándares de calidad, se realizó un matching entre dicha fuente y una base de datos auxiliar, para identificar los flujos comerciales que corresponden a productos eco-etiquetados. Esa base auxiliar fue elaborada a los fines del estudio y contiene información procedente de los sitios web de los principales esquemas de eco-etiquetado vigentes. Debido a la separabilidad del modelo gravitacional, la estimación se efectúa a nivel de producto. El tratamiento de los términos de resistencia multilateral se resuelve con la incorporación de efectos fijos direccionales, por país y período. La estimación en forma multiplicativa, mediante el estimador máximo verosímil Pseudo-Poisson, soluciona el sesgo por selección muestral que genera la alta prevalencia de ceros en la variable dependiente y aborda la heteroscedasticidad de los datos, circunstancias habituales en este tipo de estudios. La aproximación a los costos comerciales bilaterales incluye la distancia marítima entre los principales puertos pesqueros, la certificación mediante eco-etiqueta y otras covariables de uso tradicional, como la contigüidad, el idioma común, la relación colonial y la pertenencia a un mismo acuerdo comercial regional, además de los aranceles bilaterales. El modelo estocástico estimado queda expresado por: $X_{ij,t} = exp[p_{ii,t} + n_{j,t} + b_1 Dist_{ij} + b_2 Cont_{ij} + b_3 Idiom_{ij} + b_4 Colon_{ij} + b_5 EcoEt_{ij,t} + b_6 Aranc_{ij,t}] x E_{ij,t}$. A partir de los coeficientes estimados, se obtuvieron los efectos de volumen o elasticidades parciales de las exportaciones, así como el efecto arancelario equivalente de la certificación mediante eco-etiqueta, es decir el arancel ad valorem cuya aplicación o remoción (según cada producto) equivale al efecto, estadísticamente significativo, que la eco-etiqueta produce en el comercio de estos alimentos. }
