\A
{EXPERIENCIA PEDAGÓGICA EN LA ENSEÑANZA DE ESTADÍSTICA APLICADA EN LA CARRERA DE RELACIONES DEL TRABAJO: EL ÁMBITO DEL TRABAJO Y TECNOLOGÍAS DE LA INFORMACIÓN}
{GRACIELA DURET\index{DURET, G}}
{\Afilliation{FACULTAD DE CIENCIAS SOCIALES. UBA}
\\\Email{gcduret@gmail.com}
}
{enseñanza; estadística; gestión; conocimiento; investigación; pedagógica} 
 {Enseñanza de la estadística} 
 {Otras categorías metodológicas} 
 {18} 
 {40-1}
{Durante el período 2013-2015 se llevó a cabo en la Cátedra de Estadística Aplicada (I y II) de la Carrera de Relaciones del Trabajo, en el marco del Programa de Reconocimiento Institucional de Investigaciones, siendo la especialidad temática Estadística y Gestión del Conocimiento, el proyecto “El ámbito del trabajo y su integración con las Tecnologías de la Información y Comunicación (TIC)”. El objetivo general del proyecto fue introducir a los alumnos en la utilización de métodos estadísticos para llevar a cabo investigaciones relacionados al mundo laboral, dando los primeros pasos en ambos campos, el de la estadística y el de la investigación. Se plantearon dos desafíos: una experiencia pedagógica y llevar a cabo una investigación cuantitativa. El objetivo pedagógico fue explorar la factibilidad de formar equipos de investigación que involucraran alumnos de grado y realizar trabajos que continuasen después del cursado de las materias. El objetivo de la investigación cuantitativa fue relevar y describir nuevas formas de trabajar mediadas por las TIC. Dada la relevancia que tiene para los profesionales de Relaciones del Trabajo el conocer la problemática de la incidencia de las TIC en el mundo laboral, a pedido de los alumnos, se decidió investigar acerca del teletrabajo, y sobre todo, indagar acerca de cuánto sabían los estudiantes de este tema. Así se generó la encuesta acerca de “El ámbito del trabajo y tecnologías de la información”, que se suministró a los alumnos de la Cátedra que cursaron durante el primer cuatrimestre de 2015. Se analizaron 268 encuestas relevadas entre el 30 de abril y el 29 de mayo de 2015. Con respecto a la experiencia pedagógica se concluye que la respuesta que tuvo la convocatoria de participar en un trabajo de investigación fue mínima ya que implica un compromiso en el tiempo, más allá de la cursada de las materias, que no todos están en condiciones de asumir. Con respecto a la investigación cuantitativa algunos resultados son: el 78\% de los encuestados fueron mujeres. El 90\% considera al teletrabajo como mecanismo de inclusión social y laboral para personas discapacitadas y el 75\%, un beneficio. Se propone fomentar la difusión del teletrabajo y seguir en este camino de la experiencia pedagógica en la formación de investigación, ya que la misma propicia la reflexión, la innovación y producción en los alumnos de la carrera. }
