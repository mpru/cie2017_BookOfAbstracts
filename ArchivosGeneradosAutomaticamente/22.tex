\A
{ANÁLISIS POR DISEÑO EXPERIMENTAL DE LA REMOCIÓN DE CASEINATO DE SODIO POR EC-EF}
{\Presenting{PAULA V SARMIENTO}\index{SARMIENTO, P}, MAURICIO MATTALIA\index{MATTALIA, M}, DIEGO G SEMPRINI\index{SEMPRINI, D} y MIGUEL A ROSA\index{ROSA, M}}
{\Afilliation{GIDAIQ - UNIVERSIDAD TECNOLÓGICA NACIONAL, FACULTAD REGIONAL VILLA MARÍA, VILLA MARÍA, CÓRDOBA}
\\\Email{paula.victoria.sarmiento@gmail.com}}
{diseño de experimentos; electrocoagulación; efluentes; caseinato de sodio} 
 {Industria y mejoramiento de la calidad} 
 {Diseño de experimentos} 
 {22} 
 {65-1}
{En este trabajo se aplicó un diseño de experimentos para obtener un modelo matemático representativo de la remoción de proteínas en efluentes mediante el proceso de electrocoagulación-electrofloculación (EC-EF). Además, se determinó un modelo que permitiese estimar el consumo de energía del proceso. Se aplicó el método de superficie de respuesta con un diseño de Box-Behnken el cual facilita la posterior optimización del modelo y presenta un menor número de experimentos respecto al diseño central compuesto. Las experiencias se realizaron empleando un efluente sintético, compuesto por caseinato de sodio (CS) y agua, el contenido de proteínas se determinó aplicando el método de Bradford. El reactor empleado consiste en una cuba de acrílico y una celda de 4 electrodos de aluminio con separación de 1 cm, área efectiva de 270 cm2 y volumen útil de 1,7 L. La operación de la celda se realizó mediante una conexión bipolar, inversión de polaridad cada 5 minutos y agitación magnética. Se registró la intensidad de corriente que circulaba por la celda cada 1 minuto para calcular la energía consumida. El post-tratamiento consistió en centrifugar las muestras durante 10 minutos y determinar la absorbancia correspondiente al contenido de proteínas remanente. Los factores seleccionados fueron concentración de CS, voltaje y tiempo con tres niveles. Se obtuvo un diseño con 17 corridas en un bloque. Para la remoción de proteínas se obtuvo un modelo cuadrático reducido donde los términos más significativos fueron el voltaje, el tiempo y el término cuadrático de la concentración. La falta de ajuste presentó un valor-p de 0.2759, lo cual indica que la misma no es significativa y el modelo ajusta correctamente los datos. El valor de R2 fue de 93.87\%, y el modelo desarrollado puede emplearse para uso predictivo dentro del espacio de trabajo analizado. Para obtener un modelo adecuado para la energía consumida, se debieron transformar los valores reales mediante una función raíz cuadrada. El mejor ajuste de los datos transformados se obtuvo con un modelo que incluye las interacciones entre dos factores, donde los más significativos fueron voltaje, tiempo, y la interacción de segundo orden entre los mismos. Este modelo arrojó un valor-p para la falta de ajuste de 0.8403 y un R2 de 98.42\%. El análisis de los gráficos de diagnóstico permitió confirmar su validez. Como próxima actividad se plantea realizar la optimización de estos modelos y verificar su eficacia de predicción.}
