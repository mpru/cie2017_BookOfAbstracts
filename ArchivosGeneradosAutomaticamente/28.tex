\A
{APLICACIÓN DE TÉCNICAS DE METAANÁLISIS PARA LA DETECCIÓN DE EXPRESIÓN DIFERENCIAL DE GENES EN LA CANCROSIS DE LOS CÍTRICOS}
{\Presenting{LAURA PISKULIC}$^1$\index{PISKULIC, L}, MARCOS PRUNELLO$^{1,3}$\index{PRUNELLO, M} y LUCAS DAURELIO$^{1,2,3}$\index{DAURELIO, L}}
{\Afilliation{$^1$ÁREA ESTADÍSTICA Y PROCESAMIENTO DE DATOS, FACULTAD DE CIENCIAS BIOQUÍMICAS Y FARMACÉUTICAS, UNR}
\Afilliation{$^2$CÁTEDRA DE FISIOLOGÍA VEGETAL, FACULTAD DE CIENCIAS AGRARIAS, UNL}
\Afilliation{$^3$CONICET}
\\\Email{lpiskulic@yahoo.com}}
{metaanálisis; expresión génica; microarreglos; interacción planta-patógeno; cancrosis de los cítricos} 
 {Ciencias agropecuarias} 
 {Otras categorías metodológicas} 
 {28} 
 {76-1}
{Introducción: Se denomina metaanálisis (MA) al uso de técnicas estadísticas para combinar resultados de experimentos independientes pero referentes al mismo problema. El MA tiene la virtud tanto de aumentar la potencia como de generalizar conclusiones a partir de estudios únicos. En lo que respecta a los experimentos de expresión génica, los cuales miden la cantidad de transcripto o ARN mensajero que se genera a partir de un gen codificado en el ADN, el MA permite aumentar la potencia para detectar genes diferencialmente expresados (DE), es decir aquellos que cambian su expresión en forma significativa al comparar dos o más tratamientos. La interacción entre las plantas y los patógenos, que puede dar como resultado que la planta presente defensa o enfermedad, involucra cambios en la expresión génica de ambos. Las técnicas de análisis de expresión génica como los microarreglos, junto con la disponibilidad de secuencias genómicas para algunas especies vegetales, han permitido un significativo progreso en la caracterización de genes implicados en la interacción planta-patógeno. La producción de cítricos, principal cultivo frutícola en la Argentina, se ve afectada por la cancrosis, enfermedad producida por la bacteria Xanthomonas citri pv. citri (Xcc). Objetivo: Aplicar diferentes técnicas estadísticas de MA sobre experimentos de expresión génica para detectar genes de cítricos que responden a la infección por Xcc. Metodología: Se aplicaron 7 métodos de MA basados en la combinación de los valores p obtenidos en 5 experimentos de microarreglos, realizados sobre hoja de Citrus sinensis durante la infección por cancrosis. Los métodos aplicados fueron: mínimo valor p, máximo valor p, valor p de orden 3, Fisher, Stouffer, Suma de Rangos y Producto de Rangos. Resultados: De los 7167 genes estudiados, 232 resultaron DE por lo menos en 5 de las 7 técnicas aplicadas. De estos, 68 fueron DE en todas las técnicas, y 43 en 6 de ellas. El estudio funcional de este conjunto de genes mediante Singular Enrichment Analysis permitió identificar 23 categorías significativas, entre las cuales se destacan traducción y fotosíntesis, y un número importante de genes relacionados a estrés biótico, regulación y hormonas. Conclusión: El uso de técnicas de MA permitió identificar genes DE en 5 experimentos diferentes, cuyo análisis contribuirá a la interpretación de los mecanismos moleculares involucrados en la interacción planta-patógeno que conduce a la cancrosis de los cítricos. Estos resultados además podrán ser utilizados para buscar alternativas que permitan combatir esta importante enfermedad. }
