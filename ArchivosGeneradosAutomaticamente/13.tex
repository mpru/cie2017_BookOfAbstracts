\A
{DEMORA EN LA ELECCIÓN DE CARRERA EN LA FACULTAD DE CIENCIAS ECONÓMICAS Y ESTADÍSTICA UNA ESTIMACIÓN MEDIANTE CADENAS DE MARKOV}
{\Presenting{GABRIELA BOGGIO}\index{BOGGIO, G}, NORA ARNESI\index{ARNESI, N} y LUCIANA RUIZ\index{RUIZ, L}}
{\Afilliation{INSTITUTO DE INVESTIGACIONES TEÓRICAS Y APLICADAS DE LA ESCUELA DE ESTADÍSTICA, FACULTAD DE CIENCIAS ECONÓMICAS Y ESTADÍSTICA, UNIVERSIDAD NACIONAL DE ROSARIO}
\\\Email{gboggio@fcecon.unr.edu.ar}}
{cadenas absorbentes; estados túneles; rendimiento académico} 
 {Educación, ciencia y cultura} 
 {Probabilidad y procesos estocásticos} 
 {13} 
 {27-2}
{Uno de los problemas que enfrenta la Facultad de Ciencias Económicas y Estadística de la Universidad Nacional de Rosario en relación al rendimiento académico de los alumnos es la demora en los tiempos previstos por los planes de estudio para cumplir una primera meta relacionada con la posibilidad de elegir carrera. El plan de estudios de las tres carreras de Ciencias Económicas de esta Facultad comienza con un Ciclo Introductorio Común y la elección de la carrera es posible previa aprobación de cuatro asignaturas de este ciclo. Por ello interesó estimar el tiempo promedio que los alumnos demoran hasta alcanzar esta condición requerida para elegir carrera. A tal fin se analizó la cohorte de ingresantes del año 2008, evaluando al inicio de cada año académico la cantidad de materias aprobadas de ese ciclo introductorio hasta la finalización del año académico 2016. Se tomó como criterio que el alumno que estuvo pasivo durante dos años académicos consecutivos está en una situación no explícita asimilable al abandono de la carrera. Se recurrió a la metodología de cadenas de Markov definiendo como estados absorbentes los que representan estar en condición de elegir carrera o abandonar los estudios y como estados transitorios la cantidad de materias aprobadas al finalizar cada año académico (ninguna, una, dos o tres). Se pudo evidenciar la falta de homogeneidad en las probabilidades de transición entre los distintos estados en los 9 años siguientes al ingreso de los alumnos, fundamentalmente en el primer año de cursado. Para salvar este problema se usaron los denominados estados túneles que permiten establecer probabilidades de transición con un ajuste temporal y se pusieron a prueba distintas estructuras de cadena teniendo en cuenta reagrupamientos de los estados en la búsqueda de la mejor representación de la situación estudiada. El tiempo promedio hasta la elección de carrera bajo las diferentes estructuras consideradas resultó igual a 2,3 años y las diferentes representaciones dieron lugar a la estimación de la probabilidad de alcanzar esta condición bajo distintas circunstancias. Por ejemplo, la probabilidad de que un ingresante logre la condición en el primer año resultó igual a 0,57 mientras que aquél que, finalizado el primer año, no pudo aprobar materias tiene una probabilidad de 0,08 de lograr este objetivo. Dicha probabilidad se redujo a 0,02 ante la persistencia en esta situación pasados dos ciclos.}
