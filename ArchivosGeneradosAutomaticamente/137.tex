\A
{EFICACIA Y SEGURIDAD DE UNA FORMULACIÓN INNOVADORA DE LATANOPROST 0.005\% LIBRE DE CLORURO DE BENZALCONIO PARA EL TRATAMIENTO DEL GLAUCOMA DE ÁNGULO ABIERTO}
{\Presenting{CELINA LOGIOCO}$^1$\index{LOGIOCO, C}, VIRGINIA ZANUTIGH$^1$\index{ZANUTIGH, V}, MARÍA SILVIA PASSERINI$^1$\index{PASSERINI, M}, MARÍA DE LOURDES RODRIGUEZ$^1$\index{RODRIGUEZ, M}, MARÍA JOSÉ CASTRO$^2$\index{CASTRO, M} y MYRIAM NUÑEZ$^2$\index{NUÑEZ, M}}
{\Afilliation{$^1$CENTRO DE OJOS QUILMES}
\Afilliation{$^2$UNIVERSIDAD DE BUENOS AIRES. FACULTAD DE FARMACIA Y BIOQUÍMICA. DEPARTAMENTO DE FISICOMATEMÁTICA. CÁTEDRA DE MATEMÁTICA}
\\\Email{myriam@ffyb.uba.ar}}
{glaucoma de ángulo abierto; superficie ocular; eficacia y tolerabilidad; presión intraocular} 
 {Salud humana} 
 {Inferencia estadística} 
 {137} 
 {304-4}
{El Glaucoma de ángulo abierto, también conocido como el ladrón de la vista es, junto al glaucoma de ángulo cerrado, uno de los dos tipos de glaucoma existentes. Es una enfermedad del ojo cuyo principal factor de riesgo es la elevación de la presión intraocular. El cloruro de benzalconio (BAK) se utiliza en formulaciones de análogos prostaglandínicos debido a su acción adyuvante y conservante. Sin embargo, su acción conservante tiene efectos tóxicos conocidos sobre la superficie ocular, causando sequedad ocular y malestar a largo plazo. El objetivo de este estudio es evaluar la eficacia y tolerabilidad de una nueva emulsión latanoprost 0.005\% (LEBAK-free) respecto al a la solución tradicional de Latanoprost 0.005\% con BAK (LSBAK). Se llevó a cabo un estudio prospectivo durante 12 semanas. Un total de 16 pacientes (32 ojos) con glaucoma de ángulo abierto primario que fueron tratados con LSBAK por más de 6 meses. Luego se procedió a la suspensión de la aplicación de LSBAK para comenzar con el uso de LEBAK-free. La eficacia y la tolerabilidad de las formulaciones fueron evaluadas al momento de suspender el tratamiento con LSBAK y luego de 12 semanas de recibido el nuevo producto. El valor de la presión intraocular (PIO) se midió (n = 32 ojos) al momento de comenzar el tratamiento con LEBAK-free (semana 0), luego a las 4, 8 y 12 semanas. Se investigó la ocurrencia de efectos adversos y tolerabilidad al nuevo producto. Se analizó la hipótesis de independencia entre ambos ojos. Para la variable PIO se estudió la existencia de diferencias significativas a lo largo de las semanas, Análisis de la Varianza con Medidas Repetidas. Para esto se analizó el supuesto de normalidad de la variable en estudio que se requiere para la aplicación del método. Se comprobó que la variable seguía una distribución normal (Test de Shapiro Wilks, p = 0,359). El ANOVA reflejó que, existen diferencias significativas entre las medias de los 4 grupos (p = 0,005). Las comparaciones a posteriori mostraron la existencia de diferencias significativas entre las medias de la PIO para las semanas 1 y 4 (Método de Bonferroni). Se concluyó que la nueva fórmula de Latanoprost 0.005\%en emulsión sin BAK garantiza una distribución homogénea en la superficie ocular, asegurando como mínimo la misma eficiencia en reducir la PIO que la solución tradicional, mejorando la tolerabilidad local y con pocos efectos adversos en la superficie ocular.}
