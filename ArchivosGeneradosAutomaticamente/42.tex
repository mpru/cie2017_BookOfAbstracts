\A
{PROPUESTA PARA LA IMPLEMENTACIÓN DEL SOFTWARE R EN LAS CARRERAS DE CONTADOR PÚBLICO Y LICENCIATURA EN ADMINISTRACIÓN DE LA FACULTAD DE CIENCIAS ECONÓMICAS Y ESTADÍSTICA DE LA UNIVERSIDAD NACIONAL DE ROSARIO}
{\Presenting{MARÍA EVANGELINA ÁLVAREZ}\index{ALVAREZ, M}, JUAN JOSÉ CÁMPORA\index{CAMPORA, J@CÁMPORA, J} y NORMA BEATRIZ MEROI\index{MEROI, N}}
{\Afilliation{FACULTAD DE CIENCIAS ECONÓMICAS Y ESTADÍSTICA, UNIVERSIDAD NACIONAL DE ROSARIO}
\\\Email{normameroi@outlook.com}}
{r commander; métodos estadísticos; licenciatura en administración; ntic} 
 {Enseñanza de la estadística} 
 {Otras categorías metodológicas} 
 {42} 
 {95-1}
{La sociedad actual, llamada de la información, demanda cambios en los sistemas educativos de forma que éstos se tornen más flexibles y accesibles, menos costosos y a los que han de poderse incorporar los ciudadanos en cualquier momento de su vida. Nuestras instituciones de formación superior, para responder a estos desafíos, deben revisar sus referentes actuales y promover experiencias innovadoras en los procesos de enseñanza-aprendizaje apoyados en las Tecnologías de la Información y la Comunicación (TIC). Y, contra lo que estamos acostumbrados a ver, el énfasis debe hacerse en la docencia, en los cambios de estrategias didácticas de los profesores, en los sistemas de comunicación y distribución de los materiales de aprendizaje, en lugar de enfatizar la disponibilidad y las potencialidades de las tecnologías. Las simulaciones interactivas, los recursos educativos digitales y abiertos, los instrumentos de recolección y análisis de datos son algunos de los muchos recursos que permiten a los docentes ofrecer a sus estudiantes posibilidades, antes inimaginables, para asimilar conceptos. En la asignatura Métodos Estadísticos y Estadística para Administradores de la Facultad de Ciencias Económicas y Estadística los docentes nos estamos replanteando nuestro quehacer en el aula y evaluando la factibilidad de implementar el uso de nuevas tecnologías mediante el programa R Commander en el dictado de las mismas. En una primera etapa del proyecto de investigación llevamos a cabo un relevamiento bibliográfico para conocer esta herramienta y las ventajas y desventajas en la implementación de este software y su uso en algunas facultades de nuestro país. Además proponemos un primer material de trabajo para confeccionar cuadros y gráficos estadísticos y realizar análisis descriptivo univariado para variables cuantitativas. Uno de los objetivos de nuestro trabajo de investigación es que los alumnos conozcan la herramienta y logren optimizar el tiempo; enfatizando más en las interpretaciones que en la realización de cálculos estadísticos extensos. }
