\A
{TOMA DE DECISIONES EN EL SECTOR SALUD: ANÁLISIS DEL TIEMPO DE ESTANCIA POR ACCIDENTES DE TRÁNSITO}
{\Presenting{MARÍA VIRGINIA PISANI}$^1$\index{PISANI, M}, FERNANDA VILLARREAL$^{1,2}$\index{VILLARREAL, F} y EUGENIA ELORZA$^{1,2,3}$\index{ELORZA, E}}
{\Afilliation{$^1$DEPARTAMENTO DE MATEMÁTICA- UNIVERSIDAD NACIONAL DEL SUR}
\Afilliation{$^2$INSTITUTO DE INVESTIGACIONES ECONÓMICAS Y SOCIALES DEL SUR (CONICET-UNS)}
\Afilliation{$^3$DEPARTAMENTO DE ECONOMÍA - UNIVERSIDAD NACIONAL DEL SUR}
\\\Email{fvillarreal@uns.edu.ar}}
{estadística inferencial; tiempo de estancia; accidentes de tránsito; gestión hospitalaria} 
 {Otras ciencias económicas, administración y negocios} 
 {Inferencia estadística} 
 {118} 
 {257-2}
{La variable duración de la estancia (length of stay (LOS)) representa el tiempo que un paciente permanece internado en una institución hospitalaria y constituye un instrumento clave de gestión ya que determina un alto porcentaje del gasto hospitalario. En las internaciones asociadas a los procesos de atención de pacientes que fueron víctimas de un accidente de tránsito el comportamiento de la variable duración de la estancia brinda información que puede ser de utilidad para predecir la evolución clínica del paciente como para la toma de decisiones que mejoren la gestión de los recursos hospitalarios afectados por estos procesos de atención y por este motivo, es sumamente relevante avanzar en el análisis estadístico. El objetivo de este trabajo es estudiar la distribución del tiempo de internación de pacientes que ingresaron por accidentes de tránsito a una institución hospitalaria en la ciudad de Bahía Blanca. Se utilizó una muestra de 121 casos que llegaron a dos hospitales públicos (uno provincial y otro municipal) entre el 22 de noviembre de 2014 al 22 de mayo 2015. Se realizó en primer lugar un análisis descriptivo y posteriormente un análisis inferencial utilizando los contrastes no paramétricos de Mann-Whitney y Kruskal-Wallis para probar si existen diferencias significativas en la distribución de la variable de interés según las variables sociodemográficas (edad, género, medio de locomoción, localidad y hospital) que pueden afectar el tiempo de permanencia en el hospital. Entre los principales resultados hallados se puede mencionar que: 1) la distribución del tiempo de internación presenta asimetría positiva en ambos hospitales; 2) existen diferencias significativas entre las distribuciones del tiempo de internación según el medio de locomoción utilizado por el accidentado, observándose que la permanencia en el hospital fue mayor para los casos en donde la víctima se transportaba en un auto sin cinturón de seguridad y 3) existen diferencias significativas para la variable en estudio de acuerdo a si el accidente se produjo dentro de la ciudad o si el paciente llego derivado de otra localidad de la región. Los resultados obtenidos en base a la metodología empleada, pueden ser relevantes tanto para la gestión hospitalaria como para el diseño de las políticas de seguridad vial, al mejorar la calidad de la información en base a la cual se toman las decisiones de asignación de recursos y se estiman los costos asociados a este problema de salud pública. }
