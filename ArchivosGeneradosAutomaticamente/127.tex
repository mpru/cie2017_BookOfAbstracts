\A
{ESTRUCTURA DE COVARIANZA ESTIMADA PARA EL MODELO ESTRUCTURAL ASIMÉTRICO DE CAUDALES DE DRENAJE}
{\Presenting{NÉLIDA SUSANA OZÁN}$^1$\index{OZÁN, N}, CLAUDIA DE LOS RÍOS$^2$\index{DE LOS RÍOS, C}, GRACIELA GARCÍA$^1$\index{GARCÍA, G}, EMMA MORALES$^1$\index{MORALES, E}, MARIANA GÓMEZ$^1$\index{GOMEZ, M@GÓMEZ, M} y MARIO FERNENDEZ$^1$\index{FERNENDEZ, M}}
{\Afilliation{$^1$FACULTAD DE INGENIERÍA. UNIVERSIDAD NACIONAL DE SAN JUAN}
\Afilliation{$^2$FACULTAD DE CIENCIAS EXACTAS FÍSICAS Y NATURALES. UNIVERSIDAD NACIONAL DE SAN JUAN}
\\\Email{sozan@unsj.edu.ar}}
{estructura de covarianza; modelo estructural; distribución asimétrica} 
 {Otras ciencias de la ingeniería y arquitectura} 
 {Modelos de regresión} 
 {127} 
 {284-1}
{La estimación del caudal de drenaje de una cuenca es de gran importancia, en la instancia del diseño, cuando se realiza la construcción de caminos de montaña. Una forma de estimar los caudales en la literatura tradicional de la ingeniería vial es utilizar el Método Racional, que no considera el aspecto aleatorio del fenómeno. Estudios previos muestran que las variables de mayor influencia en la determinación de caudales, por ejemplo la intensidad de las precipitaciones fluviales, son claramente sesgadas. Por este motivo, el ajuste del modelo estructural es una alternativa viable para este caso, asumiendo una distribución asimétrica (skwe normal) para los errores.  Se ha considerado la representación estocástica propuesta por Yuan y Bentler con la denominación de distribuciones asimétricas. La ventaja de esta forma de representación para las distribuciones permite modelar específicamente la estructura de covarianzas tanto del modelo estructural aditivo como multiplicativo, según corresponda. Por lo expuesto, en este trabajo, se estima la matriz de varianzas-covarianzas de resultados asintóticos para la distribución de los estimadores de los parámetros del modelo estructural asimétrico aditivo (Arellano-Valle, Ozán y Bolfarine, 2005). Esto permite probar la significancia de los parámetros del modelo y calcular los errores de estimación cometidos. El área de estudio donde se recolectó la información es el sector frontal de la Cordillera de Los Andes, San Juan (Argentina), contando con la colaboración de la Escuela de Caminos de Montaña (UNSJ). Los resultados muestran que, a pesar de haber asumido condiciones de regularidad para evitar problema de identificabilidad del modelo, se disminuyen los errores de estimación y se pueden plantear estudios de eficiencia asintótica. }
