\A
{DIFERENCIAS EN NIVELES DE COLESTEROL HDL ENTRE NIÑOS KOYAS Y DE ESTADOS UNIDOS}
{\Presenting{CLAUDIA MOLINARI}$^1$\index{MOLINARI, C}, CLAUDIO GONZALEZ$^2$\index{GONZALEZ, C}, GUSTAVO MACCALLINI$^3$\index{MACCALLINI, G}, MARIANA HIDALGO$^3$\index{HIDALGO, M}, VALERIA HIRSCHLER$^2$\index{HIRSCHLER, V} y SAN ANTONIO DE LOS COBRES STUDY GROUP COLLABORATORS$^4$}
{\Afilliation{$^1$FACULTAD DE FARMACIA Y BIOQUÍMICA-UBA}
\Afilliation{$^2$UBA}
\Afilliation{$^3$LABORATORIOS HIDALGO}
\Afilliation{$^4$UBA y Hospital San Antonio de los Cobres (Sata, Argentina)}
\\\Email{cmolinari@ffyb.uba.ar}}
{hdl; dislipidemia; nhanes; sac; prevalencia; regresión logística} 
 {Salud humana} 
 {Inferencia estadística} 
 {143} 
 {308-1}
{La dislipidemia es uno de los factores de riesgo cardiovascular en niños, sin embargo muy pocos estudios se llevan a cabo sobre poblaciones indígenas viviendo en condiciones muy diferentes a las urbanas. El presente estudio se basó en la observación de niños de la población de San Antonio de los Cobres (SAC), localizado a una altura promedio de 3750 m sobre el nivel del mar, con una temperatura media anual de menos de 10°C. Los habitantes de esta región muestran diferencias con los individuos de otras localidades situadas a menor altura debido a la hipoxia crónica asociada con la altitud. Un estudio llevado a cabo en USA sugiere que la reducción de la temperatura en alturas elevadas puede llevar a la pérdida de peso. Consistentemente con ello, un estudio previo sobre la misma población de SAC mostró que tienen una reducción significativa en los niveles de hemoglobina y en la prevalencia de obesidad, respecto de niños viviendo al nivel del mar. En este caso se aplicaron métodos estadísticos apropiados para comparar los niveles de HDL en estos niños y los niños americanos incluidos en el estudio del National Center for Health Statistics (NHANES), que constituye un estudio de referencia internacional para parámetros poblacionales. La muestra de NHANES es una muestral nacional representativa de la población norteamericana. La muestra en que se basa el presente estudio consistió de 1232 niños de la localidad de San Antonio de los Cobres, con edad entre 4 y 14 años, que fueron comparados estadísticamente con los 1451 niños americanos incluidos en NHANES 2011-2012. Se testeó la diferencia de medidas antropométricas, la prevalencia de sobrepeso y obesidad, y la prevalencia de HDL por debajo del valor admitido como normal. Se comparó también el nivel de HDL según categorías de MIC (masa de índice corporal). Se aplicó regresión logística a fin de evaluar la diferencia de riesgo de bajo HDL según origen del niño, controlando por variables antropométricas. A partir de los análisis realizados se concluyó que los niños de SAC resultan significativamente más delgados y bajos que los niños americanos. La prevalencia de HDL bajo es significativamente más alta que en sus pares de USA. Los niños de SAC tienen 9,5 veces más de chance de tener HDL por debajo del punto de corte considerado como normal, que los niños americanos, ajustando por posibles confusores.}
