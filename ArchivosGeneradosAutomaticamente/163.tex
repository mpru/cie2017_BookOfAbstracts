\A
{IDENTIFICACIÓN DE PERFILES DE LA VOZ EN ENFERMOS DE PARKINSON CON REDES NEURONALES}
{\Presenting{GASTÓN BERRETTA}$^1$\index{BERRETTA, G}, MONICA GIULIANO$^1$\index{GIULIANO, M}, EVANGELINA MALDONADO$^2$\index{MALDONADO, E} y GABRIEL BLANCO$^1$\index{BLANCO, G}}
{\Afilliation{$^1$DEPARTAMENTO DE INGENIERÍA E INVESTIGACIONES TECNOLÓGICAS. UNIVERSIDAD NACIONAL DE LA MATANZA}
\Afilliation{$^2$DEPARTAMENTO DE CIENCIAS DE LA SALUD. UNIVERSIDAD NACIONAL DE LA MATANZA}
\\\Email{mgiuliano@unlam.edu.ar}}
{enfermedad de parkinson; voz; pdrs; redes neuronales} 
 {Salud humana} 
 {Minería de datos} 
 {163} 
 {341-1}
{Vinculación entre parámetros de la voz y avance de la enfermedad en pacientes con enfermedad de Parkinson (EP). Si bien el diagnóstico definitivo es histopatológico, se considera que un diagnóstico preliminar en etapas tempranas resulta ser un desafío y una enorme responsabilidad, considerando las implicancias de pronóstico vital y las limitaciones motrices futuras para cada paciente diagnosticado. En esta dirección, se orientan múltiples investigaciones que hacen uso de Tecnologías de la Información y las Comunicaciones en el área de salud, en este caso se trata se trata de predecir el grado de avance de la EP a través de parámetros de la voz grabada de pacientes. A partir de una base pública de enfermos ya diagnosticados (747 observaciones) donde se ha grabado un fonema y se han paramatrizado matemática a través de 38 parámetros, de los cuales seleccionamos 36. A su vez el grado de enfermedad está incluido en la base según las respuestas a una versión reducida de la escala PDRS (Parkinsons Disease Rating Scale) y mediante el valor suma de esta escala, que dividimos en 2 categorías. En nuestro trabajo realizamos 3 tipos de análisis con Redes Neuronales. Se plantea un primer acercamiento al problema con una red neuronal perceptrón multicapa con la siguiente arquitectura: capa de entrada, capa de salida y dos capas ocultas Optimizando y revisando los parámetros, llegamos a una configuración que, utilizando la base de datos completa resultó en una clasificación correcta del 89,96\%. Luego, manteniendo la configuración, decidimos armar el modelo con un 50\% de los datos y evaluando el modelo con el restante 50\%, con lo que obtuvimos un 64,34\% bien clasificado. Por último armamos el mismo modelo en base a una validación cruzada de 10 iteraciones, obteniendo un 61.32\% correctamente clasificado. En base a estos resultados, realizaremos un análisis exploratorio para profundizar en los casos que fueron predichos correctamente, analizando las variables involucradas y descubriendo si existen patrones de comportamiento en estas variables.}
