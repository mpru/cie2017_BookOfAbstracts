\A
{LA HETEROCEDASTICIDAD CONDICIONAL EN LA INFLACIÓN DE LA ARGENTINA (1943 - 2013)}
{\Presenting{MARÍA DE LAS MERCEDES ABRIL}\index{ABRIL, M} y JUAN CARLOS ABRIL\index{ABRIL, J}}
{\Afilliation{UNIVERSIDAD NACIONAL DE TUCUMÁN. CONICET}
\\\Email{mabrilblanco@hotmail.com}}
{inflación; heterocedasticidad; volatilidad; series de tiempo} 
 {Economía} 
 {Series de tiempo} 
 {20} 
 {44-1}
{Numerosas series de tiempo económicas no tienen una media constante y en situaciones prácticas, frecuentemente vemos que la varianza del error observacional está sujeta a una sustancial variabilidad a través del tiempo. Ese fenómeno es conocido como volatilidad. Para tomar en cuenta la presencia de la volatilidad en una serie económica es necesario recurrir a modelos conocidos como modelos heterocedásticos condicionales. En estos modelos, la varianza de una serie en un dado instante de tiempo, depende de la información pasada y de otros datos disponibles hasta aquel instante de tiempo, de modo que se debe definir una varianza condicional, que no es constante y no coincide con la varianza global de la serie observada. Existe una variedad muy grande de modelos no lineales en la literatura, útiles para el análisis de series de tiempo económicas con volatilidad, pero nos concentraremos para analizar la serie de interés, en los modelos de tipo ARCH introducidos por Engle (1982) y sus extensiones. Estos modelos son no lineales en lo que se refiere a la varianza. Nuestro objetivo será el estudio de los datos mensuales de inflación de la Argentina para el período enero de 1943 a diciembre de 2013. Los datos son los publicados oficialmente por Instituto Nacional de Estadística y Censos (INDEC). Si bien es un período muy largo en el cual sucedieron cambios de base, cambios de canasta e intervenciones en el INDEC, se puede apreciar que ciertos patrones generales de comportamiento han persistido en el tiempo, lo que nos permite admitir que el estudio está adecuadamente basado en la información disponible. }
