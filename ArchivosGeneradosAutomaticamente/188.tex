\A
{ESTRATEGIAS METODOLÓGICAS PARA DATOS DE PANEL EL CASO DE LOS BANCOS TÍPICOS EN ARGENTINA}
{\Presenting{MARGARITA DÍAZ}\index{DIAZ, MARGARITA@DÍAZ, MARGARITA} y JOSÉ VARGAS\index{VARGAS, J}}
{\Afilliation{UNIVERSIDAD NACIONAL DE CÓRDOBA, FACULTAD DE CIENCIAS ECONÓMICAS}
\\\Email{mdiazlujan@gmail.com}}
{rentabilidad de bancos; datos de panel; driscall and kraay; errores estándar robustos} 
 {Economía} 
 {Modelos de regresión} 
 {188} 
 {904-1}
{El objetivo de este trabajo es analizar los efectos de indicadores específicos de los bancos en su rentabilidad, analizando para ello la información financiera de los bancos Típicos en Argentina para el período 2005-2015. En una primera etapa se clasificó a los bancos en diversos clusters mediante la aplicación del algoritmo K Medias robusto. A través del mismo se identificaron 4 clusters, caracterizados como Típicos, Otros Bancos, Comerciales y Personales. En este estudio se estimó el modelo para el grupo de 25 bancos considerados Típicos, de los cuales se excluyó uno en la estimación del modelo, dado su comportamiento atípico, detectado a través de un análisis exploratorio de los datos. La rentabilidad está medida a través del retorno sobre activo (ROA) y las variables predictoras son: Capital (Patrimonio Neto/Activo) , Riesgo Crediticio (Previsiones/Préstamos), Productividad (Ingresos Brutos/Gastos Administrativos), Gestión de Gastos (Gastos Administrativos/Activo) y Tamaño (logaritmo del Activo). En una primera etapa se planteó un Modelo de Regresión Lineal que fue estimado utilizando Ordinary Least Squares (OLS), lo que permitió detectar observaciones influyentes. En una segunda etapa se trabajó con los métodos que consideran la estructura jerárquica de la información, esto es: Modelo de Efectos Aleatorios y Modelo de Efectos Fijos, con errores estándares robustos y con la corrección de Driscall and Kraay. A fin de seleccionar el método de estimación apropiado, se realizaron las pruebas de Hausman, de heterocedasticidad y de correlación transversal entre bancos. Los resultados indican que el modelo de efectos fijos con errores estandares con la corrección de Driscall and Kraay es el más apropiado para la estimación del modelo. Todos los determinantes, con la excepción de Gestión de Gastos, afectan significativamente la rentabilidad.}
