\A
{COMPARACIÓN DE ALGORITMOS DE CLASIFICACIÓN NO SUPERVISADA PARA EL ESTUDIO DE COBERTURAS FORRAJERAS EXTENSIVAS}
{\Presenting{DIEGO CARCEDO}\index{CARCEDO, D} y MIGUEL NOLASCO\index{NOLASCO, M}}
{\Afilliation{CÁTEDRA DE ESTADÍSTICA Y BÍOMETRIA - FACULTAD DE CIENCIAS AGROPECUARIAS - UNIVERSIDAD NACIONAL DE CÓRDOBA}
\\\Email{dncarcedo@gmail.com}}
{k-means; upgma; imágenes satelitales; google earth engine} 
 {Ciencias agropecuarias} 
 {Estadística espacial} 
 {107} 
 {234-1}
{El conocimiento de los tipos de cobertura de vegetación en tierras destinadas a la producción animal extensiva permite una mejor administración de la biomasa forrajera disponible. Tradicionalmente la caracterización de la oferta forrajera se realiza en base a muestreos a campo. En los últimos años, el acceso a datos provenientes de sensores remotos y a servicios para su procesamiento en línea, permiten obtener series temporales de índices de vegetación para múltiples sitios en el dominio de estudio. El objetivo del presente trabajo fue comparar el comportamiento de los métodos de conglomerados K-means (no jerárquico) y UPGMA (jerárquico) en el contexto de datos geoespaciales temporales. El área de estudio (noroeste de la Provincia de Córdoba), abarcó una superficie rectangular de 407 km2. En primera instancia, utilizando el algoritmo de segmentación denominado Meanshift, se confeccionaron objetos espaciales que respetaran la homogeneidad de las distintas coberturas presentes en el terreno. Empleando la plataforma web Google Earth Engine (GEE) se construyó una serie temporal del producto Índice de Vegetación Diferencial Normalizado (IDVN) (2013-2015 con 2 o más índices por mes). Las colecciones de imágenes satelitales Landsat 7, Landsat 8 y Sentinel 2, fueron usadas para cada objeto y tipo de cobertura. Los valores de la serie temporal resultante fueron resumidos estadísticamente (media y DE) por año y estación. Posteriormente se calculó una matriz de distancias entre los objetos, utilizando la métrica Euclídea, sobre estas propiedades de los datos. Esta matriz de distancia fue el input de los algoritmos K-means y UPGMA. Para identificar el número apropiado de grupos existentes en la estructura de los datos, se usaron los índices Ancho de silueta (IAS), Dunn (ID) y Conectividad (IC). Los valores obtenidos evidenciaron que ambos algoritmos identificaron una partición apropiada en dos grupos (K-Means: 0,76; 0.98; 1 y UPGMA: 0,76; 0.98; 1 para IAS, ID y IC respectivamente). Este agrupamiento puede explicarse por la diferencia en el patrón temporal de la serie de datos. Una de las clases presentó valores estables y periódicos en el tiempo, mientras que la restante exhibió una mayor inestabilidad. El uso de propiedades estadísticas de ventanas del tamaño de una estación en series de NDVI permitió contemplar la estructura temporal subyacente en los datos.}
