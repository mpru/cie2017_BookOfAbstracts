\A
{EL VALOR DE EDUCARSE EN ARGENTINA UN ANÁLISIS TEMPORAL}
{\Presenting{PAULA LORENA NAIDICZ}\index{NAIDICZ, P}, GABRIEL AMÓS FRIDRIJ\index{FRIDRIJ, G} y VÍCTOR FABIO LAZARTE\index{LAZARTE, V}}
{\Afilliation{UNIVERSIDAD NACIONAL DE TUCUMÁN- FACULTAD DE CIENCIAS EXACTAS Y TECNOLOGÍA}
\\\Email{lorenafridrij@hotmail.com}}
{nivel educativo; ingresos; brecha de ingresos; análisis temporal} 
 {Economía} 
 {Inferencia estadística} 
 {83} 
 {184-1}
{El valor de educarse no abarca solamente los aspectos económicos y laborales de la vida de las personas, sino que nos forma para llevar una vida mejor en general, para uno mismo y para la sociedad. Una de las maneras de cuantificar la influencia del estudio en la calidad de vida de las personas es a través de un análisis del ingreso. Existen muchos trabajos en donde se evidencia que una mejor educación contribuye a tener mejores ingresos. El presente trabajo muestra que, en realidad, esta relación fue disminuyendo con el tiempo. Es decir si bien hay un efecto positivo entre la educación y el ingreso, éste es cada vez menor. Estetrabajo presenta un análisis temporal con datos de la Encuesta Permanente de Hogares desde 2003 a la fecha, que analiza las relaciones existentes entre el nivel educativo e ingresos mensuales considerando la población económicamente activa. Los simples cocientes de los ingresos entre los diferentes niveles educativos, en cada trimestre, representan la cantidad de veces que el salario del nivel más bajo está contenido en el nivel más alto, por ejemplo si uno hace el cociente entre el ingreso mediano del secundario completo con primaria completa y el resultado es 3, quiere decir que el ingreso mediano del secundario completo es el triple que el del primaria completa. Si el cociente diera próximo a uno, significa que no hay diferencias entre completar uno u otro nivel. Este análisis permite ver que los ingresos son superiores para niveles educativos más altos en un instante de tiempo por aglomerado, pero el análisis de estos a través del tiempo, demuestra una tendencia decreciente significativa. }
