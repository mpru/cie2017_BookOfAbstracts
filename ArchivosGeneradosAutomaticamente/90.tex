\A
{RECONOCIMIENTO DE TEXTURAS EN UNA IMAGEN POLARIMÉTRICA BASADO EN DISTANCIAS ESTOCÁSTICAS}
{\Presenting{SILVINA PISTONESI}$^1$\index{PISTONESI, S}, JORGE MARTINEZ$^2$\index{MARTINEZ, J}, ÉRICA SCHLAPS$^3$\index{SCHLAPS, E} y ALEJANDRO FRERY$^4$\index{FRERY, A}}
{\Afilliation{$^1$UNIVERSIDAD NACIONAL DEL SUR Y TECNOLÓGICA NACIONAL REGIONAL BAHÍA BLANCA}
\Afilliation{$^2$UNIVERSIDAD NACIONAL DEL SUR}
\Afilliation{$^3$UNIVERSIDAD NACIONAL DE LA PAMPA}
\Afilliation{$^4$LABORATÓRIO DE COMPUTAÇÃO CIENTÍFICA E ANÁLISE NUMÉRICA. UNIVERSIDADE FEDERAL DE ALAGOAS}
\\\Email{lpistone@criba.edu.ar}}
{imagen polarimétrica; texturas; distancias estocásticas; estimador de máxima verosimilitud} 
 {Otras aplicaciones} 
 {Otras categorías metodológicas} 
 {90} 
 {202-1}
{Las imágenes de Radar de Apertura Sintética (Synthetic Aperture Radar — SAR) tanto monopolarizadas como polarimétricas, son fundamentales para el estudio y caracterización de la zona observada, dado que a partir de ellas puede obtenerse información que ningún otro tipo de sensor provee. Sin embargo, estas imágenes tienen la desventaja de que son difíciles de analizar e interpretar debido, entre otras razones, a la presencia de ruido speckle, que es inherente al proceso de captura y formación de imágenes con iluminación coherente. La extracción de características de textura, intensidad, posición espacial, color, granularidad, etc., de una imagen constituye una etapa previa para el procesamiento automático de imágenes relacionado con la segmentación, la clasificación, y la detección e identificación de objetos. Este trabajo pretende evaluar la capacidad de las distancias estocásticas en el reconocimiento de diferentes texturas (distintos tipos de cobertura terrestre) presentes en una imagen polarimétrica. Para el estudio, se selecciona una subimagen de 100x512 píxeles, de la imagen polarimétrica de 4-looks de la Bahía de San Francisco, que presenta tres texturas bien diferenciables correspondientes a bosque, zona urbana, y agua. Inicialmente se efectúa un análisis cuantitativo descriptivo de estos tres conjuntos de datos de diferentes texturas. Hecho esto, suponemos que los datos, que son de polarización completa, obedecen distribuciones Wishart Complejas. Con el fin de discriminar las regiones de interés, de cada una de ellas se seleccionan dos muestras de tamaño 20x20 píxeles y se calculan las distancias estocásticas de Kullback-Leibler (dKL), Bhattacharyya (dB) y la de Hellinger (dH), intra y entre los conjuntos de datos. Para ello fue necesario estimar previamente los parámetros de las distribuciones Wishart de dichos conjuntos de datos mediante el método de Máxima Verosimilitud. El análisis descriptivo de las distintas texturas refleja una intensa asimetría a derecha y una gran presencia de datos que no se explican bien por una ley gaussiana (outliers). En todas las comparaciones de texturas consideradas, los valores de las distancias respetaron el siguiente orden: dKL $\leq$ dB $\leq$ dH. Las distancias dKL y dB permiten discriminar con éxito las diferentes, alcanzando valores menores entre áreas cuyo grado de rugosidad es relativamente similar. En cambio, la distancia dH evidencia ausencia de poder discriminativo de las diferentes rugosidades de las texturas bajo estudio.}
