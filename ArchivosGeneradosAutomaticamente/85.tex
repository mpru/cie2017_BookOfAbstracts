\A
{ANÁLISIS DIALELO MULTIAMBIENTAL APLICADO EN R}
{\Presenting{JULIA INÉS FERNÁNDEZ}$^1$\index{FERNÁNDEZ, J}, DIEGO MARFETÁN MOLINA$^2$\index{MARFETÁN MOLINA, D} y MARÍA LUJÁN FARACE$^{1,3}$\index{FARACE, M}}
{\Afilliation{$^1$UNIVERSIDAD NACIONAL DEL NOROESTE DE LA PROVINCIA DE BUENOS AIRES}
\Afilliation{$^2$UNIVERSIDAD NACIONAL DE ROSARIO}
\Afilliation{$^3$COMISIÓN DE INVESTIGACIÓN CIENTÍFICA DE LA PROVINCIA DE BUENOS AIRES (CIC)}
\\\Email{diego.marfetan@gmail.com}}
{dialelo; griffing; multiambiente; r; diseños genéticos; anova} 
 {Genética} 
 {Diseño de experimentos} 
 {85} 
 {187-1}
{Los diseños dialélicos de Griffing (1956) son de gran utilidad para evaluar diferentes aspectos genéticos asociados con las cruzas, que permiten estimar la aptitud combinatoria, importante en el mejoramiento genético cuando se desea comparar el comportamiento de líneas endocríadas con diferentes grados de diversidad genética en combinaciones híbridas. Griffing presentó cuatro métodos en función de la inclusión de los parentales, sus cruzas directas F1 y sus recíprocos, y dos tipos de modelos: de efectos fijos o aleatorios. Singh (1973) desarrolló las sumas de cuadrados para evaluar estos efectos a través de distintos ambientes. Dicho análisis multiambiental está implementado únicamente en programas pagos y en el software AGD desarrollado por el CIMMYT. El objetivo del presente trabajo es construir una función en R que replique los resultados de dichos programas, sea de fácil acceso y pueda ser empleada en programas de mejoramiento genético. Para evaluar su implementación, esta función se aplicó a datos resultantes de un experimento dialélico Método II, Modelo 1 de Griffing entre 5 líneas de maíz (EEA INTA Pergamino) y sus cruzas F1 directas mediante un diseño en bloques anidados en localidades (Pergamino, Junín y Colón). Las variables respuestas consideradas fueron contenido de etanol y proteína, medidas con el método NIRS. Tomando como referencia los resultados proporcionados por AGD, se programó una función en R que permite realizar el análisis dialélico en el contexto de múltiples localidades. Esta función produce como resultados la tabla ANOVA correspondiente, estimaciones de las componentes de habilidades combinatorias general y específica, de la variancia fenotípica y de las heredabilidades en sentido amplio y estricto; además, proporciona estimaciones de los efectos generales y específicos acompañados del ranking de cada parental y cada cruza. Como novedad, se le sumó la opción de indicar el coeficiente de endocría (F) y estimar los parámetros genéticos de acuerdo al valor de F establecido. La implementación en R resultó satisfactoria, logrando reproducir los resultados obtenidos mediante otros paquetes estadísticos. Entre las ventajas de la función propuesta se destacan las posibilidades de aplicar el método a diseños que abarquen varias localidades y de elegir el coeficiente de endocría, opciones hasta ahora no disponibles en R. Si bien actualmente la función puede aplicarse sólo a modelos contemplados por el Método 2 de Griffing, se planea extenderla de forma tal que contemple los restantes diseños.}
