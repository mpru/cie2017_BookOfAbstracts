\A
{CÁLCULO DE PROBABILIDADES DE INCLUSIÓN PARA MUESTREO SISTEMÁTICO CON PROBABILIDAD PROPORCIONAL AL TAMAÑO CON REEMPLAZO DE UNIDADES}
{\Presenting{GONZALO MARÍ}$^1$\index{MARI, G@MARÍ, G}, LUCÍA HERNÁNDEZ$^1$\index{HERNÁNDEZ, L}, GERARDO MITAS$^2$\index{MITAS, G} y GABRIELA BARBARÁ$^3$\index{BARBARÁ, G}}
{\Afilliation{$^1$INSTITUTO DE INVESTIGACIONES TEÓRICAS Y APLICADAS DE LA ESCUELA DE ESTADÍSTICA, UNIVERSIDAD NACIONAL DE ROSARIO}
\Afilliation{$^2$DIRECCIÓN NACIONAL DE METODOLOGÍA ESTADÍSTICA, INDEC}
\Afilliation{$^3$COORDINACIÓN DE MUESTREO, INDEC}
\\\Email{mari.gonzalo@gmail.com}}
{muestreo ppt sistemático; mmuvra; probabilidades de inclusión; reemplazo de unidades} 
 {Estadísticas oficiales} 
 {Muestreo} 
 {1} 
 {1-6}
{La Muestra Maestra Urbana de Viviendas de la República Argentina (MMUVRA) es un instrumento a partir del cual es posible seleccionar muestras para los diversos operativos que se llevan a cabo en el Instituto Nacional de Estadística y Censos (INDEC). El diseño muestral consiste en uno con 2 etapas de selección. En la primera de ellas, las unidades primarias de muestreo están constituidas por los aglomerados urbanos que son seleccionados con un muestreo sistemático con probabilidad proporcional a la población total. En cada uno de los aglomerados seleccionados, se definen las unidades secundarias de muestreo como áreas, que en la mayoría de los casos equivalen a un radio censal y se seleccionan con un muestreo sistemático con probabilidad proporcional a la cantidad de viviendas del área. Cada una de las áreas seleccionadas son listadas y las viviendas que forman parte de las mismas constituyen el MMUVRA a partir de la cual se pueden seleccionar muestras para diversos operativos. Para cada uno de ellos se seleccionan viviendas, evitando la repetición de las mismas para no ocasionar un cansancio por parte de los respondentes que produciría un aumento en la no respuesta. Ante el agotamiento de una de las áreas por haber utilizado todas las viviendas, una de las soluciones que es la que utiliza el INDEC es reemplazar el área agotada por una nueva. Esto posee el inconveniente que la selección del reemplazo rompe con el diseño muestral planteado en la primera selección. Bajo esta situación, el cálculo de las probabilidades de inclusión de primer orden de las áreas seleccionadas resulta ser dificultoso. Una las soluciones posibles es realizar dicho cálculo a través de simulaciones. Thompson y Wu (2008) proponen un método basado en simulaciones que permite calcular las probabilidades de inclusión ante la ocurrencia de reemplazos para muestreo sistemático aleatorio con probabilidad proporcional al tamaño (PPT). El mismo se basa en la repetición del método de selección y del esquema de reemplazo un número grande de veces (por ejemplo, $1x10^6$) y en base a los mismos, calcular las probabilidades de inclusión. La diferencia en el caso de la MMUVRA, es que el método de selección no es sistemático aleatorio PPT, sino sistemático PPT, el cual posee soporte mínimo (Pea, Qualité, Tillé, 2007). Se plantea una solución similar que se ajuste a la situación actual, y se consideran estudios futuros con soluciones alternativas.}
