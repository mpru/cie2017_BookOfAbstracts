\A
{ESTIMACION DEL COMPORTAMIENTO ASIMETRICO DE LOS COSTOS CON UN MODELO LINEAL MIXTO COMPARACION DE EMPRESAS ARGENTINAS, CHILENAS, PERUANAS Y COLOMBIANAS}
{\Presenting{MARIA INES STIMOLO}\index{STIMOLO, M} y MAXIMILIANO IGLESIAS\index{IGLESIAS, M}}
{\Afilliation{FACULTAD DE CIENCIAS ECONOMICAS}
\\\Email{mstimolo@eco.unc.edu.ar}}
{datos longitudinales; modelos mixtos; costos asimétricos; economías diferentes} 
 {Otras ciencias económicas, administración y negocios} 
 {Modelos de regresión} 
 {176} 
 {353-3}
{En este trabajo se propone un modelo mixto para cuantificar el incremento (o disminución) de los gastos por cada incremento (o disminución) porcentual de los ingresos por venta de las empresas extendiendo el modelo de regresión log lineal por partes propuesto por Anderson et.al. (2003). Con este modelo se verifica empíricamente la falta del supuesto de proporcionalidad simétrica entre los costos y el nivel de ventas, sobre el cual se basan los modelos de costos a corto plazo y los métodos de costeo. El modelo lineal mixto permite predecir el comportamiento para cada empresa cuando los coeficientes aleatorios son significativos. El análisis se realiza sobre una muestra de empresas argentinas, chilenas, peruanas y colombianas que hacen oferta pública de sus acciones durante el período 2004-2012. El modelo de comportamiento asimétrico de los costos está basado en la teoría de las decisiones gerenciales en relación al ajuste de los costos de las empresas, siendo estas decisiones afectadas por, las expectativas económicas,entre otros. La empresas analizadas corresponden a países con ambientes y expectativas macroeconómicos diferentes. Además, todas estas economías fueron de una manera u otra afectadas por la crisis del año 2008 por lo que se incorporan en el análisis los períodos 2004-2007 y 2008-2012, encontrando diferencias en el comportamiento de los costos de las empresas relacionados con el contexto económico de cada uno. Las empresas se clasificaron en sectores, los que fueron incorporados al modelo como una manera de incorporar estructura de costos diferentes. Los sectores tienen características diferentes en los países analizados por lo que el análisis comparativo resulta más complejo.}
