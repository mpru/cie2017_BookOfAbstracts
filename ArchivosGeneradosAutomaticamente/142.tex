\A
{EFECTOS DE LAS TRANSFERENCIAS CONDICIONADAS DE INGRESO SOBRE LA PARTICIPACIÓN LABORAL DE LOS ADULTOS: EL CASO DE LA AUH EN ARGENTINA}
{\Presenting{MARIANA HEREDIA}\index{HEREDIA, M} y GABRIEL WEIDMANN\index{WEIDMANN, G}}
{\Afilliation{FACULTAD DE CIENCIAS ECONÓMICAS - UNER}
\\\Email{meriheredia@hotmail.com}}
{participación de la fuerza laboral; transferencias de ingresos; evaluación de políticas; estimador diferencias en diferencias} 
 {Economía} 
 {Modelos de regresión} 
 {142} 
 {306-1}
{En el año 2009 Argentina puso en marcha el programa Asignación Universal por hijo (AUH) para Protección Social, el cual universaliza las asignaciones familiares a los segmentos de la población más vulnerables no amparados por el Régimen de Asignaciones Familiares. Los objetivos de este programa son dos: incrementar el bienestar del sector más pobre de la sociedad y fomentar la acumulación de capital humano en los niños. En general, existe una vasta evidencia en la literatura que las Transferencias Condicionadas de Ingreso son exitosas en América Latina en cuanto a la reducción de la pobreza y producir mejoras en la distribución de ingreso. Sin embargo, existen otros aspectos que motivan debates, principalmente, el impacto de este tipo de políticas sobre la actividad laboral de los adultos en los hogares destinatarios. Tanto el modelo estándar de oferta laboral como el modelo de oferta laboral familiar, predicen que las transferencias de ingresos no laborales, producen desincentivos al trabajo dado el efecto ingreso de las transferencias. En nuestro país, la evidencia en este campo es aún escasa y en diferentes direcciones. Para investigar estos posibles efectos no deseados, se plantea evaluar los efectos de la asignación universal por hijo en la participación laboral de los adultos, la tasa desempleo y tasa de actividad, en el corto y mediano plazo, con el fin de realizar un aporte al diseño de una política pública más efectiva de protección social. La metodología aplicada es el estimador de diferencias en diferencias, combinado con un modelo logit ya que en los diferentes modelos, la respuesta es discreta. Se utilizan los microdatos de la encuesta anual que realiza el Observatorio Social de la Universidad Nacional del Litoral en la ciudad de Santa Fe en el periodo 2009-2012. Para ello, se comparan dos grupos de adultos, uno que declara ser beneficiario de la AUH en la encuesta, con otro de características muy similares pero que no recibe el beneficio. Los resultados sugieren que ni en el corto ni mediano plazo, la AUH tiene efectos no deseados en el mercado laboral.}
