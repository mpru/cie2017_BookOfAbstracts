\A
{ESTUDIO COMPARATIVO DEL TEGUMENTO DE TRES VARIEDADES DE ARVEJAS COMERCIALIZADAS EN LA CIUDAD DE HUMAHUACA. ANÁLISIS ESTADÍSTICO PARA SU DIFERENCIACIÓN MICROSCÓPICA}
{\Presenting{JUDITH MONTENEGRO BRUSOTTI}$^{1,2}$\index{MONTENEGRO BRUSOTTI, J}, GRACIELA BASSOLS$^1$\index{BASSOLS, G} y MYRIAM NUÑEZ$^2$\index{NUÑEZ, M}}
{\Afilliation{$^1$CÁTEDRA DE FARMACOBOTÁNICA, FACULTAD DE FARMACIA Y BIOQUÍMICA, UNIVERSIDAD DE BUENOS AIRES}
\Afilliation{$^2$CÁTEDRA DE MATEMÁTICA, FACULTAD DE FARMACIA Y BIOQUÍMICA, UNIVERSIDAD DE BUENOS AIRES}
\\\Email{jmontenegro@ffyb.uba.ar}}
{arveja; tegumento; esclereida; análisis estadístico} 
 {Otras aplicaciones} 
 {Inferencia estadística} 
 {138} 
 {304-3}
{En la ciudad de Humahuaca, localizada en la Provincia argentina de Jujuy, se encuentran comercializándose a granel tres variedades de arvejas. Macroscópicamente son bien diferentes, pero a nivel microscópico la diferencia es válida si se realizan métodos micrométricos y estadísticos para comparar largo y ancho de algunas esclereidas que componen el tegumento de las semillas. La necesidad de una identificación a nivel microscópico surge de que su uso y/ o comercialización suele estar en forma trozada o de harinas. El objetivo del presente trabajo es poder identificar, a través del la diferenciación microscópica del tegumento la variedad de la cual proviene la arveja. La metodología empleada para la preparación del material en estudio consiste en hidratar las arvejas, retirar los tegumentos y realizar con ellos un disociado fuerte para poder separar las células. Se observa bajo microscopio óptico y con un ocular graduado se mide el tamaño de las esclereidas. El resultado del análisis mostró 2 tipos de esclereidas (esclereidas columnares y osteoesclereidas) que se encontraban en las tres variedades de arvejas, algunas con un amplio rango de tamaño que hace necesario un tratamiento estadístico para validar diferencias. Al realizar el análisis estadístico (Análisis de la varianza) no se evidencia diferencia significativa entre las tres variedades de arvejas, columnar pero si se obtienen diferencias significativas entre las tres variedades del otro tipo celular, las osteoesclereidas. Se concluye a partir de este resultado que ante una presentación comercial de harina o pequeños trozos de arveja, es posible determinar la variedad de la que proviene.}
