\A
{CONCIENCIA DE DAÑO DEL HÁBITO TABÁQUICO EN FAMILIARES DE PACIENTES CON CÁNCER}
{\Presenting{MYRIAM NUÑEZ}$^1$\index{NUÑEZ, M}, FERNANDA RUSSO$^1$\index{RUSSO, F}, JOSÉ MARÍA LASTIRI$^2$\index{LASTIRI, J} y GUADALUPE PALLOTTA$^2$\index{PALLOTTA, G}}
{\Afilliation{$^1$UBA. FACULTAD DE FARMACIA Y BIOQUÍMICA. CÁTEDRA DE MATEMÁTICA}
\Afilliation{$^2$HOSPITAL ITALIANO DE BUENOS AIRES}
\\\Email{myriam@ffyb.uba.ar}}
{factores de riesgo; hábito tabáquico; conciencia de daño; métodos multivariados} 
 {Salud humana} 
 {Métodos multivariados} 
 {136} 
 {304-1}
{El objetivo del estudio es comparar los resultados obtenidos entre el Grupo de pacientes (GP) y el Grupo de familiares (GF) en pacientes con neoplasias asociadas y no asociadas al hábito tabáquico. Hay aspectos relacionados al consumo de tabaco que son fundamentales, especialmente si uno plantea la posibilidad de desarrollar estrategias contra esta adicción. Para este análisis se realizó una encuesta en el Hospital Italiano de Buenos Aires (HIBA) a 390 familiares de pacientes con diagnóstico de cáncer y a 350 pacientes. La misma fue efectuada tanto al paciente como al familiar por separado. Se registraron los siguientes datos, en ambos grupos; Sexo, Edad, Factores de riesgo: hábito de fumar, frecuencia de cigarrillos, tiempo que lleva fumando, Características del hábito, Conocimiento acerca de la enfermedad, Deseos de cesación, Conciencia de daño. Se observó que en el GF el consumo es levemente inferior al conocido en nuestro medio (29,3\% fumadores; 15,6\% ex fumadores y el 55,1\% no fumadores), y, seguramente relacionado con el mayor número de mujeres en este grupo (71,1\%). El 17 \% afirmó que no cesaría de fumar, aunque esto ayudara a su familiar, lo que sugiere más claramente el desconocimiento y un profundo grado de adicción al tabaco. En el GF su conocimiento parece ser algo mayor y sin embargo el consumo sigue siendo semejante al del GP. La aplicación de metodología estadística multivariada como, Análisis Factorial de Correspondencias, Regresión Logística y Árboles de Clasificación, nos permitió tener una idea del comportamiento de todos los factores de riesgo en su conjunto. Se concluyó que los mecanismos psicológicos previos, (específicamente, la falta de reconocimiento de una relación causal entre el tabaco y la patología que lo afecta), relacionados con la adicción, o reactivos a la situación de enfermedad, le impiden tanto al paciente como al familiar solucionar tal adicción. }
