\A
{IMPUTACIÓN DE DATOS FALTANTES DEL CENSO DE POBLACIÓN Y VIVIENDA UTILIZANDO TÉCNICAS DE ESTADÍSTICA ESPACIAL}
{MARÍA EUGENIA RIAÑO\index{RIAÑO, M}}
{\Afilliation{SUE}
\\\Email{rianomiranda@gmail.com}
}
{imputación de datos faltantes; modelos sar autorregresivos; validación cruzada; árboles de clasificación} 
 {Estadísticas oficiales} 
 {Estadística espacial} 
 {156} 
 {333-1}
{En general, la calidad y cobertura de los Censos de Población y Vivienda del año 2011 realizado en Uruguay, fue calificada como positiva, cumpliendo con los estándares exigidos internacionalmente. Sin embargo, su implementación no estuvo exenta de inconvenientes. No se cuenta con información de determinados hogares cuyo domicilio fue relevado, y para algunos se cuenta con sólo información parcial relativa a la composición del hogar. La omisión censal se concentra en zonas socioeconómicamente vulnerables. Esto afectaría la construcción del mecanismo utilizado por el Ministerio de Desarrollo Social para seleccionar a la población beneficiaria de los programas de transferencia monetaria. Este mecanismo se basa en la Encuesta Continua de Hogares cuyo marco muestral es el del Censo, y refleja los problemas de omisión. El trabajo se desarrolla para la ciudad de Montevideo. El patrón espacial de la población objetivo y de la propia omisión hace necesaria una regionalización previa a la imputación, dado que la distribución espacial se muestra heterogénea en el mapa. La selección de los modelos a utilizar para la imputación es muy sensible a la escala del mapa, por lo que la definición de las regiones condiciona la selección del modelo final a utilizarse para realizar la imputación. Las regiones se construyen mediante el algoritmo de árboles oblicuos de decisión, implementado en el paquete SPODT de R. Se ajustan modelos autorregresivos espaciales en cada región (SAR) que son evaluados con métodos de validación cruzada. Las imputaciones se realizan como predicción de los modelos mejores evaluados. Los resultados muestran una mayor omisión de hogares socioeconómicamente vulnerables, lo que impacta directamente en la selección de población beneficiaria de programas de transferencia monetaria. }
