\A
{ANÁLISIS DESCRIPTIVO LONGITUDINAL DE LA FRECUENCIA DE INTIMIDACIÓN ENTRE PARES Y SUS PRINCIPALES CARACTERÍSTICAS EN ESCOLARES DE BAHÍA BLANCA}
{\Presenting{MARÍA GABRIELA SERRALUNGA}\index{SERRALUNGA, M}, CECILIA VASCONI\index{VASCONI, C} y LUCAS DURÁN\index{DURÁN, L}}
{\Afilliation{UNIVERSIDAD NACIONAL DEL SUR}
\\\Email{mgserra@criba.edu.ar}}
{intimidación entre pares; encuesta preconcimei; escolares} 
 {Salud humana} 
 {Otras categorías metodológicas} 
 {124} 
 {274-2}
{INTRODUCCIÓN: El objetivo del trabajo es estimar longitudinalmente la frecuencia de la intimidación entre pares, según la percepción de los alumnos, y analizar sus principales características. METODOLOGÍA: Estudio longitudinal descriptivo realizado en niños de 8 a 12 años, de siete escuelas primarias de gestión estatal y privada de la ciudad de Bahía Blanca, seleccionadas mediante un muestreo por conveniencia, durante mayo (Toma 1) y noviembre (Toma 2) de 2016. Instrumento de recolección de datos: Cuestionario auto-administrado “Pre-Concepciones de Intimidación y Maltrato entre Iguales” (PRECONCIMEI, Avilés, 2002). Definición de roles según la participación del niño en situaciones de intimidación, recibiendo maltrato, ejerciéndolo, o ambos: Participación sostenida si manifestó haber participado en ambas tomas, Participación episódica si indicó participar en sólo una, Sin participación activa si manifestó no haber participado en ninguna toma. Las características de la intimidación se analizaron sólo en la primera toma. Programa utilizado SPSS 17. RESULTADOS: De los 458 niños/as encuestados, un 30,3\% y un 35,2\% refirieron participar en situaciones de intimidación en mayo y noviembre, respectivamente. El análisis longitudinal reveló que un 17,9\% participó sostenidamente, un 29,6\% participó episódicamente y el 52,4\% restante no participó activamente en ninguna de las tomas. Se encontraron diferencias significativas al comparar estos porcentajes a través del tiempo (años 2012, 2014, 2015 y 2016). En cuanto a sus características, el tipo de intimidación más frecuente en niños fue “Insultar y poner apodos” (58,2\%) y en niñas “Hablar mal de alguien”(48,1\%); el lugar donde ocurren con mayor frecuencia fue “En el patio cuando no vigilaba ninguna maestra” (14,4\%); para ambos sexos el interlocutor válido a quien recurren fue “la familia” (28,2\%); quien más frecuentemente suele parar estas situaciones fue “alguna maestra” (42\%); respecto a porqué lo hacen, la mayoría (53,2\%) dijo “por molestar”; qué hicieron cuando observaron tales situaciones, “Nada, no me metí” (26,2\%), y la respuesta más frecuente a qué habría que hacer para solucionar el problema, fue “Que hagan algo las maestras” (34,3\%). CONCLUSIÓN: La observación en dos momentos del ciclo escolar permitió detectar que la mayoría de los niños no participa de estas situaciones en forma activa en ningún momento del año, un 30\% cambia de perfil, entrando o saliendo de tales situaciones con un patrón dinámico, y un 18\% participa en forma sostenida. El conocimiento de esta dinámica es de utilidad tanto para el diseño de intervenciones escolares, como para el abordaje preventivo de la problemática. }
