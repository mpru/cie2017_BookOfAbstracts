\A
{PROPIEDAD DE MUESTREO DE ÍNDICES DE SIMILITUD BASADOS EN DATOS BINARIOS}
{\Presenting{PATRICIA S. TORRES}$^1$\index{TORRES, P}, MARTA B. QUAGLINO$^2$\index{QUAGLINO, M} y SERGIO CAMIZ$^3$\index{CAMIZ, S}}
{\Afilliation{$^1$CÁTEDRA DE ECOLOGÍA, FACULTAD DE CIENCIAS AGRARIAS (UNR)}
\Afilliation{$^2$ESCUELA DE ESTADÍSTICA, FACULTAD DE CIENCIAS ECONÓMICAS Y ESTADÍSTICA (UNR)}
\Afilliation{$^3$UNIVERSIDAD LA SAPIENZA, ROMA (ITALIA)}
\\\Email{patrizia662@gmail.com}}
{índices; similitud; muestreo; binarios; aleatorización} 
 {Ecología y medio ambiente} 
 {Métodos multivariados} 
 {185} 
 {901-1}
{Los índices de similitud basados en datos binarios han sido ampliamente usados en Ecología de comunidades, ya sea, para agrupar las unidades experimentales como las especies de una comunidad biótica. Índices similares o idénticos fueron aplicados subsecuentemente en otras disciplinas biológicas. Como resultado del esfuerzo por cuantificar la asociación o similitud en varios campos de la Biología, han aparecido una gran cantidad de medidas. Se encuentran actualmente en la literatura más de 60 índices de similitud, aunque algunos son simplemente funciones de otros. Existen varias propiedades que deben cumplir los índices de similitud para ser considerados “buenos índices”. Entre ellas se encuentra una que se refiere a la propiedad de muestreo de los mismos, por lo tanto el objetivo de este trabajo es evaluar la propiedad de muestreo de 61 índices de similitud encontrados en la literatura, a partir de procedimientos de simulación. Para probar esta propiedad se siguió un esquema similar al utilizado por Goodall en 1973, se consideraron 5 escenarios con distintas probabilidades de presencia para las especies. Los 5 escenarios se corresponden con valores conocidos y constantes para las frecuencias de celdas: a, b, c y d. Se tomaron muestras con repetición de cada una de las poblaciones, siguiendo el esquema de muestreo de Goodall (1973): 100 muestras de tamaño 10, 100 de tamaño 20, 100 de tamaño 40 y 100 de tamaño 80, se calcularon los índices, se promediaron y se testaron las diferencias entre los valores poblacionales y los muestrales. En la mitad de los índices se encontraron más de un 50\% de resultados viciados mientras que la otra mitad resultaron en general poco viciados. El mejor comportamiento fue para el índice de Dennis y el peor para el índice de Pierce}
