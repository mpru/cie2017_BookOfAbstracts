\A
{ESTIMACIÓN POR MUESTREO DE LAS SUPERFICIES SEMBRADAS, CON CULTIVOS EXTENSIVOS MÉTODO DE SEGMENTOS ALEATORIOS}
{DIRECCIÓN DE INFORMACIÓN AGROPECUARIA Y FORESTAL - SUBSECRETARIA DE AGRICULTURA, MINISTERIO DE AGROINDUSTRIA \index{DIRECCIÓN DE...}}
{\\\Email{ceciliaconde@gmail.com}
}
{superficie agrícola sembrada; segmentos aleatorios; uso del suelo; estimadores por simple expansión y razón} 
 {Estadísticas oficiales} 
 {Muestreo} 
 {187} 
 {903-1}
{El conocimiento de las superficies sembradas con cultivos de tipo extensivo es de relevancia estratégica para el país y se hace necesario estimarlas para todas aquellas zonas y regiones de las Provincias Argentinas que revisten importancia agropecuaria. Asimismo es la base para la determinación de la producción agrícola que es requerida por numerosos actores económicos, sociales y políticos para planificar sus acciones, reducir incertidumbre y mejorar la asignación de los recursos. Las estimaciones agropecuarias deben satisfacer una serie de requisitos para poder asegurar su confiabilidad y estar disponible en tiempo y forma es decir, en el momento en que se necesite para tomar decisiones. Como consecuencia el Ministerio de Agroindustria, desde la Dirección de Información Agropecuaria y Forestal, desarrolló e implementó desde la campaña agrícola 2011/2012 el Método de Segmentos Aleatorios (MSA). El MSA consiste en un diseño de muestra estratificado de conglomerados monoetápico dentro de cada departamento o partido del país. La unidad de muestreo es el segmento, el cual consiste en un área de 400has. con diferentes coberturas del suelo, ubicado aleatoriamente en estratos homogéneos según la intensidad de uso agrícola, la cual es analizada a través de información satelital. Los operativos de recolección de datos se realizan de forma bianual coincidente con las campañas agrícolas de cosecha fina y de cosecha gruesa. El relevamiento consiste en observaciones visuales directas de los usos del suelo sin consultar a productores o informantes de las explotaciones, minimizando así los errores no debidos al muestreo. Se definieron a priori más de 30 diferentes tipos de cobertura. Esta información es integrada en un sistema de información geográfica (SIG), el cual hace posible calcular exactamente la superficie de cada tipo de cobertura dentro del segmento. Los estimadores utilizados para expandir la superficie de las coberturas muestrales son en su mayoría estimadores basados en la metodología de Horvitz-Thompson. Circunstancialmente cuando los segmentos son de tamaño variable se usan estimadores por razón a la superficie del segmento. Los resultados son informados a través de estimadores puntuales y por intervalos de confianza, adicionando sus correspondientes errores estándar y los coeficientes de variación. El conocimiento de las superficies sembradas con cultivos de tipo extensivo es de relevancia estratégica para el país y se hace necesario estimarlas para todas aquellas zonas y regiones de las Provincias Argentinas que revisten importancia agropecuaria. Asimismo es la base para la determinación de la producción agrícola que es requerida por numerosos actores económicos, sociales y políticos para planificar sus acciones, reducir incertidumbre y mejorar la asignación de los recursos. Las estimaciones agropecuarias deben satisfacer una serie de requisitos para poder asegurar su confiabilidad y estar disponible en tiempo y forma es decir, en el momento en que se necesite para tomar decisiones. Como consecuencia el Ministerio de Agroindustria, desde la Dirección de Información Agropecuaria y Forestal, desarrolló e implementó desde la campaña agrícola 2011/2012 el Método de Segmentos Aleatorios (MSA). El MSA consiste en un diseño de muestra estratificado de conglomerados monoetápico dentro de cada departamento o partido del país. La unidad de muestreo es el segmento, el cual consiste en un área de 400has. con diferentes coberturas del suelo, ubicado aleatoriamente en estratos homogéneos según la intensidad de uso agrícola, la cual es analizada a través de información satelital. Los operativos de recolección de datos se realizan de forma bianual coincidente con las campañas agrícolas de cosecha fina y de cosecha gruesa. El relevamiento consiste en observaciones visuales directas de los usos del suelo sin consultar a productores o informantes de las explotaciones, minimizando así los errores no debidos al muestreo. Se definieron a priori más de 30 diferentes tipos de cobertura. Esta información es integrada en un sistema de información geográfica (SIG), el cual hace posible calcular exactamente la superficie de cada tipo de cobertura dentro del segmento. Los estimadores utilizados para expandir la superficie de las coberturas muestrales son en su mayoría estimadores basados en la metodología de Horvitz-Thompson. Circunstancialmente cuando los segmentos son de tamaño variable se usan estimadores por razón a la superficie del segmento. Los resultados son informados a través de estimadores puntuales y por intervalos de confianza, adicionando sus correspondientes errores estándar y los coeficientes de variación. El conocimiento de las superficies sembradas con cultivos de tipo extensivo es de relevancia estratégica para el país y se hace necesario estimarlas para todas aquellas zonas y regiones de las Provincias Argentinas que revisten importancia agropecuaria. Asimismo es la base para la determinación de la producción agrícola que es requerida por numerosos actores económicos, sociales y políticos para planificar sus acciones, reducir incertidumbre y mejorar la asignación de los recursos. Las estimaciones agropecuarias deben satisfacer una serie de requisitos para poder asegurar su confiabilidad y estar disponible en tiempo y forma es decir, en el momento en que se necesite para tomar decisiones. Como consecuencia el Ministerio de Agroindustria, desde la Dirección de Información Agropecuaria y Forestal, desarrolló e implementó desde la campaña agrícola 2011/2012 el Método de Segmentos Aleatorios (MSA). El MSA consiste en un diseño de muestra estratificado de conglomerados monoetápico dentro de cada departamento o partido del país. La unidad de muestreo es el segmento, el cual consiste en un área de 400has. con diferentes coberturas del suelo, ubicado aleatoriamente en estratos homogéneos según la intensidad de uso agrícola, la cual es analizada a través de información satelital. Los operativos de recolección de datos se realizan de forma bianual coincidente con las campañas agrícolas de cosecha fina y de cosecha gruesa. El relevamiento consiste en observaciones visuales directas de los usos del suelo sin consultar a productores o informantes de las explotaciones, minimizando así los errores no debidos al muestreo. Se definieron a priori más de 30 diferentes tipos de cobertura. Esta información es integrada en un sistema de información geográfica (SIG), el cual hace posible calcular exactamente la superficie de cada tipo de cobertura dentro del segmento. Los estimadores utilizados para expandir la superficie de las coberturas muestrales son en su mayoría estimadores basados en la metodología de Horvitz-Thompson. Circunstancialmente cuando los segmentos son de tamaño variable se usan estimadores por razón a la superficie del segmento. Los resultados son informados a través de estimadores puntuales y por intervalos de confianza, adicionando sus correspondientes errores estándar y los coeficientes de variación. }
