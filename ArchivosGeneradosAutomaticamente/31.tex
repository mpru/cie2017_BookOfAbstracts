\A
{BIVARIATE GENERALIZED MODIFIED WEIBULL DISTRIBUTION FOR LIFETIME MODELING WITH A CURE FRACTION}
{\Presenting{MARCOS VINICIUS DE OLIVEIRA PERES}\index{DE OLIVEIRA PERES, M}, JORGE ALBERTO ACHCAR\index{ACHCAR, J} y EDSON ZANGIACOMI MARTINEZ\index{ZANGIACOMI MARTINEZ, E}}
{\Afilliation{DEPARTMENT OF SOCIAL MEDICINE, UNIVERSITY OF SÃO PAULO, RIBEIRÃO PRETO MEDICAL SCHOOL, BRAZIL}
\\\Email{mvperes1991@usp.br}}
{copula functions; survival analysis; bayesian analysis} 
 {Salud humana} 
 {Datos de duración} 
 {31} 
 {78-1}
{In medical studies for lifetime modeling we can find situations where a fraction of individuals will not experience the event of interest. In this case the traditional approaches for survival analysis are not suitable, since it is assumed that all individuals under study are susceptible to the event of interest. In such cases, the survival regression models in the presence of a cure fraction can be more appropriate to fit the data. Other aspect of survival analysis that should be given attention in the medical applications is the presence of multivariate lifetimes, in particular bivariate lifetimes. We have as examples the time until a treated eye fails or the time to recurrence of each breast after cancer treatment. Survival copula functions have become widely used for modelling multivariate survival data. Copulas are functions that connect univariate distribution functions to form a multivariate distribution. For more information, read Nelson (2006) or Balakrishnan and Lai (2009). In this work we consider a bivariate model for survival data based on the four-parameter generalized modifield Weibull distribution (Carrasco et al., 2008) in the presence of a cure fraction and right-censored data. A Bayesian approach was applied and the parameter estimation was based on Markov Chain Monte Carlo (MCMC) simulation methods. We consider two types of copula functions, the Farlie-Gumbel-Morgenstern (FGM) and Clayton copulas. We also consider independence between the lifetimes. In addition, it was compared the situations where the cure fractions for lifetimes are dependent or independent of one another. Comparison between models from different formulations was assessed by the deviance information criteria (DIC) and visual comparisons betwenn the predicted marginal survival functions based on the proposed models and the respective Kaplan-Meier estimates. To illustrate the applicability of the model, we consider a real data set related to an invasive cervical cancer study. In this application, we conclude that the model based on the Clayton copula and which assumes dependence between the parameters related to the cure fractions is the most appropriate to the data.}
