\A
{EVALUACIÓN DE IDEAS PARA PROMOVER EL COMPROMISO Y EL APRENDIZAJE DE LA ESTADÍSTICA}
{\Presenting{GABRIELA DAMILANO}$^1$\index{DAMILANO, G} y DAIANA RIGO$^2$\index{RIGO, D}}
{\Afilliation{$^1$UNIVERSIDAD NACIONAL DE RÍO CUARTO}
\Afilliation{$^2$UNIVERSIDAD NACIONAL DE RÍO CUARTO - CONICET}
\\\Email{gdamilano@gmail.com}}
{alfabetización estadística; innovación; motivación; evaluación} 
 {Enseñanza de la estadística} 
 {Otras categorías metodológicas} 
 {152} 
 {326-1}
{Durante el período 2015-2016 desde la cátedra de Estadística en Ciencias Sociales se desarrolló una propuesta de innovación pedagógica con la finalidad de favorecer y promover el compromiso, la motivación y la comprensión de las herramientas básicas por parte de los estudiantes. La misma consistió en una configuración didáctica organizada en distintos ejes, en uno de los cuales los estudiantes desarrollaron un proyecto de investigación (TCI) original e inesperado, desde la formulación del problema hasta la presentación y discusión de resultados. En ambos años, la innovación se evaluó con un diseño longitudinal de grupo. El seguimiento del desempeño de los estudiantes se realizó, a través del rendimiento académico obtenido durante el cursado de la asignatura; la valoración del proceso y defensa del TCI junto a entrevistas semiestructuradas para indagar la participación y el compromiso con la propuesta; y la aplicación de un instrumento antes y después de la innovación, para saber qué conocimientos tienen los estudiantes y cómo los aplican para atender a la lectura e interpretación de información estadística. En el primer año, se consideró como instrumento de evaluación la proyección del video Desmontando mitos sobre el mundo (entrevista a Hans Rosling) y una serie de preguntas que buscaban que los estudiantes relacionarán el contenido de la charla con conceptos básicos de estadística: problema de investigación, población, muestra, unidad de análisis, variables y la interpretación de información estadística a través nociones centrales como tendencia, variabilidad y predicción. En el segundo año, con el fin de que comprendieran la importancia del contexto para interpretar los datos, se utilizó como disparador la actividad Papeles en la espalda, donde los estudiantes debían registrar las respuestas que sus compañeros brindaban a la pregunta que tenían pegada en su espalda; con la información recolectada debían comentar los aspectos más destacados, realizar un gráfico lo más creativamente posible y en función de ello intentar adivinar cuál era la pregunta formulada en su espalda. Además y para valorar la compresión de la información brindada, debían conectar a través de una línea, una serie de enunciados curiosos que circulan en la web con el gráfico estadístico que consideraran mejor representaba la información proporcionada y le permitiera responder a las preguntas asociadas. Los resultados muestran que aprender bajo propuestas pedagógicas innovadoras, originales y novedades no sólo favorece el interés y la participación sino que además contribuye a creer un contexto promisorio para el aprendizaje de las estadísticas.}
