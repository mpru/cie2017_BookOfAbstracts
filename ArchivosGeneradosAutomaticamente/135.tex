\A
{APLICACIÓN DE METODOLOGÍA NO PARAMÉTRICA Y MULTIVARIADA PARA EL ANÁLISIS DE ATRIBUTOS DE UN CONJUNTO DE YERBAS COMERCIALES}
{\Presenting{MYRIAM NUÑEZ}$^1$\index{NUÑEZ, M}, MARCELA FERNÍCOLA$^1$\index{FERNICOLA, M@FERNÍCOLA, M}, FABIÁN DRUNDAY$^2$\index{DRUNDAY, F} y AMALIA CALVIÑO$^3$\index{CALVIÑO, A}}
{\Afilliation{$^1$UNIVERSIDAD DE BUENOS AIRES, FACULTAD DE FARMACIA Y BIOQUÍMICA, CÁTEDRA DE MATEMÁTICA}
\Afilliation{$^2$UNIVERSIDAD DE BUENOS AIRES, FACULTAD DE FARMACIA Y BIOQUÍMICA, CÁTEDRA DE FISIOLOGÍA}
\Afilliation{$^3$IQUIMEFA, UBA-CONICET}
\\\Email{myriam@ffyb.uba.ar}}
{cata; atributos sensoriales; análisis de correspondencias; pruebas no paramétricas} 
 {Otras ciencias de la salud} 
 {Métodos multivariados} 
 {135} 
 {304-2}
{La yerba mate presenta un elevado índice de consumo doméstico y también se diversifica su empleo en numerosos mercados a nivel mundial (Bracesco y col., 2010, Filip, 2011). En los últimos años el sector yerbatero ofrece productos diferenciados, con mayor variedad de sabores y amplios beneficios para la salud que incluye un porcentaje de numerosas hierbas agregadas dando lugar a la categoría de yerba mate compuesta. La diversidad de productos ofrecidos por el mercado hace relevante la descripción y el estudio de los mismos basándonos en la percepción de los consumidores, más representativos que los paneles entrenados. Los objetivos de este trabajo consisten en: a- analizar 5 tipos de yerba mate elaborada y compuesta al ser consumida por un grupo de 85 panelistas no entrenados de la zona metropolitana mediante un listado que incluye atributos sensoriales, emocionales y sociales, b- posicionar y agrupar los productos en función de las características mencionadas, c- dar algunos descriptores del producto considerado ideal y su ubicación relativa dentro del grupo analizado. El estudio se lleva a cabo a través de la metodología CATA (check-all-that-apply) y que consistió, en este caso, en una encuesta de 24 términos asociados al producto que los consumidores consideran relevantes en la elección de las yerbas. Como primera medida se aplicó la prueba Q de Cochran sobre cada atributo Se obtuvieron así descriptores significativos en la diferenciación de las yerbas, entre los que se destacan los sensoriales (“nota a menta”, a “peperina”, “refrescante”) y los emocionales/sociales (“calma tensiones”, “para tomar al aire libre”), etc. Luego se realizaron pruebas de a pares para identificar las muestras significativamente diferentes. A través Análisis de Correspondencias se obtuvo un mapa sensorial de las yerbas y términos de CATA que permite visualizar las similitudes y diferencias entre las muestras así como sus principales características sensoriales, determinado así la posición relativa de las mismas. Se identificaron tres grupos, el primero relacionado con atributos como “nota a hierbas”,”poco amargo” y “calma tensiones”, otro asociado a “mucho polvo”, “muy amargo” y “sabor persistente al tragar”, y por último una yerba con “nota ahumada” y “sin polvo”. Finalmente el estudio reúne información para identificar los principales atributos sensoriales percibidos que nos permitan acercarnos al producto ideal. }
