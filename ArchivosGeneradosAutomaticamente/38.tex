\A
{REVISIÓN SISTEMÁTICA DE LOS EFECTOS PRODUCIDOS POR CAMBIOS EN LA DIETA ALIMENTARIA SOBRE LA PRESIÓN ARTERIAL}
{\Presenting{LUCÍA CARABALLO}\index{CARABALLO, L}, AYLÉN AVILA\index{AVILA, A}, DARÍO WEITZ\index{WEITZ, D}, FERNANDO DÍAZ PACÍFICO\index{DIAZ PACIFICO, F@DÍAZ PACÍFICO, F}, MARTA MARZI\index{MARZI, M}, LAURA PISKULIC\index{PISKULIC, L} y MARÍA BELÉN ALLASIA\index{ALLASIA, M}}
{\Afilliation{FACULTAD DE CIENCIAS BIOQUÍMICAS Y FARMACÉUTICAS, UNIVERSIDAD NACIONAL DE ROSARIO}
\\\Email{mallasia@fbioyf.unr.edu.ar}}
{revisión sistemática; hipertensión arterial; dieta alimentaria} 
 {Salud humana} 
 {Otras categorías metodológicas} 
 {38} 
 {88-2}
{La hipertensión arterial es el principal factor de riesgo para la enfermedad cardiovascular. La adopción de una dieta con bajo contenido de sodio, grasas e hidratos de carbono y abundante en fibras naturales ha sido descripta como una estrategia para ayudar a prevenir o controlar la hipertensión. El objetivo del presente trabajo fue recolectar y sistematizar la evidencia acerca de los efectos que cambios en la dieta producen sobre la presión arterial. Se realizó una Revisión Sistemática (RS) de estudios publicados hasta abril de 2016 a partir de búsqueda electrónica en Medline/PubMed, SciELO y LILACS. Siguiendo los lineamientos del Manual Cochrane de Revisiones Sistemáticas de Intervenciones, se incluyeron ensayos controlados aleatorizados (ECAs) que evaluaron los efectos de modificaciones en la dieta sobre la presión arterial, comparando una dieta intervención (DI), rica en frutas y verduras y reducida en grasas, carnes rojas y azúcares, con la ingesta habitual de los pacientes (dieta control, DC). Criterios de elegibilidad: población adulta sin restricción de edad, sexo, etnia; presión arterial mayor a 120/80 mmHg sin medicación antihipertensiva; sometidos a tratamientos dietarios entre 2 semanas y 6 meses de duración. Se identificaron 1.077 estudios, de los cuales 10 satisfacían los criterios de elegibilidad y fueron incluidos en la RS, contribuyendo con un total de 1.541 participantes (DC: n=801, DI: n=783). El efecto, medido como la diferencia de la reducción media de la presión arterial entre el grupo intervención y el grupo control, fue significativo en 9 de los 10 estudios individuales tanto para la presión sistólica (PS) como para la presión diastólica (PD). En 8 de los 10 estudios, la magnitud de dicho efecto fue mayor para la PS que para la PD, tanto en pacientes prehipertensos como hipertensos. La RS reveló que una dieta rica en frutas, verduras y productos lácteos bajos en grasa, con granos enteros y carnes magras, pero reducida en dulces y carnes rojas tiene significativos efectos favorables sobre la presión sistólica y diastólica como primera línea de tratamiento antihipertensivo. Sin embargo, la heterogeneidad estadística entre los resultados publicados en los ECAs, la imprecisión de la información publicada en algunos artículos y la imposibilidad de contactar a los autores para obtener los datos originales, impidió realizar un Metanálisis a fin de cuantificar el efecto global promedio.}
