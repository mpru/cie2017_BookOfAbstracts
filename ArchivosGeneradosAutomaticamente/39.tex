\A
{COMPARACIÓN DEL ORDENAMIENTO DE PREDICCIONES BLUP EN MODELOS DE EVALUACIÓN GENÉTICA PARA CONEJOS}
{\Presenting{E N FERNÁNDEZ}$^1$\index{FERNÁNDEZ, E}, N N ABBIATI$^1$\index{ABBIATI, N}, J P SÁNCHEZ$^2$\index{SANCHEZ, J@SÁNCHEZ, J}, R D MARTÍNEZ$^1$\index{MARTINEZ, R@MARTÍNEZ, R}, C P GONZALEZ$^1$\index{GONZALEZ, C} y M S ROVEGNO$^1$\index{ROVEGNO, M}}
{\Afilliation{$^1$FACULTAD DE CIENCIAS AGRARIAS, IIPAAS, UNIVERSIDAD NACIONAL DE LOMAS DE ZAMORA}
\Afilliation{$^2$GENETICA I MILLORA GENETICA ANIMAL, INSTITUT DE RECERCA I TECNOLOGIA AGROALIMENTÀRIES}
\\\Email{ednfer@yahoo.com}}
{modelos mixtos; coeficiente de spearman; respuesta a la selección; cunicultura} 
 {Ciencias agropecuarias} 
 {Métodos no paramétricos} 
 {39} 
 {89-1}
{El modelo mixto aditivo con repetibilidad es el más empleado para mejorar el número de gazapos destetados (ND) en conejos para carne. Estudios recientes detectaron una fuerte asociación entre el efecto fijo año-estación (AE) y la consanguinidad, que condujo a una sobreestimación de parámetros y tendencias genéticas. El AE incluido como aleatorio en el modelo produjo estimaciones más razonables. El objetivo de este trabajo fue comparar el ranking de los valores genéticos aditivos predichos según se considere el AE fijo o aleatorio en el análisis del carácter ND en conejas de la línea A pertenecientes a la Universidad Politécnica de Valencia. Se dispuso de una base de 15671 registros para ND y de una genealogía de 5668 conejos que comprendieron 38 generaciones. Los datos fueron analizados por dos modelos, diferenciados por la inclusión de AE como fijo (M1) o aleatorio (M2). Además, en ambos se consideró como efecto fijo al estado fisiológico de la hembra y la covariable consanguinidad, y como efectos aleatorios el valor genético aditivo y el efecto permanente de la coneja. La relación entre los rankings de ambos modelos se estudió empleando dos aproximaciones: la correlación de Spearman (para todas y las veinte o cuarenta mejores conejas (T20 o T40), en cada generación) y el porcentaje de discordancias en la elección de las T20/T40, según generación. Al considerar la generación completa, las correlaciones fueron altas con un promedio por generación de 0.99 (±0.01), disminuyendo a 0.90 (±0.08) y 0.94 (±0.03) en las T20 y T40 respectivamente. El porcentaje de discordancias promedio en T20 fue 8.55\% (±5.05\%) y de 6.18 \% (±3.3\%) en T40. Se detectaron generaciones con el mismo porcentaje de discordancia y correlaciones de Spearman diferentes, como en las generaciones 5 y 11 de T20 con un porcentaje de discordancia del 5\% y correlaciones de 0.72 y 0.99, respectivamente. También se presentaron generaciones con \% de discordancia similares y correlaciones diferentes. Sin embargo, la respuesta estimada a la selección a través de las hembras elegidas por los M1 o M2, evaluadas por M2, fue de 0.05 gazapos por generación tanto para las T20 como las T40. En nuestro caso y poniendo énfasis en los resultados del proceso selectivo, el estudio de los rankings entre modelos mediante correlaciones y \% de discordancias no resultó concluyente respecto a la respuesta esperada a la selección. }
