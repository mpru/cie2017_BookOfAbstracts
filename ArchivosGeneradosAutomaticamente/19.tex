\A
{MODELO DINA APLICADO A LA EVALUACIÓN DE MATEMÁTICA EN ESTUDIANTES DE SEGUNDO GRADO DE SECUNDARIA}
{\Presenting{YURIKO KIRILOVNA SOSA PAREDES}\index{SOSA PAREDES, Y} y LUIS VALDIVIESO SERRANO\index{VALDIVIESO SERRANO, L}}
{\Afilliation{PONTIFICIA UNIVERSIDAD CATÓLICA DEL PERÚ}
\\\Email{yuriko.sosa.p@gmail.com}}
{modelo dina; mdc; muestreador de gibbs; modelo rasch; variables latentes; educación} 
 {Educación, ciencia y cultura} 
 {Métodos bayesianos} 
 {19} 
 {43-1}
{Los modelos de diagnóstico cognitivo (MDC) tienen como finalidad describir o diagnosticar el comportamiento de los evaluados por medio de clases o perfiles latentes, de tal manera que se obtenga información más específica acerca de las fortalezas y debilidades de ellos. Uno de los modelos más populares de esta gran familia es el llamado modelo DINA, el cual tuvo su primera aparición en Haertel (1989) enfocado principalmente en el campo educacional. Este modelo considera solo respuestas observadas dicotómicas de parte de los individuos y tiene como restricción principal que ellos deben dominar necesariamente todas las habilidades requeridas por cada ítem; aquellas que se resumen en una matriz llamada Q. Asimismo, el modelo estima parámetros para los ítems, los cuales son denominados de ruido: Adivinación y Desliz. En este trabajo se desarrolla teóricamente el modelo expuesto; es decir, sus fundamentos y principales propiedades desde el enfoque bayesiano. Específicamente, las estimaciones se realizan mediante el Muestreador de Gibbs. Se presenta una aplicación enfocada en educación, donde se analiza una muestra de 3040 alumnos del 2do grado de secundaria, evaluados en una prueba de 48 ítems de la competencia matemática realizada por la Oficina de Medición de la Calidad de los Aprendizajes (UMC) en el 2015. A esta prueba se le aplica el modelo de Rasch y el modelo DINA bajo el enfoque bayesiano, con el fin de estudiar la correspondencia entre indicadores de ambos modelos, tanto para los parámetros de los alumnos (habilidad y perfiles latentes) como de los ítems (dificultad y parámetros de ruido).}
