\A
{PROBABILIDAD DE MOROSIDAD EN LOS CRÉDITOS DE CONSUMO DE UNA COOPERATIVA DE AHORRO Y CRÉDITO DE PARAGUAY}
{\Presenting{MARIO DAMIÁN VÁZQUEZ}$^1$\index{VAZQUEZ, M@VÁZQUEZ, M} y MARÍA CRISTINA MARTÍN$^2$\index{MARTÍN, M}}
{\Afilliation{$^1$DIRECCIÓN GENERAL DE INVESTIGACIÓN Y ESTUDIOS ESPECIALES. UNIVERSIDAD NACIONAL DE VILLARRICA DEL ESPÍRITU SANTO. PARAGUAY}
\Afilliation{$^2$DEPARTAMENTO DE MATEMÁTICA. UNIVERSIDAD NACIONAL DEL SUR. ARGENTINA}
\\\Email{mscientiae@gmail.com}}
{regresión logística; técnica stepwise; odds ratio; curva roc; morosidad; cooperativa} 
 {Economía} 
 {Datos categóricos} 
 {87} 
 {195-1}
{El objetivo del trabajo de investigación es estimar la morosidad de los socios de una cooperativa de ahorro y crédito de Paraguay al solicitar un crédito de consumo. Analizando la base de datos de la cooperativa y al aplicar la Regresión Logística para respuesta binaria con la técnica Stepwise se seleccionan las variables que componen el modelo final. Para la interpretación del modelo, lo que se analiza son los cocientes de ventaja (Odds Ratio), dada por la expresión OR=$e^{\beta_k}$, en términos probabilísticos, refleja el cambio marginal de pasar de la categoría de referencia a otra categoría de esta variable. Si la OR resulta inferior a la unidad, se calcula su inversa $OR^{-1}$ para una mejor interpretación, siendo ahora la ventaja a favor de la categoría de referencia. Clasificando a los socios con probabilidad de ser morosos mayor a 0,16 como socios morosos, se obtienen los índices para medir la bondad de ajuste del modelo final estimado por Regresión Logística, siendo la Precisión igual a 70,55\%, la Especificidad igual a 70,54\% y la Sensibilidad igual a 70,59\%, por lo que la capacidad predictiva del modelo matemático propuesto es aceptable. Similarmente, el valor 0,721 del área bajo la curva ROC, permite considerar que el modelo tiene capacidad de discriminación aceptable. Se concluye que hay evidencias estadísticas significativas para afirmar que el plazo solicitado en el crédito de consumo en categorías, la antigüedad en la cooperativa en categorías, el estado civil del socio y el monto solicitado en el crédito, estiman la probabilidad de morosidad de los socios. Los resultados son relevantes para las políticas de utilización de modelos matemáticos en la estimación de la morosidad en los créditos de consumo en la cooperativa.}
