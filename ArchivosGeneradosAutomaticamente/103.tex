\A
{IDENTIFICACIÓN DE ZONAS PARA MANEJO DIFERENCIADO EN FRUTICULTURA DE PRECISIÓN EN FRUTOS DE PERA EN EL ALTO VALLE DE RÍO NEGRO Y NEUQUÉN}
{\Presenting{S OCAMPO}$^1$\index{OCAMPO, S}, D FERNÁNDEZ$^1$\index{FERNÁNDEZ, D}, M CURETTI$^1$\index{CURETTI, M} y S BRAMARDI$^2$\index{BRAMARDI, S}}
{\Afilliation{$^1$INTA}
\Afilliation{$^2$UNCOMA}
\\\Email{ocampo.santiago@inta.gob.ar}}
{geoestadística; pronósticos de rendimiento; datos geo-referenciados; análisis de la variancia} 
 {Otras aplicaciones} 
 {Otras categorías metodológicas} 
 {103} 
 {226-2}
{El mapa de rendimiento es una de las herramientas más utilizadas en agricultura de precisión. Con el objetivo de diferenciar diferentes zonas de manejo se trabajó en una parcela de 1.8 Ha de pera cv. Williams. La cosecha se realizó durante cinco temporadas (2014 a 2017) con una plataforma autopropulsada que cuenta con un sistema de pesado de la fruta. La parcela cuenta con 57 filas con 36 plantas y está plantada a 4x2 m. Los datos se colectaron acumulando el peso de la fruta de 12 plantas vecinas (3 datos por fila). Cada grupo de 12 plantas conformó una unidad estadística (UE) a la que se le asoció un dato de producción acumulada y una coordenada en el sistema UTM (168 UE). A su vez se conformaron grupos de UE vecinas para establecer diferentes sectores dentro del cuadro (por ejemplo, agrupando las 168 UE de a 4, se establecen 42 sectores dentro del cuadro). Se estudiaron las diferencias en la producción a lo largo de las cuatro temporadas y en los diferentes sectores del cuadro, y se evaluó la presencia de interacción entre la temporada y el sector. El estudio se llevó a cabo utilizando análisis de la varianza sobre la producción, utilizando el año y el sector de la parcela como factores fijos del modelo. También se utilizó un término de interacción año-sector en el modelo. Si para cierto sector se evalúa la relación entre su nivel de producción respecto al promedio de toda la parcela y se compara esta relación a lo largo de diferentes años, se puede evaluar la estabilidad o inestabilidad de cada sector de la parcela a lo largo de los años. Es posible medir si existe inestabilidad de los sectores a través del término de interacción del análisis de la varianza. No se encontraron interacciones significativas para los años 2015 a 2017, es decir que los sectores mostraron comportamiento estable (P<0,5). Se vio un marcado efecto del año de cosecha y del sector de la parcela sobre la producción (ambos P<0,001). Cuando se incorporó el año 2014, solo algunos de los sectores mostraron inestabilidad ocasionando que la interacción sea ligeramente significativa (P<0,003). Con todo esto fue posible identificar sectores de baja, media y alta productividad y establecer regiones dentro de la parcela para realizar prácticas culturales diferenciadas en el futuro.}
