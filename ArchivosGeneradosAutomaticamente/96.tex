\A
{COMPARACIÓN DE MÉTODOS PARA COMPLETAR DATOS FALTANTES EN VARIABLES CLIMATOLÓGICAS EN PATAGONIA SUR}
{\Presenting{DORA SILVIA MAGLIONE}$^1$\index{MAGLIONE, D}, JOSÉ LUIS SÁENZ$^1$\index{SAENZ, J@SÁENZ, J}, JULIO SOTO$^1$\index{SOTO, J}, OSCAR BONFILI$^2$\index{BONFILI, O}, CARLOS TALAY$^1$\index{TALAY, C}, MARISA SANDOVAL$^1$\index{SANDOVAL, M} y MIGUEL LLANCALAHUEN$^1$\index{LLANCALAHUEN, M}}
{\Afilliation{$^1$UNIVERSIDAD NACIONAL DE LA PATAGONIA AUSTRAL}
\Afilliation{$^2$SERVICIO METEOROLÓGICO NACIONAL, OFICINA METEOROLÓGICA RÍO GALLEGOS}
\\\Email{dmaglione@disytel.net}}
{datos faltantes; series temporales; variables climatológicas} 
 {Otras aplicaciones} 
 {Datos faltantes} 
 {96} 
 {211-1}
{La región Patagónica sur se caracteriza por su amplia extensión y por la escasez de estaciones meteorológicas en donde recabar información. En este trabajo se utilizan los datos de estaciones meteorológicas del INTA de la provincia de Santa Cruz y sur de Chubut (todas distantes y con presencia de datos faltantes), para comparar seis métodos propuestos en la literatura para completar series mensuales, como son el método de Karl, el método CLP, el método de la razón q, el método de la razón normal, la regresión lineal simple, la correlación múltiple y el método de las distancias inversas para las variables temperatura media, máxima y mínima, humedad relativa, precipitación acumulada y presión atmosférica. Para poder llevar adelante las comparaciones de estos métodos se realiza un estudio sobre los errores obtenidos al predecir los datos de valores existentes al ser eliminados de la serie, usando distintas medidas de selección y de interpretación en cada una de las estaciones. Para las cuatro primeras variables, la correlación es alta entre todas las estaciones climatológicas, por lo que son consideradas todas ellas al aplicar los siete métodos; se observa que los mejores métodos para completar datos faltantes resultan ser el de regresión lineal, el de correlación múltiple y el de la distancia inversa, con variaciones según la variable y las ubicaciones geográficas de las estaciones meteorológicas. En el caso de las variables presión atmosférica y precipitación acumulada mensual, no existe correlación alta entre todas las estaciones, por lo que los siete métodos se aplican sólo a grupos más reducidos que presentan correlación moderada entre las series mensuales; en este caso se observa que los mejores métodos para completar las series de presión atmosférica resultan ser el de regresión lineal, el de correlación múltiple y el de la distancia inversa, y para la precipitación acumulada aparece mayor diversidad de métodos dependiendo de la ubicación geográfica como son la razón q, regresión lineal, distancia inversa, razón normal y el método CLP. Para estas dos últimas variables si se utilizan métodos que no tengan en cuenta las correlaciones entre estaciones sino toda la información disponible en la región, se obtiene que el método de la distancia inversa es el que mejor completa la información faltante para la precipitación acumulada mensual en todas las estaciones climatológicas, en tanto que para la presión atmosférica principalmente el de CLP.}
