\A
{PROPUESTA DE DISEÑO DE UN SISTEMA PARA EL SEGUIMIENTO ACADÉMICO DE ESTUDIANTES Y DOCENTES DE LA CARRERA DE ING. EN SISTEMAS DE INFORMACIÓN - UTN FRC}
{\Presenting{CECILIA SAVI}\index{SAVI, C}, ANDREA FABIANA RIGHETTI\index{RIGHETTI, A}, ANA MARÍA STRUB\index{STRUB, A}, CLARISA LILIANA STEFANICH\index{STEFANICH, C} y IRENE ESTHER ROMOLI\index{ROMOLI, I}}
{\Afilliation{FACULTAD REGIONAL CÓRDOBA. UNIVERSIDAD TECNOLÓGICA NACIONAL}
\\\Email{ceciliasavi@hotmail.com}}
{indicadores estadísticos; seguimiento académico de docentes; seguimiento académico de estudiantes; dimensiones de análisis de calidad educativa} 
 {Educación, ciencia y cultura} 
 {Otras categorías metodológicas} 
 {110} 
 {239-1}
{La acreditación es un proceso por medio del cual un programa o institución educativa brinda información sobre sus operaciones y logros a un organismo externo que evalúa y juzga. La evaluación y la acreditación son procesos relacionados cuya práctica se entrecruzan, ya que se acredita conforme y como consecuencia de un proceso de evaluación y seguimiento. La acreditación no sólo provee un diagnóstico que conduce a la acción por parte de la propia institución, sino que constituye una constancia de credibilidad por parte de la sociedad demandante de los servicios educativos. Estos servicios educativos implican, conceptos relativos, intangibles y muchas veces subjetivos, que no permiten una evaluación concreta y/o absoluta, lo que obliga al diseño de mecanismos de control diferentes. Una opción posible es la creación de indicadores estadísticos que permiten relacionar funcionamiento, recursos y resultados respecto a actividades, procesos, unidades organizacionales y otros componentes de una institución. La evaluación de instituciones y de carreras de educación superior, en sí misma, constituye un valioso instrumento que proporciona elementos de juicio para analizar a fondo los procesos educativos, generando información para promover y asegurar la mayor calidad, productividad y pertinencia de las acciones y resultados. En este contexto y como parte del diseño de un Sistema de Seguimiento Académico de estudiantes y docentes de la carrera Ingeniería en Sistemas de Información UTN-FRC, se presenta una propuesta de indicadores estadísticos específicos. Para ello en una primera etapa se analizan, definen y conceptualizan dimensiones de análisis de la calidad educativa, las variables intervinientes y los indicadores, precisando nombre, definición, explicación, fuentes de obtención de la información de base, fórmula de cálculo, y elementos que se consideren necesarios para la interpretación de los resultados. Se propone además, construir una aplicación informática que permita capturar y procesar datos del Sistema Académico de la FRC e ingresar los que se generen a través de instrumentos de recolección específicos. El diseño del software prevé la presentación de información mediante tablas, cuadros comparativos, gráficos estadísticos, tableros de comando y archivos exportables, para distintos usuarios de la gestión académica. La construcción de la herramienta de software para el tratamiento de datos masivos pretende contribuir a la necesidad de manipular enormes cantidades de datos y también a la creación de modelos predictivos para ser utilizados específicamente en la carrera propuesta.}
