\A
{ANÁLISIS DE ESTRÉS EN POTRILLOS AL APLICAR LAS TÉCNICAS DE VINCULACIÓN Y APRENDIZAJE NO TRAUMÁTICO EMPLEANDO MODELOS MIXTOS}
{\Presenting{N ABBIATI}$^1$\index{ABBIATI, N}, S PAZ$^2$\index{PAZ, S}, M TOPAYAN$^2$\index{TOPAYAN, M}, D REFOJO$^2$\index{REFOJO, D}, O MEDINA$^1$\index{MEDINA, O}, A HARBAR$^2$ \index{HARBAR, A} y S VIAMONTE$^2$ \index{VIAMONTE, S}}
{\Afilliation{$^1$CÁTEDRA DE BIOMETRÍA, FACULTAD DE CIENCIAS AGRARIAS, UNLZ}
\Afilliation{$^2$CÁTEDRA DE EQUINOTECNIA, FACULTAD DE CIENCIAS AGRARIAS, UNLZ}
\\\Email{norabbi2000@gmail.com}}
{modelos mixtos; potrillos; técnicas de sociabilización} 
 {Ciencias agropecuarias} 
 {Modelos de regresión} 
 {36} 
 {86-1}
{Las Técnicas de Vinculación y Aprendizaje No Traumático (TVANT) son un conjunto de maniobras realizadas a potrillos entre los días 7 y 14 de nacidos, para establecer un vínculo hombre–animal; generando desensibilización a situaciones estresantes, que impiden una sociabilización temprana. Las variables frecuencia cardíaca y temperatura rectal, después de actividades específicas, permiten cuantificar el estrés. El objetivo del trabajo fue estudiar si existe un incremento en dichas variables luego de aplicar las TVANT en condiciones a campo. Se seleccionaron aleatoriamente 21 potrillos de raza criolla, de un haras de Luján, y 5 Silla Argentino, del campo experimental Santa Catalina de la UNLZ. Las taras se efectuaron en instalaciones aptadas para tal fin, muy diferentes, en condiciones de campo. Entre los días 7 a 14, antes y después del tratamiento, se midió temperatura rectal con termómetro digital y frecuencia cardíaca con estetoscopio Littman. El análisis de la información se realizó mediante Modelos Mixtos para contemplar una estructura de correlaciones entre mediciones de un mismo animal y/o heterogeneidad de varianzas. El modelo contempló los factores: Raza-Lugar (Criollos-Luján, CLu, Silla Argentino-Santa Catalina SASC); Momento (previo y posterior a las TVANT) y Día (7 a 14) con sus interacciones dobles y triple. Para la elección de la estructura de covarianzas de la matriz de errores se empleó el criterio de información de Akaike; para la comparación de las medias mínimo cuadráticas, Tukey-Kramer. Se usó el software SAS con $\alpha$=0,05. Con relación a temperatura, el modelo elegido contempló heterogeneidad de varianzas para Raza-Lugar y detectó diferencias entre momentos siendo el momento previo inferior al posterior. También detectó interacción Raza-Lugar y Día. Se efectuó la apertura de Día dentro de Raza-Lugar, detectándose en CLu una disminución significativa de medias entre los días 7 y 10; para SASC detectó una disminución entre la temperatura media del día final y los días iniciales (7, 8 y 10). En frecuencia cardíaca, el modelo contempló una matriz desestructurada para el error y detectó interacción Raza-Lugar y Día. Al efectuar la apertura de Día dentro de Raza-Lugar, en CLu no había diferencias significativas para los dos primeros, pero disminuían y se mantenían estables a partir del día 9. Para SASC, se presentaron diferencias entre la totalidad de días (excepto 9) y el último. Se concluye existencia de comportamiento diferencial Razas-Lugar, con disminución en las medias de ambas variables a lo largo del tiempo; presentándose mayor estabilidad en CLu.}
