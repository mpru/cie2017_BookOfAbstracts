\A
{ASOCIACIÓN ENTRE VARIABILIDAD MOLECULAR Y FENOTÍPICA EN TOMATE}
{\Presenting{A P DEL MEDICO}$^1$\index{DEL MEDICO, A}, V G CABODEVILA$^1$\index{CABODEVILA, V}, M S VITELLESCHI$^2$\index{VITELLESCHI, M}, A LAVALLE$^3$\index{LAVALLE, A} y G R PRATTA$^1$\index{PRATTA, G}}
{\Afilliation{$^1$IICAR (INSTITUTO DE INVESTIGACIONES EN CIENCIAS AGRARIAS DE ROSARIO), CONICET/UNR}
\Afilliation{$^2$IITAE (INSTITUTO DE INVESTIGACIONES TEÓRICAS Y APLICADAS DE LA ESCUELA DE ESTADÍSTICA), CIUNR/UNR}
\Afilliation{$^3$DEPARTAMENTO DE ESTADÍSTICA. UNIVERSIDAD NACIONAL DEL COMAHUE}
\\\Email{delmedico@iicar-conicet.gob.ar}}
{análisis factorial múltiple; mejoramiento genético; aflp; ssr} 
 {Genética} 
 {Métodos multivariados} 
 {33} 
 {83-2}
{Actualmente, en los programas de mejoramiento genético resulta de interés conocer las asociaciones entre la variabilidad genética y fenotípica para caracteres de interés agronómico. Existen diferentes tipos de marcadores moleculares que pueden aplicarse para estimar la variabilidad genética. Entre ellos, la técnica de AFLP (Amplified Fragment Length Polymorphism o Polimorfismo en la Longitud de los Fragmentos Amplificados) no requiere un conocimiento previo de la secuencia de ADN y brinda marcadores de tipo dominante y multiloci. Por otro lado, los SSR (Single Sequence Repeats o Repeticiones de Secuencia Simple, también llamadas microsatélites) requieren un conocimiento previo de la secuencia de ADN, siendo marcadores codominantes y de único locus. El objetivo de este trabajo fue comparar, mediante Análisis Factorial Múltiple (AFM), la utilidad de ambos tipos de marcadores para caracterizar la variabilidad molecular y estimar su asociación con la variación fenotípica en una generación segregante de tomate. Se evaluaron 66 individuos provenientes de la F2 del híbrido de segundo ciclo F1 ToUNR18xToUNR1, cuyos perfiles de AFLP y SSR fueron obtenidos por técnicas estándares. Los caracteres fenotípicos medidos fueron peso, diámetro, altura, forma, vida poscosecha, porcentaje de reflectancia, firmeza, coeficiente a/b, sólidos solubles, pH y acidez titulable. Se aplicaron dos AFM para relacionar en forma independiente cada tipo de marcador molecular que segregó en la forma mendeliana esperada según su comportamiento genético, con los caracteres fenotípicos bajo estudio. 29 marcadores AFLP y 17 marcadores SSR segregaron en forma mendeliana. En el AFM, la proporción de variación explicada por los dos primeros ejes principales fue levemente superior para AFLP (30,9\% vs. 25,0\% para SSR). Si bien la representación de las correlaciones entre las variables cuantitativas resultó similar en ambos análisis, se observaron algunas discrepancias que podrían estar explicadas por su diferente asociación con la variación molecular. Además, el agrupamiento de los individuos de acuerdo a la caracterización fenotípica y molecular fue diferente, lo que se correspondería con el diferente tipo de variación en el ADN que mide cada marcador. La asociación global, medida por el coeficiente RV, fue mayor para SSR que para AFLP (0,22 y 0,13, respectivamente). Se concluye que, a pesar de que los dos tipos de marcadores considerados brindan información diferente respecto a la variación en el ADN, ambos son útiles para caracterizar la variabilidad molecular y estimar su asociación con la variación fenotípica en la generación segregante de tomate analizada.}
