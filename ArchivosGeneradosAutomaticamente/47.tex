\A
{EL PROTOCOLO Y SU EFECTO SOBRE LA CONTINUIDAD DE LA EMPRESA FAMILIAR A TRAVÉS DE MODELOS DE ECUACIONES ESTRUCTURALES}
{\Presenting{MARÍA DE LOS ÁNGELES LUCERO BRINGAS}$^1$\index{LUCERO BRINGAS, M} y NORMA PATRICIA CARO$^2$\index{CARO, N}}
{\Afilliation{$^1$UNIVERSIDAD CATÓLICA CÓRDOBA}
\Afilliation{$^2$UNIVERSIDAD NACIONAL DE CÓRDOBA}
\\\Email{pacaro@eco.unc.edu.ar}}
{ecuaciones estructurales; alpha de cronbach; empresas familiares; protocolo} 
 {Otras ciencias económicas, administración y negocios} 
 {Modelos de regresión} 
 {47} 
 {102-1}
{Las empresas familiares son representativas de la economía en la provincia de Córdoba, ellas se caracterizan por una fuerte vocación de continuidad como parte del legado familiar. Sin embargo, su permanencia se haya obstaculizada en la mayoría de los casos por la falta de gestión de la familia empresaria. El alto índice de desaparición obedece en la mayoría de los casos a contrariedades de índole familiar. El protocolo familiar es una herramienta que aporta a la continuidad de las empresas familiares facilitando la gestión empresarial y de la familia empresaria. Se materializa a través de un acuerdo escrito donde los miembros de la familia se anticipan a los tópicos que pueden afectar la continuidad, previendo principios y normas cuyo cumplimiento estén orientados a una mayor unidad familiar y fortalecimiento de la empresa. El protocolo es un instrumento de gestión indispensable para la supervivencia de la empresa familiar y familia empresaria. Este trabajo tiene como objetivo, encontrar un análisis causal entre el comportamiento de las empresas familiares en relación al protocolo familiar como herramienta de gestión en pos a la continuidad del negocio familiar. La muestra está formada por 220 empresas familiares protocolizadas (16\%) y no protocolizadas (84\%), donde el 46,8\% se encuentra en la segunda generación, un 32,7\% aún está en primera y un 20,5\% se encuentra en tercera o superior. Se aplicaron los modelos de ecuaciones estructurales (structural equations models SEM), donde por un lado, se define la variable latente “continuidad” que indica la percepción que tienen los miembros de las empresas familiares y por otro lado si poseer protocolo tiene implicancia en la continuidad. El Apha de Cronbach indica que el instrumento es adecuado para lo que desea medir (0,92) y los p-value obtenidos por el SEM indican que las afirmaciones fueron significativas en la definición del constructo y confirman la hipótesis de que investir un protocolo familiar influye a la continuidad de la empresa familiar. Las medidas de bondad de ajuste: RMSEA (0,064) y SRMR (0,058) resultaron aceptables. Poseer un protocolo familiar que incluya los lineamientos para una gestión familiar y empresarial contribuye a la continuidad de la empresa familiar con todos los beneficios que ello implica tanto para la propia familia empresaria como para la sociedad y economía en la cual se encuentra inmersa.}
