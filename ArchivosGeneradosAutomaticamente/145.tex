\A
{METODOLOGÍA PARA LA PREDICCIÓN DEL COEFICIENTE DE RETENCIÓN DE UN HERBICIDA EN SUELO}
{\Presenting{FRANCA GIANNINI KURINA}$^{1,2}$\index{GIANNINI KURINA, F}, SUSANA HANG$^1$\index{HANG, S}, ARIEL RAMPOLDI$^1$\index{RAMPOLDI, A} y MÓNICA BALZARINI$^{1,2}$\index{BALZARINI, M}}
{\Afilliation{$^1$FACULTAD DE CIENCIAS AGROPECUARIAS, UNIVERSIDAD NACIONAL DE CÓRDOBA}
\Afilliation{$^2$CONICET}
\\\Email{francagianninikurina@gmail.com}}
{árboles de regresión; glifosato; suelos; predicción} 
 {Ciencias agropecuarias} 
 {Modelos de regresión} 
 {145} 
 {312-1}
{La dinámica de un fitosanitario en suelo es un proceso complejo que depende de propiedades edafoclimáticas del sitio donde se aplica. Uno de los principales mecanismos que caracterizan esta dinámica es la retención parametrizada por el coeficiente de adsorción (Kd), relación entre lo que queda retenido en el suelo y lo que se deriva. El objetivo de este trabajo fue generar un modelo predictivo del Kd de glifosato para suelos de la provincia de Córdoba. Se calculó el Kd en 90 muestras de suelo distribuidas en el territorio y seleccionadas según un HLc basado en variables edafoclimaticas en búsqueda de variabilidad de las co-variables a usar en la modelación. El Kd de glifosato se distribuyó como una gama (0,035; 1,129). Mediante árboles de regresión mejorados por re-muestreo (Boosting RT) aplicados sobre 20 co-variables edafoclimáticas y Kd como variables dependiente, se seleccionaron las variables de mayor capacidad contribución en la explicación de la variabilidad del Kd. Se ajustó un modelo lineal generalizado de regresión múltiple, con distribución gama y enlace log y un modelo lineal al Logaritmo del Kd. El Error Cuadrático Medio de Predicción relativo a la media, del modelo en escala logarítmica fue del 7\%. Se priorizó este modelo sobre el modelo lineal generalizado por su simplicidad. Los resultados indicaron que los efectos del pH del suelo y la textura fueron las propiedades edáficas de mayor relevancia para predecir la retención de glifosato en los suelos muestreados. El impacto del Al fue modelado con un polinomio de segundo orden, mientras que para pH, Arena y Limo se usaron términos de primero orden. La ecuación resultante fue usada como valor esperado del Kd con la finalidad de mapear e identificar áreas de mayor vulnerabilidad a la contaminación por este herbicida.}
