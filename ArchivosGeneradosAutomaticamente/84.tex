\A
{EXPLORACIÓN DE EFECTOS GENÉTICOS EN GENOTIPOS DE PRIMER Y SEGUNDO CICLO DE TOMATE}
{\Presenting{PAOLO CACCHIARELLI}$^1$\index{CACCHIARELLI, P}, VLADIMIR CAMBIASO$^1$\index{CAMBIASO, V}, JAVIER PEREIRA DA COSTA$^1$\index{PEREIRA DA COSTA, J}, ELIZABETH TAPIA$^2$\index{TAPIA, E}, GUSTAVO R. RODRÍGUEZ$^1$\index{RODRÍGUEZ, G} y GUILLERMO R. PRATTA$^1$\index{PRATTA, G}}
{\Afilliation{$^1$IICAR (INSTITUTO DE INVESTIGACIONES EN CIENCIAS AGRARIAS DE ROSARIO), CONICET/UNR}
\Afilliation{$^2$CIFASIS (CENTRO INTERNACIONAL FRANCO ARGENTINO DE CIENCIAS DE LA INFORMACIÓN Y SISTEMAS), CONICET/UNR}
\\\Email{cacchiarelli@iicar-conicet.gob.ar}}
{selección artificial; recombinación genética; heterosis; dirección de cruzamiento} 
 {Genética} 
 {Inferencia estadística} 
 {84} 
 {186-1}
{Los programas de mejoramiento genético de tomate (Solanum lycopersicum L.) pueden disponer de Genotipos de Primer Ciclo (GPC, progenitores cultivados y exóticos de constitución genotípica homocigota y sus híbridos, F1 heterocigotas) y Genotipos de Segundo Ciclo (GSC): nuevas líneas homocigotas derivadas de la generación segregante de los GPC mediante un proceso de Selección Artificial (SA), que presentan por Recombinación Genética (RG) características deseables de ambos progenitores, y sus respectivas F1. Por otro lado los híbridos suelen presentar Heterosis (H) o diferencias con respecto a la media entre sus progenitores para caracteres cuantitativos. También pueden encontrarse en ellos efectos recíprocos según la Dirección de Cruzamiento (DC), es decir cuál progenitor se usó como madre y cuál como padre. El objetivo fue explorar, mediante la comparación de GPC y GSC de tomate, los efectos de SA, RG, H y DC sobre caracteres de calidad de fruto. Se evaluaron los GPC: cultivar Caimanta de S. lycopersicum (C), LA0722 de S. pimpinellifolium (P) y sus F1 recíprocas (CxP y PxC), y los GSC: ToUNR1 (L1) y ToUNR18 (L18) y sus F1 recíprocas (L1xL18 y L18xL1). En invernadero se cultivaron 8 plantas de cada genotipo (n=64), en cuyos frutos se evaluaron peso, vida poscosecha, forma, sólidos solubles (SS), pH y firmeza. Para cada carácter cuantitativo se compararon todos los genotipos mediante ANOVA a un criterio de clasificación. Para explorar los diferentes efectos, se plantearon los siguientes contrastes ortogonales: 1- SA y RG: C, P, CxP, PxC vs L1, L18, L1xL18, L18xL1; 2- H en GPC: CxP, PxC vs C, P; 3- H en GSC: L1xL18, L18xL1 vs L1, L18; 4- DC en GPC: CxP vs PxC; 5- DC en GSC: L1xL18 vs L18xL1. Para todos los caracteres, la distribución fenotípica ajustó a la normal (excepto para peso, que fue transformado mediante logaritmo natural) y se encontraron diferencias significativas entre todos los genotipos. Respecto a los diferentes efectos explorados, los de SA y RG fueron significativos para todos los caracteres excepto para SS. H en GPC fue significativa para todos los caracteres excepto para pH; mientras que H en GSC solo fue significativa para peso. DC fue no significativa para todos los caracteres en GPC y significativa sólo para vida poscosecha y peso en GSC. Se concluye que comparando GPC y GSC mediante contrastes ortogonales, fue posible determinar los efectos de SA, RG, H y DC sobre caracteres cuantitativos de calidad de fruto.}
