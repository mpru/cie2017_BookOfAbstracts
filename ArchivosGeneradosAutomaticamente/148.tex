\A
{REDUCCIÓN DE DIMENSIÓN EN UN PROBLEMA DE HUMEDALES DEL PARANÁ MEDIO}
{\Presenting{STELLA MARIS VAIRA}$^1$\index{VAIRA, S}, LAURA MODINI$^2$\index{MODINI, L}, MARIEL ZERBATTO$^2$\index{ZERBATTO, M} y LETICIA MESA$^3$\index{MESA, L}}
{\Afilliation{$^1$DEPARTAMENTO DE MATEMÁTICA - FBCB - UNL}
\Afilliation{$^2$SECCIÓN AGUAS - DPTO. DE CS. BIOLÓGICAS - FBCB - UNL}
\Afilliation{$^3$INSTITUTO NACIONAL DE LIMNOLOGÍA (CONICET-UNL)}
\\\Email{stella.vaira@gmail.com}}
{análisis de componentes principales; biplot; humedal; coliformes} 
 {Ecología y medio ambiente} 
 {Métodos multivariados} 
 {148} 
 {317-1}
{La ganadería extensiva, con modalidad rotacional, en humedales de la cuenca media del río Paraná, es una estrategia de pastoreo común en las últimas décadas. Sin embargo, su efecto sobre la calidad bacteriológica de este sistema acuático es escasamente conocido. El objetivo de este trabajo fue analizar conjuntamente variables (p = 6) profundidad de la laguna, nivel hidrométrico del río, pH, cobertura de macrófitas, turbiedad y temperatura del agua en muestras (n = 51) de tres humedales que fueron considerados como factor, así como los indicadores bacteriológicos (Coliformes totales, Coliformes termotolerantes y E. coli) y la presencia o no de ganado, para la ubicación en el biplot de las nuevas componentes principales. La finalidad es identificar el número y composición de componentes necesarios para resumir las puntuaciones observadas en un conjunto grande de variables observadas; se aplicó un Análisis de Componentes Principales (ACP). Este método explica el máximo porcentaje de varianza observada en cada ítem a partir de un número menor de componentes que resume la información inicial. Los resultados muestran que la matriz de correlaciones iniciales tiene asociado un valor de covariancia generalizada de 0,095; el valor del estadístico KMO (Kaiser-Meyer-Olkin) de 0,382, si bien es bajo este valor, la prueba de esfericidad de Bartlett resultó altamente significativa. Los autovalores asociados a las dos primeras componentes dieron: 2,22 y 1,84; con un porcentaje acumulado de variancia del 68\%. El biplot o representación punto-vector nos proporcionó la información de las variancias y covariancias de las variables y las distancias entre las muestras de agua. El ACP permitió distinguir, según los coeficientes de la combinación lineal, que las variables Temperatura, Nivel Hidrométrico y Turbiedad (propiedades del agua) explican la primera componente principal mientras la cobertura de macrófitas es la variable que mejor define la segunda componente principal. El software utilizado fue R. }
