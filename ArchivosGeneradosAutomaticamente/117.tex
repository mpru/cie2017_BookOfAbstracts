\A
{ANALISIS ESTADISTICO DE PUNTAS LITICAS Y SU CORRESPONDENCIA CON DIFERENTES SISTEMAS DE ARMAS}
{\Presenting{ALICIA HERNÁNDEZ}$^1$\index{HERNÁNDEZ, A}, LILIANA A GARCÍA$^2$\index{GARCÍA, L}, RODRIGO VECCHI$^3$\index{VECCHI, R} y CRISTINA BAYÓN$^3$\index{BAYÓN, C}}
{\Afilliation{$^1$UTN FRBB - UNS}
\Afilliation{$^2$UNS}
\Afilliation{$^3$CONICET - UNS}
\\\Email{aliciahe@criba.edu.ar}}
{análisis multivariado; anova; comparaciones múltiples; puntas líticas; sistemas de armas; características técnicas} 
 {Otras ciencias sociales y humanas} 
 {Métodos multivariados} 
 {117} 
 {252-1}
{Las puntas líticas son el remanente perdurable de un artefacto conformado por varias tecnounidades, que sólo puede estudiarse de manera indiciaria. Para el abordaje de las puntas líticas, un buen punto de partida consiste en la realización de estudios tecno-morfológicos de artefactos procedentes de microrregiones y relacionar el análisis de los diseños con las asociaciones contextuales y cronológicas. La descripción y clasificación de las puntas líticas permiten reconocer ciertas tendencias o patrones generales. En la pampa bonaerense, las puntas líticas se han hallado en distintos sitios, desde las primeras ocupaciones (ca. 12.000 años AP) hasta momentos previos al contacto con los europeos, con características tecno-morfológicas variadas. La presencia de estos diferentes tipos de puntas líticas generó interés sobre los sistemas de armas y sus componentes y sobre las características tecno-tipológicas. El objetivo de este trabajo es analizar las puntas de proyectil (78 piezas) provenientes de ocho sitios de dicho sector a fin de caracterizarlas y determinar si provienen de diferentes sistemas de armas. En primer término, se registraron las variables métricas y en base a éstas se realizó la segmentación del conjunto en grupos mediante análisis estadístico, descriptivo e inferencial, aplicando métodos univariados y multivariados. Del análisis descriptivo de 78 piezas pudo observarse una variabilidad relativa superior al 10 \% para las variables longitud, ancho y espesor. Dado la heterogeneidad de las piezas, mediante análisis multivariado se las segmentó en cuatro grupos. Para determinar si los grupos presentan diferencias estadísticamente significativas para las variables métricas en estudio, se realizaron pruebas de comparación de medias, paramétricas o no paramétricas, según se cumpla o no el requisito de homocedasticidad. Las mismas resultaron altamente significativas (valor p<0,01), lo que permitió concluir que, para cada variable, al menos uno de los grupos difería de los demás. Posteriormente, se realizó la prueba T3 de Dunnet de comparaciones múltiples, para cada variable, concluyendo que todas las medias difieren entre sí (valor p < 0,01). Es decir que los tres grupos presentaron diferencias estadísticamente significativas. El análisis estadístico fue realizado mediante el software Infostat versión 2013. Como conclusión podría plantearse, inicialmente, que las diferencias entre los grupos estarían relacionadas con la presencia de diferentes sistemas de armas. }
