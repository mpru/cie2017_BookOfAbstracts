\A
{POBLACIONES OCULTAS: KIDS ONLINE, MUESTRA DE HOGARES CON MARCO IMPERFECTO}
{JUAN JOSÉ GOYENECHE\index{GOYENECHE, J}}
{\Afilliation{FACULTAD DE CIENCIAS ECONÓMICAS}
\\\Email{jjgoye@iesta.edu.uy}
}
{poblaciones ocultas; probabilidad desiguales de inclusión; estudio kids online} 
 {Otras aplicaciones} 
 {Muestreo} 
 {15} 
 {34-1}
{En muestreo los marcos imperfectos son la regla más que la excepción. En general hay un cierto grado de desajuste entre el marco y la población de estudio, lo que lleva a temas de sub- o sobre-cobertura, o problemas en las probabilidades de selección. La ausencia de marcos de lista nos lleva a usar alternativas para el estudio de poblaciones ocultas. Kids Online (Niños Conectados, en español) es un estudio realizado en varios países. El objetivo es estudiar como niños entre 9 y 17 años se relacionan con Internet, como y para que la usan, desde que dispositivos, los riesgos asociados con el uso y que oportunidades les da el acceso a la red. “El estudio Kids Online fue realizado por primera vez en países de la Unión Europea en el año 2006 ante la iniciativa de un equipo de académicos del London School of Economics (LSE). A la fecha continúa realizándose de manera regular en varios continentes. En América del Sur lo han realizado Brasil, Chile y Argentina, y este año por primera vez Uruguay” (tomado de la carta de presentación a las familias seleccionadas) En el caso que se presenta se desean estudiar hogares con niños en las edades de 9 a 17 años (Hogares KO). Estos hogares KO son parte de los incluidos en el Censo (aproximadamente un 25\% de los hogares tienen niños en esas edades), pero no se dispone de sus ubicaciones actuales. Se desarrolla un mecanismo para ubicar estos Hogares KO y se estudia la eficiencia del procedimiento y los posibles sesgos que implique. El procedimiento diseñado implica que las probabilidades de inclusión de los Hogares KO son desconocidas y deben ser aproximadas a partir de información censal.}
