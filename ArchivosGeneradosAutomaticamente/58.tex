\A
{EXPERIMENTOS DE SCREENING BASADOS EN FRACCIONES NO REGULARES}
{\Presenting{MAITE SAN MARTIN}\index{SAN MARTIN, M}, LUCIA HERNÁNDEZ\index{HERNÁNDEZ, L} y JOSE PAGURA\index{PAGURA, J}}
{\Afilliation{FACULTAD DE CIENCIAS ECONÓMICAS Y ESTADÍSTICA - UNIVERSIDAD NACIONAL DE ROSARIO}
\\\Email{mai.sanmartin@gmail.com}}
{experimentos a dos niveles; diseños factoriales; arreglos ortogonales} 
 {Industria y mejoramiento de la calidad} 
 {Diseño de experimentos} 
 {58} 
 {138-1}
{En muchas investigaciones, y en particular en las primeras etapas de las mismas, el objetivo se centra en determinar cuáles factores, de un gran grupo, afectan a una variable de interés. Los diseños factoriales completos a dos niveles, conocidos como diseños $2^k$, son planes experimentales que permiten estimar los efectos lineales asociados a k factores y todas las interacciones posibles entre dichos efectos. Sin embargo, su ejecución puede resultar muy costosa dado que, en general, requieren un gran número de pruebas. La alternativa tradicionalmente utilizada es la de llevar a cabo un diseño factorial fraccionario $2^{k-p}$, el cual consiste un subconjunto de los ensayos del diseño $2^k$ seleccionado apropiadamente. Estos diseños son construidos definiendo relaciones entre los efectos que conducen a estructuras de confusión simples (pares de efectos ortogonales o completamente confundidos). El número de tratamientos asociados a estos diseños es una potencia de 2, lo que resulta en cierta inflexibilidad en cuanto al número de pruebas a realizar. En los últimos años, los diseños factoriales fraccionarios no regulares han ganado popularidad en los experimentos de screening debido principalmente a su flexibilidad con respecto al número de ensayos que requieren y al número de niveles de los factores que pueden considerar. Estos diseños, a diferencia de los diseños $2^{k-p}$, pueden exhibir estructuras de confusión complejas donde los efectos pueden estar parcialmente confundidos introduciendo una complejidad adicional a su análisis. En este trabajo se presenta una clase particular de diseños factoriales fraccionarios no regulares llamados arreglos ortogonales y en particular a dos niveles y se enuncian algunos conceptos definidos en la literatura para estudiar sus propiedades. Además, se construyen diseños para distintos números de factores y de pruebas tanto mediante el enfoque tradicional como el de arreglos ortogonales con el objetivo de mostrar las ventajas y desventajas de cada clase.}
