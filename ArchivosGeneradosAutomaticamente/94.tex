\A
{INTEGRANDO LA ESTIMACIÓN BAYESIANA A LOS MODELOS DE ECUACIONES ESTRUCTURALES (SEM)}
{ANDREA DE LOS ÁNGELES CÉSPEDES SOLÍS\index{CESPEDES SOLÍS, A@CÉSPEDES SOLÍS, A}}
{\Afilliation{UNIVERSIDAD DE COSTA RICA. SAN JOSÉ – COSTA RICA}
\\\Email{ancelis.07@gmail.com}
}
{ecuaciones estructurales; cadenas de markov vía monte carlo; estadística bayesiana; distribuciones a priori; hiperparámetros} 
 {Otras ciencias sociales y humanas} 
 {Métodos bayesianos} 
 {94} 
 {210-1}
{El enfoque bayesiano es considerado una alternativa atractiva para la estimación de modelos que tiene información previa sobre un fenómeno de interés, así como un tamaño de muestra pequeño o problemas de convergencia en las estimaciones: recientemente las poderosas herramientas computacionales relacionadas con la simulación le han dado una gran aplicabilidad a esta técnica. Por otra parte los modelos de ecuaciones estructurales (SEM), conforman otra opción de análisis estadístico, su principal particularidad es generar una explicación que apoya o refuta un modelo definido bajo un marco teórico sólido. A pesar de que éstas técnicas son útiles ante determinadas situaciones, su aplicación de manera integrada es escasa en la literatura, por lo anterior, en el presente trabajo realiza la estimación de un modelo de ecuaciones estructurales (SEM) bajo el enfoque bayesiano, a partir de las especificaciones dadas por Sik-Yum Lee (2007) en “Structural Equation Modeling: A Bayesian Approach”. La técnica es aplicada a un modelo previamente estimado con el enfoque frecuentista, denominado la violencia contra las mujeres por parte de sus esposos o compañeros: Un modelo estructural desde el enfoque de género y la psicología evolucionaria, inicialmente propuesto propuesto por la Dra. Eiliana Montero Rojas de la Universidad de Costa Rica y el Dr. Aurelio José Figueredo de la Universidad de Arizona, USA, en el 2005, y replicado por la Dra. Eiliana Montero Rojas en conjunto con Br. Mauricio Campos Fernández, Br. Rebeca Sura Fonseca y Br. Andrea Céspedes Solís de la Universidad de Costa Rica en el 2016. Las estimaciones obtenidas en ambos casos son comparadas de tal manera que se dan a conocer los posibles beneficios al utilizar la estadística bayesiana en relación con la estimación convencional. Para la estimación del modelo SEM bayesiano, son asignadas a las distribuciones priori varianzas pequeñas, específicamente la distribución Gamma con parámetros $\alpha$=4 y $\beta$=9 y la Wishart. Además, se elaboran dos cadenas considerando 4 000 iteraciones de burn-in y 6 000 más para la estimación de los parámetros. Es importante mencionar que parte del conocimiento previo del modelo considerado para la definición de los hiperparámetros, es tomado de los resultados de la estimación frecuentista con la que ya se contaba. Para la estimación del modelo se utilizan los programas LISREL 8.80 y OpenBUGS 3.2.3. Finalmente, en la estimación de algunos de los parámetros, el resultado es similar en ambos enfoques. }
