\A
{EFECTO DE VALORES FALTANTES EN ESTUDIOS LONGITUDINALES EN ADULTOS MAYORES}
{FERNANDO MASSA\index{MASSA, F}}
{\Afilliation{INSTITUTO DE ESTADÍSTICA, UNIVERSIDAD DE LA REPÚBLICA}
\\\Email{fmassa@iesta.edu.uy}
}
{datos longitudinales; modelo conjunto; datos faltantes; deteriro cognitivo} 
 {Salud humana} 
 {Datos faltantes} 
 {179} 
 {366-1}
{Comprender el proceso de deterioro cognitivo global de los adultos mayores es de gran relevancia para una mejor planificación no sólo a nivel económica/familiar, repercutiendo de forma directa en los individuos afectados, sino también a nivel del sistema de salud y seguridad social de un país. Puesto que el deterioro se manifiesta a lo largo del tiempo, para su estudio son implementados modelos longitudinales. Sin embargo estos modelos tienen como desventaja que a lo largo del período de seguimiento, por diversas causas, algunos individuos desertan del estudio. Este abandono no sólo reduce el tamaño muestral sino que genera importantes sesgos en los resultados de los análisis estadíasticos. El objetivo de este trabajo fue modelar el deterioro cognitivo global de un conjunto de adultos mayores a lo largo del tiempo. Para llevar a cabo este objetivo se trabajó con datos provenientes del estudio Origins of Variance in the Old-old: Octagenarian Twins que contaba con una cohorte de 702 individuos entre 79 y 98 años de edad. Para medir el deterioro global se utilizó el Mini Mental State Examination (MMSE), evaluado a lo largo del tiempo en intervalos de dos años. Se implementaron modelos conjuntos puesto que estos parten de la base de que el tiempo de sobrevida de los sujetos y sus resultados de MMSE están relacionados. Se observó que los valores de MMSE son afectados tanto por la edad de los individuos al inicio del estudio como por el nivel educativo de los mismos. Incluir el análisis de sobrevida como parte del proceso de deterioro cognitivo corrige las inferencias. No existen trabajos previos en el Uruguay relacionados con la metodología desarrollada en este trabajo por lo que se considera que este trabajo representa una primera aproximación a la metodología a ser aplicada tanto para la planificación en el sistema de salud y seguridad social como para el sistema educativo e incluso para el contexto de trasplante de órganos.}
