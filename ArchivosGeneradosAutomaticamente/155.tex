\A
{ANÁLISIS ESPACIAL DE PARÁMETROS MAGNÉTICOS}
{\Presenting{MARCELA NATAL}$^1$\index{NATAL, M}, MAURO CHAPARRO$^2$\index{CHAPARRO, M}, LILA RICCI$^1$\index{RICCI, L}, MARCOS CHAPARRO$^3$\index{CHAPARRO, M} y DÉBORA MARIÉ$^4$\index{MARIE, D@MARIÉ, D}}
{\Afilliation{$^1$CENTRO MARPLATENSE DE INVESTIGACIONES MATEMÁTICAS (CEMIM-UNMDP), MAR DEL PLATA, ARGENTINA}
\Afilliation{$^2$CENTRO MARPLATENSE DE INVESTIGACIONES MATEMÁTICAS (CEMIM-UNMDP), MAR DEL PLATA, ARGENTINA. CONICET}
\Afilliation{$^3$CENTRO DE INVESTIGACIONES EN FÍSICA E INGENIERÍA DEL CENTRO DE LA PCIA. DE BUENOS AIRES. (CIFICEN, CONICET-UNCPBA), TANDIL, ARGENTINA}
\Afilliation{$^4$CENTRO DE INVESTIGACIONES EN FÍSICA E INGENIERÍA DEL CENTRO DE LA PCIA. DE BUENOS AIRES. (CIFICEN), TANDIL, ARGENTINA}
\\\Email{mnatal@mdp.edu.ar}}
{medidas de asociación; modelo geoestadístico; transformación gpot; contaminación ambiental; grillado regular} 
 {Ecología y medio ambiente} 
 {Estadística espacial} 
 {155} 
 {330-1}
{La modelización de datos espaciales es de gran interés en muchas áreas del conocimiento, entre ellas en las ciencias ambientales. Particularmente el estudio de la contaminación del aire en ambientes urbanos a través de técnicas del magnetismo ambiental ha sido ampliamente estudiado durante la última década. La ciudad de ciudad de Tandil, provincia de Buenos Aires tiene como una de las principales actividades económicas la industria metalúrgica existiendo 7 de las mismas emplazadas dentro del área urbana que pueden ser fuentes de contaminación. El objetivo de esta contribución es construir modelos geoestadísticos para analizar la contaminación atmosférica por medio de un índice de contaminación magnética (ICM) y parámetros magnéticos. El conjunto de datos de la ciudad de Tandil consiste de 180 muestras de líquenes a los que, además de su ubicación geográfica, se le realizaron determinaciones magnéticas relativas a concentración, mineralogía y tamaño del material magnético en laboratorio. La descripción espacial de diferentes zonas de estudio permite determinar fuentes de contaminación y representar gráficamente las zonas más afectadas por los contaminantes. Se diseñó una estrategia para construir una grilla regular a partir de los datos obtenidos dado que los puntos muestreados pertenecen a un grillado irregular. Para la construcción de los modelos se analizó normalidad y se determinaron los índices de Geary y Morán para corroborar la existencia de autocorrelación espacial. Además se determinaron puntos atípicos espaciales. Se utilizaron las transformaciones gpot, dado que las variables consideradas son asimétricas. El semivariograma experimental se ajustó por los modelos: exponencial, gaussiano, esférico y wave. Para la selección del mejor modelo se utilizó el método clásico de validación cruzada “leave-on-out” y sus estadísticos de diagnóstico: el error medio (ME), el error cuadrático medio (MSE) y el cociente de la desviación cuadrática media (MSDR). La predicción espacial fue realizada por medio de un kriging ordinario. Se determinó una grilla regular con una distancia aproximada de 200 mts, se obtuvo normalidad luego de la transformación gpot con un parámetro de 1.38, se detectaron tres puntos espaciales atípicos. A partir de los modelos ajustados se construyeron los mapas de predicción. En las representaciones, en general, se observó que las zonas más contaminadas están cercanas a las metalurgias. Para los cálculos y gráficos se utilizó el paquete geoR del software libre R. }
