\A
{MODELOS ADITIVOS GENERALIZADOS UNA APLICACIÓN EN EL CAMPO DEL MARKETING DIRECTO}
{\Presenting{RODRIGO LÓPEZ}\index{LOPEZ, R@LÓPEZ, R} y FERNANDA MENDEZ\index{MENDEZ, F}}
{\Afilliation{ITTAE. FACULTAD DE CIENCIAS ECONÓMICAS Y ESTADÍSTICA. UNIVERSIDAD NACIONAL DE ROSARIO}
\\\Email{fmendez@fcecon.unr.edu.ar}}
{modelo aditivo generalizado; suavizado; regresión logística; marketing directo} 
 {Otras ciencias económicas, administración y negocios} 
 {Modelos de regresión} 
 {11} 
 {18-1}
{Cada vez más empresas utilizan modelos predictivos para optimizar sus campañas de marketing directo; estos modelos, usualmente conocidos como modelos de propensión de compra, permiten estimar la probabilidad que tiene una persona de aceptar una determinada oferta, en base a un conjunto de características particulares. A través de los modelos de propensión de compra las empresas pueden orientar sus ofertas hacia aquellos individuos con mayor probabilidad de aceptación, incrementando las tasas de respuesta de las campañas y disminuyendo los costos asociados a las mismas. Tradicionalmente, los modelos de regresión logística han sido los más utilizados para modelar resultados binarios como es el caso de la respuesta a una campaña de marketing. Estos modelos poseen importantes virtudes: son capaces de combinar simplicidad con buena performance, son fáciles de interpretar y relativamente robustos. Los modelos de regresión logística imponen una forma paramétrica (muchas veces lineal) para el efecto de cada variable predictora, por lo que han sido criticados por fallar en la captación de relaciones no lineales más complejas que a menudo se presentan en la práctica, conduciendo en algunos casos a resultados no satisfactorios. Como alternativa, los modelos aditivos generalizados (GAM), introducidos por Hastie y Tibshirani (1986, 1990), omiten supuestos sobre la forma en que los predictores se relacionan con la respuesta y de esta manera proveen la capacidad de detectar relaciones más o menos complejas que de otro modo se perderían. Estos modelos utilizan funciones no paramétricas para describir el efecto de cada predictor, permitiendo un ajuste más flexible que el que se puede obtener con los modelos de regresión logística. En este trabajo se presentan los modelos aditivos generalizados y su aplicación en el campo del marketing directo. Se utilizan datos reales de una campaña de marketing de un reconocido banco nacional, con el objetivo de mostrar la utilidad de los GAM como alternativa y/o complemento de otros métodos tradicionales como la regresión logística. La comparación realizada entre los GAM y los modelos de regresión logística no pretende ser una comparación exhaustiva entre las dos metodologías. Han quedado sin considerar características importantes tales como robustez, estabilidad, tiempos de procesamiento, etc., que podrán incluirse en posteriores estudios. Del mismo modo sucede con ciertos aspectos propios de los GAM, han quedado pendientes análisis más profundos sobre la selección de los parámetros de suavizado del modelo así como la sensibilidad de los GAM ante la elección de distintos métodos de suavizado.}
