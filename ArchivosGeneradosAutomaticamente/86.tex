\A
{ANÁLISIS DE LA INFORMACIÓN FINANCIERA Y DE MERCADO EN EMPRESAS LATINOAMERICANAS A TRAVÉS DE LA APLICACIÓN DE MODELOS MIXTOS}
{\Presenting{LETICIA TOLOSA}\index{TOLOSA, L} y MARÍA PAULA ROJO\index{ROJO, M}}
{\Afilliation{UNIVERSIDAD NACIONAL DE CÓRDOBA-FACULTAD DE CS.ES.}
\\\Email{leticiaetolosa@gmail.com}}
{modelos mixtos; indicadores contables; indicadores de mercado; retornos trimestrales} 
 {Otras ciencias económicas, administración y negocios} 
 {Modelos de regresión} 
 {86} 
 {192-1}
{Resumen Ejecutivo ¿Cuáles son los indicadores que explican los retornos de las acciones en el mercado de capitales? Según Chaopricha (2007) la respuesta a esta pregunta ha sido muy buscada por los investigadores e inversionistas, y la mayoría de los estudios hasta la fecha son de naturaleza empírica y proporcionan una visión muy limitada del problema. Las conclusiones generales son, que ciertas características de la empresa sí tienen un efecto en las existencias de retornos. Entre estas características, las más estudiadas incluyen relación de precio-valor libro, tamaño y precio-ganancia. En el presente trabajo se realiza, en una primera etapa, un estudio descriptivo del mercado latinoamericano a los fines de mostrar la situación relativa de cada país, con el fin de identificar las particularidades sobre cantidad de empresas cotizantes y capitalización bursátil de Argentina, Chile, Perú, Colombia y Brasil. A continuación en una segunda etapa, a través de la aplicación de modelos estadísticos, se analizan datos de mercado e información contable de empresas no financieras que conforman los índices bursátiles más representativos de los países mencionados anteriormente. El horizonte temporal de análisis es 2011 a 2016. Para explicar la variación de precios de las acciones en el período mencionado, el panel de datos conformado de cada país se analiza a través de la aplicación de modelos lineales mixtos (Tolosa, 2013). Los resultados obtenidos de acuerdo a los datos muestrales, evidencian que para explicar la variación de precios de las acciones en el mercado, resulta significativo y con relación directa la Rentabilidad del Patrimonio Neto (RPN) en el mercado de Brasil, Perú, y Chile, aunque el ratio de Solvencia (RSOL) solo resulta significativo en forma directa en Brasil. En el mismo sentido el cociente entre el Precio y el Valor de Libros (PVL), resultan significativo para los mercados de Argentina, Brasil y Chile El tamaño de las empresas medido a través de la Capitalización Bursátil (CBU) es significativo de manera inversa en Argentina y Colombia. El Año resulta significativo para Argentina y Chile considerado como efecto fijo y en todos los modelos se lo incluye como efecto aleatorio. El ratio precio – ganancia (PE) no aparece como significativo en ninguno de los mercados analizados en este espacio temporal Al no resultar estadísticamente significativos indicadores comunes para las empresas domésticas de cada país, es de interés el análisis de las diferencias observadas, con el resultado del análisis descriptivo realizado en la primera parte del trabajo. }
