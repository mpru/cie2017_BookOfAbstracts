\A
{EVALUACIÓN DE FACTORES INFLUYENTES EN LA PRESIÓN ARTERIAL EN PACIENTES HIPERTENSOS UTILIZANDO UN MODELO LINEAL MIXTO}
{\Presenting{MARÍA DEL CARMEN GARCÍA}$^1$\index{GARCÍA, M D C}, NOELIA CASTELLANA$^1$\index{CASTELLANA, N}, MARA CATALANO$^1$\index{CATALANO, M}, CECILIA RAPELLI$^1$\index{RAPELLI, C} y CAROLINA CHACÓN$^2$\index{CHACÓN, C}}
{\Afilliation{$^1$INSTITUTO DE INVESTIGACIONES TEÓRICAS Y APLICADAS. ESCUELA DE ESTADÍSTICA / FACULTAD DE CS ECONÓMICAS Y ESTADÍSTICAS (UNR), ARGENTINA}
\Afilliation{$^2$FUNDACIÓN ECLA, ARGENTINA}
\\\Email{mgarcia@fcecon.unr.edu.ar}}
{hipertensión arterial; adherencia al tratamiento farmacológico; modelo lineal mixto; elección de covariabes} 
 {Otras ciencias de la salud} 
 {Modelos de regresión} 
 {130} 
 {291-1}
{La hipertensión arterial, que consiste en el aumento de la presión arterial, es una de las principales causas de las enfermedades cardiovasculares en el mundo y sus complicaciones son responsables de millones de muertes anuales. A pesar de que la hipertensión arterial es un factor de riesgo modificable y que existe una gran cantidad de medicamentos efectivos para su tratamiento, esta enfermedad crónica sigue teniendo una alta prevalencia asociada a altas tasas de desconocimiento y a un mal control de los valores de presión arterial en el tiempo (valores por encima de los óptimos). Esta falta de control adecuado obedece a diversas razones, entre las que se destacan los hábitos poco saludables y la mala adherencia terapéutica, es decir, el paciente no cumple con el tratamiento indicado. El objetivo de este trabajo es evaluar el efecto del grado de adherencia al tratamiento farmacológico y de otras variables sobre la presión arterial sistólica de un grupo de pacientes hipertensos que forman parte de un programa de atención y control de la hipertensión iniciado en el año 2014 en Rosario. Este programa contempló un seguimiento exhaustivo de los mismos en el tiempo, registrándose tanto características basales como los valores de presión arterial en cada una de las visitas realizadas. A partir del año 2016, con la incorporación del cuestionario validado de Morisky-Green, se comenzó a medir la adherencia al tratamiento farmacológico. Como consecuencia de esto, los pacientes que ya participaban del programa no cuentan con esta información previo a esta fecha. Una herramienta valiosa para el análisis de este tipo de datos son los modelos lineales mixtos que permiten modelar las múltiples mediciones de la presión arterial sistólica en función de las covariables y la correlación entre las mismas. El análisis de los resultados obtenidos permite evidenciar que el valor de la presión arterial sistólica basal, el tipo de tratamiento, el sexo del paciente y el grado de adherencia al tratamiento farmacológico influyen en los valores medios de la presión arterial.}
