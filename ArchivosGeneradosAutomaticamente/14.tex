\A
{SEGUIMIENTO DE PACIENTES TRATADOS POR UNA INFECCIÓN PARASITARIA CRÓNICA: REVISIÓN SISTEMÁTICA Y METAANÁLISIS DE PARTICIPANTES INDIVIDUALES}
{\Presenting{KAREN ROBERTS}$^1$\index{ROBERTS, K}, GUILLERMINA HARVEY$^2$\index{HARVEY, G}, CRISTINA CUESTA$^2$\index{CUESTA, C}, YANINA SGUASSERO$^1$\index{SGUASSERO, Y}, AGUSTÍN CIAPPONI$^3$\index{CIAPPONI, A}, SERGIO SOSA-ESTANI$^4$\index{SOSA-ESTANI, S} y GRUPO DE COLABORADORES}
{\Afilliation{$^1$CENTRO ROSARINO DE ESTUDIOS PERINATALES}
\Afilliation{$^2$FACULTAD DE CIENCIAS ECONÓMICAS Y ESTADÍSTICA, UNR}
\Afilliation{$^3$INSTITUTO DE EFECTIVIDAD CLÍNICA Y SANITARIA}
\Afilliation{$^4$INSTITUTO NACIONAL DE PARASITOLOGÍA FATALA CHABEN-ANLIS / CONSEJO NACIONAL DE INVESTIGACIONES CIENTÍFICAS Y TÉCNICAS}
\\\Email{robertskarenn@gmail.com}}
{metaanálisis de datos de pacientes individuales; tasa de supervivencia; análisis de regresión; trypanosoma cruzi; enfermedad crónica} 
 {Salud humana} 
 {Datos de duración} 
 {14} 
 {31-1}
{La enfermedad de Chagas es una infección endémica de América Latina causada por el parásito Trypanosoma cruzi (T. cruzi). En la fase crónica de la enfermedad, el éxito del tratamiento es incierto porque la seronegativización (desaparición de anticuerpos contra T. cruzi) puede tardar décadas. Con el propósito de conocer más acerca del tiempo hasta la seronegativización, se realizó una revisión sistemática con meta-análisis de datos de participantes individuales (N° PROSPERO: CRD42012002162). Se incluyeron estudios de cohortes e ICAs. Las bases de datos originales de los pacientes individuales fueron solicitadas a los investigadores principales de cada estudio. Los resultados primarios se dicotomizaron como positivos o negativos para las pruebas de inmunoabsorción enzimática (ELISA), inmunofluorescencia indirecta (IIF) y análisis de hemaglutinación indirecta (IHA). Se estimaron las funciones de supervivencia y de riesgo aplicando el método de Kaplan-Meier, y se ajustaron modelos de regresión de riesgos proporcionales de Cox. A los fines de considerar la variabilidad entre las distintas poblaciones, se consideró un efecto aleatorio asociado a cada estudio. El exceso de riesgo o fragilidad para distintos estudios se describió utilizando el componente de variancia del efecto aleatorio y el correspondiente coeficiente Tau de Kendall. Los análisis de subgrupos se basaron en edad al tratamiento (menor vs mayor de 19 años) y región donde se adquirió la infección (Argentina, Bolivia, Chile, Paraguay con predominio de cepa TcV vs Brasil con predominio de cepa TcII). Los objetivos del trabajo fueron determinar, mediante la aplicación de métodos estadísticos apropiados, la evolución de las pruebas serológicas en sujetos con infección crónica por T. cruzi tratados con fármacos tripanocidas y describir el efecto de los factores explicativos sobre la seronegativización. Se incluyeron 27 estudios y se obtuvieron datos de 1.296 sujetos. El riesgo de sesgo fue bajo en 17 estudios (63\%). El valor del coeficiente Tau de Kendall fue de 0.63, 0.58 y 0.47 para ELISA, IIF y IHA, respectivamente. Basados en el conjunto de datos disponible, los hallazgos muestran una interacción entre la edad al tratamiento y la región. En la región TcII la posibilidad de seronegativización fue mayor en los sujetos tratados entre 1-19 años comparados con adultos; HR 6.60 (IC 95\% 2.03-21.42) para ELISA, HR 9.37 (3.44-25.50) para IIF y HR 5.55 (1.46-21.11) para IHA.}
