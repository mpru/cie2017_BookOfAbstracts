\A
{ESTUDIO DEL IMPACTO REDISTRIBUTIVO DE LA ASIGNACIÓN UNIVERSAL POR HIJO EN ENTRE RÍOS. UNA APLICACIÓN DE LA METODOLOGÍA DE DESCOMPOSICIÓN DEL COEFICIENTE DE GINI}
{\Presenting{STEFANIA D'IORIO}\index{D'IORIO, S} y OLGA AVILA\index{AVILA, O}}
{\Afilliation{FACULTAD DE CIENCIAS ECONOMICAS - UNIVERSIDAD NACIONAL DE ENTRE RIOS}
\\\Email{stefaniadiorio@gmail.com}}
{impacto redistributivo; asignación universal por hijo; metodología de descomposición del coeficiente de gini} 
 {Economía} 
 {Otras categorías metodológicas} 
 {111} 
 {240-1}
{La desigualdad en la distribución del ingreso se ha convertido en un grave problema de los argentinos desde el comienzo de la última dictadura militar, en 1976. Este problema responde a la interacción entre distintos factores -diferencias salariales, niveles de empleo, instituciones laborales, políticas sociales, condiciones macroeconómicas, entre otros-. En este marco, los estudios de las políticas públicas sobre las distintas fuentes de ingreso y su impacto sobre la desigualdad resultan relevantes. Este trabajo tuvo como objetivo estimar el impacto distributivo de la Asignación Universal por Hijo (AUH) en la desigualdad del ingreso en los aglomerados de Gran Paraná y Concordia, mediante la aplicación de una metodología de descomposición del coeficiente de Gini por fuentes de ingresos, entre el 3er trimestre de 2009 -trimestre anterior a su puesta en vigencia-, hasta el 2do trimestre de 2015. Mediante la metodología de descomposición propuesta por Lerman y Yitzhaki (1985) se buscó ahondar en la desigualdad total, mediante el cálculo de la participación relativa de la fuente transferencias en el ingreso total, la desigualdad observada en dicha fuente, y la correlación entre ésta y el ingreso del hogar. También, con el cálculo de la elasticidad-Gini del ingreso propuesto por Wodon y Yitzhaki (2002), se evaluó cómo un cambio marginal en la composición del ingreso por transferencias modificó la desigualdad total, de manera de cuantificar el aporte de esta fuente en la distribución del Ingreso Per Cápita Familiar en la dinámica que este último ha seguido desde la implementación de la AUH. Esta investigación contribuyó al estudio de la desigualdad a nivel subprovincial, limitándose a cuantificar el aporte de las transferencias en la variación de la desigualdad total y su evolución en los aglomerados mencionados, sin intenciones de establecer de manera definitiva las causas de los cambios en la desigualdad del ingreso. Se observó un aumento del impacto redistributivo de esta fuente en la desigualdad total en ambos aglomerados debido a la AUH: la variación en el coeficiente de Gini debido a las transferencias aumentó en el trimestre de su implementación, en los posteriores a cada aumento del monto, y entre las puntas del período analizado, con mayor énfasis en Concordia. Pero debido a su baja participación en el ingreso total se concluye que, en la medida en que los objetivos de este tipo de políticas busquen mejorar la distribución del ingreso, serán precisos mayores esfuerzos fiscales para incrementar el monto de la transferencia.}
