\A
{ESTUDIO DE LOS “NI-NI” DEL GRAN CÓRDOBA A TRAVÉS DE UN MODELO DE REGRESIÓN LOGÍSTICA}
{\Presenting{NANCY STANECKA}\index{STANECKA, N}, MACARENA CANCINOS\index{CANCINOS, M} y PATRICIO CANALIS\index{CANALIS, P}}
{\Afilliation{FAC. DE CIENCIAS ECONÓMICAS. UNC}
\\\Email{nstanec@gmail.com}}
{regresión logística; joven urbano; jóvenes “ni-ni”} 
 {Otras ciencias sociales y humanas} 
 {Modelos de regresión} 
 {158} 
 {335-1}
{Si bien la trayectoria juvenil hacia la adultez se observa cada vez más heterogénea y compleja, es esperable que en esta etapa de la vida los jóvenes transiten por ciertos eventos claves que incluyen, entre otros, la finalización de los estudios y/o el ingreso al ámbito laboral. No obstante, los registros censales permiten afirmar que en el año 2010 el 16,7\% de los habitantes de la provincia de Córdoba, que tenían entre 15 y 29 años, no trabajaba ni estudiaba. Este fenómeno, de jóvenes que ni trabajan ni estudian, es conocido con el nombre de la generación de los “ni-ni”, pudiendo pensarse la situación como de alto riesgo social. En este marco, el siguiente estudio presenta una caracterización general de los jóvenes e indaga sobre factores socio-demográficos que inciden en la condición de “ni-ni” de aquellos que habitan el aglomerado urbano Gran Córdoba. Para ello se recurre a la información que brinda el Censo Nacional de Población, Hogares y Vivienda 2010 y la Encuesta Permanente de Hogares (EPH) del último semestre del mismo año. En una primera etapa se realiza un análisis descriptivo de la población en estudio y posteriormente se utiliza un modelo de regresión logística binaria donde la variable respuesta indica si el individuo presenta el status “ni-ni“, considerando como variables predictoras a factores socio-demográficos correspondientes al joven y socio-económicos del hogar en el que el mismo habita. Este último análisis nos permite identificar como variables estadísticamente significativas en la condición de “ni-ni”, al sexo del joven, su nivel educativo y ciertas necesidades básicas insatisfechas del hogar donde él vive, entre otras. }
