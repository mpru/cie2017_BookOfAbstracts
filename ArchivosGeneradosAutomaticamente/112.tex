\A
{LA POBREZA E INDIGENCIA MULTICAUSAL: UNA PROPUESTA METODOLÓGICA PARA LA PROVINCIA DE MISIONES}
{\Presenting{SILVANA DEA LABAT}\index{LABAT, S} y DARIO EZEQUIEL DÍAZ\index{DIAZ, D@DÍAZ, D}}
{\Afilliation{INSTITUTO PROVINCIAL DE ESTADÍSTICA Y CENSOS DE LA PROVINCIA DE MISIONES}
\\\Email{slabat24@hotmail.com}}
{pobreza multicausal; medición; bienestar; ingresos} 
 {Economía} 
 {Otras categorías metodológicas} 
 {112} 
 {244-1}
{Resumen En los últimos años, la pobreza e indigencia se han vuelto un fenómeno complejo para lograr una medición confiable, válida y lo más objetiva posible. En la literatura académica existe una multiplicidad de enfoques y abordajes con el objetivo concreto de pretender comprender lo que implica ser pobre, para evitar la subestimación en el recuento de los mismos. Recientemente, la “multidimensionalidad” de la pobreza ha alcanzado un profundo reconocimiento, pues se divisa una inquietud en adoptar metodologías con posibilidad de diferenciar tópicos de índole social o nuevas dimensiones humanas. Generalmente, en la literatura, la pobreza se ha vinculado de manera habitual con los conceptos de necesidad estándar de vida, o insuficiencia de recursos, pues se trata de atributos sobre los que pueden obtenerse medidas relativamente aceptables. En tanto, el marco teórico que se utilice y su correspondiente metodología están estrechamente relacionados con los objetivos de la investigación que se esté llevando a cabo y sus resultados pueden contribuir a identificar aspectos particulares de la pobreza. Aun cuando existe una gran pluralidad de aproximaciones teóricas para reconocer qué hace pobre a un individuo, hay un acuerdo cada vez más amplio sobre la naturaleza multidimensional de este concepto, el cual afirma que los elementos que toda persona precisa para tomar una decisión de manera libre, informada y con igualdad de oportunidades sobre sus alternativas vitales, no pueden ser reducidos a una sola de las características o dimensiones de su existencia. Los abordajes alternativos para medir la pobreza, el nivel de vida y el desarrollo que parten del rechazo explícito a la posibilidad de hallar un patrón de medición única y universal, se transforman inevitablemente en enfoques multidimensionales. El presente trabajo tiene como objetivo primordial presentar la metodología utilizada por el Instituto de Estadística y Censos de la Provincia de Misiones para el diseño del cálculo del Índice de Pobreza Multicausal para la Provincia de Misiones. El Índice de Pobreza e Indigencia Multidimensional se refiere a una estructura metodológica que busca dar respuesta a un problema a partir de la identificación de un conjunto de causas posibles que lo generan. El mismo surge del promedio ponderado de tres métodos: • Método Indirecto Absoluto (MIA) • Método Indirecto Relativo (MIR) • Método Directo de Satisfacción de Necesidades Básicas (MDS) La ponderación de esta tasa de pobreza e indigencia multicausal fue la siguiente: 25\% de MIA, 25\% MIR y 50\% MDS.}
