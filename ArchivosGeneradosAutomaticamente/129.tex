\A
{R-LADIES: MEJORANDO LA DIVERSIDAD DE GÉNERO EN LA COMUNIDAD R}
{\Presenting{NATALIA DA SILVA}$^1$\index{DA SILVA, N} y LAURA ACIÓN$^2$\index{ACIÓN, L}}
{\Afilliation{$^1$IESTA, UDELAR}
\Afilliation{$^2$UBA}
\\\Email{ndasilva@iastate.edu}}
{comunidad r; diversidad de género; software libre; talleres de r; brecha de género; mujeres liderando} 
 {Otras aplicaciones} 
 {Otras categorías metodológicas} 
 {129} 
 {285-2}
{R es un lenguaje de programación de código abierto, está desarrollado por la comunidad y es ampliamente utilizado en educación, investigación, y la aplicación de la estadística tanto en la academia como la industria. De acuerdo a resultados presentados en la conferencia Mujeres en Estadística y Ciencia de Datos 2016 (organizada por la Asociación Americana de Estadística), las mujeres y todas las personas que se identifican significativamente como mujer están subrepresentadas dentro de la comunidad R. Aproximadamente, sólo 17\% de los paquetes de R son creados y/o mantenidos por mujeres. Además, de los integrantes ordinarios de la R Foundation, el organismo que apoya el desarrollo de R, sólo 13.5\% son mujeres. Este último porcentaje superó la barrera del 10\% sólo recientemente. Finalmente,  el equipo de liderazgo de R cuenta sólo con una mujer entre sus 7 integrantes. R-Ladies surge como una iniciativa para revertir la subrepresentación de las mujeres en la comunidad R. El objetivo de R-Ladies es mejorar la participación femenina no sólo como usuarias, desarrolladoras, y oradoras en conferencias, sino también como líderes en la comunidad. R-Ladies es una organización internacional que incentiva la diversidad de género en la comunidad R a través de grupos de usuarios en distintas ciudades. R-Ladies organiza reuniones presenciales, actividades virtuales y mentorías en un entorno informal y amigable. Organizaciones similares en ciencias, tecnología, ingeniería y matemática muestran que este tipo de iniciativas ayudan a mejorar la representación de las minorías en estos ámbitos. El primer grupo de R-Ladies se organizó en Octubre 2012. En 2016 surgió la organización madre R-Ladies Global a partir de la cual R-Ladies se expandió a más de 45 grupos en ciudades de más de 20 países. R-Ladies cuenta actualmente con más de 6000 integrantes. Cada grupo reúne a mujeres y hombres tanto de la academia como de la industria. Los integrantes provienen de diferentes disciplinas, tienen diferentes grados de conocimiento de R y están en diferentes etapas de su carrera. En esta charla, presentaremos estadísticas de los grupos de R-Ladies en Uruguay y Argentina. También mostraremos estadísticas sobre el desarrollo de R-Ladies en América Latina y en el resto del mundo. Además, compartiremos nuestras experiencias relacionadas con la logística de la gestión de grupos locales. Presentaremos historias de éxito en todo el mundo, algunos de los desafíos que se enfrentan y cómo hacer para sumarse a un grupo existente o para crear uno nuevo. }
