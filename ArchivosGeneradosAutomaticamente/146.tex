\A
{EFICIENCIA RELATIVA DE DISEÑOS EXPERIMENTALES USADOS PARA ESTIMAR REPUESTA A LA FERTILIZACIÓN EN AGRICULTURA DE PRECISIÓN}
{\Presenting{PABLO PACCIORETTI}\index{PACCIORETTI, P}, MARIANO CÓRDOBA\index{CORDOBA, M@CÓRDOBA, M} y MÓNICA BALZARINI\index{BALZARINI, M}}
{\Afilliation{UNIVERSIDAD NACIONAL DE CÓRDOBA - CONSEJO NACIONAL DE INVESTIGACIONES CIENTÍFICAS Y TÉCNICAS}
\\\Email{pablopaccioretti@agro.unc.edu.ar}}
{ensayos; regresión; simulación; estratificación} 
 {Ciencias agropecuarias} 
 {Diseño de experimentos} 
 {146} 
 {313-1}
{Conocer la respuesta de los cultivos a la fertilización es crucial para optimizar económica y ambientalmente el uso de recursos en agricultura. Esta función puede estimarse a partir de ensayos conducidos con tecnologías de precisión, que hacen posible obtener numerosas repeticiones de la respuesta del cultivo a distintas dosis de fertilización y asociar a cada medición la información sobre su posición en el terreno (georreferencia). Así, se han comenzado a implementar en Argentina, diseños experimentales en campos de productores que incluyen numerosas unidades experimentales (parcelas) aleatorizadas o estratificadas en un número alto de bloques con o sin delimitación de áreas homogéneas dentro del lote. Otro diseño utilizado es el arreglo de parcelas en franjas que atraviesan la variabilidad del lote. La respuesta al interrogante sobre que diseño de experimento es el más adecuado para estimar funciones de rendimiento, no está del todo clara. El objetivo de este trabajo es evaluar el desempeño de arreglos de parcelas según un diseño completamente aleatorizado (DCA), en bloques completamente aleatorizado (DBCA) y alternativamente en franjas. La comparación se realizó en base a 600 simulaciones de ensayos de fertilización con una estructura de correlación espacial cuyos parámetros fueron determinados a partir de ensayos de fertilización nitrogenada en cultivo de maíz. En cada realización se adicionó un efecto estadísticamente significativo de zona (área homogénea) con dos niveles. El rendimiento se estimó a partir de una función conocida, cuyos parámetros se fijaron a través de distribuciones de probabilidad estimadas con datos empíricos. Para cada diseño se ajustaron modelos de regresión lineal, con o sin efecto zona y/o contemplando la estructura de correlación espacial del término de error. El DBCA fue el diseño más eficiente tanto en modelos de análisis que incorporaron el efecto de la zona como en aquellos donde se omite este efecto, situación donde la eficiencia relativa del DBCA fue de hasta 5 veces mayor respecto a los otros diseños.}
