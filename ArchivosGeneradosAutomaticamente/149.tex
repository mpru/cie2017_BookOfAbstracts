\A
{INGRESANTES DE LA UNIVERSIDAD NACIONAL DE ROSARIO: PERFIL SOCIO-ECONÓMICO Y RENDIMIENTO ACADÉMICO}
{\Presenting{DIEGO MARFETÁN MOLINA}\index{MARFETÁN MOLINA, D}, MARÍA EUGENIA TESSER\index{TESSER, M}, BEATRIZ LUCÍA MEINARDI\index{MEINARDI, B} y NERINA LÓPEZ\index{LOPEZ, N@LÓPEZ, N}}
{\Afilliation{DIRECCIÓN GENERAL DE ESTADÍSTICA UNIVERSITARIA - UNR}
\\\Email{estadistica@unr.edu.ar}}
{cluster; rendimiento académico; perfil; ingresantes; unr} 
 {Educación, ciencia y cultura} 
 {Métodos multivariados} 
 {149} 
 {318-1}
{El estudio y la correcta interpretación de los perfiles sociales de los alumnos es una herramienta indispensable a la hora de lograr una eficiente gestión académica a nivel universitario. En este sentido, conocer las características de los propios estudiantes se convierte en un aspecto fundamental del diseño de cualquier plan estratégico, influenciando la toma de decisiones y definiendo las políticas a seguir por parte de las autoridades universitarias. Si bien este tema ha sido tratado con anterioridad en la Universidad Nacional de Rosario (UNR), principalmente en Moscoloni et al. (2007), el contexto social general y las realidades de los estudiantes de la UNR poseen una naturaleza dinámica que vuelve inevitable la permanente actualización de estos perfiles. En consecuencia, el principal objetivo de este trabajo es identificar grupos (clusters) de estudiantes con características similares, y al mismo tiempo analizar si estos atributos se encuentran correlacionados con el posterior rendimiento académico del alumno. En el presente trabajo se tiene en cuenta la cohorte de estudiantes ingresantes en el año académico 2014 a carreras de grado y pregrado dictadas en la UNR. Las variables analizadas están referidas a datos personales y al rendimiento académico de los alumnos, obtenidas a partir de la información más actualizada y confiable a disposición de la Dirección General de Estadística Universitaria de la UNR. En un primer paso se transformaron las variables cualitativas a indicadoras, a partir de las cuales se construye una matriz de distancias entre alumnos utilizando el coeficiente de similitud definido por Gower (1971). Posteriormente se aplicó una técnica conocida como análisis de cluster jerárquico, la cual identifica grupos de estudiantes con características similares y los agrupa progresivamente en diversos clusters. Para llevar a cabo estos análisis se utiliza la librería MASS (Venables \& Ripley, 2002) disponible en el paquete estadístico R. Los resultados del presente trabajo indican que, en líneas generales, no existe un perfil socio-económico claramente definido que pueda ser asociado al rendimiento académico del alumno. No obstante, existen tendencias que sugieren una relación entre la deserción universitaria, la edad del ingresante y el área de estudio dentro de la cual se encuadra la carrera elegida. Además, variables referidas a la categoría ocupacional y al máximo nivel de estudios alcanzado por los padres también ejercen cierta influencia sobre el rendimiento académico de los alumnos de la UNR. }
