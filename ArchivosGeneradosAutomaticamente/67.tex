\A
{ANÁLISIS DE LA EFICIENCIA DE DIFERENTES DOSIS DE ENZIMAS CON MODELOS LINEALES GENERALIZADOS EN ENSAYOS DE NUTRICIÓN EN POLLOS PARRILLEROS}
{\Presenting{LUCIANO PALACIOS}$^1$\index{PALACIOS, L}, MARÍA LAURA VIGNERA$^1$\index{VIGNERA, M}, MAURICIO DE FRANCESCHI$^2$\index{DE FRANCESCHI, M} y SUSANA FILIPPINI$^1$\index{FILIPPINI, O}}
{\Afilliation{$^1$DIVISIÓN ESTADÍSTICA, DPTO CS BÁSICAS, UNIVERSIDAD NACIONAL DE LUJÁN}
\Afilliation{$^2$AVICULTURA, DPTO TECNOLOGÍA, UNIVERSIDAD NACIONAL DE LUJÁN}
\\\Email{lucianofepalacios@hotmail.com}}
{enzimas; modelos lineales generalizados; devianzas; bonferroni secuencial} 
 {Ciencias agropecuarias} 
 {Modelos de regresión} 
 {67} 
 {158-1}
{El empleo de modelos matemáticos para la explicación de fenómenos probabilísticos ha sido imprescindible en la investigación científica. No obstante, es frecuente tener que trabajar con variables que no cumplen con las características requeridas por el Modelo Lineal General, siendo el modelo Lineal generalizado, el que responde adecuadamente a los problemas generados por la métrica de las variables. El objetivo de este trabajo es modelar la eficiencia de diferentes dietas de nutrición en pollos parrilleros, a través del uso del indicador FEP (Factor de Eficiencia de Producción), cuyo cálculo es el siguiente: FEP es igual al producto entre el Peso y la Viabilidad de grupo de animales, dividido por el producto entre la Conversión alimenticia y la Edad del lote, todo multiplicado x 100. El criterio para dicotomizar los resultados de este índice se llevó a cabo, estipulando como punto de corte el valor teórico de FEP=390 que debería obtener un grupo de pollos parrilleros a los 42 días de edad, 2.953 kg de peso y 1.69 de conversión alimenticia. Estos parámetros son prestablecidos por el proveedor de la línea genética Cobb a través del Suplemento Informativo Sobre Rendimiento Y Nutrición De Pollos De Engorde Cobb500. Se observa la falta de ajuste de los residuos a la media, es decir el no cumplimiento de la independencia de las observaciones. Se prueban diferentes modelos Lineales Generalizados de dosis respuesta y funciones de enlace como Probits, Logit y Log- log complementario. Se emplea el análisis de las Devianzas de los distintos modelos, como una generalización del Análisis de la Varianza para una secuencia de modelos, cada uno incluyendo más términos que los anteriores-Se comparan las bondades de ajustes con el Criterio de Información de Akaike. Se detectan diferencias significativas entre los tratamientos que poseen diferentes dosis de enzimas fitasas a partir de la comparación de medias con Bonferroni secuencial. Se concluyó que el mejor tratamientos es el que posee una dosis de enzima fitasa a razón de 2500 FTU con diferencias estadísticas sobre las dietas que no poseen. }
