\A
{DETERMINANTES DE LA INNOVACION UN ANÁLISIS A NIVEL DE FIRMA POR TAMAÑO Y SECTOR}
{\Presenting{VERÓNICA ARIAS}\index{ARIAS, V}, LAURA ISABEL LUNA\index{LUNA, L}, MARÍA INÉS AHUMADA\index{AHUMADA, M} y NORMA PATRICIA CARO\index{CARO, N}}
{\Afilliation{FACULTAD DE CIENCIAS ECONÓMICAS, UNIVERSIDAD NACIONAL DE CÓRDOBA.}
\\\Email{veroarias1@gmail.com}}
{innovación; modelo logístico mixto; industria manufacturera; provincia de córdoba} 
 {Otras ciencias económicas, administración y negocios} 
 {Datos categóricos} 
 {52} 
 {114-1}
{A nivel de empresa, la innovación conduce a un uso más eficiente de los recursos, generando así ventajas competitivas sostenibles en el tiempo. Se reconoce a la innovación como uno de los factores claves de la competitividad y el crecimiento, por lo cual existe gran interés en comprender las variables que la afectan. La “Encuesta sobre innovación y conducta tecnológica de la Provincia de Córdoba” llevada a cabo por la Dirección General de Estadística y Censos, es fuente de información y los datos que provee son las observaciones y mediciones presentadas por cada empresa a través del tiempo, logrando conformar un panel de datos para este estudio. Estos cuentan con una estructura jerárquica (empresas medidas repetidamente en el tiempo), la que introduce dependencia en las respuestas múltiples dentro de cada unidad. El presente trabajo tiene como objetivo generar un aporte a la comprensión del proceso de innovación en las empresas industriales manufactureras de la provincia de Córdoba para el período 2011 a 2014, incorporando nuevas dimensiones de análisis a los resultados obtenidos en trabajos previos sobre los principales determinantes de la innovación en las empresas. En esta oportunidad de análisis se introducen variables de clasificación de las empresas según el sector al que pertenecen y el tamaño de las mismas. En el primer caso se considera la clasificación realizada por agrupación sectorial a saber: intensivas en I+D, escala, trabajo y en recursos naturales. Para la clasificación por tamaño se realizó una división entre empresas grandes y PyMes. Luego, se buscaron los principales determinantes de la innovación ajustando modelos logísticos mixtos, donde la variable respuesta considera si la empresa logró nuevos productos y/o procesos o bien mejoras significativas. Mientras que los indicadores vinculados a la estructura de las firmas, factores sectoriales y determinantes sistémicos se asumen como variables predictoras. Resultaron significativos en el estado innovador de la empresa, dependiendo del sector y tamaño considerado, la proporción de profesionales ocupados en la empresa, los gastos efectuados en actividades de innovación interna y externa, exportar, la utilización de fondos de programas oficiales de innovación y la vinculación con organismos públicos del sistema nacional de innovación y el tamaño. }
