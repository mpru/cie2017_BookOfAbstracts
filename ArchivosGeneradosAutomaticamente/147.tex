\A
{APLICACIÓN DE MODELOS LINEALES GENERALIZADOS MIXTOS PARA LA EVALUACIÓN DE EFECTOS DE MEZCLAS BINARIAS DE PLAGUICIDAS}
{P M DEMETRIO\index{DEMETRIO, P}}
{\Afilliation{CENTRO DE INVESTIGACIONES DEL MEDIO AMBIENTE, FACULTAD DE CIENCIAS EXACTAS, UNLP.  LA PLATA, BUENOS AIRES, ARGENTINA. CONICET}
\\\Email{pablo.demetrio@gmail.com}
}
{ecotoxicología; mezclas; glifosato; 2,4-d; mlg; mlgm} 
 {Ecología y medio ambiente} 
 {Modelos de regresión} 
 {147} 
 {316-1}
{La Ecotoxicología como disciplina tiene a los bioensayos de toxicidad como una de las herramientas fundamentales. Esta herramienta ha sido utilizada ampliamente para evaluar el efecto biológico de un tóxico y, posteriormente, mezclas de tóxicos sobre distintos organismos modelo. En la región pampeana, la ocurrencia simultánea de plaguicidas en cuerpos de agua superficiales es un escenario de exposición recurrente para la biota acuática; llegando estos xenobióticos a partir de la escorrentía superficial y/o deriva de las zonas de cultivos. Por esta razón la evaluación del impacto de mezclas de plaguicidas que se presentan con alta frecuencia en estos tipos de sistemas es ambientalmente relevante. En la actualidad, dentro de la Ecotoxicología se encuentran en discusión las metodologías implementadas para el análisis de datos de los bioensayos de toxicidad y de las mezclas de sustancias en particular. El objetivo del presente trabajo fue evaluar de manera preliminar la toxicidad de la mezcla de los herbicidas glifosato y 2,4-D sobre el cladócero Daphnia magna en ensayos estáticos a 48 h en condiciones controladas de laboratorio. Se realizaron pruebas de cada herbicida por separado y posteriormente con las mezclas binarias. El análisis de resultados de los bioensayos de los herbicidas individuales se realizó en el contexto de los Modelos Lineales Generalizados (MLG) y en el caso de la mezcla binaria en el contexto de los Modelos Lineales Generalizados Mixtos (MLGM). En estos últimos se evaluaron modelos con y sin interacción entre los herbicidas. Como ventajas de este abordaje en relación a los tradicionales utilizados cabe destacar que: a) no es necesaria la transformación de la variable binomial, b) es posible utilizar criterios estadísticos para la elección de los modelos (aditividad vs interacción) y c) es viable incorporar la variabilidad inter-ensayos, permitiendo modelarla debido a la no simultaneidad temporal de las pruebas. Los resultados indican que no hay evidencia de interacción de los herbicidas glifosato y 2,4-D para las condiciones experimentales evaluadas, considerando un efecto aditivo sobre la supervivencia de D. magna. El abordaje del efecto de las mezclas mediante GLM y sus extensiones es una alternativa para este tipo de estudios en el ámbito de la Ecotoxicología. Agradecimiento: PICT 2013-2393}
