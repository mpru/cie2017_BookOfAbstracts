\A
{ALTERNATIVA PARA LA ASIGNACIÓN DE VALORES DE REFERENCIA EN LA DIMENSIÓN SALUD REPRODUCTIVA DEL ÍNDICE DE DESIGUALDAD DE GÉNERO}
{\Presenting{ALDO VIOLLAZ}\index{VIOLLAZ, A}, JORGELINA MENA\index{MENA, J} y VIVIANA LENCINA\index{LENCINA, V}}
{\Afilliation{INSTITUTO DE INVESTIGACIONES ESTADÍSTICAS, FACULTAD DE CIENCIAS ECONÓMICAS, UNT}
\\\Email{jorgelinamena@yahoo.com.ar}}
{desigualdad de género; salud reproductiva; regresión de cuantiles} 
 {Estadísticas oficiales} 
 {Modelos de regresión} 
 {23} 
 {66-2}
{En la agenda de desarrollo sostenibles, acordados por los países miembros de la ONU en septiembre de 2015, el lograr la igualdad entre los géneros y empoderar a todas las mujeres y las niñas sigue siendo un objetivo a alcanzar. Los índices multidimensionales han demostrado ser instrumentos valiosos para captar desigualdades, colaboran generando atención, estimulan el debate y análisis de políticas desarrolladas en este sentido además ayudan a monitorear los progresos logrados. Una de las medidas desarrolladas y publicadas por Human Development Report desde 2010, es el Índice de Desigualdad de Género (GII) compuesto por tres dimensiones: Salud Reproductiva, Empoderamiento y Participación del mercado laboral. El GII ha tenido grandes repercusiones tanto positivas como negativas, puesto que incorporó nuevas dimensiones y conceptos omitidos en otros índices, pero también fue criticado por la complejidad de su definición y la incorporación en su construcción tanto de indicadores que se calculan para ambos géneros, como otros que sólo están definidos para el sexo femenino, específicamente, los utilizados en la dimensión Salud Reproductiva. Puesto que en esta dimensión los indicadores incorporados no son medidos para el género masculino se le asigna un valor ideal, de manera independiente a la situación general de la región en donde habitan, característica que penaliza a regiones económicamente menos favorecidas, dado que los indicadores utilizados en la contraparte femenina están asociados al bienestar económico de la región donde se aplica. En este trabajo se realiza una propuesta de modificación en la definición de la dimensión Salud Reproductiva. Haciendo uso de regresión de cuantiles se construye una frontera inferior de posibilidades para cada indicador incorporado en esta dimensión y este valor es el asignado a los hombres en la construcción del índice. Se analiza el comportamiento de esta propuesta metodológica en las provincias argentinas.}
