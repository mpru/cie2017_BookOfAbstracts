\A
{CURVAS ROC COMO MEDIDA DE DIAGNÓSTICO EN MODELOS}
{\Presenting{SHRIKANT BANGDIWALA}$^1$\index{BANGDIWALA, S}, ANA MARÍA SFER$^2$\index{SFER, A} y M A D'URSO VILLAR$^2$\index{D'URSO VILLAR, M}}
{\Afilliation{$^1$MCMASTER UNIVERSITY}
\Afilliation{$^2$UNIVERSIDAD NACIONAL DE TUCUMÁN}
\\\Email{asfer@herrera.unt.edu.ar}}
{modelos mixtos; curvas roc; concordancia} 
 {Otras aplicaciones} 
 {Otras categorías metodológicas} 
 {25} 
 {67-1}
{El ajuste de modelos lineales generalizados mixtos (GLMM) ha dado respuestas a una amplia gama de estudios observacionales y experimentales cuando las observaciones están correlacionadas. En estos modelos, el análisis de los residuos no tiene una discusión acabada en la bibliografía y no todos los paquetes disponibles tienen en cuenta el efecto aleatorio en las predicciones. El modelo más usado dentro de los GLMM es el de respuesta binaria. Para el análisis de los residuos, en este trabajo, se utiliza la curva ROC y su respectiva área bajo la curva (AUC), herramientas clásicas del modelo logístico. También se usa el coeficiente de concordancia Kappa de Cohen y el Test de McNemar. El análisis de la curva ROC, permite seleccionar los modelos óptimos independientemente de la distribución de éxitos y fracasos. EL AUC se usa para medir el poder de discriminación del modelo. El coeficiente Kappa de Cohen, permite ver si las predicciones clasifican bien a las observaciones. El Test de McNEmar compara las proporciones de correctamente clasificados por el modelo. Los objetivos son estudiar, mediante simulaciones, el comportamiento de las predicciones de GLMM con respuesta binaria teniendo en cuenta o no los efectos aleatorios, y compararlas con las correspondientes a las del modelo logístico. Se genera una variable (Y1,Y2,Y3,X1,X2) $\sim$ Normal Multivariada (100, 0, $\Sigma$). Se considera Yi como las respuestas en 3 repeticiones y Xi las covariables. Se categorizan las variables respuestas considerando punto de corte en cero, y una de las covariables considerando punto de corte su mediana. Se ajusta un modelo logístico y un modelo logístico mixto. Se generan las predicciones en la escala link y de probabilidad en cada modelo. En el modelo mixto se generan con y sin efecto aleatorio. Se construyen las correspondientes curvas ROC. Los resultados indican que las mayores áreas bajo la curvas ROC las tiene el modelo mixto que incluye en la predicción los efectos aleatorios. La concordancia entre la respuesta observada y la predicción es fuerte, aproximadamente igual a 0.70, en estos modelos y moderada, próxima a 0.56, en el modelo mixto cuando no se incluyen los efectos aleatorios. Los resultados más pobres se obtienen con el modelo logístico (conceptualmente erróneo). Las pruebas de McNemar muestran resultados similares, aunque son mejores para el modelo mixto cuando ellas consideran los efectos aleatorios. También, se concluye que los resultados son mejores cuando se considera las repeticiones por separado.}
