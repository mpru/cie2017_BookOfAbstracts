\A
{MODELO ESTRUCTURAL DE LAS RELACIONES ENTRE VARIABLES SOCIOECONÓMICAS Y LA DECISIÓN DE COMPRA FAMILIAR DE ALIMENTOS EN ADULTOS MAYORES}
{\Presenting{ANTONIO HUMBERTO CLOSAS}$^{1,2}$\index{CLOSAS, A} y JORGE GUILLERMO ODRIOZOLA$^1$\index{ODRIOZOLA, J}}
{\Afilliation{$^1$UNIVERSIDAD NACIONAL DEL NORDESTE}
\Afilliation{$^2$UNIVERSIDAD TECNOLÓGICA NACIONAL}
\\\Email{hclosas@hotmail.com}}
{modelización estadística; ecuaciones estructurales; consumo; adultos mayores; marketing} 
 {Otras ciencias económicas, administración y negocios} 
 {Métodos multivariados} 
 {91} 
 {206-1}
{En un contexto globalizado, de economía de libre mercado, el consumo de productos en general, y la adquisición de alimentos para el ámbito familiar en particular, se presenta como un fenómeno multidimensional, modulado por diversos factores que van desde la disponibilidad económica hasta la influencia de los soportes comunicacionales. Estos determinantes participan a través de diferentes relaciones que inciden en forma directa e indirecta en la motivación, planificación y realización final de compra de las personas. A su vez, algunas características individuales de los consumidores, p. ej., los distintos segmentos sociales, ameritan que el estudio de esta temática se realice de manera diferenciada. Así pues, el objetivo del presente trabajo consiste en desarrollar mediante ecuaciones estructurales un modelo que explique de qué manera ciertos factores socioeconómicos influyen en la decisión de compra familiar de alimentos por parte de adultos mayores en ciudades del nordeste de Argentina. La investigación responde a un diseño explicativo, de estilo descriptivo mediante encuesta, de línea cuantitativa, de corte transversal y perfil correlacional. La muestra estuvo conformada por 460 individuos (263 mujeres, 197 hombres) de 60 a 85 años, que residen en las ciudades de Resistencia y Corrientes. En el procedimiento utilizado para extraer la muestra hemos combinado los métodos estratificado (las ciudades representan los estratos) y no probabilístico (la elección de las personas se realizó de manera intencionada e incidental). Las variables observadas o indicadores se han obtenido de aplicar un cuestionario socioeconómico, elaborado por un equipo interdisciplinario de académicos, investigadores y profesionales, con vasta experiencia en la región en relevamiento de consumo de diversos bienes y categorías de productos y servicios. Se propone un modelo conceptual que es contrastado a nivel empírico mediante la técnica Ecuaciones Estructurales, dando lugar a una representación ajustada a los datos, proveniente del procedimiento de estimación ERLS (Elliptical Reweighted Least Square), utilizando EQS, en la que intervienen seis constructos: Nivel socioeconómico, Situación en el hogar (independientes), Tecnologías de la información y la comunicación, Tiempo libre, Gastos (intermediarios) y Compras (dependiente). Tanto la metodología empleada, como el modelo final planteado, podrían ser recursos válidos para abordar con eficiencia la obtención de resultados y conclusiones que posibiliten comprender de manera integral, unificada y cercana a la realidad una cuestión de central interés en sectores productivos, administrativos, institucionales, comerciales y empresariales del área objeto de estudio.}
