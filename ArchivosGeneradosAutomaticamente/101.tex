\A
{APLICACIÓN DE ANALISIS FACTORIAL MÚLTIPLE SOBRE DATOS DE CALIDAD DE AGUA MEDIDA MEDIANTE DIFERENTES GRUPOS DE INDICADORES}
{\Presenting{L MAYANS}$^1$\index{MAYANS, L}, S OCAMPO$^1$\index{OCAMPO, S}, J MOLINA$^1$\index{MOLINA, J}, L LAFFITTE$^2$\index{LAFFITTE, L}, J MUÑIZ SAAVEDRA$^2$\index{MUÑIZ SAAVEDRA, J} y A LAVALLE$^1$\index{LAVALLE, A}}
{\Afilliation{$^1$UNIVERSIDAD NACIONAL DEL COMAHUE}
\Afilliation{$^2$SUBSECRETARÍA DE AMBIENTE}
\\\Email{lau\_mayans@hotmail.com}}
{análisis multivariad; análisis a tres vías; calidad de agua} 
 {Otras aplicaciones} 
 {Otras categorías metodológicas} 
 {101} 
 {226-3}
{La cuenca Lacar – Hua Hum, constituye la única cuenca de vertiente pacífica de la Provincia del Neuquén. Durante el año 2016 se llevó a cabo el monitoreo bio-hidrológico de la misma, en colaboración entre la Dirección General de Biología Acuática, Subsecretaria de Ambiente de Neuquén, quien financió el monitoreo y el Organismo de Control Municipal de San Martín de los Andes. Se muestrearon 29 sitios de diferentes tipos de cuerpos y cursos de agua (arroyos, ríos, lago, canales) entre los meses de marzo y abril de 2016. Se determinaron parámetros físico- químicos, biológicos y de hábitat de cada sitio. Los resultados se utilizaron para determinar calidad de sitios en base a bioindicadores y para comparar la evolución de la calidad en base a datos históricos de algunos de los sitios. En el presente trabajo se utilizaron datos de 21 de los sitios mencionados. Se realizó un análisis factorial múltiple sobre 6 grupos de variables diferentes. Estos grupos se conformaron con 5 variables fisicoquímicas relevadas in situ, 14 variables referidas a minerales y nutrientes, 2 variables bacteriológicas, 11 referidas al hábitat acuático, 6 que expresan la composición del suelo y una indicadora de calidad de agua (Índice BMPS cualitativa). Se buscó determinar relaciones entre los grupos de variables, estudiar la configuración de los sitios al utilizar diferentes grupos de variables simultáneamente y determinar la relación entre los diferentes grupos de variables y la calidad del agua. Se vio una clara distinción de la variable suplementaria indicadora de la calidad del agua con la primer dimensión. Se encontró una relación entre el tipo de suelo de cada sitio y la presencia de iones y nutrientes, entre otras.}
