%%%%%%%%%%%%%%%%%%%%%%%%%%%%%%% naslovnica %%%%%%%%%%%%%%%%%%%%%%%%
\thispagestyle{empty}
{\center
{\Large \bf \DocConferenceTitleA} \\[14mm]
{\LARGE \bf \DocConferenceTitleB} \\ [6mm]
{\LARGE \bf \DocYear} \\ [14mm] %[34mm]
{\large \DocTitle} \\
\begin{center} \setlength{\unitlength}{1cm}
  \includegraphics[width=\linewidth,  clip]{\DocFigCover}
\end{center}
%\vfill
\DocDate \\ %[3mm]
\DocPlace \\[3mm]%[3mm]
\DocURL \\
}
\newpage

%leave an empty page so the numbering is correct also in the pdf file

\thispagestyle{empty}
\input{empty.tex}


%%%%%%%%%%%%%%%%%%%%%%%%%%%%%%% Prva stran %%%%%%%%%%%%%%%%%%%%%%%%

\thispagestyle{empty}
\setcounter{page}{1}
{\center
%{\Large \bf \DocConferenceTitleA}\\  [14mm]

\vspace*{32 mm}
{\LARGE \bf Congreso Interamericano de Estadística} \\ [6mm]
%{\LARGE \bf \DocYear} \\ [6mm]
{\Large \DocTitle}  \\ [16mm]
\vspace{32 mm}
{\large \DocYear} \\[3mm]
{\large \DocPlace} \\[6mm]
{\large \DocURL}\\[10mm]

\vfill

{\Large Organizado por:} \\ [3mm]
{\large Facultad de Ciencias Económicas y Estadística} \\[10mm] %[16mm]
{\large Universidad Nacional de Rosario} \\[10mm] %[16mm]


%Urad vlade R Slovenije za informiranje

%{\Large Supported by} \\ [3mm]
%\DocSponsors 
}
\newpage

%%%%%%%%%%%%%%%%%%%%%%%%%%%%%%% Druga stran %%%%%%%%%%%%%%%%%%%%%%%%


\newpage
\thispagestyle{empty}

%\vspace*{11cm}
%\vspace*{9cm}
%\DocCenterPageTwo
%\DocBottomPageTwo

\fbox{
\parbox{0.9\textwidth}{%
Apellido, Nombre \\
\hspace*{2em} Libro de Resúmenes Congreso Interamericano de Estadística. - 1a ed.\\
\hspace*{2em} Rosario: Editora, 2017. X p. ; 21x16 cm.\\
\hspace*{2em} ISBN 000-000-0000-00-0 \\
\hspace*{2em} 1. Estadísticas. I. Título. \\
\hspace*{2em} CDD 310.4
}
}

\bigbreak
\footnotesize Fecha de catalogación: 01/01/2017

\vfill

\small{
\textcopyright{} 2017 Congreso Interamericano de Estadística \\
ISBN 000-000-0000-00-0

\bigbreak
\textbf {Publicado por:} \\
Sociedad Argentina de Estadística \\
Av. Corrientes 4264, 9° piso A \\
C1195AAO Ciudad Autónoma de Buenos Aires

\bigbreak
\textbf {Compilado por:}\\
Nombre Apellido \\
email@gmail.com \\
Rosario - Argentina

\bigbreak
\textbf {Organizadores del Congreso:} \\
Sociedad Argentina de Estadística (SAE) \\
Grupo Argentino de Biometría (GAB) \\
Instituto Interamericano de Estadística (IASI)

\bigbreak
\textbf {Impresión:} \\
Poner el nombre de la editora si se imprime
\bigbreak
Fecha de impresión: Octubre de 2017 \\
Producido con adaptaciones a partir del paquete \textit{generbook} de R.
}

%%%%%%%%%%%%%%%%%%%%%%% PRELIMINARES %%%%%%%%%%%%%%%%%%%%%%%%%%%%%%%%%%%%%%%%

\clearpage
\newpage
\noindent\\

\thispagestyle{empty}
 \begin{center}
  \Large
   % \textbf{Program} \\ [0.5cm]
   \begin{flushright}
   \vspace{17cm} {\Huge \em{ \textbf{PRELIMINARES}}} \\ [0.5cm]
   \end{flushright}
   \normalsize
 \end{center}
%\noindent  \hrulefill \\[0.5cm]
\small
\clearpage

%%%%%%%%%%%%%%%%%%%%%%% AUSPICIANTES %%%%%%%%%%%%%%%%%%%%%%%%%%%%%%%%%%%%%%%%

\clearpage
\newpage
\thispagestyle{empty}
\input{empty.tex}

\newpage
\noindent
\pagestyle{fancy}
\setlength\parindent{16pt}
\SetHeader{Auspiciantes}{\textit{Congreso Interamericano de Estadística}}

\vspace*{1cm}
\centerline{\textbf{\LARGE{Auspiciantes}}}

\begin{center} \setlength{\unitlength}{1cm}
  \includegraphics[width=\linewidth,  clip]{../Graficos/sponsors}
\end{center}

%%%%%%%%%%%%%%%%%%%%%%%%%%%%%%% AUTORIDADES %%%%%%%%%%%%%%%%%%%%%%%%

\clearpage
\newpage
\pagestyle{fancy}
\SetHeader{Autoridades}{\textit{Congreso Interamericano de Estadística}}
%\thispagestyle{empty}

\vspace*{1cm}

%\par\hbox{\LARGE{\textbf{Autoridades}}}\hrule
\centerline{\textbf{\LARGE{Autoridades}}}

\vspace{1cm}

% \par\hbox{\Large{\textbf{Comité Organizador Local}}}\hrule
\noindent {\Large\textbf{Comité Organizador Local}}
\bigbreak
\noindent \textbf {Presidente:} Cristina Cuesta (Facultad de Ciencias Económicas y Estadística, UNR) \\
\textbf {Viceresidente:} Alberto Trevizan (Facultad de Ciencias Agrarias, UNR) \\
\textbf {Miembros:} \\
Docentes de la Facultad de Ciencias Económicas y Estadística. UNR \\
Docentes de la Facultad de Ciencias Agrarias. UNR \\
Docentes de la Facultad de Ciencias Exactas, Ingeniería y Agrimensura. UNR \\

\vspace{1cm}

% \par\hbox{\Large{\textbf{Comité Científico}}}\hrule
\noindent {\Large\textbf{Comité Científico}}
\bigbreak
\noindent Arce, Osvaldo (UNT) \\
Beltran, Celina (UNR) \\
Barbona, Ivana (UNR) \\
Blaconá, María Teresa (UNR) \\
Boggio, Gabriela (UNR) \\
Bussi, Javier (UNR) \\
Casparri, María Teresa (UBA) \\
Castro Kuriss, Claudia (ITBA) \\
Cendoya , María Gabriela (UNMdP) \\
Charre de Trabuchi, Clyde (IASI) \\
Cosolito, Patricia (UNR) \\
Cuesta, Cristina Beatriz (UNR) \\
De Alba,Enrique (IASI) \\
Diaz, María del Pilar (UNC) \\
Di Blasi, Angela (UNCu) \\
Di Rienzo, Julio (UNC) \\
García, María del Carmen (UNR) \\
Giménez, Laura (UNNE) \\
Hachuel, Leticia (UNR) \\
Kelmansky, Diana (UBA) \\
Llera, Joaquín (UNC) \\
Marí, Gonzalo (UNR) \\
MacDonald, Alphonse (IASI) \\
Mendez, Fernanda (UNR) \\
Pagura, José A. (UNR) \\
Quaglino, Marta (UNR) \\
Quintslr, Marcia (IASI) \\
Ricci, Lila (UNMdP) \\
Rosa, Ernesto (UNTREF) \\
Trevizan, Alberto (UNR) \\
Urrutia, María Inés (UNLP) \\

%%%%%%%%%%%%%%%%%%%%%%%%% ANTECEDENTES %%%%%%%%%%%%%%%%%%%%%%%%%%%%%%%%%%%%%%

\newpage
\pagestyle{fancy}
\setlength\parindent{16pt}
\SetHeader{Antecedentes}{\textit{Congreso Interamericano de Estadística}}

\vspace*{1cm}

\centerline{\textbf{\LARGE{Antecedentes}}}
%\par\hbox{\LARGE{\textbf{Antecedentes}}}\hrule

\bigbreak

\noindent \textbf{\textit{De la Escuela de Estadística de la UNR}}

La carrera de Estadística de la Facultad de Ciencias Económicas y Estadística de la UNR fue creada en el año 1948 por iniciativa del Profesor Carlos Eugenio Dieulefait en la entonces Facultad de Ciencias Económicas de la Universidad Nacional del Litoral. Este evento significó un paso trascendental para la estadística de nuestro país ya que  fue la primera carrera en el área de Latinoamérica.

A lo largo de su historia egresaron numerosos profesionales, de los cuales muchos se distinguieron en el país y en el extranjero desempeñándose en el ámbito académico, reparticiones públicas y privadas, en distintas áreas tales como  estadísticas oficiales, ciencias económicas, biología y salud, agronomía, industria, demografía, entre otras.

Actualmente la carrera de grado, junto a la Maestría en Estadística Aplicada, creada en el año 2002, y el Doctorado en Estadística, creado en el año 2014, ambos de la UNR, forman parte de uno de los principales centros del país de reconocido prestigio, en formación de recursos humanos en el área. 
Cabe señalar que la Escuela de Estadística posee una extensa trayectoria en investigación desarrollada en los institutos de investigaciones. Las tareas en investigación no sólo han enriquecido la formación docente, sino que también motivan la participación de los alumnos en las mismas. 

Las actividades de transferencia, que se desarrollan a través de convenios firmados entre la Escuela de Estadística y organismos públicos y privados, permiten a docentes y estudiantes la vinculación con el medio y la resolución de problemas reales.
Estas características y la evolución de los planes de estudio han permitido lograr un perfil de graduado de fácil inserción laboral en distintos ámbitos del país y del exterior, con posibilidades, además, de realizar con éxito estudios de posgrado en universidades nacionales y extranjeras.

La Escuela de Estadística ha organizado en varias oportunidades congresos, jornadas y eventos similares al que se llevará a cabo próximamente. En el año 2006 se realizaron las Jornadas Internacionales de Estadística que convocaron a unos 700 participantes, investigadores, docentes y profesionales nacionales y extranjeros.

\bigbreak
\noindent\textbf{\textit{De la Sociedad Argentina de Estadística}}

La Sociedad Argentina de Estadística (SAE) es un organismo técnico - científico que promueve el desarrollo de la Estadística en el país. Fue creado en junio de 1952 y entre sus objetivos se encuentran: : agrupar y vincular entre sí a todas las personas relacionadas con la Ciencia Estadística y con sus aplicaciones; fomentar y promover el estudio, investigación, desarrollo y perfeccionamiento de la Estadística y disciplinas conexas, mediante reuniones científicas y técnicas, cursos, becas, premios, concursos, publicaciones sin fines de lucro y otros medios adecuados; contribuir al mejoramiento de la enseñanza de la Estadística y mantener relaciones con instituciones afines, nacionales y extranjeras.

Cada año la SAE organiza Coloquios que intercaladamente comparte con otros países de Latinoamérica.

\bigbreak
\noindent\textbf{\textit{Del Instituto Interamericano de Estadística}}

El Instituto Interamericano de Estadística (IASI) fue fundado en el año 1940, a iniciativa de un grupo de miembros del Instituto Internacional de Estadística (ISI), interesados en mantener y desarrollar en la región americana actividades y programas que a nivel mundial venía llevando a cabo el ISI, y que se habían interrumpido a consecuencia de la II Guerra Mundial.

El Instituto Interamericano de Estadística (IASI) es una organización profesional cuyo propósito es promover el desarrollo de la estadística en la región americana.  Persigue los siguientes objetivos: 

(a) desarrollo y fortalecimiento de la profesión estadística;

(b) promoción y divulgación de avances en teoría y métodos estadísticos;

(c) perfeccionamiento de la metodología en la producción de estadísticas, tanto gubernamentales como no gubernamentales;

(d) promoción de medidas que tiendan a mejorar la comparabilidad y el aprovechamiento de las estadísticas económicas y sociales entre las naciones de la región; y

(e) colaboración con las organizaciones nacionales e internacionales en actividades orientadas al mejoramiento de la estadística en la región

\bigbreak
\noindent\textbf{\textit{Del Grupo Argentino de Biometría}}

El Grupo Argentino de Biometría (GAB) nace en 1995, con el objeto de relacionar personas que investigan y/o enseñan aspectos de la estadística aplicada a la biología, promoviendo la investigación, la docencia y la difusión de estos conocimientos a través de diversas actividades científicas.

Sus objetivos principales son: fortalecer y enriqucer los vínculos y propósitos que dieron origien al Grupo Argentino de Biometría (GAB); proveer un ámbito para el intercambio y discusión de ideas motivadoras e inspiradoras para la comunidad de biometristas; hacer público y difundir los resultados del trabajo de los biometrístas argentinos; promover y/o fortalecer la generación de vínculos inter instituciones; fortalecer la formación de recursos humanos y la actividad de los biometristas en la Argentina. 

%%%%%%%%%%%%%%%%%%%%%%%%%%%% PROGRAMA %%%%%%%%%%%%%%%%%%%%%%%%%%%%%%%%%%%

\newpage
\pagestyle{fancy}
\setlength\parindent{16pt}
\SetHeader{Programa}{\textit{Congreso Interamericano de Estadística}}

\vspace*{1cm}
\centerline{\textbf{\LARGE{Programa}}}

\begin{center} \setlength{\unitlength}{1cm}
  \includegraphics[width=\linewidth,  clip]{../Graficos/programa}
\end{center}
