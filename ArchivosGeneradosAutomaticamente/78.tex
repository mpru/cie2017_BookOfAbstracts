\A
{MEDIDAS TIPO R-CUADRADO PARA MODELOS ESPACIALES}
{\Presenting{LILA RICCI}$^1$\index{RICCI, L} y SILVIA OJEDA$^2$\index{OJEDA, S}}
{\Afilliation{$^1$CENTRO MARPLATENSE DE INVESTIGACIONES MATEMÁTICAS, UNIVERSIDAD NACIONAL DE MAR DEL PLATA, ARGENTINA}
\Afilliation{$^2$FACULTAD DE MATEMÁTICA ASTRONOMÍA Y FÍSICA. UNIVERSIDAD NACIONAL DE CÓRDOBA}
\\\Email{lricci@mdp.edu.ar}}
{estadística espacial; medidas de ajuste global; modelos lineales} 
 {Ciencias exactas y naturales} 
 {Estadística espacial} 
 {78} 
 {174-1}
{En el proceso de ajuste de modelos espaciales a un conjunto de observaciones, resultará de gran utilidad contar con medidas que permitan evaluar el ajuste global y ayuden en el proceso de selección de covariables. Para tener un criterio de admisibilidad de las medidas propuestas, Cameron y Windmeijer (1995) enunciaron las cinco propiedades que debe tener una medida de ajuste global para ser considerada “tipo R2”. Para Modelos Lineales Generalizados Zheng (2000) analiza cuatro coeficientes posibles; todos ellos se reducen al R2 clásico cuando los errores son normales. Es de particular interés la medida basada en remplazar las distancias euclídeas, que aparecen en el modelo clásico dentro de las sumas de cuadrados, por la deviance que es equivalente a la distancia de Kullback-Leiber. Para el modelo normal multivariado, una medida de ajuste global está dada por el conocido estadístico lambda de Wilks, que es un cociente entre determinantes de las matrices de covarianza de dos hipótesis anidadas; varía entre 0 y 1 y es proporcional a la estadística F para la hipótesis de nulidad de todos los coeficientes. Nagelkerke (1991) amplió la definición del coeficiente de determinación a todo aquel que posea propiedades similares a las definidas por Cameron y Windmeijer. En este trabajo se resumen las diferentes medidas de ajuste global que han sido propuestas para modelos espaciales y se propone una medida que tiene las propiedades necesarias para ser considerada una medida de tipo R2, la misma está basada en la deviance, y tiene como antecedentes medidas similares propuestas para Modelos Lineales.}
