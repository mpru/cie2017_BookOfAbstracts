\A
{REGIONES E INDICES SINTÉTICOS POR APLICACIÓN DE MÉTODOS ESTADÍSTICOS MULTIVARIADOS COMO BASE PARA LA COMPARACIÓN DE COMPONENTES DEMOGRÁFICAS}
{\Presenting{MARTÍN SAINO}\index{SAINO, M}, MARÍA CAROLINA TROGLIERO\index{TROGLIERO, M}, DIEGO CONTRERAS\index{CONTRERAS, D} y JOSÉ CASTILLO SOSA\index{CASTILLO SOSA, J}}
{\Afilliation{INSTITUTO DE ESTADÍSTICA Y DEMOGRAFÍA, FACULTAD DE CIENCIAS ECONÓMICAS, UNIVERSIDAD NACIONAL DE CÓRDOBA}
\\\Email{martinsaino@gmail.com}}
{componentes principales; indices sintéticos; condiciones socioeconómicas; componentes demográficas} 
 {Otras aplicaciones} 
 {Métodos multivariados} 
 {161} 
 {339-1}
{Las condiciones de vida de una población están fuertemente ligadas tanto a la distribución espacial de las actividades económicas como a sus niveles de desarrollo regional, motivos por los cuales éstas también presentan diferenciales sociodemográficas. Conforme a estas perspectivas tales asimetrías pueden ser exploradas y/o explicadas merced a una adecuada selección de variables interactuantes, susceptibles al mismo tiempo de un correcto agrupamiento conforme niveles de homogeneidad. Así, sobre la base de un conjunto de indicadores elaborados con datos censales de 2001, y mediante la aplicación de Análisis de Componentes Principales (ACP) y Cluster se efectúa una regionalización y se construyen índices socioeconómicos sintéticos que reflejen la posición relativa de las provincias y esas regiones. Esos índices fueron construidos para cada jurisdicción a partir de la agregación de las componentes retenidas ponderadas por su peso y normalizadas por su máximo. Luego para cada región se tomó el promedio de los índices de las jurisdicciones que las conforman. Las regiones así constituidas y los indices de referencia sirven de base para la comparación de componentes demográficas en entornos básicamente desiguales. En ese sentido y en el marco de la transición demográfica, con información procedente de Estadísticas Vitales, se analizó niveles de mortalidad y el cambio en los patrones de fecundidad entre 2001 y 2015. A partir de los resultados obtenidos no cabe dudas que si bien globalmente Argentina se encuadra dentro de los países que presentan una etapa de transición avanzada, las regiones empero se encuentran todavía en diversos estadios de transición. Dan cuenta de ello las bajas tasas de natalidad como de mortalidad. En apoyo de estas ideas vale la pena consignar que mientras la Región 1 está comenzando la Segunda Transición, las restantes regiones se ubican en distintos momentos y en estado avanzado de la tercera etapa de la Primera Transición.}
