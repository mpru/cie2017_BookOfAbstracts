\A
{ASISTENCIA OBLIGATORIA EN CURSOS DE ESTADÍSTICA PARA ESTUDIANTES DE CIENCIAS BIOLÓGICAS Y QUÍMICAS: ENTRE LA PERCEPCIÓN DE LOS ALUMNOS Y EL LOGRO DE LOS OBJETIVOS CURRICULARES}
{\Presenting{LAURA PISKULIC}\index{PISKULIC, L}, MARCOS PRUNELLO\index{PRUNELLO, M}, LILIANA CUCCIARELLI\index{CUCCIARELLI, L}, MARÍA BELÉN ALLASIA\index{ALLASIA, M}, SABRINA SILVA QUINTANA\index{SILVA QUINTANA, S}, JUAN JOSÉ IVANCOVICH\index{IVANCOVICH, J}, LILIANA RACCA\index{RACCA, L} y HEBE BOTTAI\index{BOTTAI, H}}
{\Afilliation{ÁREA ESTADÍSTICA Y PROCESAMIENTO DE DATOS, FACULTAD DE CIENCIAS BIOQUÍMICAS Y FARMACÉUTICAS, UNIVERSIDAD NACIONAL DE ROSARIO}
\\\Email{lpiskulic@yahoo.com}}
{enseñanza-aprendizaje; asistencia; regularización} 
 {Enseñanza de la estadística} 
 {Otras categorías metodológicas} 
 {29} 
 {76-2}
{El conocimiento de los fundamentos del cálculo de probabilidades, del diseño de experimentos y del análisis descriptivo e inferencial son indispensables en múltiples aspectos de la Biología y la Química. La asignatura Estadística en la Facultad de Ciencias Bioquímicas y Farmacéuticas de la Universidad Nacional de Rosario persigue el objetivo de introducir al estudiante en la Estadística, resaltando su papel en la construcción del conocimiento científico, y desarrollar una actitud crítica y reflexiva en el análisis e interpretación de un estudio de investigación. Esta asignatura forma parte del ciclo de formación general de todas las carreras dictadas en la facultad, contando con más de 400 inscriptos. El cursado consiste en clases teórico-prácticas de asistencia obligatoria. La exposición verbal se integra con la presentación de situaciones problemáticas cuya resolución conduce al alumno a buscar información, discutir con sus pares, analizar datos y arribar a conclusiones. El alumno alcanza la condición de “Regular” aprobando un examen práctico o su recuperatorio, “Abandonó” si no se presentó a ninguna actividad evaluativa o “Insuficiente” en otro caso. En 2016 se eliminó la asistencia obligatoria. Se realizó una encuesta a los estudiantes, encontrándose que el 71\% estuvo a favor de la no obligatoriedad, el 60\% opinó que la misma no modificaría el rendimiento y el 50\% que la obligatoriedad no contribuye a llevar la asignatura al día. A pesar de que el 76.8\% reconoció que faltar a clase dificulta la regularización, la asistencia mediana disminuyó a 58\%, siendo de 81.8\% y 82.5\% en 2014 y 2015, respectivamente. Al comparar los resultados académicos se observó un peor desempeño de los estudiantes. El porcentaje de regulares fue 46.6\% en 2014, 52.1\% en 2015 y 34.2\% en 2016, mientras que el de insuficientes fue 46.1\%, 40.3\% y 53.5\%. Entre aquellos que asistieron al menos al 80\% de las clases, no hubo diferencias en la distribución de la condición final a través de los tres años. Esta experiencia pone de manifiesto una discordancia entre la creencia de los estudiantes sobre los beneficios del cursado y los resultados obtenidos cuando la obligatoriedad es removida, tanto en el porcentaje de asistencia como en el rendimiento académico. Además, resalta la importancia del trabajo presencial y en conjunto entre el docente y los estudiantes para alcanzar los objetivos de la enseñanza-aprendizaje de Estadística en carreras de grado no afines. Como resultado, la asistencia obligatoria a clases fue instaurada nuevamente para el ciclo lectivo 2017.}
