\A
{DESARROLLO DE UN NUEVO MAPA DE RIESGO EPIDEMIOLÓGICO PARA LEPTOSPIROSIS EN LA CIUDAD DE ROSARIO}
{\Presenting{LUCRECIA CORALLO}$^1$\index{CORALLO, L}, LAURA BALPARDA$^1$\index{BALPARDA, L}, XIMENA PORCASI$^2$\index{PORCASI, X}, MARIO LAMFRI$^2$\index{LAMFRI, M}, ANDREA FRETES CHÁVEZ$^1$\index{FRETES CHÁVEZ, A} y DIEGO LÓPEZ$^3$\index{LOPEZ, D@LÓPEZ, D}}
{\Afilliation{$^1$SISTEMA MUNICIPAL DE EPIDEMIOLOGÍA. SECRETARÍA DE SALUD PÚBLICA. MUNICIPALIDAD DE ROSARIO.}
\Afilliation{$^2$INSTITUTO DE ALTOS ESTUDIOS “MARIO GULICH”. COMISIÓN NACIONAL DE ACTIVIDADES ESPACIALES.}
\Afilliation{$^3$ÁREA DE SENSORES REMOTOS. ESCUELA DE AGRIMENSURA. FACULTAD DE CIENCIAS EXACTAS, INGENIERÍA Y AGRIMENSURA. UNIVERSIDAD NACIONAL DE ROSARIO.}
\\\Email{lbalparda@hotmail.com}}
{mapa; riesgo; epidemiológico; leptospirosis; imágenes satelitales} 
 {Salud humana} 
 {Estadística espacial} 
 {126} 
 {278-2}
{Este trabajo describe mejoras de un mapa preliminar de riesgo para leptospirosis de la ciudad de Rosario, considerando un brote ocurrido en el año 2007 luego de un evento de inundación en la ciudad. El mapa preliminar fue presentado en publicaciones previas (Corallo et.al., 2010). Para la generación del mapa que aquí se presenta, se consideraron 21 espacios territoriales donde se localizaron casos confirmados de leptospirosis ocurridos entre el 04/04/2007 al 16/05/2007 (grupo A) y 70 nuevos espacios territoriales elegidos por muestreo simple al azar donde no se localizaron casos (grupo B). Se incluyó la correspondiente corrección atmosférica y calibración radiométrica a las imágenes satelitales Landsat 5TM (17/03/2007, 18/04/2007 y 20/05/2007). Se calcularon los índices de vegetación, agua y suelo, y sus correspondientes diferencias. Se crearon áreas de influencia hasta 11.000 metros de las zonas de interés (inundadas/anegadas, vegetación y ferrocarriles). Específicamente, los aspectos mejorados o incorporados para el nuevo mapa fueron: a) modificación del criterio para la construcción de las áreas de influencia de las zonas de interés; b) incorporación de la corrección atmosférica y calibración radiométrica, en cada una de las imágenes (Chander et. al., 2009); c) profundización del análisis descriptivo, adición del análisis de componentes principales y ajustes en el análisis de regresión logística. Resultados: Análisis descriptivo con estadísticas básicas y gráficos boxplot, según grupos A y B. En el análisis de componentes principales se identificó, para cada una de las imágenes y sus índices, que las dos primeras componentes explicaron más del 80\% de la variabilidad total. Según los ajustes en el análisis de regresión logística, las variables de mayor poder predictivo fueron: la distancia a las zonas inundadas/anegadas, la primer componente principal de la imagen del mes de mayo y la segunda de la imagen de marzo (sensibilidad 52\%; especificidad 90\%). Considerando el modelo de regresión, las variables y sus coeficientes, se construyó el nuevo mapa de riesgo para leptospirosis de la ciudad de Rosario para el año 2007, con un procedimiento optimizado más riguroso que permitió salvar las dificultades preliminares. El trabajo muestra los mapas preliminar y actualizado en el que se evidencian mejoras en la especificidad de predicción del riesgo. El procedimiento de generación del nuevo mapa, puede replicarse a futuro ante la ocurrencia de intensas lluvias que pudieran provocar la inundación o el anegamiento de terrenos, importante factor de riesgo para la aparición de un nuevo brote de leptospirosis en la ciudad.}
