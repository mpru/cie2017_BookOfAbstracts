\A
{PRUEBA PARA SIMETRÍA ORIENTADA POR DATOS EN DISTRIBUCIONES CON MEDIANA DESCONOCIDA}
{\Presenting{JIMMY CORZO}$^1$\index{CORZO, J}, MYRIAM VERGARA$^2$\index{VERGARA, M} y GIOVANY BABATIVA$^3$\index{BABATIVA, G}}
{\Afilliation{$^1$Universidad Nacional de Colombia, Bogotá, Colombia}
\Afilliation{$^2$Universidad de La Salle, Bogotá, Colombia}
\Afilliation{$^3$Universidad Santo Tomás, Bogotá, Colombia}
\\\Email{}}
{pruebas de rachas; pruebas para simetría; potencia de una prueba; estocásticamente mayor} 
 {Otras Ciencias Sociales y Humanas} 
 {Métodos no paramétricos} 
 {191} 
 {907-1}
{Se ordenan los valores absolutos de las observaciones centradas con la mediana muestral, y por medio de los anti-rangos se transforman en una sucesión dicótoma en la que están representadas por unos las observaciones positivas y por ceros las observaciones negativas. A partir de esta sucesión dicotomizada se define una sucesión de contadores y con estos se obtiene el número de rachas hasta cada observación ordenada. Con el número de rachas se construye una estadística ponderada por signos, dependiendo de si la observación a que corresponde este número de rachas proviene de una observación positiva o de una observación negativa. La prueba propuesta para la hipótesis de simetría con alternativa “estocásticamente mayor” surge de que valores extremos positivos de la estadística son indicadores de que las observaciones más grandes en valor absoluto positivas y por tanto que la distribución muestreada tiene más larga la cola derecha. Con un estudio de Monte Carlo se compara la potencia empírica de la prueba con la de otras pruebas para la misma hipótesis. Se calibra tamaño de la prueba utilizando seis casos simétricos de la distribución lambda gerneralizada (DLG) y se muestra que la prueba es insesgada. El estudio de la potencia empírica realizado para ocho casos asimétricos de la DLG mostró que la potencia empírica de la prueba sobrepasa la de las otras pruebas comparadas, exceptuando una de las pruebas en algunos casos específicos. Para orientar la prueba por los datos se calculó el tamaño de la prueba propuesta para varios casos simétricos de la DLG y se observó que la prueba tiende a sesgarse con el grado de aplastamiento y la proporción de observaciones recortadas. Entonces se diseñó un algoritmo con el cual se puede elegir una proporción de recorte a partir del grado de apuntamiento de la distribución muestreada. Se concluyó también que la proporción de recorte que optimiza la potencia empírica varía con el tamaño de las colas, siendo mayor para distribuciones de colas pesadas y menor para distribuciones de colas suaves.}