\A
{VARIABILIDAD EN LA EXPRESIÓN DE POLIPÉPTIDOS Y ATRIBUTOS FENOTÍPICOS DURANTE LA MADUREZ DEL TOMATE}
{\Presenting{A P DEL MEDICO}$^1$\index{DEL MEDICO, A}, M S VITELLESCHI$^2$\index{VITELLESCHI, M}, A LAVALLE$^3$\index{LAVALLE, A} y G R PRATTA$^1$\index{PRATTA, G}}
{\Afilliation{$^1$IICAR (INSTITUTO DE INVESTIGACIONES EN CIENCIAS AGRARIAS DE ROSARIO), CONICET/UNR}
\Afilliation{$^2$IITAE (INSTITUTO DE INVESTIGACIONES TEÓRICAS Y APLICADAS DE LA ESCUELA DE ESTADÍSTICA), CIUNR/UNR}
\Afilliation{$^3$DEPARTAMENTO DE ESTADÍSTICA. UNIVERSIDAD NACIONAL DEL COMAHUE}
\\\Email{delmedico@iicar-conicet.gob.ar}}
{análisis factorial múltiple; solanum lycopersicum l.; expresión génica} 
 {Biología} 
 {Métodos multivariados} 
 {34} 
 {83-1}
{Durante la madurez del fruto de tomate (Solanum lycopersicum L.) se producen cambios fisiológicos genéticamente regulados, que determinan un producto cuya calidad se adecue a los requerimientos de los consumidores. En esta transición se reconocen varios estados, de los cuales los más contrastantes son el Verde Maduro (VM, definido como aquél en el que el fruto inmaduro alcanza el tamaño máximo) y el Rojo Maduro (RM, en el cual el fruto presenta más del 90\% de su superficie con el color de madurez). Una forma de evaluar la expresión génica diferencial es mediante el análisis de la Variabilidad de la Frecuencia con la cual determinados Polipéptidos (VFP) están presentes a lo largo de la madurez. Esta variabilidad en el nivel molecular podría estar relacionada con la Variabilidad para Atributos Fenotípicos (VAF) del fruto. El objetivo de este trabajo fue estudiar, mediante Análisis Factorial Múltiple (AFM), la posible asociación de VFP en VM y RM con VAF. El material vegetal estuvo compuesto por 75 individuos de tomate de la generación segregante $F_2$ del híbrido de segundo ciclo $F_1$ ToUNR15xToUNR9. De cada individuo se cosecharon 2 frutos en VM y 2 frutos en RM, de cuyo pericarpio se extrajeron los polipéptidos totales. Los perfiles moleculares se determinaron por electroforesis convencional. Para evaluar VFP, se seleccionaron en VM y RM polipéptidos con frecuencias extremas (0 y 1) y que ajustaron a una segregaron mendeliana conocida. Para evaluar VAF, en cada individuo se midieron peso, diámetro, altura, forma y vida poscosecha. Se aplicaron dos AFM, el primero para relacionar VFP en VM con VAF y el segundo para relacionar VFP en RM con VAF. Los resultados de ambos AFM se compararon a fin de verificar si alguno de los estados de madurez presenta mayor asociación con los atributos fenotípicos. En VM, 12 polipéptidos cumplieron los requisitos para ser seleccionados, mientras que 7 polipéptidos lo hicieron en RM. La proporción de variación explicada por los dos primeros ejes principales fue mayor en el AFM correspondiente a RM (39,8\% vs. 28,1\% en VM). Las representaciones de los individuos y de las variables en función de VFP y VAF difirieron entre VM y RM, lo que indica que la asociación entre la variabilidad molecular y la fenotípica fue diferente en ambos estados. Sin embargo, el coeficiente RV fue 0,05 en VM y 0,06 en RM, evidenciando una baja magnitud de tales asociaciones.}
