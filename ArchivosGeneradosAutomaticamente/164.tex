\A
{BAYESIAN ESTIMATION IN THE ADDITIVE HAZARDS MODEL}
{\Presenting{ENRIQUE E ÁLVAREZ}$^1$\index{ALVAREZ, E} y MAXIMILIANO L RIDDICK$^2$\index{RIDDICK, M}}
{\Afilliation{$^1$UNLP - UBA.IC - CONICET}
\Afilliation{$^2$UNLP - CONICET}
\\\Email{enriqueealvarez@fibertel.com.ar}}
{bayesian inference; survival analysis; statistical inference; additive hazards model} 
 {Ciencias exactas y naturales} 
 {Métodos bayesianos} 
 {164} 
 {342-1}
{Suppose we have a sample of n individuals who may experience an terminal event over a window [0,u]. We denote by by Ti the true, possibly latent, time to occurrence for the i-th individual. Because some individuals experience censoring at times Ci, their duration until the event is observed only when Ci >= Ti. In classical Survival Analysis, it is of interest to study the soujourn times as related to observed individual covariates, which we assume time-independent and denote by Zi. In the literature, models for survival data typically focus on the so-called hazard rate, which we assume takes the additive form l(t,b)=l0(t) + zb due to Aalen (1980), where l0(.) is the baseline hazard function and b is a vector of unknown coefficients. Alternative approaches abound in the literature, the most common being Coxs (1972) proportional hazards model an the Accelerated failure time model. It is our goal in this study to propose a Bayesian method of estimation for the semiparametric Additive Hazards Model (AHM) under right-censoring. With this aim, we review the AHM and introduce the likelihood function, so that we comment on the challenges posed by estimation from the full likelihood. Thus we discuss an alternative approach based on a hybrid Bayesian method that exploits Lin and Yins (1994) estimating equation approach and simple tractable priors for the parameters. For the simplest case, we obtain the posterior distributions and we provide algorithms for the estimators. We illustrate our method with a simple dataset in the medical field. References Aalen, O. O. (1980). A model for non-parametric regression analysis of counting processes. Lecture Notes on Mathematics Statistics and Probability. 2: 1-25. Cox, D.R. (1972). Regression Models and Life-Tables (with discussion). Journal of the Royal Statistical Society. Series B. 34: 2, 187-220. Lin, D. Y., Ying, Z. (1994). Semiparametric analysis of the additive risk model. Biometrika, Vol. 81, No. 1, pp. 61-71.}
