\A
{UNA APLICACION ACTUARIAL DEL MODELO DE AZBEL}
{\Presenting{MARIA TERESA CASPARRI}\index{CASPARRI, M}, MARIA ALEJANDRA METELLI\index{METELLI, M} y CAROLINA ARTUSO\index{ARTUSO, C}}
{\Afilliation{FCE-UBA}
\\\Email{ametelli@gmail.com}}
{proyecciones; azbel; mortalidad} 
 {Otras aplicaciones} 
 {Inferencia estadística} 
 {106} 
 {232-2}
{Resumen En diversas circunstancias muchos profesionales, entre ellos especialmente los actuarios se enfrentan con la necesidad de conocer, no sólo cuál es la fuerza de mortalidad a la que está sometida una población determinada en un momento dado del tiempo, sino también conocer su evolución futura. Por dicho motivo, en el presente trabajo nos propusimos proyectar en el tiempo las tablas de mortalidad de Argentina para ambos sexos. Dado que la información disponible en nuestro país es escasa, fue necesario recurrir a fuentes externas. Desde principios del siglo XX, año tras año, muchos países han venido construyendo sus propias tablas de mortalidad. Haciendo uso de dicha información, buscamos vincular la última tabla de mortalidad local disponible con alguna de las tablas norteamericanas que se han publicado desde el año 1933 al presente. El objetivo fue encontrar entre todas ellas y para cada sexo, una a la cual, la nuestra podría ser equivalente. Para ello, hemos recurrido al uso de tests estadísticos, y una vez completados los mismos hemos podido decidir por cuál podría ser reemplazada. Por otro lado, con el objetivo de analizar la mortalidad de la población de Estados Unidos, hemos utilizado al modelo de Azbel. Para cada una de las tablas mencionadas discriminadas por sexo, y mediante el método de mínimos cuadrados, hemos realizado una regresión lineal entre el logaritmo natural de la tasa anual de mortalidad y la edad. Una vez obtenida la estimación de los parámetros para cada uno de los años involucrados, y mediante la utilización de series de tiempo, éstos fueron proyectados de manera tal de conocer sus valores futuros. Completado dicho proceso, hemos podido construir las tablas de mortalidad venideras para dicho país. Por último, haciendo uso de dicha predicción, y suponiendo que el desarrollo temporal de ambas tablas es similar, logramos obtener las tablas de mortalidad proyectadas para Argentina. Palabras clave: Proyecciones, Azbel, Mortalidad, Área temática: Inferencia estadística, aplicaciones actuariales. }
