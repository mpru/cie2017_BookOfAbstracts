\A
{MODELACIÓN DEL COMPORTAMIENTO ASIMETRICO DE LOS COSTOS EN EMPRESAS ARGENTINAS INCORPORANDO CASOS ATIPICOS CON GAMLSS}
{\Presenting{MARIA INES STIMOLO}\index{STIMOLO, M} y JOSE VARGAS\index{VARGAS, J}}
{\Afilliation{FACULTAD DE CIENCIAS ECONOMICAS}
\\\Email{mstimolo@eco.unc.edu.ar}}
{modelos gamlss; regresión robusta; costos asimétricos} 
 {Otras ciencias económicas, administración y negocios} 
 {Modelos de regresión} 
 {177} 
 {353-2}
{La función de costos de las empresas supone una proporcionalidad simétrica entre los costos y el nivel de ventas, es decir que los costos aumentan y disminuyen en la misma proporción ante el mismo cambio del nivel de actividad. Sin embargo, esta proporcionalidad no se cumple en la práctica y una de estas asimetrías, denominadas costos pegadizos, se manifiesta cuando los costos disminuyen en menor proporción que lo que se hubieran incrementado ante un aumento en el nivel de actividad. Anderson et.al. (2003) propone un modelo log lineal por partes para medir este comportamiento en las empresas y es muy común encontrar valores atípicos. Los trabajos empíricos sobre la problemática han eliminado esos valores extremos considerando distintos criterios. Aunque esos criterios pueden fallar cuando existe enmascaramiento de los mismos, y en consecuencia la estimación resultante no es apropiada. Ello se advierte cuando se realiza el diagnostico de los residuos. Así, a partir de una muestra de empresas argentinas que realizan oferta pública de sus acciones durante el período 2004-2012, se estima el comportamiento de los costos utilizando modelos de regresión robusta y GAMLSS (Generalized Additive Models for Location, Scale and Shape) y se comparan parámetros estimados y empresas identificadas como atípicas proponiendo alternativas para encontrar un modelo que las logrando una mejor estimación. El modelo de regresión robusta identifica valores atípicos, que no difieren significativamente de los encontrados por los modelos tradicionales de detección a partir de las medidas de análisis de residuos del modelo de regresión, la estimaciones encontradas con GAMLSS resultaron superadoras ya que los valores considerados como atípicos en modelos basados en normalidad no resultaron tal cuando se consideraron otras distribuciones. }
