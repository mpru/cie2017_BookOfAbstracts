\A
{ANÁLISIS DE VARIANZA ROBUSTO EN ENSAYOS DE NUTRICIÓN EN POLLOS PARRILLEROS}
{\Presenting{DIANA ALEJANDRA GIORGINI}\index{GIORGINI, D}, STELLA MARIS ZABALA\index{ZABALA, S}, SUSANA FILIPPINI\index{FILIPPINI, O} y LUCIANO PALACIOS\index{PALACIOS, L}}
{\Afilliation{UNIVERSIDAD NACIONAL DE LUJÁN (UNLU)}
\\\Email{dgiorgin@agro.uba.ar}}
{datos outliers; análisis de varianza; métodos robustos} 
 {Ciencias agropecuarias} 
 {Métodos robustos} 
 {72} 
 {163-1}
{La producción avícola en la República Argentina afronta actualmente la prohibición del uso de antimicrobianos como promotores del crecimiento con el empleo de acidificantes, enzimas, antioxidantes, probióticos, beta-adrenérgicos, pigmentadores, estimulantes y extractos vegetales, con el propósito de mantener en niveles aceptables la productividad y eficiencia. El objetivo de esta investigación es estimar y predecir diferencias en el comportamiento de un parámetro productivo en el área avícola, a partir del empleo del Modelo Lineal General (ML) para un análisis de Varianza, en ensayos que pongan a prueba enzimas fitasas. El ensayo se realiza en la Granja Experimental de una empresa ubicada en San Andrés de Giles, utilizándose pollos parrilleros en condiciones de crianza comercial. Se aplicó un diseño Completamente Aleatorizado. La aleatorización, fue realizada considerando, 16 (dieciséis) repeticiones con 4 (cuatro) tratamientos. Se comprobó la homogeneidad de Variancias entre tratamientos. Asimismo, se analizó por métodos robustos si existían datos outliers. Cuando se encontraron, se aplicaron Métodos Robustos o Técnicas de acomodación de los datos anómalos. Estos tienen mayor sensibilidad ante la presencia de datos anómalos o outliers. Están diseñados para realizar inferencias sobre el modelo, reduciendo la posible influencia que pudiera tener la presencia de datos anómalos. La información obtenida se estudió con un Análisis de Variancia para un diseño en Completamente aleatorizado y por métodos robustos. Se utilizaron además de los Métodos Clásicos como la prueba de Tukey ó Dunnett (existe un control o testigo), pruebas robustas para las comparaciones entre grupos y dentro de grupo cuando no existía homogeneidad de variancias, analizando la concordancia entre todas las pruebas. Se deja constancia que las pruebas se plantearon con un nivel de significación del cinco por ciento. }
