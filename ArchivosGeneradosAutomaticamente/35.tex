\A
{GRÁFICOS DE CONTROL MULTIVARIADOS PARA PROCESOS DE ATRIBUTOS DE ALTA CALIDAD}
{\Presenting{SILVIA JOEKES}$^1$\index{JOEKES, S}, ANDREA RIGHETTI$^1$\index{RIGHETTI, A} y MARCELO SMREKAR$^2$\index{SMREKAR, M}}
{\Afilliation{$^1$INSTITUTO DE ESTADÍSTICA Y DEMOGRAFÍA, U.N.C}
\Afilliation{$^2$LAB. DE INGENIERÍA Y MANTENIMIENTO INDUSTRIAL, U.N.C}
\\\Email{silviajks@gmail.com}}
{procesos de alta calidad; gráficos multivariados; gráfico de control multivariado} 
 {Industria y mejoramiento de la calidad} 
 {Control de procesos} 
 {35} 
 {84-1}
{El estudio de múltiples características es importante en los procesos de alta calidad. En las cercanías de cero no conformidades la mayoría de los ítems serán conformes, especialmente cuando se considera una sola característica crítica. En el caso de considerar varias características, los productos pueden ser mejor diferenciados y las oportunidades de mejoras suelen ser más fácilmente identificables, permitiendo obtener mayor información acerca de un producto o proceso. Los gráficos de control de atributos son utilizados para el monitoreo de procesos donde las características de calidad no pueden medirse en una escala continua. Esta situación se observa en procesos industriales, procesos relacionados con la salud, y en procesos de servicios, entre otros. Muchos de estos ejemplos implican el monitoreo simultáneo de múltiples atributos lo que conduce a la aplicación de métodos de control de calidad multivariados que han mostrado ser más adecuados que el uso simultáneo de múltiples procedimientos univariados. En este estudio se presenta en primer lugar una breve revisión de la metodología existente para procesos multivariados de alta calidad. A continuación se muestra la metodología y aplicación de un gráfico de control multivariado, el gráfico Mnp que surgió como una extensión de los gráficos np univariados de Shewhart. Este gráfico emplea una estadística que se obtiene como la suma ponderada de los conteos de unidades no conformes para todas las características de calidad, considerando además la correlación entre atributos. Al mismo tiempo, se muestra un procedimiento sencillo para identificar la presencia de causas asignables ante una señal de fuera de control. Por último se realiza la comparación del gráfico Mnp con los gráficos np univariados, que son los más comúnmente empleados.}
