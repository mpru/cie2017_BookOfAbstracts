\A
{CARACTERIZACIÓN CLIMATOLÓGICA PARA UNA REGIÓN DE LA PROVINCIA DE SANTA CRUZ DESPROVISTA DE INFORMACIÓN}
{\Presenting{JULIO SOTO}$^1$\index{SOTO, J}, DORA SILVIA MAGLIONE$^1$\index{MAGLIONE, D} y OSCAR BONFILI$^2$\index{BONFILI, O}}
{\Afilliation{$^1$UNIVERSIDAD NACIONAL DE LA PATAGONIA AUSTRAL}
\Afilliation{$^2$SERVICIO METEOROLÓGICO NACIONAL, OFICINA METEOROLÓGICA RÍO GALLEGOS}
\\\Email{dmaglione@disytel.net}}
{variables climatológicas; método de la distancia; mapas} 
 {Otras aplicaciones} 
 {Datos faltantes} 
 {95} 
 {211-2}
{Muchas actividades económicas están supeditadas a las condiciones climáticas de la región en la que se desarrollan ya que todo ser vivo depende de las condiciones de la temperie. Sin contar con datos meteorológicos representantivos de una zona, difícilmente se puedan planificar adecuadamente actividades de toda índole. La provincia de Santa Cruz, y en particular la región central es una de las zonas más desprovista de información meteorológica de largo plazo (climatológica) a nivel mundial, y a su vez es una de las áreas mineras más importante de la Patagonia. Con la finalidad de poder caracterizar un sector de la región (de la cual no se poseen registros) se trabajó con información proveniente de distintas fuentes tales como el Servicio Meteorólógico Nacional, la Administración General de Vialidad Provincial, el Instituto Nacional de Tecnología Agropecuria, el Instituto Nacioaan de Investigación Agropecuara de Chile, entre otros. Algunas fenómenos discontinuos como las lluvias tienen una variabilidad temporal mayor por lo que se puede llegar a necesitar más datos. En este trabajo se utiliza la metodología usada por Cressman (1959) para lograr una aproximación iterativa para la elaboración de la variable objetivo en cada punto de una región usando como funciones de pesos las distancias inversas de las redes de observaciones que se poseen. A partir de este método se analizan y caracteriza de forma aproximada a las siguientes variables meteorológicas en forma anual y estacional: Temperatura Media; Temperatura Máxima; Temperatura Mínima; Precipitación; Humedad Relativa; Presión Atmosférica y Velocidad Media de Viento en una región de la provincia de Santa Cruz de la que no existen registros meteorológicos. Además se analizó el Índice de Concentración de Precipitación (ICP); Índice de Aridez De Demartonne y la Evapotranspiración de la FAO (ETo – FAO).}
