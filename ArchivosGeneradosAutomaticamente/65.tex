\A
{ANÁLISIS DE COMPONENTES PRINCIPALES CON DATOS GEORREFERENCIADOS. UNA APLICACIÓN EN LA INDUSTRIA TURÍSTICA}
{\Presenting{LAURA ISABEL LUNA}\index{LUNA, LI} y FERNANDO GARCÍA\index{GARCÍA, F}}
{\Afilliation{FACULTAD DE CIENCIAS ECONÓMICAS, UNC}
\\\Email{lauraisabel.luna@gmail.com}}
{análisis multivariado; multispati-pca; pca; actividades económicas} 
 {Economía} 
 {Métodos multivariados} 
 {65} 
 {156-2}
{Existen nuevos métodos multivariados que permiten mapear el espacio geográfico según la estructura espacial de indicadores económicos como son los datos de actividades relacionadas al turismo que conforman el producto bruto. Dray et al. (2008), proponen un método de análisis multivariado que incorpora la información espacial previo al análisis multivariado, el método es conocido como MULTISPATI-PCA. La técnica ha mostrado ser eficiente en estudios de agricultura de precisión (Córdoba, 2014), y en este trabajo se prueba para variables económicas. Existen interacciones subyacentes entre las diferentes actividades económicas, por ello el análisis de las covariaciones o correlaciones entre las mismas es un aspecto que debe ser considerado en estudios económicos de naturaleza multivariada. No obstante, es importante remarcar que la estructura de covariación reflejada por un análisis multivariado clásico puede verse afectada por los patrones espaciales subyacentes en los datos. Las componentes principales (PC) son apropiadas sólo para resumir variabilidad y no están diseñadas para revelar patrones espaciales. Por ello es necesario utilizar una metodología que resuma la variabilidad y revele estructuras espaciales al mismo tiempo; existen hoy métodos que abarcan estos dos objetivos. En este trabajo se exponen dos técnicas y se realiza un análisis comparativo de los resultados obtenidos, con la implementación de un PCA clásico y de una versión restringida espacialmente (MULTISPATI-PCA, Dray et al., 2008). Se utilizaron datos procesados por la Dirección de Estadísticas Económicas (DEE) de la provincia de Córdoba para el periodo 2001 – 2014. Los mapas de variabilidad espacial construidos a partir de la primera componente de ambas técnicas fueron similares; no así los de la segunda componente debido a cambios en la estructura de co-variación identificada, al corregir la variabilidad por la autocorrelación espacial de los datos.}
