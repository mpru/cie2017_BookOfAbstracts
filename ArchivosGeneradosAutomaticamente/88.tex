\A
{ESTUDIO DE LA DINÁMICA DEL BENEFICIO ENERGÉTICO BRUTO ANUAL EN SISTEMAS DE PASTOREO TRASHUMANTE}
{\Presenting{NATALIA PEREZ}\index{PEREZ, N}, MARCOS EASDALE\index{EASDALE, M} y OCTAVIO BRUZZONE\index{BRUZZONE, O}}
{\Afilliation{CONICET - INTA}
\\\Email{nperezleon@gmail.com}}
{conectividad; fragmentación; ndvi; socio-ecológicos; transformada de fourier; variabilidad} 
 {Ecología y medio ambiente} 
 {Series de tiempo} 
 {88} 
 {196-1}
{El objetivo de este trabajo fue desarrollar un método para estimar el beneficio energético bruto anual (BEBa) para la producción ganadera trashumante, al moverse entre sitios de invernada y veranada. En el pastoreo trashumante las personas y sus rebaños se desplazan estacionalmente desde las tierras bajas, que son pastoreadas durante el invierno (invernadas), hacia las tierras altas situadas en zonas montañosas durante el verano para aprovechar pastos de alta calidad (veranadas). Se seleccionaron cuatro casos biofísicamente contrastantes entre sitios de pastoreo trashumante de la Patagonia noroccidental, Argentina, luego se delimitaron las invernadas y veranadas. Series de imágenes MODIS compuestas cada 16 días, desde febrero de 2000 a septiembre de 2016 (píxeles de 250m x 250m) conteniendo el Índice de Vegetación Diferencial Normalizado (NDVI), fueron usadas como estimador indirecto de la productividad primaria neta aérea (Rouse et al., 1974). Luego estimamos el BEBa como las diferencias absolutas entre las curvas anuales de NDVI de cada veranada e invernada asociadas a un productor trashumante, respectivamente. Luego, el BEB parcial (BEBp) de pasar de una invernada a una veranada (y viceversa), es la diferencia en la productividad estacional calculada cuando un lugar es más productivo que el otro. Se hizo un filtrado de paso de múltiples bandas para obtener el ciclo anual y sus armónicos, así como un filtrado de paso bajo (frecuencias menores a un año), para obtener la variabilidad interanual de la estimación de BEB. Sobre las series filtradas, se estimó el BEBa, para cada año. Entre los resultados más relevantes obtuvimos que el BEBa fue siempre positivo en la mayoría de los casos estudiados, sugiriendo que la movilidad del ganado entre sitios de invernada y veranada es beneficiosa. La variabilidad entre valores sucesivos del BEBa fue baja (menor al 5\%) y estable en el tiempo, corroborando que el grado de incertidumbre para los productores sea relativamente pequeño. Este método permite una estimación rápida del BEBa y su variabilidad en el tiempo (a lo largo de los años). El mismo podría servir de base para desarrollar un sistema de seguimiento y alerta temprana de cambios en el BEB en un momento determinado, con el fin de anticipar decisiones productivas. }
