\A
{GRÁFICOS DE CONTROL DE PROCESOS DE ALTA CALIDAD MULTI-ATRIBUTO MEDIANTE LA DISTRIBUCION MULTINOMIAL}
{\Presenting{ANDREA RIGHETTI}$^1$\index{RIGHETTI, A}, SILVIA JOEKES$^1$\index{JOEKES, S} y MARCELO SMREKAR$^2$\index{SMREKAR, M}}
{\Afilliation{$^1$INSTITUTO DE ESTADÍSTICA Y DEMOGRAFÍA, U.N.C}
\Afilliation{$^2$LAB. DE INGENIERÍA Y MANTENIMIENTO INDUSTRIAL, U.N.C}
\\\Email{analizamos@yahoo.com.ar}}
{procesos de alta calidad; gráficos multi-atributos; gráfico de  control} 
 {Industria y mejoramiento de la calidad} 
 {Control de procesos} 
 {46} 
 {101-1}
{El mejoramiento continuo es esencial para la supervivencia y el crecimiento de las empresas que requieren de una permanente reducción en el nivel de no conformidades de sus procesos. Para lograr altos estándares de calidad, las empresas necesitan de la aplicación de programas de gestión y de procedimientos estadísticos adecuados. En este aspecto, el Control Estadístico de Procesos (CEP) sigue desempeñando un rol fundamental en el esfuerzo hacia el mejoramiento continuo. Controlar y asegurar la conformidad del producto con las especificaciones en niveles cercanos a cero defectos, requiere de procedimientos estadísticos, tales como gráficos de control cada vez más complejos. A pesar de que en la última década ha habido un creciente interés por el desarrollo de gráficos de control de procesos de alta calidad, existen a la fecha pocos procedimientos para controlar procesos cuya calidad está determinada por la presencia o ausencia de varias características no medibles, es decir, procesos en los que cada ítem es susceptible de ser clasificado en dos o más categorías simultáneamente. Un método de generalización de la distribución binomial es la distribución multinomial que puede ser empleada en la construcción de un gráfico de atributo multivariado. Este gráfico llamado gráfico de atributo multinomial (MAC) tiene en cuenta la correlación definiendo adecuadamente las clases de calidad de un producto. En este trabajo se presenta una aplicación considerando dos características de calidad binomiales. El procedimiento consiste en adicionar una nueva variable que surge del conteo de no conformidades para ambas variables simultáneamente más una categoría adicional que considera los ítems conformes. Para ello se aplica la estadística $\chi^2$, se determina el atributo responsable de la salida de control del proceso mediante una estadística Z y se evalúa su rendimiento mediante la longitud media de corrida (ARL).}
