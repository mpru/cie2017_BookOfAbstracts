\A
{BÚSQUEDA DE TAMAÑO DE MUESTRA Y DIMENSIÓN DE UNIDAD MUESTRAL ADECUADOS PARA LA CONFECCIÓN DE MAPAS DE RENDIMIENTO EN MANZANOS EN EL ALTO VALLE DE RÍO NEGRO Y NEUQUÉN}
{\Presenting{S OCAMPO}$^1$\index{OCAMPO, S}, D FERNÁNDEZ$^1$\index{FERNÁNDEZ, D}, M CURETTI$^1$\index{CURETTI, M} y S BRAMARDI$^2$\index{BRAMARDI, S}}
{\Afilliation{$^1$INTA}
\Afilliation{$^2$UNCOMA}
\\\Email{ocampo.santiago@inta.gob.ar}}
{agricultura de precisión; geoestadística; pronósticos de rendimiento; datos geo-referenciados} 
 {Otras aplicaciones} 
 {Otras categorías metodológicas} 
 {102} 
 {226-1}
{El mapa de rendimiento es una de las herramientas más utilizadas en agricultura de precisión. Con el objetivo de determinar el tamaño de muestra más apropiado para generar mapas de rendimiento con la menor pérdida de información utilizando la técnica de geoestadística, se relevaron datos de manera exhaustiva de una parcela de árboles de manzano cv. Crisp Pink durante la temporada 2014-2015. Se cosecharon y pesaron todos los frutos de cada planta de una parcela de 9 filas con 61 plantas en un marco de plantación de 4 x 1,5 m. A cada una de las plantas se les asignó una coordenada en el sistema UTM. Para la confección de los mapas se utilizó el software GS+. En un primer estudio se conformaron grupos de plantas adyacentes dentro de cada fila (ej.: 274 grupos de 2 plantas, 183 de 3, etc., hasta 18 grupos de 30 plantas). Se calculó la producción promedio de cada grupo y su coordenada promedio. Para cada tamaño de agrupamiento, se realizó un mapa de rendimiento y se evaluó su capacidad de representación respecto del mapa obtenido con el muestreo exhaustivo de las 549 plantas. La comparación se hizo mediante el coeficiente de correlación r. Se encontró un descenso de la capacidad de representación a medida que el nivel de agrupamiento se incrementa, con un r=0,98 si se mapea con los 549 datos, un r=0,63 si se mapea con los datos agrupados de a 6 plantas vecinas y un r=0,30 si se mapea con los datos agrupados de a 30 plantas vecinas. De este modo, no parece conveniente confeccionar mapas utilizando esta técnica, agrupando más de 6 plantas. En un segundo estudio, de los 549 datos de rendimiento disponibles, se seleccionaron diferentes muestras de manera aleatoria y se confeccionaron mapas geoestadísticos a partir de los datos de cada muestra. Se utilizaron tamaños de muestra de 50, 100, 150 y hasta 450. Para cada caso se comparó la capacidad de representación de los mapas resultantes vs. el mapa obtenido con los 549 datos, mediante el coeficiente de correlación. Como es de esperar, la capacidad de representación se incrementa con el tamaño de muestra, obteniéndose un r=0,36 para una muestra de tamaño 50, r=0,60 para 150, r=0,70 para 250, entre otros. La muestra aleatoria de 150 unidades parecería ser la menor a partir de la cual se logra una representación satisfactoria del rendimiento del cuadro.}
