\A
{PONDERACIÓN DE LA TÉCNICA “SUPPORT VECTOR MACHINE” (SVM) PARA EL MEJORAMIENTO EN LA PREDICCIÓN DE LA DISTRIBUCIÓN DE TAMAÑOS A COSECHA DE FRUTOS DE PERAS CULTIVAR BEURRE D’ ANJOU}
{\Presenting{GUSTAVO GIMÉNEZ}$^1$\index{GIMÉNEZ, G}, NATALIA RUBIO$^1$\index{RUBIO, N}, DOLORES DEL BRÍO$^2$\index{DEL BRÍO, D} y SERGIO BRAMARDI$^1$\index{BRAMARDI, S}}
{\Afilliation{$^1$DEPARTAMENTO DE ESTADÍSTICA, FACULTAD DE ECONOMÍA, UNIVERSIDAD NACIONAL DEL COMAHUE}
\Afilliation{$^2$CONSEJO NACIONAL DE INVESTIGACIONES CIENTÍFICAS Y TÉCNICAS-INSTITUTO NACIONAL DE TECNOLOGÍA AGROPECUARIA.}
\\\Email{gustavo.gimenez@faea.uncoma.edu.ar}}
{datos multicategóric; "statistical learning"; penalización; vectores soporte} 
 {Ciencias agropecuarias} 
 {Minería de datos} 
 {105} 
 {229-2}
{La técnica denominada “Support Vector Machine (SVM)” se basa en la teoría de “statistical learning” y consiste en aplicar un método lineal sobre un conjunto de datos en un espacio de alta dimensionalidad. Es una técnica que se aplica a distintos problemas del aprendizaje estadístico como: clasificación, regresión y detección de datos nuevos (“Novelty Detection”). La SVM se ha aplicado en pronóstico de diversos cultivos con resultados muy precisos. Su aplicación no requiere conocer la relación matemática entre variables de entrada (predictoras) y de salida (respuesta) ni la distribución de probabilidad subyacente y es aplicable a enormes bases de datos, con bajo costo computacional. El presente trabajo retoma un conjunto de curvas de crecimiento de peras D’Anjou medidas en milímetros y evaluados en días a partir de la floración donde cada fruto fue categorizado por el tamaño comercial a cosecha. En dichos datos se implementó SVM 30 y 40 días previos a la cosecha para predecir los tamaños de los frutos a partir del diámetro de los mismos y su clasificación en tamaños comerciales. El SVM se calibró con un kernel lineal y el parámetro C =1. El parámetro C controla la penalidad por las clasificaciones erróneas del SVM en datos de entrenamiento. Cuanto mayor es el valor de C, mayor la reducción de los errores de clasificación, incrementado la rigidez del algoritmo y la posibilidad de sobreajustar los datos. De esta forma se logró una precisión (“accuracy”) de 0.60, una tasa de no información de 0.26 y el test de no información vs precisión un p-valor <0,0001. No obstante, la matriz de confusión mostró que la predicción de frutos grandes y pequeños alcanzaba valores de 0.54 y 0.5 respectivamente. Una solución a este desbalance fue ponderar los vectores soporte por las proporciones de los tamaños comerciales de los datos de entrenamiento. Las ponderaciones permiten sesgar el modelo afectando el parámetro C para cada categoría de clasificación y compensar las categorías que están menos representadas en los datos. Éstas pueden ser realizadas según distintos criterios: 1/w, w-1/2 ,1500*w-1/2, donde w es la proporción de cada categoría. En los datos de peras cultivar D’Anjou, el criterio de ponderación que obtuvo mayor precisión fue w -1/2 alcanzando el 0.63, mejorando la precisión en los tamaños pequeños y grandes a valores de 0.93 y 0.81 respectivamente, muy superiores a los valores sin ponderar.}
