\A
{VARIABILIDAD ESPACIO-TEMPORAL DE LAS PRECIPITACIONES AL OESTE DE LA REGIÓN PAMPENA – IMPLICANCIAS PARA LA PRODUCCIÓN DE GRANOS}
{\Presenting{TERESA BOCA}$^1$\index{BOCA, T} y PABLO CIPRIOTTI$^2$\index{CIPRIOTTI, P}}
{\Afilliation{$^1$INSTITUTO NACIONAL DE TECNOLOGÍA AGROPECUARIA (INTA). DPTO. DE MÉTODOS CUANTITATIVOS Y SISTEMAS DE INFORMACIÓN, FACULTAD DE AGRONOMÍA, UNIVERSIDAD DE BUENOS AIRES (FA-UBA)}
\Afilliation{$^2$DPTO. DE MÉTODOS CUANTITATIVOS Y SISTEMAS DE INFORMACIÓN, FACULTAD DE AGRONOMÍA, UNIVERSIDAD DE BUENOS AIRES (FA-UBA) INSTITUTO DE INVESTIGACIONES FISIOLÓGICAS Y ECOLÓGICAS VINCULADAS A LA AGRICULTURA}
\\\Email{tereboca@gmail.com}}
{autocorrelación; climatología; geoestadística; precipitaciones; variograma; producción de granos} 
 {Ciencias agropecuarias} 
 {Estadística espacial} 
 {73} 
 {165-1}
{Durante las últimas décadas ha habido un corrimiento de la frontera agrícola hacia al oeste de la Región Pampeana sobre áreas más limitantes para los cultivos en secano. Este desplazamiento de la agricultura se dio hacia regiones de mayor variabilidad inter-anual en las precipitaciones, como es el este de la provincia de La Pampa, con un claro impacto sobre el rinde y la seguridad de cosecha de los principales cultivos. El objetivo de este trabajo fue describir la variabilidad espacio-temporal de las precipitaciones en dicha zona, que ocupa más de 130,000 km2. Se trabajó con series históricas de precipitaciones mensuales de 85 sitios con registros entre 50-94 años en el período 1921-2015 de la red provincial pampeana del agua. Se definieron 16 variables de respuesta por sitio: precipitación anual, precipitaciones acumuladas para tres cultivos: trigo (jun-dic), maíz (sep-mar) y soja (nov-abr), y 12 precipitaciones mensuales. Para cada variable respuesta se describió la variabilidad espacio-temporal a través de: 1- estimar las tasas de cambio a través del tiempo mediante análisis de regresión lineal en cada sitio; 2- modelar la estructura de correlación espacial del estimador de la pendiente temporal a partir de variogramas; Y 3- mapear la estructura espacial de la pendiente en la región a través de kriging ordinario. De las 1360 regresiones analizadas (85 sitios x 16 precipitaciones) el 28\% presentaron pendientes significativas (alfa=5\%), sin embargo los resultados dependieron mucho de la variable de precipitación considerada. Tanto la precipitación anual como la de los cultivos de verano (maíz y soja) tuvieron más del 75\% de las pendientes significativas, aunque con valores de R2 bajos (<30\%) y estimadores puntuales entre 1,2–4,6 mm/año. Esto se debió a que los cambios más consistentes en las precipitaciones mensuales a través del tiempo ocurrieron en los meses de verano y otoño (Ene-Abr), sosteniendo el corrimiento de la frontera agrícola. El análisis geo-estadístico mostró una correlación espacial estructurada variable en las tasas de cambio temporales (20-80\%), siendo mayor para las precipitaciones mensuales de Enero, Febrero, Marzo y el periodo del cultivo de soja, con alcances entre 50-150 km. Los resultados encontrados están en línea con la tercera comunicación nacional de la República Argentina para la Convención de las Naciones Unidas sobre el Cambio Climático (IPCC) que afirma que los mayores cambios se registraron en el este del país y con las variaciones más importantes en algunas zonas semiáridas. }
