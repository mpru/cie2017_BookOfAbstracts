\A
{DESARROLLO DE UN MODELO PREDICTIVO PARA EL ÍNDICE DE RETENCIÓN DE COMPUESTOS ORGÁNICOS VOLÁTILES}
{\Presenting{RICARDO BAQUERO}$^1$\index{BAQUERO, R}, ROMMEL LARGO$^1$\index{LARGO, R}, VIVIANA MARQUEZ$^1$\index{MARQUEZ, V}, CRISTIAN ROJAS$^2$\index{ROJAS, C}, PIERCOSIMO TRIPALDI$^3$\index{TRIPALDI, P} y MÓNICA BALZARINI$^4$\index{BALZARINI, M}}
{\Afilliation{$^1$FACULTAD DE CIENCIAS AGROPECUARIAS, FACULTAD DE CIENCIAS ECONÓMICAS Y FACULTAD DE MATEMÁTICA, ASTRONOMÍA, FÍSICA Y COMPUTACIÓN. UNIVERSIDAD NACIONAL DE CÓRDOBA, CÓRDOBA, ARGENTINA}
\Afilliation{$^2$INSTITUTO DE INVESTIGACIONES FISICOQUÍMICAS TEÓRICAS Y APLICADAS (INIFTA), CONICET, UNLP, LA PLATA, ARGENTINA}
\Afilliation{$^3$VICERRECTORADO DE INVESTIGACIONES. UNIVERSIDAD DEL AZUAY, CUENCA, ECUADOR}
\Afilliation{$^4$FACULTAD DE CIENCIAS AGROPECUARIAS. UNIVERSIDAD NACIONAL DE CÓRDOBA. CONICET. CÓRDOBA, ARGENTINA}
\\\Email{crojasvilla@gmail.com}}
{perfiles aromáticos; regresión; qspr} 
 {Ciencias agropecuarias} 
 {Modelos de regresión} 
 {168} 
 {345-1}
{El objetivo de este trabajo es el desarrollo de una relación cuantitativa estructura-propiedad (QSPR) para predecir los índices de retención de Kovats de compuestos volátiles orgánicos. Se trabajó con 63 compuestos detectados mediante cromatografía de gases/espectrometría de masas en la pimienta negra (Piper nigrum L.). Para cada compuesto volátil se calcularon 1642 descriptores moleculares, independientes de la conformación, en el programa Dragon versión 7. La base de datos se dividió de forma aleatoria en grupos de calibración y validación con 25 moléculas cada uno y un grupo de predicción con 13 compuestos. Seguidamente, se usó un método secuencial de selección de variables sobre los 50 compuestos para la búsqueda de los mejores modelos de regresión de 1 a 7 descriptores. El criterio de selección fue la desviación estándar residual (S) durante validación. En cada regresión se han analizado el coeficiente R2 y S como parámetros estadísticos de calidad. El mejor modelo QSPR quedó constituido por 3 descriptores moleculares (R2 = 0.93 y S = 60), cuya calidad ha sido verificada mediante las técnicas de validación interna de dejar-uno-fuera (R2 = 0.88 y S = 81.5) y dejar-varios-fuera de ventanas venecianas con 5 grupos de validación (R2 = 0.88 y S = 76.6). Adicionalmente, se evaluó la capacidad predictiva del modelo QSPR mediante la predicción de los índices de retención de los 13 compuestos externos al ajuste del modelo. Finalmente, se ha definido el dominio de aplicabilidad mediante el cálculo del valor de influencia crítico (h* = 0.24), con el cual se ha verificado que todos los índices de retención predichos pertenecen al dominio de ajuste del modelo. La relación cuantitativa estructura-propiedad estuvo constituida por los descriptores índice anillo distancia/detour de orden 4 (D/Dtr04), índice tipo Randic; promedio a partir de la matriz de distancias topológicas (ChiAD) y autocorrelación de Moran a desplazamiento 1 ponderado por el volumen de van der Waals (MATS1v).}
