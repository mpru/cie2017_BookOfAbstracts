\A
{MODELO LOG-ACUMULADO MIXTO APLICADO A LA TASA RESPIRATORIA BAJO ESTRES SEVERO DEBIDO A TEMPERATURAS EXTREMAS EN OVINOS}
{\Presenting{NATALIA RUBIO}$^1$\index{RUBIO, N}, M FERNANDA LOPEZ ARMENGOL$^2$\index{LOPEZ ARMENGOL, M@LÓPEZ ARMENGOL, M} y GUSTAVO GIMÉNEZ$^1$\index{GIMÉNEZ, G}}
{\Afilliation{$^1$DEPARTAMENTO DE ESTADÍSTICA, FACULTAD DE ECONOMÍA Y ADMINISTRACIÓN UNCO}
\Afilliation{$^2$CENTRO DE INVESTIGACIONES EN TOXICOLOGÍA AMBIENTAL Y AGROBIOTECNOLOGÍA DEL COMAHUE}
\\\Email{rubionatalia@yahoo.com.ar}}
{estrés térmico; respuesta ordinal; modelo multicategórico} 
 {Ciencias agropecuarias} 
 {Datos categóricos} 
 {113} 
 {245-1}
{El clima es uno de los factores más importantes a tener en cuenta en la cría de animales debido a que los cambios en las variables climáticas como temperatura, humedad, movimiento del aire, fotoperiodo y radiación solar no pueden ser controlados, pero si estudiados. Los mismos afectan de forma directa el aspecto productivo y reproductivo de ovejas. Se ha estudiado que las mismas cuentan con una gran habilidad para adaptarse a distintas condiciones climáticas. La adaptación y habituación han hecho y siguen haciendo posible la vida en los distintos ambientes. En relación con ello es notable que la oveja se considere un animal con un espectro extremadamente amplio de adaptación a los diversos climas con diferentes condiciones micro ambientales y de disponibilidad de alimentos; destacándose su alta capacidad de adaptación en ambientes donde prevalecen climas fríos o templados, áridos o secos y tropicales y húmedos. Sin embargo, a pesar de esta versatilidad fisiológica y biológica de tolerar diversas condiciones ambientales, el estrés térmico repercute negativamente en los procesos vitales y funciones relacionadas con la reproducción animal. Se desarrolló una experiencia en la localidad de Cinco Saltos, Río Negro, que cuenta con condiciones climáticas controladas en una cámara de calor, resultando éste estudio de relevancia regional. Se midió la tasa respiratoria, número de jadeos por minuto, a seis carneros en condiciones físicas similares a diferentes temperaturas para evaluar el estrés generado por el calor. Para analizar el efecto que tiene la lana en los animales, tres de ellos fueron esquilados y tres se dejaron sin esquilar. También se registró la posición del animal durante el estudio, debido a que puede afectar su tasa respiratoria. Se consideró la respuesta según cinco categorías ordinales que permitió construir un modelo log-acumulado mixto, que contempla la variabilidad entre carneros, posición del carnero (parado – sentado), condición (esquilado – no esquilado) e índice de temperatura-humedad (ITH).Se observó un efecto significativo de los individuos. En la estimación de los parámetros del modelo se observa una interacción triple significativa. Se continuó analizando dentro de la condición de los carneros. Para los esquilados a medida que se incrementa el ITH, la chance de aumentar de categoría respiratoria es 2 veces mayor; mientras que, para los no esquilados es 1,5 veces. En ambos casos no importa la posición en que se encuentra el carnero.}
