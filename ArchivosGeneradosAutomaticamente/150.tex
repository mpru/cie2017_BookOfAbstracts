\A
{ESTIMACIÓN DEL TRASVASE DE VOTOS ENTRE PARTIDOS APLICACIÓN DE UN MODELO MATEMÁTICO A LOS RESULTADOS DE LA ELECCIONES DE LA PROVINCIA DE SANTA FE AÑOS 2011-2015}
{\Presenting{GREGORIO GARCÍA}\index{GARCÍA, G} y JOSÉ ALBERTO PAGURA\index{PAGURA, J}}
{\Afilliation{ESCUELA DE ESTADÍSTICA. FACULTAD DE CIENCIAS ECONÓMICAS Y ESTADÍSTICA. UNIVERSIDAD NACIONAL DE ROSARIO}
\\\Email{jpagura@fcecon.unr.edu.ar}}
{movilidad electoral; programación lineal; simplex} 
 {Sociología} 
 {Otras categorías metodológicas} 
 {150} 
 {321-1}
{Un tema relevante a la hora del análisis de los resultados de las elecciones, es el desplazamiento de votos de un partido hacia otro conocido como “trasvase de votos”. Quienes se interesan en conocer el comportamiento de este fenómeno, por lo general buscan la información por medio de encuestas. Además del costo que representan estos estudios y el tiempo que transcurre hasta la obtención de los resultados finales, puede haber falta de respuesta, o respuestas falsas por olvido u omisión. En el año 2014, el académico valenciano Rafael Romero Villafranca publicó un procedimiento matemático para estimar el trasvase de votos entre partidos a partir de los resultados electorales en diferentes unidades territoriales de una región de interés, aplicándolo con éxito en diferentes elecciones europeas hasta autonómicas. Dicho método se basa en programación lineal y se puede aplicar sobre los resultados electorales de una elección y la anterior. En este procedimiento es crucial la hipótesis de homogeneidad, esto es, la probabilidad de pasar de votar un partido a votar otro, es la misma para todas las unidades territoriales. Otra hipótesis necesaria para ejecutar el proceso de optimización, es el establecimiento de un “umbral de lealtad” es decir, la proporción de votantes que en ambas elecciones votan al mismo partido. Un aparente problema es la forma en la que se consideran aquellos votantes que se incorporan a la elección actual y aquellos que votaron en la elección anterior y ya no forman parte del padrón electoral, cuya solución es la creación de dos partidos ficticios llamados “altas” y “bajas”. Para la inclusión de votos en blanco y nulos se emplea el mismo recurso. En el presente trabajo, se muestran las estimaciones de trasvase de votos entre los años 2011 y 2015 en las elecciones a gobernador, para cada departamento de la provincia de Santa Fe tomando como unidad territorial cada localidad. El procedimiento fue implementado computacionalmente en R. }
