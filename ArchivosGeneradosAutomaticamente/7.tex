\A
{MUESTREO ESTRATIFICADO Y ENFOQUE DE MODELOS PARA ESTIMAR TOTALES CON DATOS ESPACIALES EN ESTUDIOS SOCIOECONÓMICOS}
{\Presenting{VIRGINIA LAURA BORRA}\index{BORRA, V}, CÉSAR ANTONIO MIGNONI\index{MIGNONI, C} y JOSÉ ALBERTO PAGURA\index{PAGURA, J}}
{\Afilliation{FACULTAD DE CIENCIAS ECONÓMICAS Y ESTADÍSTICA-UNIVERSIDAD NACIONAL DE ROSARIO}
\\\Email{vborra@fcecon.unr.edu.ar}}
{datos espaciales; muestreo estratificado; enfoque de modelos; modelos de semivariograma} 
 {Economía} 
 {Muestreo} 
 {7} 
 {12-1}
{En los estudios por muestreo, es frecuente encontrarse con la necesidad de realizar inferencias en poblaciones cuyas unidades se encuentran ubicadas en el espacio. En esta situación, por lo general, se observará la presencia de autocorrelación espacial y el plan de muestreo puede volverse más eficiente si se tiene en cuenta esta característica. En estos casos, se ha propuesto la selección de la muestra mediante muestreo sistemático o muestreo estratificado y podrán lograrse importantes mejoras en la precisión de las estimaciones si se emplea la información proporcionada por un modelo que exprese dicha autocorrelación espacial (modelo de semivariograma) en dicha selección. Diferentes autores han presentado expresiones matemáticas para evaluar la eficiencia de los métodos de selección mencionados con respecto al muestreo aleatorio simple como funciones de la correlación espacial. Por otra parte, la información contenida en el semivariograma, también puede utilizarse en el proceso de estimación recurriendo al enfoque inferencial basado en modelos, el cual basa la estimación de los valores poblacionales en la predicción de los valores de la variable, en las unidades no incluidas en la muestra, por medio de la utilización de un modelo teórico. Este modelo puede contener variables auxiliares relacionadas con la variable en estudio, así como la información proporcionada por el semivariograma. Como medida de precisión, se estima el error cuadrático medio del predictor en la superpoblación. Borra (2015), Pagura y otros (2015) presentan la aplicación de los métodos de selección mencionados en estudios socioeconómicos, mostrando mejor eficiencia en los estimadores de simple expansión con respecto al muestreo aleatorio simple, considerando el error cuadrático medio en la población finita (enfoque de diseño). Además muestran que el empleo del predictor utilizando modelos de semivariograma, produce mejoras en el error cuadrático medio para la población finita (enfoque asistido por modelos). Por otra parte, la estratificación espacial permite definir grupos homogéneos, sin embargo las variables que se estudian podrían aun presentar correlación espacial dentro de cada estrato. Por este motivo, puede plantearse el uso de semivariogramas para obtener, con el enfoque de modelos, estimaciones separadas o combinadas de totales y promedios. En este trabajo se presentan los resultados de esta propuesta para la estimación del total de hogares con necesidades básicas insatisfechas en la ciudad de Rosario en el 2010, tomando como unidad muestral el radio censal, y se aprovecha la disponibilidad de la información censal, para la realización de comparaciones. }
