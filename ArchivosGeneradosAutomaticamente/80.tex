\A
{UNIT ROOT TESTING UNDER STRUCTURAL BREAKS}
{\Presenting{SERGIO MARTÍN BUZZI}$^1$\index{BUZZI, S} y SILVIA MARÍA OJEDA$^2$\index{OJEDA, S}}
{\Afilliation{$^1$FCE - UNC}
\Afilliation{$^2$FAMAF}
\\\Email{sergiomartinbuzzi@gmail.com}}
{unit root; time series; structural breaks; stock markets} 
 {Economía} 
 {Series de tiempo} 
 {80} 
 {181-2}
{In this paper are analyzed eleven stock market indices in order to conclude about their integration orders. With this objective in mind, three tests are performed. The first test, is the standard Augmented Dickey-Fuller (ADF) unit root test. It is known that this test can face problems of lack of power, failing to reject the null hypothesis being the series in fact integrated of order zero, I(0), erroneously concluding the existence of unit roots. The second test, is the Kwiatkowski, Phillips, Schmidt and Shin (KPSS) stationarity test, which complements the ADF test, provided that its null hypothesis is that the series under analysis is stationary. The third test posed is the Kapetanios unit root test, which is an extension of Zivot and Andrews’ unit root test for the case of up to m structural breaks. This third test is intended to solve another problem faced by standard ADF test which could conclude the existence of a unit root, when in fact the series is integrated of order zero with a broken deterministic tend. The estimations are performed using daily data for a long time period, for the nine greater world stock markets indices plus Bovespa and Merval indices. The testing procedures are run in the open source statistical programming language R. Moreover, an R procedure is written in order to perform the Kapetanios test, modifying the existing ur.za function from urca package. Finally the results from those tests are compared and interpreted, reaching the conclusion that the series are integrated of order one, I(1).}
