\A
{HETEROGENEIDAD DE LAS PREFERENCIAS DE LOS CONSUMIDORES DE ALIMENTOS DE CALIDAD DIFERENCIADA}
{BEATRIZ LUPÍN\index{LUPÍN, B}}
{\Afilliation{GRUPO ECONOMÍA AGRARIA, FCEYS-UNMDP}
\\\Email{beatrizlupin@gmail.com}
}
{alimentos; consumidores; calidad diferenciada; preferencias; heterogeneidad} 
 {Economía} 
 {Datos categóricos} 
 {3} 
 {3-1}
{El objetivo es estudiar la heterogeneidad en las preferencias de los consumidores. A tal fin, se desarrollan aspectos conceptuales de los Modelos Logit Condicional (MLC) y Logit Mixto (MLM), presentando resultados obtenidos en la primera etapa de la investigación. Mediante los experimentos de elección (EE), los consumidores declaran sus preferencias, seleccionando, por bloques, alternativas de un bien -combinación de niveles de atributos-. Se basan en el Modelo de Utilidad Aleatoria (MUT):        Unik = B´ Xnik + Enik -Unik = utilidad proporcionada por la alternativa i del bloque k al individuo n; B = vector de coeficientes no observados; Xnik = vector de variables observadas; Enik = término de error, iid Valor-Extremo Tipo I (Gumbel)-. El MLC estima un parámetro común para cada atributo. La heterogeneidad es incorporada realizando estimaciones por segmentos de consumidores según variables socio-económicas o a través de la interacción de los atributos con dichas variables. Por su parte, el MLM captura directamente la heterogeneidad, presentando dos versiones, conforme el tratamiento dado al MUT: Parámetros Aleatorios: centrada en las variaciones de las preferencias: Unik = Bn´ Xnik + Enik -Bn = vector de coeficientes no observados para cada individuo, varían en la población-. Componentes del Error: centrada en los patrones de sustitución de las alternativas: Unik = B´ Xnik + (Nnik + Enik) -Nnik = término aleatorio, correlacionado sobre las alternativas-. En ambos casos, mediante simulación, se aproxima la probabilidad de elección, asumiendo una determinada distribución para los coeficientes o para N, según corresponda. La selección de la distribución constituye un reto empírico. Con datos de un EE referido a papa producida con bajo impacto ambiental (Mar del Plata, octubre 2012, 402 casos), se aplicó un MLC de efectos principales, segmentando la muestra por nivel socio-económico. Luego, con los coeficientes estimados, se calculó la disposición a pagar por los atributos “contenido de agroquímicos” y “aptitud culinaria”. Los participantes de nivel socio-económico medio (45\% de la muestra) son los están dispuestos a pagar más: \$ 4 y \$ 2, adicionales, por papa con bajo contenido de agroquímicos y de buena calidad culinaria, respectivamente, de lo que pagan por papa sin dichas cualidades. Estas especificaciones describen limitadamente la heterogeneidad pues no indican la variación en las preferencias de individuos con iguales características. Dado que una adecuada identificación de la heterogeneidad mejora la potencia explicativa, se tiene previsto continuar la investigación con la aplicación del MLM. }
