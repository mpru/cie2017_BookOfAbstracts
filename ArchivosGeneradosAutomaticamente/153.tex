\A
{SELECCIÓN DE ESTRUCTURAS DE (CO)VARIANZAS EN EL ANÁLISIS DE DATOS EN ENSAYOS SILVICULTURALES}
{\Presenting{CESAR GASTÓN TORRES}$^1$\index{TORRES, C}, ESTEBAN JIMÉNEZ ALFARO$^2$\index{JIMÉNEZ ALFARO, E} y DANIELA LAURA PICARDI$^3$\index{PICARDI, D}}
{\Afilliation{$^1$EEA INTA BELLA VISTA}
\Afilliation{$^2$ESCUELA DE CIENCIAS AGRARIAS, UNIVERSIDAD NACIONAL COSTA RICA}
\Afilliation{$^3$DEPARTAMENTO DE PRODUCCIÓN ANIMAL, FACULTAD DE AGRONOMÍA - UBA}
\\\Email{torresgastonc@gmail.com}}
{medidas repetidas; máxima verosimilitud; modelos lineales mixtos; observaciones no equiespaciadas} 
 {Ciencias agropecuarias} 
 {Series de tiempo} 
 {153} 
 {327-1}
{El estudio de la producción forestal requiere de estimaciones de crecimiento que brinden información de los recursos actuales, además dichas estimaciones deben predecir de manera fiable el comportamiento futuro de los mismos. Para ello, se establecen ensayos que funcionan durante períodos preestablecidos y generan información en el tiempo de variables dasométricas de interés, como es el caso de la Diámetro Normal (Dn). Durante el ajuste se utilizan múltiples mediciones en una misma unidad de observación, ello determina la falta de independencia entre mediciones provenientes de la misma unidad, las cuales están autocorrelacionadas con un grado determinado por el método de medición. Mediante técnicas que maximizan la verosimilitud del modelo es posible modelar la autocorrelación implícita o explícitamente. En el presente trabajo, se llevó a cabo el ajuste de modelos mixtos de datos longitudinales con diferentes estructuras de varianzas y covarianzas, para el Dn. Se analizó datos de un ensayo de preparación de suelo (4 tipos) con tres materiales comerciales de pino, en un diseño Split Plot con cuatro bloques completamente aleatorizados. Sobre la parcela principal se asignó a la preparación del terreno y los materiales a la subparcela. La variable analizada se midió a las edades de 34, 49, 74, 93 y 114 meses. Se evaluó un modelo mixto con efectos fijos (edad, tratamiento y materiales) con sus respectivas interacciones y efectos aleatorios de bloque (b) y parcela anidada en bloque (u). En el modelo alternativo se eliminó las interacciones no significativas de los efectos fijos. Ambos modelos se ajustaron considerando homocedasticidad y varianzas homogéneas dentro de edad. Las estructuras de correlación evaluadas corresponden a: autorregresiva de primer orden (AR1), autorregresiva continua de orden uno (CAR1) y totalmente desestructurada (UN). La selección del modelo se realizó mediante el Criterio de Información de Akaike (AIC). El modelo que mostró el menor valor de AIC (665.5) fue la variante reducida con componentes de varianza homogéneas dentro de edad, al ajustar la autocorrelación de forma autorregresiva a nivel de subparcela. Los parámetros estimados fueron u2=0.41(varianza de u), e2=2.85 (varianza residual) y $\rho$ = 0.558 (coeficiente de correlación). Los coeficientes de ponderación de los componentes de la varianza fueron 1, 0.39, 0.39, 0.54 y 0.58 para las cinco edades respectivas. Los resultados muestran que la inclusión de la matriz de (co)varianzas disminuye 99 puntos el valor de AIC.}
