\A
{ANÁLISIS DE COMPONENTES PRINCIPALES GEOGRÁFICAMENTE PONDERADAS UNA PROPUESTA DE ÍNDICE DE BANCARIZACIÓN PARA LA REGIÓN CENTRO (ARGENTINA)}
{FERNANDO GARCÍA\index{GARCÍA, F}}
{\Afilliation{FACULTAD DE CIENCIAS ECONÓMICAS - UNIVERSIDAD NACIONAL DE CÓRDOBA}
\\\Email{fgarcia.unc@gmail.com}
}
{componentes principales geográficamente ponderadas; bancarización; índice} 
 {Economía} 
 {Estadística espacial} 
 {4} 
 {7-1}
{En este trabajo se propone la construcción de un Índice de Bancarización (IB) para la Región Centro (Argentina). Se utiliza información a nivel departamental del Censo 2010 proporcionada por el Instituto Nacional de Estadísticas y Censos e información sobre el sistema bancario provista por el Banco Central de la República Argentina. El indicador considera aspectos referentes a las diferentes dimensiones de análisis: Magnitudes Agregadas, Disponibilidad y Cobertura Geográfica y Utilización y Acceso. Como los datos están georreferenciados, el Análisis de Componentes Principales (PCA) puede verse afectado por los patrones espaciales subyacentes en los datos. En tal sentido, resulta adecuado aplicar PCA Geográficamente Ponderadas (GWPCA) que permite no solo incorporar la dimensión espacial de los datos sino también considerar situaciones donde los datos espaciales no son bien descriptos por un modelo global. Los resultados muestran que las Componentes Principales (CPs) seleccionadas permitirían una mejor visualización de la heterogeneidad espacial y corroborar que GWPCA constituye una estrategia superadora en relación a PCA. De esta manera, se recomienda avanzar en la construcción del IB a partir de las CPs obtenidas a través de esta metodología. Un índice de este tipo es importante en tanto facilita la comprensión de la bancarización y contribuye a que exista un reconocimiento respecto a su trascendencia como elemento que puede apoyar al crecimiento y desarrollo económico. Se destaca la provincia de Santa Fe, exhibiendo un mayor nivel de bancarización. Sigue en importancia la provincia de Córdoba, pero con un comportamiento más heterogéneo que contrasta con el de la provincia de Entre Ríos, que aunque presenta un nivel de bancarización menor exhibe un comportamiento más homogéneo.}
