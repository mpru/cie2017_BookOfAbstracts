\A
{SELECCIÓN DE VARIABLES PARA LA MODELACIÓN DE LA DISTRIBUCIÓN POTENCIAL DE ECHINOPSIS ATACAMENSIS UTILIZANDO MAXENT}
{\Presenting{SILVIA SÜHRING}$^{1, 2}$\index{SÜHRING, S@SÜHRING, S}, JESÚS SAJAMA$^{1, 3}$\index{SAJAMA, J}, SERGIO FONTEÑEZ$^2$\index{FONTEÑEZ, S}, ANDREA BARRIONUEVO$^{1, 2, 3}$\index{BARRIONUEVO, A}, PABLO GOROSTIAGUE$^1$\index{GOROSTIAGUE, P} y PABLO ORTEGA-BAES$^1$\index{ORTEGA-BAES, P}}
{\Afilliation{$^1$LABORATORIO DE INVESTIGACIONES BOTÁNICAS (LABIBO), FAC. DE CS. NATURALES, U.N. DE SALTA}
\Afilliation{$^2$CÁTEDRA DE ESTADÍSTICA Y DISEÑO EXPERIMENTAL, FAC. DE CS. NATURALES, U.N.D E SALTA}
\Afilliation{$^3$CÁTEDRA DE CÁLCULO ESTADÍSTICO, SEDE REGINAL ORÁN, FAC. DE CS. NATURALES, U.N.DE SALTA}
\\\Email{ssuhring@gmail.com}}
{análisis de componentes principales; mapas de distribución potencial ajustada; cactaceae; colinealidad} 
 {Biología} 
 {Otras categorías metodológicas} 
 {144} 
 {311-1}
{Los modelos de distribución potencial permiten predecir las áreas de presencia de una especie con base en variables ambientales. MaxEnt es uno de los modelos más precisos, sólo requiere puntos de registro georeferenciados de la especie y coberturas geográficas de variables ambientales que podrían limitar su supervivencia. El programa asigna a cada celda de un mapa un valor continuo de probabilidad de aptitud de hábitat, luego se determina un valor umbral de corte para reclasificar esta probabilidad y obtener mapas de presencia/ausencia (mapas de distribución potencial ajustada, MDPA). En muchos casos la selección de variables depende de la disponibilidad de su representación geográfica. El número de variables incluidas en los modelos afecta las predicciones finales, en general el área predicha disminuye si se incluyen más variables. Por otro lado, algunas variables pueden resultar redundantes, dada la asociación que existe entre sus valores. En este trabajo se evaluó el efecto de la cantidad de variables incluidas en la modelación y de la colinealidad existente entre ellas, sobre los MDPA obtenidos para Echinopsis atacamensis, una cactácea columnar nativa de Argentina. Se comparó el rango geográfico, latitudinal, longitudinal y altitudinal de los MDPA, y se evaluó la similaridad de nicho utilizando el estadístico D de Warren. Se utilizaron las 19 variables climáticas tomadas de WORLDCLIM, referidas principalmente al comportamiento de la temperatura y las precipitaciones, y la altitud. Con estas variables se realizó un ACP para obtener componentes ortogonales. Se generaron siete modelos utilizando distintos conjuntos de variables: 1) todas; 2) excluyendo las correlacionadas con más variables; 3) excluyendo las correlacionadas con las que más contribuyeron al modelo 1; 4) las de mayor peso en el CP1; 5) las de mayor peso en CP1 y CP2; 6) las CP 1 a 4 (autovalores $>$1); 7) las CP 1 a 7 ($>$99\% de varianza explicada). A cada modelo se le aplicó el umbral de corte tal que incluya el 90 \% de los registros de presencia (percentil 10). Los MDPA correspondientes a los modelos 3 y 4 tuvieron mayor área (1,5 veces respecto del MDPA 1), y mayor rango altitudinal, latitudinal y longitudinal (1,1 a 1,46 veces respecto del MDPA 1). El área fue menor para los modelos que incluyeron mayor cantidad de variables. Todos los MDPA tuvieron una alta superposición de nicho (D $>$ 0,87), resultando mayor entre los MDPA 1, 2 y 5, los que además presentaron menor rango latitudinal y longitudinal.}
