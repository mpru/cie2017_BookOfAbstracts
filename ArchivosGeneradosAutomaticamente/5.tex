\A
{MUESTRAS ALEATORIAS CON UNIDADES MUESTRALES IMPUESTAS: ¿QUÉ HACER?}
{ERNESTO A ROSA\index{ROSA, E}}
{\Afilliation{UNTREF - DEPARTAMENTO DE METODOLOGÍA, ESTADÍSTICA Y MATEMÁTICA}
\\\Email{ear@fibertel.com.ar}
}
{muestreo; diseño muestral; muestras de objetivos múltiples; inferencia estadística} 
 {Otras aplicaciones} 
 {Muestreo} 
 {5} 
 {9-2}
{Con el objeto de analizar el funcionamiento del sistema público sanitario argentino, un organismo encargado de la promoción y financiamiento del mismo, encargó la realización en todo el país de un estudio muestral complejo (es decir, con varias etapas de muestreo), destinado a cubrir una variedad de unidades de muestreo diferentes (centros de salud, sus profesionales y empleados, los usuarios, etc.), es decir con objetivos múltiples. Finalizado el Diseño Muestral y seleccionadas las unidades básicas del estudio (los centros de salud), pero antes de iniciarse el “trabajo de campo”, algunos representantes provinciales del Ministerio de Salud, consideraron que, debido a su conocimiento del sistema sanitario provincial, estaban en condiciones de opinar en relación a los centros de salud seleccionados aleatoriamente para la Muestra, y proponer modificaciones en los que debían ser finalmente encuestados. Naturalmente el Muestrista se negó rotundamente a realizar modificaciones en los centros seleccionados, argumentando en relación a la aleatoriedad utilizada para hacerlo, las bondades y el respaldo teórico de los métodos basados en el azar, y todos los fundamentos provistos por la Inferencia Estadística, pero los funcionarios nacionales eran permeables a las presiones provinciales, y no estaban dispuestos a perder el apoyo de personas “expertas e influyentes” en las provincias, por mantener los “caprichos” teóricos del Muestrista contratado. Esta imposición, que influía de forma diferente en las diversas provincias cubiertas, impulsó al Muestrista a convocar a otros profesionales del Muestreo, ante los cuales expuso el problema, debatieron sobre el mismo, se rebelaron rotundamente contra las influencias externas, pero encontraron una forma simple de mantener la “pureza” de un Diseño Muestral aleatorio, con los compromisos de los funcionarios del organismo central, y los antojos y preferencias de los personajes provinciales: considerar a los centros de salud “impuestos”, como casos de inclusión forzosa (que ya existían por otras razones), y excluirlos de la población de referencia representada con la Muestra aleatoria. En este trabajo se expone sucintamente la forma en que se enfrentaron las imposiciones de los funcionarios, para lograr tener una muestra probabilística, con el único problema conceptual de tener que modificar los factores de expansión de cada una de las unidades seleccionadas al azar, para que representen a la población finalmente considerada. }
