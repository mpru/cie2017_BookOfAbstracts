\A
{MUESTREO DE POBLACIONES BERNOULLI: RELACIÓN ENTRE PARÁMETROS POBLACIONALES Y MUESTRALES EN PRESENCIA DE FUENTES DE VARIABILIDAD ADICIONALES AL MUESTREO}
{VIVIANA B LENCINA\index{LENCINA, V}}
{\Afilliation{INSTITUTO DE INVESTIGACIONES ESTADÍSTICAS, FACULTAD DE CIENCIAS ECONÓMICAS, UNT}
\\\Email{lencina\_viviana@yahoo.com.ar}
}
{población bernoulli; errores de respuesta; muestreo} 
 {Enseñanza de la estadística} 
 {Muestreo} 
 {24} 
 {66-1}
{En un muestreo probabilístico los parámetros de las variables observadas (muestra) se relacionan con los parámetros de la población muestreada, y es así como se usa la información de la muestra para inferir sobre la población. En este trabajo se aborda el problema de Inferencia en Poblaciones Bernoulli, se extiende la aplicación de la distribución binomial a situaciones donde no son idénticamente distribuidas las variables Bernoulli en la población objetivo y se muestra cómo se relacionan los parámetros poblacionales con los muestrales ante la presencia de errores de respuestas endógenos y exógenos. Por población Bernoulli en presencia de errores se entiende una población que posee una característica dicotómica que se presenta de manera aleatoria, errores de respuesta endógeno, o una característica dicotómica fija, pero que al introducirse un error exógeno en el proceso de medición se modifica su identificación. Específicamente se presenta cómo, en el proceso de muestreo, las diferentes fuentes de variabilidad involucradas, variabilidad intra individuo (errores de respuesta endógenos), inter individuo (errores de respuesta exógenos), o del proceso probabilístico para realizar el muestreo (errores muestrales) combinan los parámetros de las variables aleatorias que componen la población para obtener los parámetros de las variables realizadas en la muestra. Se describen situaciones prácticas donde queda justificada la utilización de la distribución binomial aún cuando el clásico modelo de urna no pueda aplicarse. Es importante resaltar que el análisis de la relación entre los parámetros muestrales y poblacionales es fundamental para poder especificar operacionalmente los objetivos de cualquier problema aplicado y seleccionar la estrategia de inferencia estadística a ser utilizada.}
